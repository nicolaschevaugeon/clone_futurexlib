

L'ensemble des lignes de codes de \code{xfem}, \code{xext} ou
\code{xfem} utilise des notions propre au C++, \`a la librairie STL ou
m\^eme la librairie boost. Pour les d\'ebutants en C++ qui ont des
notions de programmation dans un autre langage (en Fortran, par
exemple), cette partie a pour vocation d'introduire bri\`evement  ce
nouveau langage et ces librairies en d\'egageant se qu'il faut en
retenir.


Les habitu\'es du C++ peuvent passer cette section.

\section{Les grandes lignes}


\subsection{la synthaxe :}

Dans les exemples qui suivent, certains termes sont mis entre
crochets ( \code{< >}). Cette notation d\'esigne un param\`etre \`a
l'interieur d'une syntaxe. Par exemple, le type d'une variable not\'ee
\code{<type>} peut \^etre remplac\'ee par :
\begin{center}
\begin{tabular}{ p{3cm} p{9cm} }
\hline
Type      &      Description \\
\hline
\code{int} & entier   \\
\code{unsigned int} & entier positif   \\
\code{long} & entier double pr\'ecision  \\
\code{unsigned long} & entier double pr\'ecision positif   \\
\code{float} & r\'eel  \\
\code{double} & r\'eel double pr\'ecision \\
\code{char} & caract\`ere   \\
\code{string} & cha�ne de caract\`eres    \\
\code{bool}  & bool\'een    \\
\hline
\end{tabular}
\end{center}


\vspace{2em} le corps du texte s'\'ecrit en tenant compte des r\`egles
suivantes :
\begin{center}
\begin{tabular}{ p{3cm} p{9cm} }
\hline
Signe      &      Description \\
\hline
\code{;} & les lignes se terminent par un point-virgule \\
 \{ \hspace{1cm} \} & les fonctions sont \'ecrites entre accolades \\
\code{//} & les commentaires s'\'ecrivent apr\`es un double slash\\
\code{const} & devant une d\'eclaration, permet de d\'efinir une constante valable dans tout le programme  \\
\hline
\end{tabular}
\end{center}

\vspace{2em} Le programme principal s'\'ecrit comme une fonction :

\begin{verbatim}
                int main(int argc, char *argv[])
                {
                     ...
                }
\end{verbatim}



%\subsection{Les op\'erateurs d'affectation}
\vspace{2em} En plus des op\'erations classiques sur les variables, le
C++ poss\`edent quelques raccourcis syntaxique pour les  op\'erateurs
d'affectation :
\begin{center}
\begin{tabular}{ p{3cm} p{9cm} }
\hline
Op\'erateur      &      Action \\
\hline
a++      &      incr\'ementation de \code{a} \'equivalent \`a \code{a=a+1} \\
a\-\-             &      d\'ecr\'ementation de \code{a} \'equivalent \`a \code{a=a-1} \\
a += b             &      \'equivalent \`a \code{a=a+b} \\
b=a++      &       \code{b} re�oit la valeur \code{a} puis \code{a} est incr\'ement\'e \\
b=++a       &       \code{a} est incr\'ement\'e puis \code{b} re�oit la valeur \code{a}   \\
\hline
\end{tabular}
\end{center}



\vspace{2em}
%\subsection{Les op\'erateurs logiques}
les op\'erateurs logiques ressemblent  \`a ceux de nombreux codes :
\begin{center}
\begin{tabular}{ p{3cm} p{9cm} }
\hline
Op\'erateur      &      Signification \\
\hline
$==$             &      \'egal \`a\\
$!=$             &      diff\'erent de \\
$>$ \hspace{1cm} $<$      &      plus grand que, plus petit que       \\
$>=$ \hspace{.75cm} $<=$      &       plus grand ou \'egal que, plus petit ou \'egal que             \\
\&\&       &      ET logique (AND)\\
$\|\|$             & OU logique (OR)      \\
!             & contraire de {NOT}\\
\hline
\end{tabular}
\end{center}


\vspace{2em}
%\subsection{Les pointeurs}
Les pointeurs sont largement utilis\'es en C++. Leur utilisation   se
fait de la synthaxe suivante :
\begin{center}
\begin{tabular}{ p{3cm} p{9cm} }
\hline
Op\'erateur      &      Signification \\
\hline
\code{\&a}       &      adresse de  \code{a}) \\
\code{int* p}       &       d\'eclaration d'un pointeur de \code{int} nomm\'e \code{p}\\
\code{*p}      &       la chose point\'ee par  \code{p})\\
\code{<type>*}       &       d\'eclaration d'un pointeur de \code{<type>}\\
\code{*(...\&)}  &       = identit\'e. Le contenu de l'adresse d'une variable est la variable elle-m\^eme. \\
\hline
\end{tabular}
\end{center}


\vspace{2em}
%\subsection{Les controleurs}
les controleurs sont classiques. La syntaxe est la suivante :
\begin{center}
\begin{longtable}{ p{6cm} p{6cm} }
\hline
Contr�leur      &      Syntaxe \\
\hline
tant que... &  \code{while( \emph{ cond }) \{   }         \\
            &  \code{       \emph{action}   }            \\
            &  \code{                      \} }            \\
\\pour...   &  \code{for( int i=0 ; i < max ; i++) }       \\
            &  \code{       \emph{action}   }            \\
            &  \code{                      \} }            \\
\\ sortie de boucle pour   &  \code{break;}        \\
\\ contituation imm\'ediate de boucle pour   &  \code{continue;} \\
\\ selon... faire  &  \code{switch( nValeur)\{}            \\
                &  \code{    case \emph{val1}:     }            \\
                &  \code{                           ...; }            \\
                &  \code{                           break; }            \\
                &  \code{    case \emph{val2}:  }            \\
                &  \code{                           ...; }            \\
                &  \code{                           break; }            \\
                 &  \code{    defaut; ... }            \\
                &  \code{ \} }            \\
\hline
\end{longtable}
\end{center}



\subsection{Les fonctions :}

Le C++ ne conna�t que les fonctions. L'\'equivalent d'un
\emph{sous-programme} ou \emph{subroutine} serait une fonction de
type vide (\code{void}) dont  les param\`etres sont les variables de
passage. Une fonction s'\'ecrit sous la forme :
\begin{verbatim}
        <type_fcn> nom(<type_arg1> argument1,<type_arg2> argument2)
        {
            ...
        return <expression>
        }
\end{verbatim}
Une fonction peut \^etre utilis\'ee sans \^etre \'ecrite \emph{a priori} ;
le compilateur a alors besoin du \emph{prototype} de la fonction,
c'est \`a dire les lignes de d\'eclaration :
\begin{verbatim}
        <type_fcn> nom (<type_arg1> argument1,<type_arg2> argument2)
        {
            //  un petit commentaire peut \^etre le bienvenu tout de m\^eme...
        };
\end{verbatim}
Le \emph{prototype} doit \^etre inclus via la commande \code{\#include
"fonction"}   lorsqu'il est \'ecrit dans le fichier \code{fonction.h}.


La particularit\'e des fonctions en C++ est qu'elles ne modifient pas
la valeur des arguments, tout comme $y=f(x)$ fournit une valeur \`a
$y$ sans modifie $x$. C'est la valeur de $x$ au moment de l'appel
qui est transmise \`a $f()$. Pour pouvoir modifier une valeur il
convient de construire la fonction $f()$ en passant l'adresse de
$x$, par exemple, si \code{pVal} utilise l'adresse de \code{val}:
\begin{verbatim}
        void fcn(double* pVal)
        {
        };
\end{verbatim}
Ci-dessus, le param\`etre d\'eclar\'e \`a la fonction est un pointeur.

L'appel se fait par l'adresse de la valeur :
\begin{verbatim}
        double val=10.5
        fcn(val&)
\end{verbatim}


Une autre fa�on est la passage par \emph{r\'ef\'erence}, c'est \`a dire
par l'adresse (le pointeur). La r\'ef\'erence n'est pas \`a proprement
parl\'e un vrai type, mais une syntaxe particuli\`ere pour indiquer que
l'on peut modifier le contenu d'un param\`etre (argument) :
\begin{verbatim}
        void fonction(double& val)
        {
        }
\end{verbatim}
Ci-dessus, le param\`etre d\'eclar\'e \`a la fonction est une r\'ef\'erence \`a la
valeur.


L'appel se fait par la valeur elle-m\^eme :
\begin{verbatim}
        double val=10.5
        fcn(val)
\end{verbatim}


Il convient de faire attention \`a la port\'ee des variables internes
des fonctions:
\begin{itemize}
\item une variable locale d'une fonction est inconnue du programme principal ;
\item une variable locale \code{static}  est connue du programme principal ;
\item une variable locale d'une fonction d\'eclar\'ee par \code{new} est gard\'ee en m\'emoire est utilisable dans le programme principal (peut \^etre d\'etruite par la fonction \code{delete}).
\end{itemize}

\subsection{Les classes}

Il existe, comme en fortran, la notion de \code{structure}, mais est
et peu utilis\'ee car contenu dans la notion de \code{class}. Seule la
notion de \code{class} est donc \`a retenir. Les classes sont
largement utilis\'ees en C++. C'est la base m\^eme de la Programmation
Orient\'ee Object (POO).


La syntaxe de d\'eclaration d'une classe est la suivante :
\begin{verbatim}
        class <non_de_la_classe>
        {
            public:
            static <type_1> <nom_1> ;
            <type_...> <nom_...> ;
            ...
            <type_fcn1> <nom_fcn1>
            {
            }
            protected:
            <type_2> <nom_2> ;
            <type_...> <nom_...> ;
            ...
            <type_fcn2> <nom_fcn2>
            {
            }
            private:
            <type_3> <nom_3> ;
            <type_...> <nom_...> ;
            ...
            <type_fcn3> <nom_fcn3>
            {
            }
            ...
            ...
        };
\end{verbatim}
Les objets sont des instances d'une classe, tout comme une variable
est l'instance d'un r\'eel (par exemple), c'est \`a dire un \'el\'ement
d'un certain ensemble (type). Les objets peuvent donc \^etre point\'es
ou r\'ef\'erenc\'e pour une fonction comme n'importe quel autre type de
variable. On peut \'egalement affect\'e la m\'emoire sur le tas en
utilisant un \code{new} lors de la d\'eclaration \`a l'int\'erieur d'une
fonction.

\note{les membres de donn\'ees \code{static} sont des membres commun \`a
tous les objets de la classe (par exemple, un num\'ero qui peut \^etre
incr\'ementer \`a chaque cr\'eation d'objet). C'est un variable globale
qui doit \^etre initialiser dans le programme principal.}

L'utilisation d'une fonction \code{demarrer} ou variable publique
\code{longueur} d'un  objet \code{maVoiture} de classe
\code{vehicule} se fait  par l'interm\'ediaire d'un point \guil{.}
selon la synthaxe  \code{objet.fonction()}, par exemple:
\begin{verbatim}
        vehicule maVoiture;
        vmaVoiture.demarrer();
        maVoiture.longueur = 10.;
\end{verbatim}
Cependant, il est pr\'ef\'erable de travailler avec uniquement des
fonctions. L'id\'eal \'etant  d'avoir une fonction publique \code{fixeLongueur} (la variable \code{longueur} sera alors rendue \code{protected})
dans la classe du type \code{vehicule}  pour \'ecrire :
\begin{verbatim}
        vehicule maVoiture;
        maVoiture.demarrer();
        maVoiture.fixeLongueur(10.);
\end{verbatim}
les fonctions internes \`a la classe sont appel\'ees fonctions
\emph{membres}. L'ensemble des fonctions membres   d'une classe
d\'efinit l'\emph{interface} de la classe. Cette interface permet
l'utilisation de la classe, m\^eme si l'impl\'ementation des fonctions
\'evolue. Le prototype d'une classe peut \^etre (fortement conseill\'e) \'ecrit dans un fichier
inclus \code{*.h}. Ainsi, il est possible de d\'eclarer une fonction
membre sans la programmer. Elle est d\'eclar\'ee et utilisable. La
programmation pouvant se faire au fur et \`a mesure des cas \`a traiter
ou des d\'erivations de la classe. Le code source de la classe peut
\^etre \'ecrit dans un autre fichier \code{*.cc} comme n'importe quelle
autre fonction moyennant l'utilisation de sont nom complet  :
\begin{verbatim}        nomDeLaClassnom::DeLaFonction()  \end{verbatim}
(les fonctions membres peuvent \^etre surcharg\'ees si n\'ecessaire).


Lorsqu'on utilise le pointeur d'un objet, l'acc\`es aux fonctions
membres se fait par l'op\'erateur \code{->} :
\begin{verbatim}
        vehicule maVoiture;
        vehicule* ptmaVoiture = &maVoiture;

        ptmaVoiture->demarrer();        // est \'equivalent \`a  :
        maVoiture.demarrer();

        // alors que ceci est faut :
        *ptmaVoiture.demarrer();

        // mais ceci serait acceptable bien que peu employ\'e :
        (*ptmaVoiture).demarrer();
\end{verbatim}


Une fonction particuli\`ere appel\'ee \emph{constructeur} ou
\emph{op\'erateur parenth\`ese} permet d'initialiser un objet \`a �a
d\'eclaration.
\begin{verbatim}
        class vehicule
        {
            public:
            vehicule()
            {
                // en sortie d'usine : initialiser le compteur \`a z\'ero
                ...
            }
        }
\end{verbatim}
Il peut en exister plusieurs selon le nombre et le type d'argument pass\'es pour construire l'objet.


Une autre fonction particuli\`ere appel\'ee \emph{destructeur} ou
\emph{op\'erateur ~} permet de lib\'erer la m\'emoire contenant l'objet.
Elle doit g\'en\'eralement contenir la fonction \code{delete} si le
cr\'eateur contient  la commande \code{new}.


Encore une autre fonction particuli\`ere appel\'ee \emph{constructeur de
copie}  permet au C++ de construire une copie d'un object lorsqu'il
est pass\'e dans une fonction. Il permet, en fait, de construire la
r\'ef\'erence de l'objet et donc de faire une copie \guil{parfaite} de l'objet, en particulier lorsque la structure de l'object est complexe ou de taille variable, ce que ne ferait pas un constructeur par d\'efaut. Il s'\'ecrit :
\begin{verbatim}
        class vehicule
        {
            public:
            vehicule(vehicule& v)
            {
               ...
            }
        }
\end{verbatim}

La d\'eclaration des classes d\'eriv\'ees se fait de la mani\`ere suivante :
\begin{verbatim}
        class voiture : public vehicule
        {
           public:
            voiture(...) : ... , ... // ajouter les constructeurs
                                     // des classes composant
                                     // la classe voiture
            {...}
        }
\end{verbatim}


Pour beneficier du polymorphisme, les functions membres doivent \^etre
d\'eclar\'ee \code{virtuel}.
\begin{verbatim}
        class vehicule
        {
            public:
            vitual void FaireLePlein()
            ...
        };
\end{verbatim}
La fonction virtuelle sera alors d\'efinie au cas par cas en fonction
des classes d\'eriv\'ees (par exemple : vehiculeDiesel, vehiculeEssence,
vehiculeGPL,...) car la fa�on de FaireLePlein diff\`erent  en fonction
du type de v\'ehicule, alors que la fonction de FaireLePlein doit
toujours exister (en tout cas pour les v\'ehicules motoris\'es).


Les fonctions membres \code{statique}s et les constructeurs ne
peuvent pas \^etre d\'eclar\'ees virtuelle alors que les d\'estructeurs
devraient \^etre toujours virtuel.



Les classes \guil{abstraites} sont des classes qui ne peuvent \^etre
utilis\'ees telles quelles. Seule leurs classes d\'eriv\'ees peuvent \^etre
instanci\'ee. Ce sont des classes contenant une ou plusieurs
\guil{fonctions virtuelles pures}, c'est \`a dire ne contenant pas
d'impl\'ementation. Ce type de fonction est d\'eclar\'ees par :
\begin{verbatim}
        virtual void fonction_toto(...) = 0;
\end{verbatim}
Les classes \guil{abstraites}, m\^emes si elles ne peut pas \^etre
instanci\'ee, peuvent \^etre pass\'ee comme param\`etre de fonction car lors
de l'appel de cette fonction la classe r\'eellement pass\'ee sera une
sous-classes concr\`ete.


Les fichiers inclus d\'ecrivant les classes prennent la forme suivante
:
\begin{verbatim}
       #ifndef _MaClasse_
       #define _MaClasse_

       #include <...>
       #include "xToto.h"
       ...

       namespace xfem
       {
              class MaClasse {
              private:
                  ...
              public:
                  ...
              };
       }
       #endif
\end{verbatim}
Ici, le \code{namespace xfem} signifie que la classe est d\'efinie
dans la librairie \code{xfem}.

Le programme principale, quant \`a lui s'\'ecrira sous la forme suivante
:
\begin{verbatim}
       #include <...>
       #include "..."
       ...

       using namespace xfem;
       int main(int argc, char *argv[])
       {
             ...
       }
\end{verbatim}
Ici,  \code{using namespace xfem} pr\'ecise que les classes appel\'ee
sans pr\'ecision appartiennent \`a l'espace de nom \code{xfem}.

\section{Les \code{Templates}}

Les \code{Template} sont une fa�on de d\'eclar\'ee une classe (ou une fonction) de mani\`ere \guil{g\'en\'erique}, c'est \`a dire sans donner explicitement le type des variables ou classes utilis\'ees \`a l'int\'erieur.
Une classe \code{template <class T> class liste} permet de d\'efinir une classe de liste sans pr\'eciser le type du contenu de la liste. C'est un \guil{patron de classe}. Elle sera \guil{typ\'ee} lors de l'instanciation d'une liste particuli\`ere, par exemple :
\code{liste<int> ma\_liste\_d\_entier}.

La classe peut \^etre templatis\'ee plusieurs fois, si elle n\'ecessite l'utilisation de plusieurs types diff\'erents d'objets : \code{template <class T, class U> class A}.

(voir les livres sp\'ecialis\'es pour plus d'information)




\section{gcc \index{gcc}}

voir les site officiel de gcc :
\web{http://www.gnu.org/software/gcc/gcc.html}


\section{STL: Standard Template Library \index{Library!stl}}

The Standard Template Library (STL) is a software library included
in the C++ Standard Library. It provides \textbf{containers},
\textbf{iterators}, and \textbf{algorithms}. More specifically, the
C++ Standard Library is based on the STL published by SGI. Both
include some features not found in the other. SGI's STL is rigidly
specified as a set of headers, while ISO C++ does not specify header
content, and allows implementation either in the headers, or in a
true library.

The  STL provides many of the basic algorithms and data structures
of computer science. The STL is a generic  library, meaning that its
components are heavily parameterized: almost every component in the
STL is a template. You should make sure that you understand how
templates work in C++ before you use the STL. Containers and
algorithms

Like many class libraries, the STL includes container classes:
classes whose purpose is to contain other objects. The STL includes
the classes \textbf{vector}, \textbf{list}, \textbf{deque},
\textbf{set}, \textbf{multiset}, \textbf{map}, \textbf{multimap},
\textbf{hash\_set}, \textbf{hash\_multiset}, \textbf{hash\_map}, and
\textbf{hash\_multimap}. Each of these classes is a template, and
can be instantiated to contain any type of object.

SGI documentation can be found here:
\web{http://www.sgi.com/tech/stl/}

A large French documentation can be found here:
\web{http://www.moteurprog.com/Tutoriaux/VisiteTuto.php?ID_tuto=92}




\section{La librairie BGL: Boost Graphic Library}


\textit{(\`a faire...)}
