
\section{External Installation of \code{xfem} with CVS}

\subsection{Installation on a linux machine}

Remote access to ECN's CVS server is possible using ssh. To get
\code{xfem} library installed and functional on an external machine,
some other (\guil{downloadable}) tools are needed:
\begin{itemize}
    \item \code{boost} (\web{www.boost.org})
    \item \code{mtl} (\web{www.osl.iu.edu/research/mtl})
    \item \code{Trellis} (\web{www.scorec.rpi.edu/Trellis})
\end{itemize}
\begin{enumerate}
\item  Create your development directory (exp. \code{develop/} ). Then set your environment variables:
\begin{itemize}
\item  For versions sh, bash,.. :
\begin{verbatim}
export CVS_RSH=ssh export CVSROOT=:ext:public@cvs.ec-nantes.fr:/cvs
\end{verbatim}
\index{CVS!CVS RSH}\index{CVS!CVSROOT} In your \code{.bashrc},
define path to your development directory (exp:  \code{export
DEVROOT /scratch/user/develop} )

\item For versions csh, tcsh,.. :
\begin{verbatim}
setenv CVS_RSH ssh setenv CVSROOT :ext:$public@cvs.ec-nantes.fr:/cvs
\end{verbatim}


In your \code{.bashrc}, define path to your development directory
(exp:  \code{setenv DEVROOT /scratch/user/develop} )
\end{itemize}
\item Check out the needed tools:  (see \code{Tools} for description)

\begin{verbatim}
cd $DEVROOT cvs checkout Util
\end{verbatim}



\item Check out the library \code{xfem}:

\begin{verbatim}
cd $DEVROOT cvs checkout Xfem
\end{verbatim}

\item You need now to get the tools required for pre-compilation. Start by \code{Util} (which you've just installed)

\begin{verbatim}
cd $DEVROOT cd Util/buildUtil/buildUtil
\end{verbatim}

\begin{verbatim}
make setup NODEP=1 make
\end{verbatim}

\item Compile Loki:

\begin{verbatim}
cd $DEVROOT cd Util/Loki/Loki
\end{verbatim}

\begin{verbatim}
make setup NODEP=1 make
\end{verbatim}

\item Download boost and \guil{unzip} it in \code{/usr/local/include/} :
\begin{verbatim}
cp your-boost-version.tar.gz /usr/local/include/ cd
/usr/local/include/ tar -xzvf your-boost-version.tar.gz
\end{verbatim}


Create the link that will allow xfem find boost's includes (\code{in
/usr/local/include/}):

\begin{verbatim}
ln -s your-boost-version/boost boost
\end{verbatim}

\item Download mtl and install it in a repository \code{Solver/} :

\begin{verbatim}
cd $DEVROOT mkdir Solver
\end{verbatim}

\begin{verbatim}
cd Solver/ tar -xzvf your-mtl-version.tar.gz
\end{verbatim}
Then compile it:

\begin{verbatim}
cd your-mtl-version ./configure make make install
\end{verbatim}

As for boost, create a link to the repository where mtl is
installed:
\begin{verbatim}
cd $DEVROOT/Solver ln -s your-mtl-version mtl
\end{verbatim}

\item  \code{C++} includes \code{hash\_fct}, \code{hash\_map} and \code{hash\_set} are available as \code{RPM} or \code{tar}
archives. Copy them in \code{/usr/local/include}

\item Download and unzip Trellis:

\begin{verbatim}
cd $DEVROOT tar -xzvf your-Trellis-version.tar.gz
\end{verbatim}

Compile the needed libraries (AOMD and Model)

\begin{verbatim}
cd Trellis/Trellis/model/model make setup NODEP=1 make
\end{verbatim}

\begin{verbatim}
cd Trellis/Trellis/AOMD/AOMD make setup NODEP=1 make
\end{verbatim}
\end{enumerate}



\subsection{Installation on a Windows machine:}
\begin{enumerate}
\item Install cygwin (www.redhat.com )
\item  In a cygwin shell, follow the previous commands
\end{enumerate}








\section{Le serveur CVS de l'ECN (pour administrateur)}

Le serveur CVS est situ\'e \`a l'adresse : \code{cvs.ec-nantes.fr}. Il
s'agit d'une machine d'architecture Intel avec un syst\`eme
d'exploitation standard Red Hat linux 9.

\subsection{Syst\`eme CVS}
Le syst\`eme CVS est utilis\'e pour le d\'eveloppement de code. Il est un
outil de collaboration pour les d\'eveloppeurs travaillant sur des
projets communs. De plus, il permet de garder une trace de toutes
les modifications effectu\'ees (stockage incr\'emental). Du point de vue
utilisateur, son interface est standard \`a partir du moment ou
l'authentification est faite, nous ne nous attarderons pas sur ce
point dans la suite. Un manuel d'utilisation est disponible \`a
l'adresse suivante :

\web{ http://www.cvshome.org/}.

L'authentification est faite par un acc\`es ssh, du point de vue
utilisateur (sur linux ou unix) il faut simplement d\'efinir les
variables d'environnement suivantes :

\begin{verbatim}
CVSROOT = :ext:<point d'acc\`es au serveur CVS>
\end{verbatim}


\begin{verbatim}
CVS_RSH=ssh
\end{verbatim}


Le point d'acc\`es au serveur CVS est d\'ependant du mode d'acc\`es et
sera d\'etermin\'e dans la suite. Lors de l'acc\`es au serveur, un mot de
passe sera \'eventuellement demand\'e pour toute commande, par exemple
\code{cvs checkout} ou encore \code{cvs commit} . Pour \'eviter la
demande syst\'ematique d'un mot de passe il faut implanter dans le
serveur CVS les clef de la machine locale. Ceci peut \^etre fait \`a
l'aide d'une commande sp\'eciale.

\subsection{Implantation du serveur CVS \`a l'ECN}

Le serveur CVS est dot\'e d'un certain nombre de scripts  (voir le
fichier \code{/usr/local/cvs/doc/README} sur le serveur CVS). Il
s'agit d'un serveur s\'ecuris\'e permettant un acc\`es anonyme sur
certains projets, et un acc\`es contr�l\'e sur d'autres. La protection
des donn\'ees et de l'int\'egrit\'e du serveur sont assur\'es de deux fa�ons
: d'une part par une authentification et des \'echanges de donn\'ees
crypt\'es (seul le protocole ssh est utilis\'e lors des requ\^etes
r\'eseau), et d'autre part par un masquage du syst\`eme de fichiers
racine de la machine ( commande \code{chroot} ). Les scripts d\'ecrits
dans la suite sont invoqu\'es pour les actions suivantes : cr\'eation
d'un nouveau projet, l'ajout d'utilisateurs aux droits restreints \`a
un seul projet, la cr\'eation d'un module dans un projet
(sous-projet). Cette derni\`ere action peut aussi \^etre effectu\'ee
directement par un utilisateur ayant des droits d'\'ecriture dans un
projet. Des utilisateurs privil\'egi\'es peuvent avoir un acc\`es \`a
plusieurs modules \`a la condition de ne pas faire de \code{chroot}.
Ceci sera explicit\'e dans la suite.

\subsection{Installation}

Dans le cas d'une r\'einstallation, il faut utiliser l'archive
(version 1.5) des scripts d'administration:
\code{chrooted-ssh-cvs.1.5.tar.gz} et le d\'ecompresser dans le
r\'epertoire \code{/usr/local} :
\begin{verbatim}
cd /usr/local tar -zxf <chemin>/chrooted-ssh-cvs.1.5.tar.gz
\end{verbatim}

Une documentation est fournie avec (\'equivalente \`a
\code{/usr/local/cvs/doc/README} sur le serveur CVS) :
\code{chrooted-ssh-cvs.1.5.README}


\subsection{Modes d'acc\`es}
Il existe 2 modes d'acc\`es, chacun permettant un acc\`es en lecture
uniquement ou en lecture et \'ecriture. Le choix est d\'etermin\'e \`a la
cr\'eation de l'utilisateur mais peut \^etre modifi\'e ult\'erieurement.
\begin{enumerate}
\item  \underline{mode restreint :}
Ce mode est destin\'e aux utilisateurs d'un seul projet et disposant
de droits de lecture (\'eventuellement d'\'ecriture) dans ce projet. La
connexion est accord\'ee apr\`es authentification par \code{ssh} sous
forme d'un acc\`es \`a certains outils dans un environnement restreint
(apr\`es \code{chroot}) : \code{cvs} (commande serveur), et quelques
outils de modification de mot de passe (\code{password}) et d'import
de clef cryptographique pour acc\`es sans mot de passe
(\code{authorize}). Il n'y a pas de shell accessible. Dans ce mode
d'acc\`es, le \code{<point d'acc\`es au serveur CVS>} prend la forme
suivante:

\begin{verbatim}
CVSROOT=:ext:<user>@cvs.ec-nantes.fr:/cvs
\end{verbatim}



Le projet est d\'etermin\'e par \code{<user>} (dans ce mode, l'usager
n'a acc\`es qu'a un seul projet). C'est le mode d'acc\`es le plus
s\'ecuritaire, il est en particulier possible de l'utiliser en lecture
seule pour un acc\`es anonyme aux projets de nature libre (au sens
logiciel libre)

\item \underline{mode \'etendu :}
Ce mode permet \`a un utilisateur d'acc\'eder \`a un ou plusieurs projets,
en lecture ou lecture et \'ecriture. De ce fait, cet utilisateur ne
passera pas par une \'etape de\code{chroot} mais sera tout de m\^eme
authentifi\'e par \code{ssh}. Cet acc\`es ne devrait jamais \^etre utilis\'e
pour les acc\`es anonymes. Si possible (et en particulier si un usager
ne d\'esire travailler que sur un projet), il est pr\'ef\'erable
d'utiliser le mode restreint. Dans le mode d'acc\`es \'etendu, le
\code{<point d'acc\`es au serveur CVS>} prend la forme suivante:


\begin{verbatim}
CVSROOT=:ext:<user>@cvs.ec-nantes.fr:/usr/local/cvs/<proj>/chrooted-cvs/cvs
\end{verbatim}


Ici l'usager \code{<user>} ne d\'etermine pas le projet, c'est la
variable \code{<proj>} qui permet de s\'electionner le projet dans
lequel il d\'esire travailler. Selon la configuration du serveur, les
usagers \'etendus peuvent avoir acc\`es \`a un shell et donc \`a toutes les
commandes unix. Les projets sont en principe accessibles en lecture,
mais l'acc\`es en \'ecriture n\'ecessite d'appartenir au groupe du projet
en question.
\end{enumerate}



\subsection{Administration}
Toutes les actions d'administration se font en tant que \code{root}
sur la machine cvs :

\begin{verbatim}
ssh -X root@cvs.ec-nantes.fr
\end{verbatim}



\subsubsection{Cr\'eation d'un nouveau projet}

La cr\'eation d'un nouveau projet se fait \`a l'aide du script
\code{make-project}, celui-ci se chargeant de faire toutes les
actions n\'ecessaires : cr\'eation de groupes, de r\'epertoires, et mise
en place de l'espaces restreint (pour \code{chroot}). La syntaxe de
la commande est la suivante pour la cr\'eation du projet \code{<proj>}
:
\begin{verbatim}
/usr/local/cvs/sbin/make-project [-a] <proj>
\end{verbatim}

L'option \code{-a} permet l'acc\`es anonyme sur ce projet. Il est
possible de modifier cela par la suite (droits de lecture des
fichiers par tout utilisateur). Il est pr\'ef\'erable de mettre cette
option par d\'efaut puisque l'authentification est faite
syst\'ematiquement. Pour une aide en ligne sur cette commande :

\begin{verbatim}
/usr/local/cvs/sbin/make-project -h
\end{verbatim}

Note : Certains noms de projet sont interdits pour des raisons de
conflits possibles avec des r\'epertoires existants, il s'agit de :

\begin{verbatim}
chrooted-cvs home sbin doc src
\end{verbatim}

\subsubsection{Destruction d'un projet existant}

La destruction du projet \code{<proj>} se fait en deux temps
\begin{enumerate}
\item  Destruction de tous les usager \`a acc\`es restreint sur ce projet (la liste est disponible en faisant la
commande : \code{cat /etc/passwd $\vert$ grep \guil{Project <proj>}}
) voir commande de destruction plus bas
\item  Destruction du projet
\begin{verbatim}
rm -rf /usr/local/cvs/<proj> groupdel <proj>
\end{verbatim}

\end{enumerate}

\subsubsection{Cr\'eation d'usagers \`a acc\`es restreint}
La cr\'eation d'un usager \`a acc\`es restreint se fait gr�ce au script
\code{make-user}. Un usager \`a acc\`es restreint ne peut acc\'eder que \`a
un projet \`a la fois. La syntaxe est la suivante:
\begin{verbatim}
/usr/local/cvs/sbin/make-user <proj> [-ro|-rw] <user>
\end{verbatim}
La variable \code{<proj>} est \`a remplacer par le nom du projet
(existant) et la variable \code{<user>} par le nom d'usager
(inexistant). Les options \code{-ro} et \code{-rw} servent
respectivement \`a sp\'ecifier un acc\`es en lecture seule ou en lecture
et \'ecriture. Ceci peut \^etre modifi\'e \`a la main par la suite en
pla�ant l'usager dans le groupe du projet ou non (commande
\code{usermod}), mais il est pr\'ef\'erable de d\'etruire le compte et de
le recr\'eer.


\subsubsection{Modification d'usagers \`a acc\`es restreint}
La fa�on la plus simple de modifier un usager \`a acc\`es restreint
(pour changer le projet sur lequel il a des droits par exemple)
consiste \`a le d\'etruire et \`a le recr\'eer avec les bons arguments. La
destruction d'un usager restreint est identique \`a la destruction
d'un usager \`a acc\`es \'etendu :
\begin{verbatim}
userdel -r <user>
\end{verbatim}

\subsubsection{Cr\'eation ou modification d'usagers \`a acc\`es \'etendu}
Ces usagers ne sont pas cr\'e\'es ou modifi\'es par un script mais doivent
l'\^etre � \`a la main � par des commandes unix standard. Le fait
d'appartenir \`a certains groupes conditionne l'acc\`es en lecture ou en
lecture et \'ecriture aux projets. La cr\'eation est faite avec la
commande \code{useradd}, et la modification avec \code{usermod}. Un
exemple vaut mieux que mille mots : Cr\'eation de l'usager \code{toto}
avec acc\`es en \'ecriture aux projets \code{public} et \code{fissure} :
\begin{verbatim}
useradd -G cvsdev,public,fissure toto
\end{verbatim}


Modification : toto a de plus acc\`es en \'ecriture au projet ecn :
\begin{verbatim}
usermod -G cvsdev,public,fissure,ecn toto
\end{verbatim}

Maintenant, toto n'a plus acc\`es a rien (mais on veut garder
l'usager):
\begin{verbatim}
usermod -G "" toto
\end{verbatim}

On redonne de nouveau l'acc\`es en lecture \`a tous les projets :
\begin{verbatim}
usermod -G cvsdev toto
\end{verbatim}

On d\'esire d\'etruire l'usager toto :
\begin{verbatim}
userdel -r toto
\end{verbatim}

% \subsubsection{Configuration des usagers}
% Ces actions se font en dehors du serveur cvs, par l'usager, mais elles sont sp\'ecifiques au serveur cvs
% de l'ECN.

\subsubsection{Acc\`es sans mot de passe}\label{acces_sans_passwd}
Il est parfois fastidieux de rentrer un mot de passe pour tout acc\`es
au serveur CVS. Il existe un moyen de mettre \`a jour le fichier de
clefs permettant une authentification sans mot de passe. Il faut
d'abord cr\'eer la clef sur la machine cliente; ceci se fait par la
commande suivante :

\begin{verbatim}
ssh-keygen -t -dsa
\end{verbatim}


Par l'interm\'ediaire de l'utilitaire authorize, il est donc possible
de transmettre cette clef, ceci se fait de la mani\`ere suivante:
\begin{verbatim}
ssh <user>@cvs.ec-nantes.fr authorize < ~/.ssh/id_dsa.pub
\end{verbatim}

Une fois cette action effectu\'ee, plus aucun mot de passe ne sera
demand\'e (en fait, uniquement pour les clients sur lesquels l'action
a \'et\'e faite). Pour les usagers \'etendus, il faudra en plus mettre les
bons droits d'acc\`es du fichier
\code{$\thicksim$/.ssh/authorized\_keys2} :

\begin{verbatim}
ssh <user>@cvs.ec-nantes.fr "chmod 600 ~/.ssh/authorized_keys2"
\end{verbatim}

Pour effacer les clefs d\'ej\`a entr\'ees (et revenir \`a une
authentification par mot de passe pour tous les clients), il suffit
de faire :
\begin{verbatim}
ssh <user>@cvs.ec-nantes.fr authorize delete
\end{verbatim}

Modification de mot de passe La modification de mot de passe se fait
avec la commande password :
\begin{verbatim}
ssh <user>@cvs.ec-nantes.fr password
\end{verbatim}

Attention, le mot de passe entr\'e s'affiche en clair (il est
toutefois transmis sous forme crypt\'ee) Quelques commandes usager
utiles pour initialiser un projet Ces commandes se font cot\'e usager
et permettent de manipuler un projet. Ce sont des commandes standard
que toute installation cvs reconna�t. Pour plus d'information, voir
le site : \web{http://www.cvshome.org/} Charger le contenu du
r\'epertoire courant dans le projet en cours, dans le sous-projet
nomm\'e \code{<sproj>} :
\begin{verbatim}
cvs import <sproj> vendor_tag release_tag
\end{verbatim}

R\'ecup\'erer le sous-projet \code{<sproj>} (pour un nouvel usager par
exemple) :
\begin{verbatim}
cvs checkout <sproj>
\end{verbatim}



