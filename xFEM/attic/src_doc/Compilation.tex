\section{The library  \code{Util}\index{Library!Util}}\label{Util}

\textnormal{(contribution de Y. Guetari)} 

The library \code{Util/} contains:
\begin{itemize}
\item The library \code{buildUtil/} contains utilities for easier library and executable building.
\item specific utility scripts for ECN (\code{ecnUtil/}\index{Library!ecnUtil}).
\end{itemize}




\subsection{Use of \code{buildUtil} \index{Library!buildUtil}: Basic
operations}

The library \code{Util} provides utilities for developpers. These
utilities are accessible with the \code{make} command. This command
\code{make}  compile sources files and links them  to create a executable. In order to help developpers, new commands have been
added:\index{Compilation!standard}

\begin{center}
% use packages: array
\begin{longtable}{p{7.75cm}  p{7.75cm}}
\code{make distclean NODEP=1}   & To fully clean a distribution. Compulsory if a \code{.h} file is added or removed \\
& \\
\code{make setup NODEP=1}   & To create links before compiling \\
& \\
\code{make}             & To actually compile and create the library \\
& \\
\code{make VERS=opt}        & Same as above but with optimization flags \\
& \\
\code{make check}       & To check all test cases. See results in \code{verif\_result.txt} \\
& \\
\code{make check VERS=opt}      & To check all test cases with the optimized code version \\
& \\
\code{make checkO}      & equivalent as \code{make check VERS=opt}    \\
& \\
\code{make checkone DIR=<directory>}   &    To check only the <directory> problem, depending on its location. See results in \code{verif\_result.txt} \\
& \\
\code{make checkone VERS=opt DIR=<directory>}   &    same as above in the optimized version\\
& \\
\code{make checkone\_update DIR=[directory]}   &    To check only the [directory] problem,  and update the reference file with the result file  \\
& \\
\code{make checkone DIR=test/toto VERS=opt}     &   Same as above but with the optimized code version \\
& \\
\code{make doc}     &   create the Doxygen documentation of the module (library).\\
& \\
\code{make latex}     &   compile this LaTeX documentation (in \code{\$DEVROOT/Xfem/Xfem/Xfem/src\_doc}).
\end{longtable}
\end{center}


You must add the option \code{METIS=1} for compilation
\index{Compilation!cluster} on the cluster \code{Pegase}.


\subsection{Utility scripts of \code{ecnUtil}}\label{ecnUtil} \index{Library!\code{ecnUtil}}


Specific tools are developed at ECN to automate some tasks related
to directories management. They are located in
\code{/scratch/softserver/LATEST/Util/ecnUtil/} and can be called
like the following example:
\begin{verbatim}
cd /scratch/<username>
/scratch/softserver/LATEST/Util/ecnUtil/<script>
\end{verbatim}

where \code{<script>} is one of the scripts discribed in the
following table:

\begin{center}
% use packages: array
\begin{longtable}{p{4cm} p{5.2cm} p{5.4cm}}
\textbf{Script name} & \textbf{Descriptio}n &   \textbf{Use} \\
& & \\
 \code{create-devel}  & Creates an executable test-case under
the subdirectory \code{devel/} of the current module. The new
test-case will contain an empty skeleton of a test-case with:
\begin{itemize} \item an empty directory \code{data/}
\item two minimal files \code{main.cc} and \code{main.h}
\item an empty directory \code{reference/}
\item the link directory \code{SVN/} (not to be touched)
\end{itemize} & \begin{enumerate} \item Go to your subdirectory \code{devel/} : \code{cd devel/}
\item  Type out: \code{create-devel [name]}
\item The system will ask you for your password for every new item to be created  \end{enumerate} \\
& & \\
\code{create-module} & Creates a new module (\code{appli}) in the
existant repository (e.g. \code{My\_First\_Module/} in the
repository \code{Applis/}) The new module will contain the minimal
components of the new appli  &
\begin{enumerate}
\item Go to the subdirectory \code{develop/}
\item Type out: \code{create-module Applis [name]}
\item  The system will ask you for your password for every new item to be created
\end{enumerate} \\
& & \\
\code{move-devel-to-test} & Moves a test-case directory from \code{devel/} to \code{test/} so that it is automatically executed when a \code{make check} is made & TO BE TESTED\\
& & \\
\code{update-all-reference} & Replaces each files in \code{reference} by the corresponding file in \code{results}.  &  TO BE TESTED \\
& & \\
\code{update-reference} & Same as above but only for the current
directory \code{reference/} &  TO BE TESTED
\end{longtable}
\end{center}


To easily use the development environment, you need to call options
defined in bashrc.partial. 
A canevas of the configuration file
\code{.bashrc} is available in the home directory. There's no need to get the
whole file bashrc.partial, but the path to the development directory
must (at-least!) be specified in it (\code{export
DEVROOT=<development directory>}, for example: \code{export
DEVROOT=/scratch/\$USER/develop}). Otherwise, compilation wouldn't
be possible in this directory (see the whole instructions in
installation).





\section{Compiling and executing a test-case}




To compile \index{Compilation!standard} and execute a program
located in a test-case directory (for instance
\code{Xfem/Xfem/Xfem/test/} for the \code{xfem} library). 
Go to the
directory containing the \code{Makefile} file and   perform the
following sequence of instructions:

On standard server (\code{mosix1, mosix2, ...}):
\begin{itemize}
\item First compile the library (upper level: the module):
    \begin{itemize}
    \item[1.] \code{make distclean NODEP=1}
    \item[2.] \code{make setup NODEP=1}
    \item[3.] \code{make}
    \end{itemize}
\item Then compile your test-case:
    \begin{itemize}
    \item[4.] \code{make checkone DIR=test/<test-name>}
\end{itemize}
\end{itemize}
(Here you did use the \code{Util}  library (see~\ref{Util}) by using
specific keywords with the \code{make} command.)\\


On a HPC cluster  \index{Compilation!cluster}
(\code{pegase}(\code{master0})):
\begin{itemize}
\item First compile the library (upper level: the module):
    \begin{itemize}
    \item[1.] \code{make distclean NODEP=1}
    \item[2.] \code{make setup  METIS=1 NODEP=1}
    \item[3.] \code{make METIS=1 }
    \end{itemize}
\item Then compile your test-case:
    \begin{itemize}
    \item[4.] \code{make checkone  METIS=1 DIR=test/<test-name>}
\end{itemize}
\end{itemize}
You can also use the option \code{make -j$n$} (with no space) to
execute the \code{make} command on $n$  processors. This make a
parallele compilation on several nodes of the cluster
(\code{profile} file needed) or a much faster compilation even on other servers (mosix1, mosix2, ...).





\subsection{Application example} \index{installation!test}

A typical example of development is presented here. It describes a
test-case existing in \code{Xfem/Xfem/Xfem/test/} \index{Xfem!test}
. In this example, a boundary condition (L2 projection) is imposed
on a border of a 2D domain. Watch here the files organisation. It is
the typical structure of the directories you may create  in your
\code{devel/} directory.\\

\begin{description}
\item[] mechanics\_bc\_l2
   \begin{description}
	\item[$\llcorner$]main.cc  : C++ file containing the main source code
	\item[$\llcorner$]main.h : file containing the declaration of the C++ classe
	\item[$\llcorner$]main : executable file of the application
	\item[$\llcorner$]data
		\begin{description}
		\item[$\llcorner$] law.mat : material parameters
		\item[$\llcorner$] main.dat : data describing the problem to treat
		\item[$\llcorner$] square\_5.msh : the mesh file (gmsh format)
		\item[$\llcorner$] .svn : automatically created by SVN (don't modify or copy it)
		\end{description}  
	\item[$\llcorner$]reference:
		\begin{description}
		\item[$\llcorner$] DISPLACEMENT.pos : result filechosen as a reference
		\item[$\llcorner$] .svn
		\end{description}  
	\item[$\llcorner$]results:
		\begin{description}
		\item[$\llcorner$] DISPLACEMENT.pos : result file of the local version of the application
		\item[$\llcorner$] dcl.dbg : debugging file created by main.cc
		\item[$\llcorner$] ess.dbg : same as above 
		\item[$\llcorner$] rcond.txt  : same as above 
		\item[$\llcorner$] sym.dbg : same as above 
		\item[$\llcorner$] res.dbg : same as above 
		\end{description}  
   \end{description}
\end{description}







\subsection{Create or modify an application}

The best way (recommended) is to create a new reposity in the SVN
module \code{Applis/}. Make a copy of an existing application which
will be the starting point of your developments, or modify an
existing application you want to improve.



\begin{enumerate}
\item Start by creating your test subdirectory. Choose your location (directory \code{devel/} or 		\code{test/}) and use
	the script \code{create-devel} to start with (details in
	section~\ref{ecnUtil}).
\item Open \code{emacs} (see~\ref{sec:emacs}) and create/modify your \code{.geo} file, save it in 		\code{data/}.
\item Now open a terminal and go to your \code{data/} subdirectory. Launch \code{gmsh}   (by command: 		\code{gmsh \&}, see~\ref{gmsh})
\item In \code{gmsh}, open your \code{.geo} file and create the corresponding mesh file. Choose \guil{mesh} 	mode and generate a \guil{2D} mesh. Save the resulting file. You won't have to specify a name or 	saving location for it. It is automatically saved in the subdirectory from which you opened your 	\code{.geo} and will have the same name.
\item Go back to the test directory and create \code{main.cc}. Make all required includes. Save.
\item Run this program (first compile dependent libraries), for that use the following command :
\begin{verbatim}
      make -C <path_of_the_Makefile_is> setup
      make -C <path_of_the_Makefile_is> 
\end{verbatim}
For a better visibility of the compilation error, use emacs and execute the command in the dialog box of emacs (or command \touche{F8}, make sure that the command above is writen and \touche{return} : see~\ref{DevelopEnvironTool} for more details (depending on the  file\code{.emacs})). 
\item  A subdirectory \code{results/} is generated in the  test-directory. Open it to see the results you asked for.
\item Choose among these results files the ones you would like to use as reference, copy them into subdirectory \code{reference/}.
\end{enumerate}




\subsection{Committing a modification of a library  to the Community}\label{common_commit}

As many users develop individually at the same time, one has to get
sure his modifications are functional \guil{harmless} before
committing them into the basic source.


A special sequence of operations is needed to be performed before
making SVN commit\index{CVS!commit}. This sequence ensures that the
manipulation will not be unsafe for the common version.

\begin{enumerate}
\item  Update the local version with svn-server's one, this will merge the local modifications with the external committed ones:
\begin{verbatim}
   svn update  <library>
\end{verbatim}
This operation \guil{patches} the modifications of the SVN version
to your own version. These patches are possible where you didn't
modify at time the same part of the source files than the ones you
patch. Else, a conflict is detected and the patches is added with
\code{+} and \code{-} character to indicate the different
modifications, yours against the SVN one. 
You are unlucky ! You have to resolve the conflict before committing and indicat to SVN that the problem is resolved by:
\begin{verbatim}
   svn resolved  <file>
\end{verbatim}
If the \code{update} was successful :
\item  Compile and run test-cases: (description)\index{run}
\begin{verbatim}
   make distclean NODEP=1
   make setup NODEP =1 
   make
\end{verbatim}

If no problem is encountered doing this, do:

\begin{verbatim}
   make check
\end{verbatim}

The last command compiles and executes all test-cases existing in
the subdirectory \code{test/}.

\item  \code{verif\_result.txt} is a file that records a comparison between the results files this execution has
generated and the reference \index{test!reference}
results files stocked in the directory \code{reference/}.\index{test!results} \index{file!reference}

Check (carefully!) the result of this execution in
\code{verif\_result.txt}. If this comparison is conclusive, you can
commit to the svn-server's files:\index{file!result}

\item  Commit the common files:
\begin{verbatim}
   cdd
   svn commit <library>
\end{verbatim}
where \code{<library>} is \code{Xfem}, \code{Xext}, \code{Xcrack} or \code{Applis}.
\end{enumerate}

The repository gets then the committed files. They will be compiled and
executed automatically by night by softserver.
