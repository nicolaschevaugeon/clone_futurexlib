



\section{Libraries}

This chapter gives a bref description of the libaries  \code{/AOMD},
\code{/Solver}, \code{/STL} and  \code{/Treillis}.


\subsection{\code{Treillis/AOMD/}: Algorithm Oriented Mesh Database library \index{Library!AOMD}\index{Library!Trellis} }

Contains the meshing library used to work with \code{xfem} .

\paragraph*{Description}

 The aim of the Algorithm Oriented Mesh Database (AOMD) is to be a mesh management library (or database) that is able
  to provide a variety of services for mesh users. The optimal form of the mesh representation is application dependant
  with different applications requiring different sets of mesh adjacencies.

 AOMD supports hybrid meshes (Triangles, Quads, Hexes, Tets, Prisms and Pyramids). The parallel paradigm used is
 the Rensselaer  Partitioning Method (RPM).

AOMD is written in C++ and it uses STL as well for containers,
iterators and algorithms. Parallel communications are made using the
Message Passing Interface (MPI). Optimal message packing is made
using the autopack library that was developed at Argonne National
Labs by Ray Loy.

AOMD provides advanced services like automatic mesh refinement and
coarsening. Mesh refinement introduces load unbalance in partitions.
This unbalance is not acceptable if one wants to achieve scalable
parallel software. The solution to this unbalance is to dynamically
re-partition the mesh. Some load balancing libraries are available
on the web. The library Zoltan from Sandia is a package which
includes four load balancing libraries based on both graph and
octree partitioning. Classically, the balancer takes as input a
representation of the parallel mesh (octree or partitioned graph)
and provides as output a partition vector telling on which partition
a given mesh entity has to be in order to restore the load balance.
The completion of dynamic re-partitioning consists of dynamically
moving the appropriate entities from one partition to another.

For DOXYGEN documentation:
\web{http://www.scorec.rpi.edu/AOMD/Doc/html/index.html}


\subsection{\code{Solver/} \index{Library!Solver}}

Contains:

\begin{itemize}
\item  The Iterative Template Library (ITL):   \web{http://www.osl.iu.edu/research/itl/}
\item  The Matrix Template Library (MTL):   \web{http://www.osl.iu.edu/research/mtl/}
\item the Portable, Extensible Toolkit for  Scientific Computation (PETSc):
 \web{http://www-unix.mcs.anl.gov/petsc/petsc-as/index.html}
\item The  SuperLU library (SuperLU): \web{http://crd.lbl.gov/~xiaoye/SuperLU/}
\item The  Taucs: a Library of Sparse Linear Solvers \web{http://www.tau.ac.il/~stoledo/taucs/}
\end{itemize}

 the solver's library Sparskit used for \code{xfem}:

\web{http://www-users.cs.umn.edu/~saad/software/SPARSKIT/sparskit.html}

\subsection{\code{Xext/} \index{Library!Xext}}

This library contains general developments for \code{xfem} which are
only available for ECN users.

\subsection{\code{Xcrack/} \index{Library!Xcrack}}

This library contains  developments for crack propagation and is
only available for ECN users.
