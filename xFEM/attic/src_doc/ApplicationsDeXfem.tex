

\section{Applications et Mise en oeuvre}
\label{sec:applications-et-mise}

L'impl\'ementation de l'approche X-FEM dans un code \'el\'ements
finis standard requiert un certain nombre d'extensions
du code cible. Ces extensions concernent au minimum les points
suivants :
\begin{itemize}
\item La gestion des level sets ;
\item L'int\'egration des matrices et des vecteurs \'el\'ementaires ;
\item La s\'election des degr\'es de libert\'e \`a enrichir ;
\item Le stockage de ces degr\'es de libert\'e ;
\item Le post-traitement.
\end{itemize}
Chacun de ces points est explicit\'e ci-dessous.
Bien entendu, le travail sp\'ecifique \`a r\'ealiser dans un code
d\'epend de son architecture et il est d\'elicat de tirer
des r\`egles g\'en\'erales.



Dans l'approche X-FEM, certaines surfaces physiques
(voire toutes) vont \^etre localis\'ees sur le maillage \`a l'aide
d'un champ \'el\'ements finis. Le code doit donc \^etre capable de
g\'erer \`a la fois des champs physiques (vitesse, pression, ...)
et des champs dont l'iso-z\'ero localise une surface
physique. La possibilit\'e d'extraire ces iso-z\'eros est
importante
si l'on veut \^etre capable d'imposer par exemple
une pression dans un trou ou
sur les l\`evres d'une fissure (dans une canalisation par exemple).

La level set permet ensuite par son changement de signe de d\'efinir
les supports \index{Support} qui doivent \^etre enrichis. Le nombre et
la nature des degr\'es de libert\'e en un noeud est variable et d\'epend
de la position relative du support de ce noeud par rapport \`a la
fissure (ou l'interface mat\'eriau). Il faut \^etre capable de g\'erer
sur le plan informatique cette diversit\'e dans les degr\'es de libert\'e.
Un \'el\'ement peut n'avoir que quelques noeuds enrichis. Le type
d'enrichissement peut aussi \'evoluer dans le temps (propagation d'une
fissure par exemple).

L'int\'egration des matrices de raideur et des forces ext\'erieures
fait intervenir des fonctions discontinues voire singuli\`eres sur les
\'el\'ements coup\'es par la fissure. De m\^eme, si un \'el\'ement
est coup\'e par le bord d'un trou, il faut limiter l'int\'egration \`a
la partie mati\`ere de l'\'el\'ement. Un soin particulier est donc \`a
apporter dans l'int\'egration des matrices de raideur \'el\'ementaires.
Notre exp\'erience sur le sujet nous a amen\'e \`a suivre l'approche
suivante. Dans un premier temps, il faut d\'ecouper l'\'el\'ement en
sous-domaines d'int\'egration sur lesquels les fonctions \`a int\'egrer
sont continues. Ensuite, sur chaque sous-domaine une r\`egle
appropri\'ee d'int\'egration doit \^etre choisie. Pour la gestion des
trous, interface mat\'eriaux ou partie Heaviside de la mod\'elisation
d'une fissure, une int\'egration par points de Gauss classique
convient. Pour les fonctions singuli\`eres de front de fissure,
l'int\'egration de Gauss converge tr\`es mal. Il faut privil\'egier une
int\'egration ad hoc \cite{Bechet05}.



Enfin, au niveau du post-traitement, il faut \^etre capable
de visualiser des champs discontinus sur les \'el\'ements pour
effectivement voir les \'el\'ements coup\'es par une fissure se
scinder.



\subsection{Application en m\'ecanique de la rupture \index{Rupture}}

Cinq exemples sont trait\'es dans ce chapitre. Les trois premiers
concernent le calcul des facteurs d'intensit\'e de contrainte pour
diff\'erentes configurations: une fissure inclin\'ee sous tension dans
une membrane, une fissure elliptique inclin\'ee dans un cube sous
tension et enfin une fissure en forme de Y (fissure \`a branches). Le
quatri\`eme exemple concerne la propagation d'une fissure en forme de
lentille de contact et le dernier la mod\'elisation de bandes de
cisaillement dans un massif perc\'e d'un tunnel.

\subsubsection{Le probl\`eme de Griffith: une fissure inclin\'ee}
Les facteurs d'intensit\'e de contrainte d'une fissure dans une plaque
infinie sous tension sont donn\'es par la solution de Griffith :
\begin{subequations}
\begin{align}
K_I &= \sigma\sqrt{\pi a}~\mbox{cos}^2(\beta) \\
K_{II} &= \sigma\sqrt{\pi a}~\mbox{sin}(\beta)~\mbox{cos}(\beta)
\end{align}
\end{subequations}
o\`u $a$ et la demi-longueur de la fissure, $\sigma$ l'intensit\'e du
chargement et $\beta$ l'angle entre le chargement et la fissure. Ce
probl\`eme est mod\'elis\'e  figure~\ref{fig:plate2} dans une plaque
carr\'ee suffisamment grande par rapport \`a la taille de la fissure.
Le maillage utilis\'e est uniforme et contient $40\times 40$
\'el\'ements. Les facteurs d'intensit\'e de contrainte obtenus avec
X-FEM pour diff\'erents angles $\beta$ sont compar\'es avec les
facteurs de r\'ef\'erence
figure~\ref{fig:rotk1k2}~\cite{Moes:discont}. L'accord est
excellent. Notons que le m\^eme maillage ($40\times 40$
\'el\'ements) est utilis\'e pour \emph{tous} les angles $\beta$. Seul
la position relative de la fissure par rapport au maillage change.



\twofigures{5cm}{plate2}{9cm}{rotk1k2}{Le probl\`eme de la plaque
fissur\'ee sous
       tension: $W = 10 in.$, $a = 0.5 in.$ ; (a), Comparatif entre les facteurs
       d'intensit\'e de
       contrainte num\'eriques (X-FEM) et de r\'ef\'erence
       en fonction de l'angle de chargement  $\beta$ pour le
       probl\`eme de la plaque fissur\'ee ; (b) }{fig:plaque_fissuree}



\subsubsection{Fissure elliptique inclin\'ee sous tension}

Le second exemple est celui d'une fissure elliptique inclin\'ee sous
tension \cite{Moes3DGrowthI}. La fissure est plane mais la courbure
du front n'est pas constante. La solution exacte de ce probl\`eme
(pour un probl\`eme infini) est donn\'ee dans \cite{KassirSih66}. Dans
le calcul, le cube \`a un cot\'e de taille $h$ et est charg\'e selon l'axe
$Z$, voir figure~\ref{fig:ellipse}. L'axe majeur de l'ellipse est
selon l'axe $X$ et de taille $a$. L'axe mineur de taille $b=a/2$ est
orient\'e selon la bissectrice du quadrant $XY$ (la fissure est donc
inclin\'ee \`a 45 degr\'es du chargement). Les dimensions sont choisies de
telle sorte que l'hypoth\`ese de milieu infini soit applicable
($h/a=10$). Le coefficient de Poisson est 0.3 et le module de Young
\'egal \`a 1. La Figure~\ref{fig:Kellipse} montre une  comparaison entre
les solutions exactes et num\'eriques obtenues avec X-FEM. Le d\'etail
du calcul des facteurs d'intensit\'e de contraintes est donn\'e
dans~\cite{Moes3DGrowthI}. La signification de l'abscisse $\theta$
(en degr\'es) est donn\'ee sur la figure~\ref{fig:angle}. Le nombre de
degr\'es de libert\'e est de 130.000 et la taille caract\'eristique des
\'el\'ements pr\`es du front de la fissure est de l'ordre du dixi\`eme de
l'axe majeur de l'ellipse.


\twofigures{7cm}{ellipse}{6cm}{angle}{Une fissure elliptique inclin\'ee dans
       un cube sous tension ; (a), d\'efinition de la position angulaire le long du
  front de la fissure ; (b) }{fig:fissure_elliptique}

\onefigure{10cm}{Kellipse}{Les facteurs d'intensit\'e de
     contraintes exacts (ligne continue) et calcul\'es (symboles)
     pour la fissure elliptique en fonction de la position
     angulaire sur le front.}



\subsubsection{Une fissure en forme de Y}

Cette exemple, tir\'e de \cite{Daux:holes}, permet de montrer la
possibilit\'e qu'offre X-FEM de g\'erer des fissures \`a branches. On
consid\`ere une fissure \`a branches sym\'etriques dans une plaque de
largeur $2w$ et de hauteur $2H$ soumise \`a une tension de valeur
$\sigma$ perpendiculaire \`a la fissure principale,
figure~\ref{fig:mesh_infinite}. La taille de la plaque est grande
par rapport \`a la taille de la fissure ($w=20$, $H=16$ et $a=1.$)
de mani\`ere \`a pouvoir utiliser la solution de r\'ef\'erence pour
une plaque infinie donn\'ee par Chen et Hasebe (1995).
\nocite{Chen:branching} Notons que dans le calcul, la fissure n'est
pas repr\'esent\'ee \`a l'aide de level sets mais de segments de droite.

Les facteurs d'intensit\'e de contrainte normalis\'es en pointes de
fissures A et B sont d\'efinis par
\begin{equation}
F_{I}^{A} = K_{I}^{A}/\sigma \sqrt{\pi c} \mbox{ }, \quad
F_{I}^{B} = K_{I}^{B}/\sigma \sqrt{\pi c} \mbox{ }, \quad
F_{II}^{B} = K_{II}^{B}/\sigma \sqrt{\pi c} \nonumber
\end{equation}
Les r\'esultats obtenus avec X-FEM pour $F_{I}^{A}$, $F_{I}^{B}$ et
$F_{II}^{B}$ sont compar\'es aux valeurs de r\'ef\'erence pour
diff\'erents rapports $b/a$ et angles $\theta$ dans la
Table~\ref{tab:infinite_br}. Le maillage utilis\'e est donn\'e
figure~\ref{fig:mesh_infinite}.

Dans le cas  $b/a = 1$ et $\theta = \pi/4$, l'influence du
raffinement est montr\'ee dans la Table~\ref{tab:robust}. La taille
moyenne des \'el\'ements en fond de fissure est $h$. On observe que
les r\'esultats num\'eriques convergent vers les r\'esultats de
r\'ef\'erence avec le raffinement du maillage. Les r\'esultats sont
d\'ej\`a excellents pour un rapport $h/a$ de 0.1


\twofigures{6cm}{mesh_infinite}{6cm}{mesh_infinite_zoom}{Le maillage
utilis\'e pour le probl\`eme de la fissure
  \`a branches (1218 noeuds, $h/a$ = 0.12) ; (a), un zoom dans la
  r\'egion de la fissure ; (b).}{fig:mesh_infinite}

\begin{table}[!p]
\begin{center}
\begin{tabular}{cccccccc} \hline
&$\theta$ & \multicolumn{2}{c}{$15^\circ$} & \multicolumn{2}{c}{$45^\circ$} &
\multicolumn{2}{c}{$75^\circ$} \\
\hline
$b/a$ & & X-FEM & $\star$ & X-FEM & $\star$ & X-FEM & $\star$ \\
\hline
&$F_{I}^{A}$  & 1.016 & 1.018 & 1.045 & 1.044 & 1.118 & 1.117 \\
1.0 & $F_{I}^{B}$ & 0.750 & 0.737 & 0.493 &
0.495 & 0.061 & 0.061 \\
& $F_{II}^{B}$ & 0.123 & 0.114 & 0.504 & 0.506 & 0.535 & 0.541 \\
\hline
& $F_{I}^{A}$ & 1.015 & 1.016 & 1.036 & 1.036 & 1.086 & 1.087 \\
0.8 & $F_{I}^{B}$ & 0.736 & 0.735 & 0.494 &
0.495 & 0.057 & 0.056 \\
& $F_{II}^{B}$ & 0.107 & 0.107 & 0.497 & 0.498 & 0.546 & 0.551 \\
\hline
& $F_{I}^{A}$ & 1.012 & 1.011 & 1.022 & 1.023 & 1.035 & 1.037 \\
0.4 & $F_{I}^{B}$ & 0.752 & 0.729 & 0.502 & 0.504 & 0.067 & 0.066 \\
& $F_{II}^{B}$ & 0.096 & 0.078 & 0.459 & 0.460 & 0.542 & 0.542 \\
\hline
\multicolumn{8}{l}{($\star$) Solution de r\'ef\'erence de Chen et Hasebe (1995)}
\end{tabular}
\end{center}
\caption{Facteurs d'intensit\'e de contrainte normalis\'es pour
         diff\'erents rapports $b/a$ et angles $\theta$ pour le
         probl\`eme de la fissure en forme de Y.}
\label{tab:infinite_br}
\end{table}

\begin{table} [htbp]
\begin{center}
\begin{tabular}{ccccccccc} \hline
\mbox{ } $h/a$ \mbox{ }& 0.40 & 0.30 & 0.22 & 0.14 & 0.12 & 0.10 &
0.05 & $\star$ \\
\hline
$F_{I}^{A}$ & 0.963 & 1.009 & 1.027 & 1.042 & 1.045 &
1.045 & 1.044 & 1.044 \\
$F_{I}^{B}$ & 0.460 & 0.468 & 0.498 & 0.494 & 0.493 &
0.495 & 0.496 & 0.495 \\
$F_{II}^{B}$ & 0.458 & 0.464 & 0.501 & 0.505 & 0.504 &
0.507 & 0.508 & 0.506 \\
\hline
\multicolumn{9}{l}{($\star$) Solution de r\'ef\'erence de Chen et Hasebe (1995)}
\end{tabular}
\end{center}
\caption{Facteurs d'intensit\'e de contrainte num\'eriques et de
  r\'ef\'erence  normalis\'es pour diff\'erents rapports $h/a$
  (probl\`eme de la fissure en forme de Y avec $b/a=1.$ et
   $\theta=45^\circ$).}
\label{tab:robust}
\end{table}


\subsubsection{Propagation d'une fissure en forme de lentille}

Dans cette exemple,  on consid\`ere une fissure en forme de lentille
plac\'ee dans un cube soumis \`a de la tension hydrostatique
\cite{Moes3DGrowthII}. La g\'eom\'etrie de la fissure,
figure~\ref{fig:im12}, est caract\'eris\'ee par un rayon $R=0.005$ et un
angle azimuthal $\alpha=45^o$. Le cube a  un cot\'e de 0.05. Le
chargement est de type fatigue et on consid\`ere une loi de
propagation de Paris. La direction de propagation est donn\'ee par la
direction dans laquelle la contrainte circonf\'erentielle
($\sigma_{\theta \theta}$) est maximale.

Le maillage utilis\'e est non structur\'e et compos\'e de 8895 tetra\`edres
  et 1767 noeuds. Ce maillage ne respecte pas la position de
la fissure. La figure~\ref{fig:im13} donne la direction et amplitude
de la vitesse initiale de la fissure (avanc\'ee par nombre de cycle de
chargement). On observe une sym\'etrie de cette distribution.

La figure~\ref{fig:im14} montre la position de la fissure apr\`es 15
pas de propagation. Il faut noter que le front de la fissure est
maintenant compos\'e de quatre fronts ind\'ependants (les quatre
faces du cube ayant \'et\'e coup\'ees). Ce changement de topologie
du front ne demande aucune attention particuli\`ere
dans l'algorithme de propagation
par level sets. On note que la convexit\'e du front a \'egalement
chang\'e.

\onefigure{9cm}{im12}{Forme initiale de la fissure lentille dans
  le cube.}

\twofigures{6cm}{im13}{6cm}{im14}{Distribution initiale de la vitesse de propagation
 sur le front de la fissure en forme de lentille ; (a),
 position de la fissure apr\`es 15 pas de propagation ; (b)}{fig:fissure3D}

\subsubsection{Bande de cisaillement autour d'un tunnel}
Comme dernier exemple de ``m\'ecanique de la rupture'',
consid\'erons le probl\`eme
d'un tunnel creus\'e dans une roche avec joints, figure \ref{fig:tunnel}.
Ce probl\`eme met en
jeu \`a la fois des surfaces de discontinuit\'e et des bords libres (tunnel)
et a \'et\'e trait\'e dans \cite{Belytschko:circle}.
Le maillage utilis\'e, tr\`es simple,
est montr\'e figure \ref{fig:tunnelmesh}. Il ne respecte ni la position
du tunnel, ni la position des joints. L'enrichissement au niveau des
joints ne fait intervenir qu'une discontinuit\'e tangentielle. La figure
\ref{fig:tunnel_disp_x} montre le d\'eplacement horizontal obtenu.
On peut noter la discontinuit\'e de ce champ au passage des joints.
Cet exemple met en \'evidence la possibilit\'e avec X-FEM de
n'enrichir et ne rendre discontinu que la composante tangentielle
du d\'eplacement pr\`es des joints.
Ces bandes de glissement se superposent sans probl\`eme, chacune
ajoutant
son enrichissement propre dans l'approximation.

\twofigures{7.75cm}{tunnel}{5cm}{tunnelmesh}{Repr\'esentation sch\'ematique du \guil{jointed
       rocks problem}; (a),  maillage utilis\'e avec la position
       des joints et du tunnel ; (b)}{fig:tunnel}

\onefigure{11cm}{tunnel_disp_x}{D\'eplacement horizontal discontinu au niveau des joints.}





\subsection{Application \`a la th\'eorie de l'homog\'en\'eisation \index{Homog\'en\'eisation}}
La possibilit\'e de repr\'esenter des interfaces mat\'eriaux
complexes sans devoir les mailler est
particuli\`erement attractive par exemple pour homog\'en\'eiser
des milieux p\'eriodiques.
Peu de simulations de cellules composites
ou al\'eatoires peuvent \^etre r\'ealis\'ees \`a l'heure actuelle
\`a cause de la difficult\'e dans le r\'ealisation du maillage.
Cette difficult\'e est d'autant plus aigu� si l'on souhaite
que le maillage soit p\'eriodique pour imposer facilement
la condition de p\'eriodicit\'e sur le champ micro inconnu.

On traitera deux exemples dans cette section :
un exemple de validation (bi-couche) et
un exemple sur l'homog\'en\'eisation de milieux al\'eatoires
(X-FEM est tr\`es attrayant car un seul maillage peut \^etre
consid\'er\'e pour toutes les configurations).
D'autres exemples peuvent \^etre trouv\'es dans~\cite{MoesCloirec02}
et \cite{Cartraud04}. En particulier, dans ce dernier papier,
un exemple d'homog\'en\'eisation de c�ble est consid\'er\'e.

\subsubsection{Bi-couche p\'eriodique}
Le cas d'un bi-couche est un probl\`eme simple dont
la solution est lin\'eaire par morceau.
L'objectif est, ici, de  v\'erifier
que l'espace X-FEM enrichi est capable de repr\'esenter exactement
la solution m\^eme si l'interface
n'est pas maill\'ee.
La p\'eriode consid\'er\'ee
est de forme cubique. Le maillage est compos\'e de
162 t\'etra\`edres (27 cellules cubiques chacune compos\'ee de
6 t\'etra\`edres). Une coupe du maillage est montr\'ee
figure~\ref{fig:periodic} avec une indication de la nature
des diff\'erents noeuds.
La
d\'eform\'ee correspondant \`a une d\'eformation macroscopique de
cisaillement dans le plan de l'interface est pr\'esent\'ee
figure~\ref{fig:bi_layer_disp}. On
v\'erifie la bonne prise en compte de la discontinuit\'e,
avec une d\'eform\'ee lin\'eaire par morceaux dans l'\'el\'ement fini o� les
deux mat\'eriaux sont pr\'esents. Enfin, on v\'erifie que la solution
num\'erique coincide avec la solution analytique du probl\`eme
donn\'ee dans \cite{Dumontet90}.

\onefigure{12cm}{periodic}{Vue d'une coupe du maillage du bi-couche
    p\'eriodique avec l'indication du type de noeud.}

%
\onefigure{7cm}{bi_layer_disp}{D\'eform\'ee du bi-couche
    p\'eriodique dans un mode de cisaillement. On peut distinguer
    les \'el\'ements finis et les sous-cellules utilis\'ees pour
    l'int\'egration de la matrice de raideur sur les \'el\'ements
    tranch\'es par l'interface.}


\twofigures{6cm}{spheres}{6cm}{spheres-bis}{Position de 32 sph\`eres dans une cellule p\'eriodique pour
  deux ``tirs'' al\'eatoires diff\'erents. Les surfaces visualis\'ees
  correspondent dans chaque cas \`a l'iso-z\'ero de la level set.}{fig:spheress}



\subsubsection{Mat\'eriau \`a inclusions sph\'eriques}
%
Nous traitons \`a pr\'esent un mat\'eriau \'elastique \`a inclusions
sph\'eriques, r\'eparties de fa�on al\'eatoire dans la matrice, cet
exemple est tir\'e de \cite{Michel99}. Nous \'etudions donc plusieurs
cellules, avec dans chacune une r\'epartition al\'eatoire (mais
p\'eriodique) des inclusions.

La m\'ethode X-FEM s'av\`ere ici particuli\`erement int\'eressante,
puisque le m\^eme maillage peut \^etre utilis\'e pour ces diff\'erentes
cellules. Il s'agit d'un maillage r\'egulier, avec
$32 \times 32 \times 32 $ cellules cubiques chacune compos\'ee
de 6 \'el\'ements t\'etra\`edriques. Pour deux configurations de
cellules
diff\'erentes, la figure~\ref{fig:spheress} donne la position
de l'iso-z\'ero de la level set.

Les caract\'eristiques m\'ecaniques de la particule et de la matrice
sont respectivement $E_p = 70$ GPa, $\nu_p = 0.2$ et $E_m = 3$
GPa, $\nu_m = 0.35$, avec un taux d'inclusion de 26,78\%. Pour la
raideur homog\'en\'eis\'ee $a^{hom}_{1111}$, en prenant la valeur
moyenne sur diff\'erentes cellules, on obtient $7,611$ GPa pour 8
particules et 7,711 pour 32 particules, soit des r\'esultats tr\`es
proches de ceux donn\'es dans \cite{Michel99}.
D'autres exemples d'homog\'en\'eisation p\'eriodique sont trait\'es dans
\cite{MoesCloirec02}.
%



\section{Conclusion et autres lectures}
\label{sec:conclusion-et-autres}

Ce document a r\'esum\'e succinctement la technologie X-FEM,
coupl\'ee \`a la puissance de la m\'ethode des level sets, pour
mod\'eliser des surfaces de discontinuit\'e dans un champ
ou sa d\'eriv\'ee (ou dans la mati\`ere : pr\'esence de trous).
Ces surfaces pouvant \^etre fixes ou mobiles.

Deux types d'applications ont \'et\'e privil\'egi\'es : la
m\'ecanique de la rupture et l'homog\'en\'eisation.

Il ne nous a pas \'et\'e permis dans ce document de balayer
tout l'apport que repr\'esente la partition de l'unit\'e
et les level sets pour la mod\'elisation de fissures.
Cet apport  s'\'etend rapidement compte tenu du nombre
grandissant d'\'equipes \`a travers le monde qui
utilisent maintenant ces techniques.
Sans \^etre exhaustif, voici quelques domaines de d\'eveloppement
important actuel en m\'ecanique num\'erique de la rupture :
\begin{itemize}
\item Passage de la m\'ecanique de l'endommagement \`a la
  m\'ecanique de la rupture : la probl\'ematique est ici
la transition de la micro-fissuration \`a une fissure macroscopique
\'etablie~\cite{patzak03,deborst04}.
 L'approche X-FEM permet de faire cette transition
  num\'erique sur un m\^eme maillage. Il est important de noter que
m\^eme si X-FEM offre des avanc\'ees \emph{num\'eriques}, le
  passage en terme de \emph{mod\`ele} reste un probl\`eme d\'elicat.
\item M\'ecanique de la rupture non lin\'eaire :
la fissuration en milieu subissant de grande transformation
(caoutchouc par exemple) a \'et\'e \'etudi\'ee
dans~\cite{dolbow04}
et~\cite{Legrain05}. La
prise en compte du contact sur les l\`evres de la fissure
a \'et\'e consid\'er\'e dans \cite{Dolbow:contact} et la
pr\'esence d'une zone coh\'esive dans~\cite{WellsSluys01}
et \cite{MoesCohesive}.
\item La multi-fissuration et le ph\'enom\`ene de percolation a
\'et\'e consid\`ere en 2D dans~\cite{BudynMulti}.
\item La fissuration dynamique est un domaine important mais difficile
  car le nombre de degr\'es de libert\'e \'evolue dans le temps et
  la stabilit\'e des sch\'emas num\'eriques standard doit \^etre
  r\'eexamin\'ee. Des avanc\'ees importantes sur ce sujet ont
  \'et\'e obtenues par  \cite{BelytschkoDyn03} et \cite{Rethore05}.
\item La possibilit\'e de la rencontre de plusieurs fissures
  coplanaires
a \'et\'e \'etudi\'ee dans \cite{chopp03}.
\end{itemize}

De m\^eme, nous n'avons pu d\'ecrire plus en d\'etail d'autres domaines
d'applications comme l'optimisation de forme \cite{BelytschkoOpti03}
ou le suivi de front de solidification
\cite{ChessaPhase2002,DolbowPhase2002}.
