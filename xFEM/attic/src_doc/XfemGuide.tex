%%%%%%%%%%%%%%%%%%%%%%%%%%%%%%%%%%%%%%%%%%%%%%%%%%%%%%%%%%%%%%%%%%%%%
%
% Complete documentation
%
%
%%%%%%%%%%%%%%%%%%%%%%%%%%%%%%%%%%%%%%%%%%%%%%%%%%%%%%%%%%%%%%%%%%%%%
               
\documentclass[a4paper,twoside,11pt]{XfemDoc}

\usepackage[latin1]{inputenc}
%\usepackage[french]{babel}
\usepackage{epsfig,subfigure}
\usepackage{amsmath}
\usepackage{amsfonts}
\usepackage{amssymb}
\usepackage{graphicx}
\usepackage{longtable}
\usepackage{color}

\usepackage{subfigure}
\graphicspath{{figures/}{:figures:}}
\usepackage{chicago}
\usepackage{graphpap}

%%%%%%%%%%%%%%%%%%%%%%%%%%%%%%%%%%%%%%%%%%%%%
%
%	commandes sp�cifiques DOC XFEM 
%
%%%%%%%%%%%%%%%%%%%%%%%%%%%%%%%%%%%%%%%%%%%%%

\DeclareGraphicsExtensions{.pdf,.png,.jpg,.eps,.mps}

% Define command to make hyper-links with relative path on the local file system 
% (command found in Google Groups)
\newcommand{\pdflaunch}[2]{\leavevmode%
  \pdfstartlink attr{/Border [0 0 0]} user{/Subtype /Link /A << %
    /S /Launch /F (#1) >>}{\textcolor{blue}{\underline{#2}}}\pdfendlink\rule{0mm}{1em}}%


% Define command to make reference and link to on-line Doxygen Xfem documentation

\newcommand{\doxygen}[1]{
\pdflaunch{../../../../doc/html/classxfem_1_1#1.html}{\texttt{#1}}}  
\newcommand{\doxygenmenu}[2]{
\pdflaunch{../../../../doc/html/#1.html}{\texttt{#2}}}  


\newcommand{\inheritgraph}[2]{
\onefigure{#1}{../../../../doc/html/classxfem_1_1#2__inherit__graph.png}{diagramme d'h\'eritage de la classe \code{#2}}}  

\newcommand{\collabgraph}[2]{
\onefigure{#1}{../../../../doc/html/classxfem_1_1#2__coll__graph.png}{diagramme de collaboration de la classe \code{#2}}}  


\newcommand{\dependgraph}[2]{
\onefigure{#1}{../../../../doc/html/#2_8h__incl.png}{diagramme de d\'ependance de la classe \code{#2}}}  


% Define command to make web link
\newcommand{\web}[1]{ \url{#1}  }
% le premier est l'adresse, le deuxi\`eme le texte \`a lier
\newcommand{\webify}[2]{ \href{#1}{#2}  }



% Define command to include TEX file automatically generated by the parser and placed into the 
% directory tex_from_h/

\newcommand{\texfromh}[1]
{ \input{../../../../doc/tex_from_h/#1.tex} }  





% code is the most difficult one...
\newcommand{\code}[1]{\textup{\texttt{#1}}}
\definecolor{gris}{gray}{0.8}
\newcommand{\touche}[1]{
\fcolorbox{black}{gris}{ \textup{\texttt{\tiny{#1}}} }
}
\newcommand{\guil}[1]{� #1 �}


\newcommand{\remarque}[1]{\vspace{1 em}   \noindent\textbf{\textit{Remarque}}~: \textit{#1}  }
\newcommand{\note}[1]{\vspace{1 em}   \noindent\textbf{\textit{A noter}}~: \textit{#1}  }
\newcommand{\ajouter}[1]{\vspace{1 em}   \noindent\textbf{\textit{A DETAILLER}}~: \textbf{\textit{#1}}  }




%%% raccourcis pour les figures
\newcommand{\onefigure}[3]{
\begin{figure}[!htb]
\centering  
\includegraphics[width=#1]{#2}   
	\caption{#3}   
	\label{fig:#2} 
\end{figure}}

%%%
\newcommand{\twofigures}[6]{
\begin{figure}[!htb]
\centering  
\subfigure[]{\label{fig:#2} \includegraphics[width=#1]{#2}  }\quad
\subfigure[]{\label{fig:#4} \includegraphics[width=#3]{#4}  }\\
	\caption{#5}   \label{#6}
\end{figure}}

%%%
%%% !!!! attention 9 arguments maxi pour \newcommand 
\newcommand{\fourfigures}[9]{
\begin{figure}[!htb]
\centering  
\subfigure[]{\label{fig:#2} \includegraphics[width=#1]{#2}  }\quad
\subfigure[]{\label{fig:#4} \includegraphics[width=#3]{#4}  }\\
\subfigure[]{\label{fig:#6} \includegraphics[width=#5]{#6}  }\quad
\subfigure[]{\label{fig:#8} \includegraphics[width=#7]{#8}  }\\
	\caption{#9} 
\end{figure}}
%%% !!!! attention 9 arguments maxi pour \newcommand 
\newcommand{\fourturnedfigures}[9]{
\begin{figure}[!htb]
\centering  
\subfigure[]{\label{fig:#2} \includegraphics[width=#1,angle=90]{#2}  }\quad
\subfigure[]{\label{fig:#4} \includegraphics[width=#3,angle=90]{#4}  }\\
\subfigure[]{\label{fig:#6} \includegraphics[width=#5,angle=90]{#6}  }\quad
\subfigure[]{\label{fig:#8} \includegraphics[width=#7,angle=90]{#8}  }\\
	\caption{#9} 
\end{figure}}








%%%%%%%%%%%%%%%%%%%%%%%%%%%%%%%%%%%%%%%%%%%%%
%
%	raccourcis des commandes pour les équations 
%
%%%%%%%%%%%%%%%%%%%%%%%%%%%%%%%%%%%%%%%%%%%%%


%\newcommand{\xfem}{X-FEM}
%\renewcommand{\arraystretch}{1.5}
\newcommand{\dessus}[2]{\mathrel{\mathop{#2}\limits^{#1}}}
\newcommand{\dessous}[2]{\mathrel{\mathop{#2}\limits_{#1}}}
\newcommand{\Dessus}[2]{\raisebox{.2ex}{\ensuremath{\dessus{#1}{#2}}}}
\newcommand{\Dessous}[2]{\raisebox{.2ex}{\ensuremath{\dessous{#1}{#2}}}}

%\newcommand{\boxedEPSF}[1]{\BoxedEPSF{#1}}
\newcommand{\indic}{\ensuremath{\Psi}}
\newcommand{\eabs}{\ensuremath{e}}
\newcommand{\eabsex}{\ensuremath{\overline{e}}}
\newcommand{\eabstil}{\ensuremath{\Dessous{\sim}{\eabs}}}
\newcommand{\tps}{tps}
\newcommand{\esp}{esp}
\newcommand{\ite}{ite}
\newcommand{\eabst}{\ensuremath{\eabs_{\mathrm{\tps}}}}
\newcommand{\eabse}{\ensuremath{\eabs_{\mathrm{\esp}}}}

\newcommand{\erel}{\ensuremath{\epsilon}}
\newcommand{\eest}{\ensuremath{\epsilon_{\mbox{est}}}}
\newcommand{\champ}[1]{\ensuremath{(\dep_{#1};\cont_{#1};
\va_{#1};\defop_{#1};\vi_{#1})}}
\newcommand{\champex}[1]{\ensuremath{(\depex_{#1};\contex_{#1};
\vaex_{#1};\defopex_{#1};\viex_{#1})}}
\newcommand{\champad}[1]{\ensuremath{(\depad_{#1};\contad_{#1};
\vaad_{#1};\defopad_{#1};\viad_{#1})}}
\newcommand{\champpr}[1]{\ensuremath{(\deppr_{#1};\contpr_{#1};
\vapr_{#1};\defoppr_{#1};\vipr_{#1})}}
%
%\newcommand{\mef}{m\'ethode des \'el\'ements finis} 
%\newcommand{\ef}{\'el\'ements finis\ } 

\newcommand{\domtps}{\ensuremath{[0,T]}}
\newcommand{\domttt}{\ensuremath{[0,t]}}
\newcommand{\dompas}{\ensuremath{[t_{n},t_{n+1}]}}
\newcommand{\domesp}{\ensuremath{\Omega}} 
\newcommand{\domele}{\ensuremath{E}} 
\newcommand{\domtpsd}{\ensuremath{[0,T]_{\Delta}}} 
\newcommand{\domtpsdk}{\ensuremath{[0,t_{k}]_{\Delta}}} 
%\newcommand{\domtpsd}{\ensuremath{\{t_{n+1}\}}} 
\newcommand{\domespd}{\ensuremath{\domesp_{h}}} 
\newcommand{\produit}[2]{\ensuremath{#1\times#2}}
\newcommand{\pscal}[2]{\ensuremath{#1.#2}}
\newcommand{\diverg}[1]{div(#1)}
\newcommand{\tn}{\ensuremath{t_{n}}}
\newcommand{\tnp}{\ensuremath{t_{n+1}}}
\newcommand{\domega}{\ensuremath{\partial\domesp}} 
\newcommand{\dunomega}{\ensuremath{\partial_{1}\domesp}} 
\newcommand{\dunomegan}{\ensuremath{\partial_{1}\domesp^{n}}} 
\newcommand{\ddeuomega}{\ensuremath{\partial_{2}\domesp}} 
\newcommand{\ddeuomegan}{\ensuremath{\partial_{2}\domesp^{n}}} 
\newcommand{\fsurf}{\ensuremath{\underline F}}
\newcommand{\fvol}{\ensuremath{\underline f}} 
\newcommand{\depimp}{\ensuremath{\dep_{\mathrm{d}}}} 
\newcommand{\fsurfimp}{\ensuremath{\bft_{\mathrm{d}}}}
\newcommand{\fvolimp}{\ensuremath{\bff_{\mathrm{d}}}}
\newcommand{\ddeuomegad}{\ensuremath{\partial_{2}\domespd}}
%\newcommand{\dt}{\ensuremath{t_{n+1}-t_{n}}}
 

%%%%%%%%%%%%%%%%%%
\newcommand{\inttps}[1]{\ensuremath{\int_{0}^{T} \! #1 \, \mathrm{d}t}}  
\newcommand{\intttt}[1]{\ensuremath{\int_{0}^{t} \! #1 \, \mathrm{d}t}}  
\newcommand{\intpas}[1]{\ensuremath{\int_{t_{n}}^{t_{n+1}} #1 \, \mathrm{d}t}}  
\newcommand{\intesp}[1]{\ensuremath{\int_{\domesp} #1 \, \mathrm{d} \domesp }}
\newcommand{\inttpsesp}[1]{\ensuremath{\int_{0}^{T} \! \int_{\Omega} #1 \, \mathrm{d} \Omega \mathrm{d} t}}  
\newcommand{\inttpsele}[1]{\ensuremath{\int_{0}^{T} \! \int_{E} #1 \, \mathrm{d} E \mathrm{d} t}}  
\newcommand{\intpasesp}[1]{\ensuremath{\int_{\Delta t} \! \int_{\Omega} #1 \, \mathrm{d} \Omega \mathrm{d} t}}  
\newcommand{\inttpsespd}[1]{\ensuremath{\int_{0}^{T} \! \int_{\domespd} 
#1 \, \mathrm{d}\domespd \mathrm{d}t}}  
\newcommand{\inttttesp}[1]{\ensuremath{\int_{0}^{t} \! \int_{\Omega} #1 
\, \mathrm{d}\Omega \mathrm{d}t}}   
\newcommand{\inttttespd}[1]{\ensuremath{\int_{0}^{t} \! \int_{\domespd} 
#1 \, \mathrm{d}\domespd \mathrm{d}t}}   
\newcommand{\intespttt}[1]{\ensuremath{ \int_{\Omega} \! \int_{0}^{t} #1 
\, \mathrm{d}t \mathrm{d}\Omega}}  
\newcommand{\intespdttt}[1]{\ensuremath{ \int_{\domespd} \! \int_{0}^{t} #1 \, \mathrm{d}t \mathrm{d}\domespd}}  
\newcommand{\intddeuom}[1]{\ensuremath{\int_{\ddeuomega} #1 \, \mathrm{d}S}}  
\newcommand{\intddeuomd}[1]{\ensuremath{\int_{\ddeuomegad} #1 \, \mathrm{d}S}}  
\newcommand{\intespd}[1]{\ensuremath{\int_{\domespd} #1 \, \mathrm{d}\domespd }}    
\newcommand{\intabr}[2]{\left\langle #2 \right\rangle_{#1}}  
\newcommand{\intabrpet}[2]{\langle #2 \rangle_{#1}}  
\newcommand{\sumtps}[1]{\ensuremath{\sum_{n}#1}}  
\newcommand{\sumesp}[1]{\ensuremath{\sum_{E}#1}}  
\newcommand{\sumtpsesp}[1]{\ensuremath{\sum_{n}\sum_{E}#1}}  
\newcommand{\intele}[1]{\ensuremath{\int_{E} #1 \, \mathrm{d}E }}
\newcommand{\inteled}[1]{\ensuremath{\int_{E_{h}} #1 \, \mathrm{d}E_{h} }}
\newcommand{\intdele}[1]{\ensuremath{\int_{\partial E} #1 \, \mathrm{d}S }}
\newcommand{\intdeleg}[1]{\ensuremath{\int_{\partial E\cap\Gamma_{\mathrm{d}}} #1 \, \mathrm{d}S }}
\newcommand{\intgamj}[1]{\ensuremath{\int_{\Gamma_{j}} #1 \, \mathrm{d}\Gamma }}

 
\newcommand{\probb}{\ensuremath{\mathcal{P}}}
\newcommand{\probh}{\ensuremath{\probb_{\mathrm{time}}}}
\newcommand{\probn}{\ensuremath{\probb_{\mathrm{space}}}}
\newcommand{\probhn}{\ensuremath{\probb_{accu}}}
\newcommand{\ptc}{\ensuremath{\probb_{t-c}}}
\newcommand{\pec}{\ensuremath{\probb_{e-c}}}
\newcommand{\pc}{\ensuremath{\probb_{c}}}

\newcommand{\plin}{\ensuremath{\mathrm{P_{lin}}}}
\newcommand{\ppui}{\ensuremath{\mathrm{P_{pui}}}}
\newcommand{\vplin}{\ensuremath{\mathrm{VP_{lin}}}}
\newcommand{\vppui}{\ensuremath{\mathrm{VP_{pui}}}}

\newcommand{\sss}{\ensuremath{s}}
\newcommand{\sssh}{\ensuremath{s_{\{h\}}}}
\newcommand{\sssn}{\ensuremath{s_{\{n\}}}}
\newcommand{\ssshn}{\ensuremath{s_{\{h,n\}}}}

\newcommand{\sex}{\ensuremath{\overline{s}}}
\newcommand{\sexh}{\ensuremath{\overline{s}_{\{h\}}}}
\newcommand{\sexn}{\ensuremath{\overline{s}_{\{n\}}}}
\newcommand{\sexhn}{\ensuremath{\overline{s}_{\{h,n\}}}}

\newcommand{\sad}{\ensuremath{\widehat{s}}}
\newcommand{\sadh}{\ensuremath{\widehat{s}_{\{h\}}}}
\newcommand{\sadn}{\ensuremath{\widehat{s}_{\{n\}}}}
\newcommand{\sadhn}{\ensuremath{\widehat{s}_{\{h,n\}}}}

\newcommand{\spr}{\ensuremath{\widetilde{s}}}
\newcommand{\sprh}{\ensuremath{\widetilde{s}_{\{h\}}}}
\newcommand{\sprn}{\ensuremath{\widetilde{s}_{\{n\}}}}
\newcommand{\sprhn}{\ensuremath{\widetilde{s}_{\{h,n\}}}}

\newcommand{\dep}{\ensuremath{\bfu}}
\newcommand{\deph}{\ensuremath{\dep_{h}}}
\newcommand{\depex}{\ensuremath{\overline{\dep}}}
\newcommand{\depad}{\ensuremath{\widehat{\dep}}}
\newcommand{\wad}{\ensuremath{\widehat{\underline{w}}}}
\newcommand{\kad}{\ensuremath{\hat{k}}}
\newcommand{\gad}{\ensuremath{\hat{g}}}
\newcommand{\had}{\ensuremath{\hat{h}}}
\newcommand{\pad}{\ensuremath{\hat{p}}}
\newcommand{\rad}{\ensuremath{\hat{R}}}
\newcommand{\fad}{\ensuremath{\underline{\hat{F}}}}
\newcommand{\fadz}{\ensuremath{\underline{\hat{F}}_{0}}}
\newcommand{\deppr}{\ensuremath{\widetilde {\dep}}}
\newcommand{\depda}{\ensuremath{\Breve{\dep}}}
%%%%%%%%%%%%%%%%%%%%%%%%%%%%%%%%%%%%%%%%%%
%\newcommand{\defo}{\ensuremath{\varepsilon\kern-.40em{\varepsilon}}}
\newcommand{\defo}{\ensuremath{\bfeps}}
\newcommand{\defoe}{\ensuremath{\defo^{\mathrm{e}}}}
\newcommand{\defop}{\ensuremath{\defo^{\mathrm{p}}}}
\newcommand{\defoex}{\ensuremath{\overline{\defo}}}
\newcommand{\defoeex}{\ensuremath{\overline{\defoe}}}
\newcommand{\defopex}{\ensuremath{\overline{\defop}}}
\newcommand{\defoead}{\ensuremath{\widehat{\defoe}}}
\newcommand{\defopad}{\ensuremath{\widehat{\defop}}}
\newcommand{\defoad}{\ensuremath{\widehat{\defo}}}
\newcommand{\defoppr}{\ensuremath{\widetilde{\defop}}}
\newcommand{\defoep}{\ensuremath{\dot{\defo}^{\mathrm{e}}}} 
\newcommand{\defopp}{\ensuremath{\dot{\defo}^{\mathrm{p}}}} 
\newcommand{\defopdp}{\ensuremath{\dot{\defo}_{D}^{\mathrm{p}}}} 
\newcommand{\defoppd}{\ensuremath{\frac{(\defop_{n+1}-\defop_{n})}{t_{n+1}-t_{n}}}}
\newcommand{\defotpd}{\ensuremath{\frac{(\defo_{n+1}-\defo_{n})}{t_{n+1}-t_{n}}}}
\newcommand{\defoppdad}{\ensuremath{\frac{(\defopad_{n+1}-\defopad_{n})}{t_{n+1}-t_{n}}}}
\newcommand{\defoppdex}{\ensuremath{\frac{(\defopex_{n+1}-\defopex_{n})}{t_{n+1}-t_{n}}}}
\newcommand{\defoppex}{\ensuremath{\overline{\defopp}}} 
\newcommand{\defoppprim}{\ensuremath{\dot{\defo}^{\mathrm{p}'}}} 
\newcommand{\defotp}{\ensuremath{\dot{\defo}}} 
\newcommand{\defotpprim}{\ensuremath{\dot{\defo}'}} 
\newcommand{\defotpex}{\ensuremath{\overline{\defotp}}} 
\newcommand{\defoepex}{\ensuremath{\overline{\defoep}}} 
\newcommand{\defoppad}{\ensuremath{\widehat{\defopp}}} 
\newcommand{\defoepad}{\ensuremath{\widehat{\defoep}}} 
\newcommand{\defopppr}{\ensuremath{\widetilde{\defopp}}} 
\newcommand{\cont}{\ensuremath{\bfsig}}
\newcommand{\conth}{\ensuremath{\cont_{h}}}
\newcommand{\contp}{\ensuremath{\dot{\cont}}}
\newcommand{\contex}{\ensuremath{\overline{\cont}}}
\newcommand{\contpex}{\ensuremath{\overline{\contp}}}
\newcommand{\contad}{\ensuremath{\widehat{\cont}}}
\newcommand{\contpad}{\ensuremath{\widehat{\contp}}}
\newcommand{\contpr}{\ensuremath{\widetilde{\cont}}}
\newcommand{\contd}{\ensuremath{\cont^{\textrm{\tiny D}}}}
\newcommand{\rmexp}{\ensuremath{R_{\textrm{\tiny M}}}}
\newcommand{\contdad}{\ensuremath{\contad^{\textrm{\tiny D}}}}
\newcommand{\contda}{\ensuremath{\Breve{\cont}}}
\newcommand{\vi}{\ensuremath{\bfalp}}
%\newcommand{\vi}{\ensuremath{X}}
%  dessous peut etre remplace par \underset{\sim}{p}
%   overset marche aussi
\newcommand{\pt}{\ensuremath{\Dessous{\sim}{p}}}
\newcommand{\vit}{\ensuremath{\Dessous{\sim}{\vi}}}
\newcommand{\viex}{\ensuremath{\overline{\vi}}} 
\newcommand{\viad}{\ensuremath{\widehat{\vi}}} 
\newcommand{\vipr}{\ensuremath{\widetilde{\vi}}} 
\newcommand{\vip}{\ensuremath{\dot{\vi}}} 
\newcommand{\vipprim}{\ensuremath{\dot{\vi}'}} 
\newcommand{\vitpprim}{\ensuremath{\dot{\vit}'}} 
\newcommand{\vitp}{\ensuremath{\dot{\vit}}}
\newcommand{\ptp}{\ensuremath{\dot{\pt}}}
\newcommand{\vipd}{\ensuremath{\frac{(\vi_{n+1}-\vi_{n})}{t_{n+1}-t_{n}}}} 
\newcommand{\vitpd}{\ensuremath{\frac{(\vit_{n+1}-\vit_{n})}{t_{n+1}-t_{n}}}} 
\newcommand{\vipdad}{\ensuremath{\frac{(\viad_{n+1}-\viad_{n})}{t_{n+1}-t_{n}}}} 
\newcommand{\vipdex}{\ensuremath{\frac{(\viex_{n+1}-\viex_{n})}{t_{n+1}-t_{n}}}} 
\newcommand{\vipex}{\ensuremath{\overline{\vip}}} 
\newcommand{\vipad}{\ensuremath{\widehat{\vip}}} 
\newcommand{\vippr}{\ensuremath{\widetilde{\vip}}} 
\newcommand{\va}{\ensuremath{\bfbet}}
\newcommand{\vah}{\ensuremath{\bfY_{h}}}
%\newcommand{\va}{\ensuremath{Y}}
\newcommand{\vat}{\ensuremath{\Dessous{\sim}{\va}}}
\newcommand{\vatp}{\ensuremath{\dot{\Dessous{\sim}{\va}}}}
\newcommand{\rt}{\ensuremath{\Dessous{\sim}{R}}}
\newcommand{\vap}{\ensuremath{\dot{\va}}}
\newcommand{\vaex}{\ensuremath{\overline{\va}}}
\newcommand{\vaad}{\ensuremath{\widehat{\va}}}
\newcommand{\vapr}{\ensuremath{\widetilde{\va}}}
\newcommand{\vapex}{\ensuremath{\overline{\vap}}}
\newcommand{\vapad}{\ensuremath{\widehat{\vap}}}
\newcommand{\khook}{\ensuremath{\mathbf{K}}}
\newcommand{\khookm}{\ensuremath{\mathbf{K}^{-1}}}
\newcommand{\lamb}{\ensuremath{\pmb{\Lambda}}}
\newcommand{\lambm}{\ensuremath{\pmb{\Lambda}^{-1}}}
\newcommand{\psie}{\ensuremath{\varphi^{*}}}
\newcommand{\psii}{\ensuremath{\varphi}}
\newcommand{\psit}{\ensuremath{\Dessous{\sim}{\varphi}}}
\newcommand{\psiet}{\ensuremath{\psit^{*}}}
\newcommand{\psiit}{\ensuremath{\psit}}
%\newcommand{\psieb}[1]{\ensuremath{\varphi^{*}(#1)}}
%\newcommand{\psiib}[1]{\ensuremath{\varphi(#1)}}
\newcommand{\moddep}{\ensuremath{\|\defopp\|}}
\newcommand{\moddepadun}{\ensuremath{\norme{\widehat{\dot{\defo}}^{\mathrm{p}}_{n+1}}}}
%
\newcommand{\pourtout}{\ensuremath{\forall t \in [0,T]}} 
\newcommand{\entnpun}{\ensuremath{\mbox{en $t_{n+1}$}}} 
\newcommand{\ip}[2]{\ensuremath{#1 \cdot #2}}
%\newcommand{\itransp}[2]{\ensuremath{#1^{t} #2}}
%\newcommand{\itransp}[2]{\ensuremath{\trace{#1}{#2}}}
\newcommand{\itransp}[2]{\ensuremath{#1\circ#2}}
\newcommand{\trace}[2]{\ensuremath{\mathrm{Tr}[#1#2]}}
%\newcommand{\normekm}[1]{\ensuremath{\|#1\|_{\khookm}}}
\newcommand{\tracel}[1]{\ensuremath{\|#1\|^{2}_{\lambm}}}
\newcommand{\tracek}[1]{\ensuremath{\|#1\|^{2}_{\khookm}}}
%\newcommand{\normek}[1]{\ensuremath{\|#1\|_{\khook}}}
\newcommand{\tracell}[1]{\ensuremath{\|#1\|^{2}_{\lamb}}}
\newcommand{\norme}[1]{\ensuremath{\|#1\|}}
\newcommand{\cet}{\ensuremath{C^{*}}}
\newcommand{\positive}[1]{\ensuremath{\langle #1 \rangle _{+}}}
\newcommand{\etaleg}[4]{\ensuremath{\eta(#1,#2,#3,#4)}}
%\newcommand{\etalegt}[4]{\ensuremath{\eta_{t}(#1,#2,#3,#4)}}
%\newcommand{\etalegtnp}[4]{\ensuremath{\eta_{t_{n+1}}(#1,#2,#3,#4)}}
%\newcommand{\etaleght}[4]{\ensuremath{\eta_{ht}(#1,#2,#3,#4)}}
\newcommand{\etalegtil}[4]{\ensuremath
{\Dessous{\sim}{\eta}(#1,#2,#3,#4)}}
\newcommand{\ineleg}[4]{\ensuremath{\psii(#1,-#2)+\psie(#3,#4)-
\trace{#3}{#1}+\itransp{#4}{#2}}}
\newcommand{\etalegb}[2]{\ensuremath{\eta(#1,#2)}}
\newcommand{\inelegb}[2]{\ensuremath{\psii(#1)+\psie(#2)+-
\trace{#1}{#2}}}
\newcommand{\dt}{\ensuremath{\Delta}}
\newcommand{\preci}{\ensuremath{\kappa}}
\newcommand{\limzero}[1]{\ensuremath{\lim_{#1 \rightarrow 0}}}
\newcommand{\eqli}[3]{
    #1 \qquad #3 \mid_{\dunomega} = \depimp  
    \mbox{\ (+conditions de r\'egularit\'e).}}
\newcommand{\eqlieng}[3]{
    #1 \qquad #3 \mid_{\dunomega} = \depimp  
    \mbox{\ (+regularity conditions).}}
\newcommand{\eqeq}[4]{
        \lefteqn{#1 \forall \dep^{*} \mbox{#3}} \nonumber \\
         & & #2{\trace{#4}{\defo(\dep^{*})}} - 
    \intesp{\ip{\fvolimp}{\dep^{*}}} -
    \intddeuom{\ip{\fsurfimp}{\dep^{*}}} = 0 }
\newcommand{\eqet}[7]{
        #1 \quad \left\{ 
    \begin{array}{ll}
        #2=\khook #3 & \qquad \mbox{o\`u} \quad #3=\defo(#6)-#7  \\
        #4=\lamb #5  & 
    \end{array} \right.}
\newcommand{\eqetbis}[7]{
        #1 \quad \left\{ 
    \begin{array}{ll}
        #2=\khook(\defo(#6)-#7) &  \\
        #4=\lamb #5  & 
    \end{array} \right.}
\newcommand{\eqtebis}[7]{
        #1 \quad \left\{ 
    \begin{array}{ll}
        #2=\khook(\defo(#6)-#7) &  \\
        #4=\mathbf{G}(#5)  & 
    \end{array} \right.}
\newcommand{\eqeteng}[7]{
        #1 \quad \left\{ 
    \begin{array}{ll}
        #2=\khook #3 & \qquad \mbox{where} \quad #3=\defo(#6)-#7  \\
        #4=\lamb #5  & 
    \end{array} \right.}
\newcommand{\eqin}[3]{
        #1 \qquad       #2=0 \quad #3=0  }
\newcommand{\eqev}[5]{
    \left( 
    \begin{array}{r}
        #1  \\
       -#2
    \end{array} \right) \in \left( 
    \begin{array}{r}
        \partial_{#3}\psie(#3,#4)  \\
        \partial_{#4}\psie(#3,#4)
    \end{array} \right) \quad #5}
\newcommand{\eqevpet}[5]{
    #5 \, \left( 
    \begin{array}{r}
        #1  \\
       -#2
    \end{array} \right) \in \left( 
    \begin{array}{r}
        \partial_{#3}\psie(#3,#4)  \\
        \partial_{#4}\psie(#3,#4)
    \end{array} \right)}
%%%%%%%%%%%%%%%%%%%%%%pour l'appendix%%%%%%%%%%%%%%%%%%%%%%%%%%%%%%
\newcommand{\deppl}{\ensuremath{u}}
\newcommand{\contpl}{\ensuremath{\sigma}}
\newcommand{\defopl}{\ensuremath{\epsilon}}
\newcommand{\defoppl}{\ensuremath{\epsilon^{\mathrm{p}}}}
\newcommand{\defoepl}{\ensuremath{\epsilon^{\mathrm{e}}}}
\newcommand{\defopplp}{\ensuremath{\dot{\defoppl}}}
\newcommand{\depdu}{\ensuremath{u}}
\newcommand{\contdu}{\ensuremath{\sigma}}
\newcommand{\defodu}{\ensuremath{\underbrace{\defopl}}}
\newcommand{\defopdu}{\ensuremath{\underbrace{\defoppl}}}
\newcommand{\defoedu}{\ensuremath{\underbrace{\defoepl}}}
\newcommand{\defopdup}{\ensuremath{\underbrace{\defopplp}}}
\newcommand{\lamp}{\ensuremath{\dot{\lambda}}}
%\newcommand{\sig}{\ensuremath{\Sigma}}
\newcommand{\gamp}{\ensuremath{\dot{\Gamma}}}
\newcommand{\depvec}{\ensuremath{\mathbf{q}}}
\newcommand{\mvec}{\ensuremath{\underline M}}
\newcommand{\nvec}{\ensuremath{\underline n}}
\newcommand{\tvec}{\ensuremath{\underline t}}
\newcommand{\vvec}{\ensuremath{\underline v}}
\newcommand{\xvec}{\ensuremath{\underline x}}
\newcommand{\nmat}{\ensuremath{\mathbf{N}}}
\newcommand{\gvi}{\ensuremath{\mathbf{G}}}
%%%%%%%%%%%%%%%%%%%%%%%%% racourci pour potentiel %%%%%%%%%%%%%%%%
\newcommand{\rac}{\ensuremath{\Gamma(\defopp,-\dot{p})}}
\newcommand{\racet}{\ensuremath{\Gamma^{*}(\cont,R)}}
%%%%%%%%%%%%%%%%%%%%%%%%% pour indicateur %%%%%%%%%%%%%%%%%%%%%%%%
\newcommand{\itpsref}{\ensuremath{\erel_{\mathrm{time}}}}
\newcommand{\iespref}{\ensuremath{\erel_{\mathrm{space}}}}
\newcommand{\itpsapp}{\ensuremath{i_{\mathrm{time}}}}
\newcommand{\iespapp}{\ensuremath{i_{\mathrm{space}}}}
\newcommand{\itpsappabs}{\ensuremath{I_{\mathrm{time}}}}
\newcommand{\iespappabs}{\ensuremath{I_{\mathrm{space}}}}
%\newcommand{\itimeabs}{\ensuremath{\mathbf{I}_{\mathrm{time}}}}
%\newcommand{\itimerel}{\ensuremath{i_{\mathrm{time}}}}
\newcommand{\dentime}{\ensuremath{D_{\mathrm{time}}}}
\newcommand{\ispaceabs}{\ensuremath{\mathbf{I}_{\mathrm{space}}}}
\newcommand{\ispacerel}{\ensuremath{i_{\mathrm{space}}}}
\newcommand{\denspace}{\ensuremath{D_{\mathrm{space}}}}
\newcommand{\iaccuabs}{\ensuremath{\mathbf{I}_{accu}}}
\newcommand{\iaccurel}{\ensuremath{i_{accu}}}
\newcommand{\denaccu}{\ensuremath{D_{accu}}}
\newcommand{\etcrel}{\ensuremath{\erel_{\mathrm{\tps}}}}
\newcommand{\eecrel}{\ensuremath{\erel_{\mathrm{\esp}}}}
\newcommand{\ecrel}{\ensuremath{\erel_{\mathrm{\ite}}}}
\newcommand{\itcrel}{\ensuremath{i_{\mathrm{\tps}}}}
\newcommand{\iecrel}{\ensuremath{i_{\mathrm{\esp}}}}
\newcommand{\icrel}{\ensuremath{i_{\mathrm{\ite}}}}
\newcommand{\tol}{\ensuremath{\delta_\mathrm{tol}}}
\newcommand{\pas}{\ensuremath{\Delta t}}
\newcommand{\tolcrit}{\ensuremath{\theta_{\mathrm{crit}}}}
\newcommand{\itcrelpa}{\ensuremath{i_{\mathrm{\tps},\Delta t}}}
\newcommand{\itcrelpaet}{\ensuremath{i_{\mathrm{\tps},\Delta t^{*}}}}
\newcommand{\iecrele}{\ensuremath{i_{\mathrm{\esp},E}}}
\newcommand{\iecreleet}{\ensuremath{i_{\mathrm{\esp},E^{*}}}}
\newcommand{\itcrelz}{\ensuremath{i_{\mathrm{\tps},0}}}
\newcommand{\iecrelz}{\ensuremath{i_{\mathrm{\esp},0}}}
\newcommand{\icrelz}{\ensuremath{i_{\mathrm{\ite},0}}}
\newcommand{\ereltc}{\ensuremath{\erel_{\mathrm{\tps}}}}
\newcommand{\erelec}{\ensuremath{\erel_{\mathrm{\esp}}}}
\newcommand{\erelc}{\ensuremath{\erel_{\mathrm{\ite}}}}
\newcommand{\itcabs}{\ensuremath{I_{\mathrm{\tps}}}}
\newcommand{\iecabs}{\ensuremath{I_{\mathrm{\esp}}}}
\newcommand{\icabs}{\ensuremath{I_{\mathrm{\ite}}}}
%%%%%%%%%%%%%%%%%%%%%%%%% pour drucker %%%%%%%%%%%%%%%%%%%%%%%%%%%
%\newcommand{\depca}{\ensuremath{\dep_{\mathrm{CA}}}}
%\newcommand{\contsa}{\ensuremath{\cont_{\mathrm{SA}}}}
%\newcommand{\defoca}{\ensuremath{\defo_{\mathrm{CA}}}}
\newcommand{\depca}{\ensuremath{\hat{\dep}}}
\newcommand{\contsa}{\ensuremath{\hat{\cont}}}
\newcommand{\defoca}{\ensuremath{\hat{\defo}}}
\newcommand{\contsah}{\ensuremath{\cont^{\mathrm{SA}_{h}}}}
\newcommand{\contca}{\ensuremath{\cont_{\mathrm{CA}}}}
\newcommand{\defosa}{\ensuremath{\defo_{\mathrm{SA}}}}
\newcommand{\defosap}{\ensuremath{\dot{\defo}_{\mathrm{SA}}}}
\newcommand{\defocap}{\ensuremath{\dot{\defo}_{\mathrm{CA}}}}
\newcommand{\defosahp}{\ensuremath{\dot{\defo}^{\mathrm{SA}_{h}}}}
\newcommand{\defocahp}{\ensuremath{\dot{\defo}^{\mathrm{CA}_{h}}}}
\newcommand{\defosah}{\ensuremath{\defo^{\mathrm{SA}_{h}}}}
\newcommand{\defocah}{\ensuremath{\defo^{\mathrm{CA}_{h}}}}
\newcommand{\dru}{\ensuremath{\delta}}
\newcommand{\intdru}[2]{\left\langle#2\right\rangle_{#1}}  
%%%%%%%%%%%% 1
%%% avec include graphics on ne peut voir les graphiques sous mac %%%
%%% draft pour visu rapide des fig

%%%%%%%%%% Pour les proprietes
\newtheorem{propp}{Propri\'et\'e}
\newtheorem{property}{Property}
\newtheorem{probmin}{Probl\`eme}
\newtheorem{probenmin}{Problem}
\newcommand{\espx}{\ensuremath{\mathbf{X}}}
\newcommand{\yet}{\ensuremath{y^{*}}}
\newcommand{\xet}{\ensuremath{x^{*}}}
\newcommand{\xbar}{\ensuremath{\overline{x}}}
\newcommand{\ybar}{\ensuremath{\overline{y}}}
\newcommand{\ypri}{\ensuremath{y'}}
\newcommand{\xpri}{\ensuremath{x'}}
%%%%%%%%%%% Pour Babuska
\newcommand{\edge}{\ensuremath{\mathcal{E}}}
%%%%%%%%%%% Pour transparent vendredi
\newcommand{\xex}{\ensuremath{\overline{x}}}
\newcommand{\yex}{\ensuremath{\overline{y}}}
\newcommand{\rele}{\ensuremath{r_{E}}}
\newcommand{\rpas}{\ensuremath{r_{\Delta t}}}
\newcommand{\rtol}{\ensuremath{r_{\tol}}}
\newcommand{\ele}{\'el\'ement}
\newcommand{\eles}{\'el\'ements}







%%%%%%%%%%For damage
\newcommand{\damadua}{\ensuremath{Y}}
\newcommand{\damavec}{\ensuremath{\underline{\dama}}}
\newcommand{\damaduaz}{\ensuremath{\damadua_{0}}}
\newcommand{\damaduac}{\ensuremath{\damadua_{\mathrm{c}}}}
\newcommand{\damaecr}{\ensuremath{\alpha}}
\newcommand{\damaecrp}{\ensuremath{\dot{\damaecr}}}
\newcommand{\damaecrdua}{\ensuremath{\beta}}
\newcommand{\khookz}{\ensuremath{\khook_{0}}}
\newcommand{\dama}{\ensuremath{d}}
\newcommand{\damap}{\ensuremath{\dot{\dama}}}

\newcommand{\ccp}{\ensuremath{C_{\mathrm{p}}}}
\newcommand{\ccd}{\ensuremath{C_{\mathrm{d}}}}
\newcommand{\ccdet}{\ensuremath{C^{*}_{\mathrm{d}}}}
\newcommand{\ccv}{\ensuremath{C_{\mathrm{v}}}}

\newcommand{\conteff}{\tilde{\cont}}
\newcommand{\defoppeff}{\ensuremath{\tilde{\dot{\defo}}^{\mathrm{p}}}} 
\newcommand{\defopeff}{\ensuremath{\tilde{\defo}^{\mathrm{p}}}}
\newcommand{\ppeff}{\ensuremath{\tilde{\dot{p}}}}
\newcommand{\peff}{\ensuremath{\tilde{p}}}
\newcommand{\reff}{\ensuremath{\tilde{R}}}

\newcommand{\itot}{\ensuremath{i_{\mathrm{tot}}}}
\newcommand{\nite}{\ensuremath{N_{\mathrm{ite}}}}





%%%%%%%%%%%%%%%%%%%%%%%%%%%%%%%%%%
%
%	conco et conco2
%
%%%%%%%%%%%%%%%%%%%%%%%%%%%%%%%%%%
% This file defines new characters to be used in Latex files 
% -------------------------------------------------------------------
%
% -------------------------------------------------------------------
% Bold letters in math style
% -------------------------------------------------------------------
\newcommand{\bfa}{\mbox{\boldmath $a$}}
\newcommand{\bfb}{\mbox{\boldmath $b$}}
\newcommand{\bfc}{\mbox{\boldmath $c$}}
\newcommand{\bfd}{\mbox{\boldmath $d$}}
\newcommand{\bfe}{\mbox{\boldmath $e$}}
\newcommand{\bff}{\mbox{\boldmath $f$}}
\newcommand{\bfg}{\mbox{\boldmath $g$}}
\newcommand{\bfh}{\mbox{\boldmath $h$}}
\newcommand{\bfi}{\mbox{\boldmath $i$}}
\newcommand{\bfj}{\mbox{\boldmath $j$}}
\newcommand{\bfk}{\mbox{\boldmath $k$}}
\newcommand{\bfl}{\mbox{\boldmath $l$}}
\newcommand{\bfm}{\mbox{\boldmath $m$}}
\newcommand{\bfn}{\mbox{\boldmath $n$}}
\newcommand{\bfo}{\mbox{\boldmath $o$}}
\newcommand{\bfp}{\mbox{\boldmath $p$}}
\newcommand{\bfq}{\mbox{\boldmath $q$}}
\newcommand{\bfr}{\mbox{\boldmath $r$}}
\newcommand{\bfs}{\mbox{\boldmath $s$}}
\newcommand{\bft}{\mbox{\boldmath $t$}}
\newcommand{\bfu}{\mbox{\boldmath $u$}}
\newcommand{\bfv}{\mbox{\boldmath $v$}}
\newcommand{\bfw}{\mbox{\boldmath $w$}}
\newcommand{\bfx}{\mbox{\boldmath $x$}}
\newcommand{\bfy}{\mbox{\boldmath $y$}}
\newcommand{\bfz}{\mbox{\boldmath $z$}}
%
\newcommand{\bfA}{\mbox{\boldmath $A$}}
\newcommand{\bfB}{\mbox{\boldmath $B$}}
\newcommand{\bfC}{\mbox{\boldmath $C$}}
\newcommand{\bfD}{\mbox{\boldmath $D$}}
\newcommand{\bfE}{\mbox{\boldmath $E$}}
\newcommand{\bfF}{\mbox{\boldmath $F$}}
\newcommand{\bfG}{\mbox{\boldmath $G$}}
\newcommand{\bfH}{\mbox{\boldmath $H$}}
\newcommand{\bfI}{\mbox{\boldmath $I$}}
\newcommand{\bfJ}{\mbox{\boldmath $J$}}
\newcommand{\bfK}{\mbox{\boldmath $K$}}
\newcommand{\bfL}{\mbox{\boldmath $L$}}
\newcommand{\bfM}{\mbox{\boldmath $M$}}
\newcommand{\bfN}{\mbox{\boldmath $N$}}
\newcommand{\bfO}{\mbox{\boldmath $O$}}
\newcommand{\bfP}{\mbox{\boldmath $P$}}
\newcommand{\bfQ}{\mbox{\boldmath $Q$}}
\newcommand{\bfR}{\mbox{\boldmath $R$}}
\newcommand{\bfS}{\mbox{\boldmath $S$}}
\newcommand{\bfT}{\mbox{\boldmath $T$}}
\newcommand{\bfU}{\mbox{\boldmath $U$}}
\newcommand{\bfV}{\mbox{\boldmath $V$}}
\newcommand{\bfW}{\mbox{\boldmath $W$}}
\newcommand{\bfX}{\mbox{\boldmath $X$}}
\newcommand{\bfY}{\mbox{\boldmath $Y$}}
\newcommand{\bfZ}{\mbox{\boldmath $Z$}}
%
% -------------------------------------------------------------------
% Greek letters
% -------------------------------------------------------------------
\newcommand{\alp}{{\alpha}}
\newcommand{\bet}{{\beta}}
\newcommand{\gam}{{\gamma}}
\newcommand{\del}{{\delta}}
\newcommand{\eps}{{\epsilon}}
\newcommand{\vareps}{{\varepsilon}}
\newcommand{\zet}{{\zeta}}
\newcommand{\thet}{{\theta}}
\newcommand{\iot}{{\iota}}
\newcommand{\kap}{{\kappa}}
\newcommand{\lam}{{\lambda}}
\newcommand{\sig}{{\sigma}}
\newcommand{\ups}{{\upsilon}}
\newcommand{\ome}{{\omega}}
%
\newcommand{\Gam}{{\Gamma}}
\newcommand{\Del}{{\Delta}}
\newcommand{\Thet}{{\Theta}}
\newcommand{\Lam}{{\Lambda}}
\newcommand{\Sig}{{\Sigma}}
\newcommand{\Ups}{{\Upsilon}}
\newcommand{\Ome}{{\Omega}}
%
% -------------------------------------------------------------------
% Bold greek letters
% -------------------------------------------------------------------
\newcommand{\bfalp}{\mbox{\boldmath $\alpha$}}
\newcommand{\bfbet}{\mbox{\boldmath $\beta$}}
\newcommand{\bfgam}{\mbox{\boldmath $\gamma$}}
\newcommand{\bfdel}{\mbox{\boldmath $\delta$}}
\newcommand{\bfeps}{\mbox{\boldmath $\epsilon$}}
\newcommand{\bfvareps}{\mbox{\boldmath $\varepsilon$}}
\newcommand{\bfvarphi}{\mbox{\boldmath $\varphi$}}
\newcommand{\bfzet}{\mbox{\boldmath $\zeta$}} 
\newcommand{\bfeta}{\mbox{\boldmath $\eta$}} 
\newcommand{\bfthet}{\mbox{\boldmath $\theta$}}
\newcommand{\bfiot}{\mbox{\boldmath $\iota$}}
\newcommand{\bfkap}{\mbox{\boldmath $\kappa$}}
\newcommand{\bflam}{\mbox{\boldmath $\lambda$}}
\newcommand{\bfmu}{\mbox{\boldmath $\mu$}}
\newcommand{\bfnu}{\mbox{\boldmath $\nu$}}
\newcommand{\bfxi}{\mbox{\boldmath $\xi$}}
\newcommand{\bfpi}{\mbox{\boldmath $\pi$}}
\newcommand{\bfrho}{\mbox{\boldmath $\rho$}}
\newcommand{\bfsig}{\mbox{\boldmath $\sigma$}}
\newcommand{\bftau}{\mbox{\boldmath $\tau$}}
\newcommand{\bfups}{\mbox{\boldmath $\upsilon$}}
\newcommand{\bfphi}{\mbox{\boldmath $\phi$}}
\newcommand{\bfchi}{\mbox{\boldmath $\chi$}}
\newcommand{\bfpsi}{\mbox{\boldmath $\psi$}}
\newcommand{\bfome}{\mbox{\boldmath $\omega$}}
%
\newcommand{\bfGam}{\mbox{\boldmath $\Gamma$}}
\newcommand{\bfDel}{\mbox{\boldmath $\Delta$}}
\newcommand{\bfThet}{\mbox{\boldmath $\Theta$}}
\newcommand{\bfLam}{\mbox{\boldmath $\Lambda$}}
\newcommand{\bfXi}{\mbox{\boldmath $\Xi$}}
\newcommand{\bfPi}{\mbox{\boldmath $\Pi$}}
\newcommand{\bfSig}{\mbox{\boldmath $\Sigma$}}
\newcommand{\bfUps}{\mbox{\boldmath $\Upsilon$}}
\newcommand{\bfPhi}{\mbox{\boldmath $\Phi$}}
\newcommand{\bfPsi}{\mbox{\boldmath $\Psi$}}
\newcommand{\bfOme}{\mbox{\boldmath $\Omega$}}
%
% -------------------------------------------------------------------
% Letters with the tilde accents
% -------------------------------------------------------------------
\newcommand{\tia}{\tilde{a}}
\newcommand{\tib}{\tilde{b}}
\newcommand{\tic}{\tilde{c}}
\newcommand{\tid}{\tilde{d}}
\newcommand{\tie}{\tilde{e}}
\newcommand{\tif}{\tilde{f}}
\newcommand{\tig}{\tilde{g}}
\newcommand{\tih}{\tilde{h}}
\newcommand{\tii}{\tilde{i}}
\newcommand{\tij}{\tilde{j}}
\newcommand{\tik}{\tilde{k}}
\newcommand{\til}{\tilde{l}}
\newcommand{\tim}{\tilde{m}}
\newcommand{\tin}{\tilde{n}}
\newcommand{\tio}{\tilde{o}}
\newcommand{\tip}{\tilde{p}}
\newcommand{\tiq}{\tilde{q}}
\newcommand{\tir}{\tilde{r}}
\newcommand{\tis}{\tilde{s}}
\newcommand{\tit}{\tilde{t}}
\newcommand{\tiu}{\tilde{u}}
\newcommand{\tiv}{\tilde{v}}
\newcommand{\tiw}{\tilde{w}}
\newcommand{\tix}{\tilde{x}}
\newcommand{\tiy}{\tilde{y}}
\newcommand{\tiz}{\tilde{z}}
%
\newcommand{\wtiA}{\widetilde{A}}
\newcommand{\wtiB}{\widetilde{B}}
\newcommand{\wtiC}{\widetilde{C}}
\newcommand{\wtiD}{\widetilde{D}}
\newcommand{\wtiE}{\widetilde{E}}
\newcommand{\wtiF}{\widetilde{F}}
\newcommand{\wtiG}{\widetilde{G}}
\newcommand{\wtiH}{\widetilde{H}}
\newcommand{\wtiI}{\widetilde{I}}
\newcommand{\wtiJ}{\widetilde{J}}
\newcommand{\wtiK}{\widetilde{K}}
\newcommand{\wtiL}{\widetilde{L}}
\newcommand{\wtiM}{\widetilde{M}}
\newcommand{\wtiN}{\widetilde{N}}
\newcommand{\wtiO}{\widetilde{O}}
\newcommand{\wtiP}{\widetilde{P}}
\newcommand{\wtiQ}{\widetilde{Q}}
\newcommand{\wtiR}{\widetilde{R}}
\newcommand{\wtiS}{\widetilde{S}}
\newcommand{\wtiT}{\widetilde{T}}
\newcommand{\wtiU}{\widetilde{U}}
\newcommand{\wtiV}{\widetilde{V}}
\newcommand{\wtiW}{\widetilde{W}}
\newcommand{\wtiX}{\widetilde{X}}
\newcommand{\wtiY}{\widetilde{Y}}
\newcommand{\wtiZ}{\widetilde{Z}}
%
\newcommand{\tibfa}{\tilde{\bfa}}
\newcommand{\tibfb}{\tilde{\bfb}}
\newcommand{\tibfc}{\tilde{\bfc}}
\newcommand{\tibfd}{\tilde{\bfd}}
\newcommand{\tibfe}{\tilde{\bfe}}
\newcommand{\tibff}{\tilde{\bff}}
\newcommand{\tibfg}{\tilde{\bfg}}
\newcommand{\tibfh}{\tilde{\bfh}}
\newcommand{\tibfi}{\tilde{\bfi}}
\newcommand{\tibfj}{\tilde{\bfj}}
\newcommand{\tibfk}{\tilde{\bfk}}
\newcommand{\tibfl}{\tilde{\bfl}}
\newcommand{\tibfm}{\tilde{\bfm}}
\newcommand{\tibfn}{\tilde{\bfn}}
\newcommand{\tibfo}{\tilde{\bfo}}
\newcommand{\tibfp}{\tilde{\bfp}}
\newcommand{\tibfq}{\tilde{\bfq}}
\newcommand{\tibfr}{\tilde{\bfr}}
\newcommand{\tibfs}{\tilde{\bfs}}
\newcommand{\tibft}{\tilde{\bft}}
\newcommand{\tibfu}{\tilde{\bfu}}
\newcommand{\tibfv}{\tilde{\bfv}}
\newcommand{\tibfw}{\tilde{\bfw}}
\newcommand{\tibfx}{\tilde{\bfx}}
\newcommand{\tibfy}{\tilde{\bfy}}
\newcommand{\tibfz}{\tilde{\bfz}}
%
% -------------------------------------------------------------------
% Letters with hat accent
% -------------------------------------------------------------------
\newcommand{\whA}{\widehat{A}}
\newcommand{\whB}{\widehat{B}}
\newcommand{\whC}{\widehat{C}}
\newcommand{\whD}{\widehat{D}}
\newcommand{\whE}{\widehat{E}}
\newcommand{\whF}{\widehat{F}}
\newcommand{\whG}{\widehat{G}}
\newcommand{\whH}{\widehat{H}}
\newcommand{\whI}{\widehat{I}}
\newcommand{\whJ}{\widehat{J}}
\newcommand{\whK}{\widehat{K}}
\newcommand{\whL}{\widehat{L}}
\newcommand{\whM}{\widehat{M}}
\newcommand{\whN}{\widehat{N}}
\newcommand{\whO}{\widehat{O}}
\newcommand{\whP}{\widehat{P}}
\newcommand{\whQ}{\widehat{Q}}
\newcommand{\whR}{\widehat{R}}
\newcommand{\whS}{\widehat{S}}
\newcommand{\whT}{\widehat{T}}
\newcommand{\whU}{\widehat{U}}
\newcommand{\whV}{\widehat{V}}
\newcommand{\whW}{\widehat{W}}
\newcommand{\whX}{\widehat{X}}
\newcommand{\whY}{\widehat{Y}}
\newcommand{\whZ}{\widehat{Z}}
%
\newcommand{\ha}{\hat{a}}
\newcommand{\hb}{\hat{b}}
\newcommand{\hc}{\hat{c}}
\newcommand{\hd}{\hat{d}}
\newcommand{\he}{\hat{e}}
\newcommand{\hf}{\hat{f}}
\newcommand{\hg}{\hat{g}}
\newcommand{\hh}{\hat{h}}
\newcommand{\hi}{\hat{i}}
\newcommand{\hj}{\hat{j}}
\newcommand{\hk}{\hat{k}}
\newcommand{\hl}{\hat{l}}
\newcommand{\hm}{\hat{m}}
\newcommand{\hn}{\hat{n}}
\newcommand{\ho}{\hat{o}}
\newcommand{\hap}{\hat{p}}
\newcommand{\hq}{\hat{q}}
\newcommand{\hr}{\hat{r}}
\newcommand{\hs}{\hat{s}}
\newcommand{\htt}{\hat{t}}
\newcommand{\hu}{\hat{u}}
\newcommand{\hv}{\hat{v}}
\newcommand{\hw}{\hat{w}}
\newcommand{\hx}{\hat{x}}
\newcommand{\hy}{\hat{y}}
\newcommand{\hz}{\hat{z}}
%
% -------------------------------------------------------------------
% Calligraphic letters
% -------------------------------------------------------------------
\newcommand{\ca}{{\cal A}}
\newcommand{\cb}{{\cal B}}
\newcommand{\cc}{{\cal C}}
\newcommand{\cd}{{\cal D}}
\newcommand{\ce}{{\cal E}}
\newcommand{\cf}{{\cal F}}
\newcommand{\cg}{{\cal G}}
\newcommand{\ch}{{\cal H}}
\newcommand{\ci}{{\cal I}}
\newcommand{\cj}{{\cal J}}
\newcommand{\ck}{{\cal K}}
\newcommand{\cl}{{\cal L}}
\newcommand{\cm}{{\cal M}}
\newcommand{\cn}{{\cal N}}
\newcommand{\co}{{\cal O}}
\newcommand{\cp}{{\cal P}}
\newcommand{\cq}{{\cal Q}}
\newcommand{\car}{{\cal R}}
\newcommand{\cs}{{\cal S}}
\newcommand{\ct}{{\cal T}}
\newcommand{\cu}{{\cal U}}
\newcommand{\cv}{{\cal V}}
\newcommand{\cw}{{\cal W}}
\newcommand{\cx}{{\cal X}}
\newcommand{\cy}{{\cal Y}}
\newcommand{\cz}{{\cal Z}}
%

% This file defines new characters to be used in Latex files 
% -------------------------------------------------------------------
%
% -------------------------------------------------------------------
% Roman letters in math style
% -------------------------------------------------------------------
\newcommand{\rma}{\ensuremath{\mathrm{a}}}
\newcommand{\rmb}{\ensuremath{\mathrm{b}}}
\newcommand{\rmc}{\ensuremath{\mathrm{c}}}
\newcommand{\rmd}{\ensuremath{\mathrm{d}}}
\newcommand{\rme}{\ensuremath{\mathrm{e}}}
\newcommand{\rmf}{\ensuremath{\mathrm{f}}}
\newcommand{\rmg}{\ensuremath{\mathrm{g}}}
\newcommand{\rmh}{\ensuremath{\mathrm{h}}}
\newcommand{\rmi}{\ensuremath{\mathrm{i}}}
\newcommand{\rmj}{\ensuremath{\mathrm{j}}}
\newcommand{\rmk}{\ensuremath{\mathrm{k}}}
\newcommand{\rml}{\ensuremath{\mathrm{l}}}
\newcommand{\rmm}{\ensuremath{\mathrm{m}}}
\newcommand{\rmn}{\ensuremath{\mathrm{n}}}
\newcommand{\rmo}{\ensuremath{\mathrm{o}}}
\newcommand{\rmp}{\ensuremath{\mathrm{p}}}
\newcommand{\rmq}{\ensuremath{\mathrm{q}}}
\newcommand{\rmr}{\ensuremath{\mathrm{r}}}
\newcommand{\rms}{\ensuremath{\mathrm{s}}}
\newcommand{\rmt}{\ensuremath{\mathrm{t}}}
\newcommand{\rmu}{\ensuremath{\mathrm{u}}}
\newcommand{\rmv}{\ensuremath{\mathrm{v}}}
\newcommand{\rmw}{\ensuremath{\mathrm{w}}}
\newcommand{\rmx}{\ensuremath{\mathrm{x}}}
\newcommand{\rmy}{\ensuremath{\mathrm{y}}}
\newcommand{\rmz}{\ensuremath{\mathrm{z}}}

\newcommand{\bfnabla}{\mbox{\boldmath $\nabla$}}

\newcommand{\bfzer}{\mbox{\boldmath 0}}
\newcommand{\bfone}{\mbox{\boldmath 1}}
\newcommand{\txton}{\mbox{\ on\ }}
\newcommand{\txtst}{\mbox{\ s.t.\ }}
\newcommand{\txtfor}{\mbox{\ for \ }}

%\newcommand{\beq}{\begin{equation}}
%\newcommand{\eeq}{\end{equation}}
\newcommand{\ov}[1]{\overline{#1}}
\newcommand{\dint}{\mbox{\, \rmd}}

%%%%%%%%%%%%%%%%%%%%%%%%%%%%%%
%
%	figu
%
%%%%%%%%%%%%%%%%%%%%%%%%%%%%%%
%%%%%%%%%%%% 1
%%% avec include graphics on ne peut voir les graphiques sous mac %%%
%%% draft pour visu rapide des fig
\newcommand{\tableau}[2]{
\begin{center}
   \include{#1}
    \caption{#2}
    \protect\label{tab:#1}
\end{center}}
%%%%
%%%%
\newcommand{\mpcm}[3]{
\begin{minipage}[#2]{#1}
#3
\end{minipage}}

\newcommand{\figw}[2]{\includegraphics[width=#1\hsize]{#2}}

\newcommand{\figsc}[2]{
\begin{minipage}[t]{#1}
\includegraphics[width=#1]{#2}
\end{minipage}}
%%%%
\newcommand{\figscc}[2]{
\begin{minipage}[c]{#1}
\includegraphics[width=#1]{#2}
\end{minipage}}
%%%%
\newcommand{\eqsref}[2]{
\begin{minipage}[c]{#1}
\vspace{-\abovedisplayskip}
\[ #2 \]
\end{minipage}}
%%%%
\newcommand{\txsref}[2]{
\begin{minipage}[c]{#1}
#2
\end{minipage}}
%%%
\newcommand{\figca}[2]{
\begin{minipage}[t]{#1}
{\begin{flushleft} #2 \end{flushleft}}
\end{minipage}}
%%%
\newcommand{\figcm}[3]{
\begin{minipage}[t]{#1}
\includegraphics[width=#1]{#2}
        \protect\caption{#3}
        \protect\label{fig:#2}
\end{minipage}}

\newcommand{\figcmsl}[3]{
\begin{minipage}[t]{#1}
\includegraphics[width=#1]{#2}
        \protect\caption{#3}
\end{minipage}}


%%%
\newcommand{\figun}[2]{
\begin{minipage}[]{10cm}
\includegraphics[width=10.cm]{#1}
        \protect\caption{#2}
        \protect\label{fig:#1}
\end{minipage}}
%%%
\newcommand{\fig}[2]{
\begin{minipage}[]{6cm}
\includegraphics[width=6.cm]{#1}
        \protect\caption{#2}
        \protect\label{fig:#1}
\end{minipage}}






%%%%%%%%%%%%
\newcommand{\figsemcm}[3]{
\begin{minipage}[b]{#1}
\includegraphics[width=#1]{#2}
        \protect\caption{\large \bf #3}
        \protect\label{fig:#2}
\end{minipage}}
%%%%%%%%%%%%
\newcommand{\figdefense}[2]{
\begin{minipage}[c]{#1}
\includegraphics[width=#1]{#2}
%       \protect\caption{\large \bf #3}
%       \protect\label{fig:#2}
\end{minipage}}
%%%%%%%%%%%%
\newcommand{\tabsemcm}[3]{
\begin{minipage}[b]{#1}
\include{#2}
\protect\caption{#3}
\protect\label{tab:#2}
\end{minipage}}
\newcommand{\figunseul}[2]{
\unitlength1cm
\HideDisplacementBoxes
%\ShowDisplacementBoxes
\begin{minipage}[]{10cm}
    \ForceWidth{10cm}
    \BoxedEPSF{#1}
        \caption{#2}
        \protect\label{fig:#1}
\end{minipage}}


\usepackage[pagebackref, colorlinks=true, hyperindex=true,
pdftitle={XfemGuide}, pdfauthor={Nicolas~Moes~et~Gilles~Marckmann}
]{hyperref}

\usepackage{makeidx}
\makeindex

%%%%%%%%%%%%%%%%%%%%%%%%%%%%%%%%%%%%%%%%%%%%%%%%%%%%%%%%%%%%%%%%%%%
%
%           The   XFEM Guide
%
%%%%%%%%%%%%%%%%%%%%%%%%%%%%%%%%%%%%%%%%%%%%%%%%%%%%%%%%%%%%%%%%%%%

\title{The \texttt{xfem} User Guide}

\author{\Large{Nicolas M\"oes et Gilles Marckmann} \\   \\
 avec les contributions de Yosra Guetari et Eric Bechet \\ \\
    \emph{Institut de Recherche en G\'enie Civil et M\'ecanique}
    \\ \emph{Ecole Central de Nantes }
    }

\date{\today}
\makeindex
\setcounter{tocdepth}{3}

\usepackage[francais]{babel}

%%%%%%%%%%%%%%%%%%%%%%%%%%%%%%%%%%%%%%%%%%%%%%%%%%%%%%%%%%%%%%%%%%%
%
%           Begin Document
%
%%%%%%%%%%%%%%%%%%%%%%%%%%%%%%%%%%%%%%%%%%%%%%%%%%%%%%%%%%%%%%%%%%%
\begin{document}

%\DeclareGraphicsExtensions{.eps,.pdf,.png,.jpg,.mps}

\maketitle

%%%%%%%%%%%%%%%%%%%%%%%%%%%%%%%%%%%%%%%%%%%%%%%%%%%%%%%%%
%
% une page de d\'edicace en commentaire
%
%%%%%%%%%%%%%%%%%%%%%%%%%%%%%%%%%%%%%%%%%%%%%%%%%%%%%%%%%

\clearpage
%
%
%  \clearpage
%  \begin{minipage}[t][10cm][b]{.8\textwidth}
%  \large
% \textbf{LOI n� 94-665 du 4 ao�t 1994 relative \`a l'emploi de la langue fran�aise }
% Art. 7. -\emph{Les publications, revues et communications diffus\'ees en France et qui \'emanent d'une personne morale de droit public, d'une personne priv\'ee exer�ant une mission de service public ou d'une personne priv\'ee b\'en\'eficiant d'une subvention publique doivent, lorsqu'elles sont r\'edig\'ees en langue \'etrang�re, comporter au moins un r\'esum\'e en fran�ais.}
%  \normalsize
%  \end{minipage}




%%%%%%%%%%%%%%%%%%%%%%%%%%%%%%%%%%%%%%%%%%%%%%%%%%%%%%%%%
%
% Insert Table of Contents; List of Figures and Tables
%
%%%%%%%%%%%%%%%%%%%%%%%%%%%%%%%%%%%%%%%%%%%%%%%%%%%%%%%%%

\small
\tableofcontents
\normalsize

%%%%%%%%%%%%%%%%%%%%%%%%%%%%%%%%%%%%%%%%%%%%%%%%%%%%%%%%%
%   PARTIES ET CHAPITRES
%%%%%%%%%%%%%%%%%%%%%%%%%%%%%%%%%%%%%%%%%%%%%%%%%%%%%%%%%

\part*{}
    \addcontentsline{toc}{chapter}{Introduction G\'en\'erale}
    \markboth{Introduction G\'en\'erale}{Introduction G\'en\'erale}
    \chapter*{Introduction G\'en\'erale}
                           
\section*{Qu'est-ce que \texttt{xfem} }\label{IntroGene}

\code{xfem} est une libraire de classes et de fonctions C++, document\'ee en partie par ce
pr\'esent document. Elle permet de mettre en oeuvre des applications
de calculs fond\'ees sur l'approche des El\'ements Finis Etendus (X-FEM en anglais pour Extended Finite Element Method, voir
chapitre~\ref{elements_finis_etendus}). Cette librairie utilise \`a la fois les possibilit\'es de la programmation objets et de la programmation g\'en\'erique (template) du C++. Elle ne constitue
pas \`a proprement parl\'e une application ou un code de calcul
particulier, mais doit \'etre consid\'er\'ee comme une boite \`a outils de
\guil{fonctions} de bases assimilables aux fonctions ou commandes
Matlab op\'erant sur des objets d\'eclar\'es par l'utilisateur. Un certain nombre d'applications particuli\'ere ont n\'eanmoins \'et\'e d\'evelopp\'ees et sont accessibles dans le r\'epertoire \code{Applis}.

\code{xfem}  est d\'evelopp\'e de mani\'ere collaborative par des
chercheurs appartenant \`a des \'equipes ou des laboratoires diff\'erents.
Pour cette raison, cette librairie est d\'evelopp\'ee avec l'aide d'un
logiciel de d\'eveloppement collaboratif ; autrefois CVS puis aujourd'hui SVN (chapitre~\ref{InstallationXfem}) permettant de contr\'eler la coh\'erence des diff\'erents
d\'eveloppements. Il permet de charger une version du code  et
de soumettre des nouveaux d\'eveloppements. 
   
Pour assurer que les d\'eveloppements ne viennent pas corrompre les
d\'eveloppements ant\'erieurs, des cas tests sont ex\'ecut\'es toutes les
nuits par des t\^aches planifi\'ees qui comparent les nouveau r\'esultats
aux r\'esultats ant\'erieurs et signale toute anomalie.  Ces m\'emes cas tests
doivent \^etre ex\'ecut\'es par les d\'eveloppeurs avant toute diffusion de
leurs nouveaux d\'eveloppements afin de v\'erifier que leur apport ne perturbent pas les d\'eveloppement pr\'ec\'edents. Ceci ce fait simplement par une commande (\code{make check}: voir le chapitre \ref{Compilation}).

La p\'erennit\'e du d\'eveloppement de la librairie xfem ne peut \^etre assur\'ee que par une
documentation syst\'ematique. Celle-ci peut \^etre assur\'ee par chaque d\'eveloppeur dans les d\'eveloppements auquel il contribue. Certains commentaires particuliers peuvent \^etre d\'estin\'es \`a la documentation automatique par DOXYGEN (voir le chapitre~\ref{doxygen}). Ce logiciel qui g\'en\'ere une documentation syst\'ematique \`a partir des sources C++ peut \'egalement  interpr\'et\'es certains commentaires qu'il ins\'ere dans la documentation (voir \web{http://intraweb.ec-nantes.fr/labnet} rubrique D\'eveloppement, puis DOXYGEN).


\section*{Objectifs de ce document}
Le but de ce document est de  fournir un premier support permettant
aux nouveaux d\'eveloppeurs de se lancer dans des d\'eveloppements C++ 
utilisant la librairie \code{xfem} et les librairies associ\'ees, en particulier avec 
les outils utilis\'es a l'ECN (svn, emacs, BuildUtil, ...).

Un autre but de cette documentation est d'offrir une vue d'ensemble
de la librairie  \code{xfem} afin d'am\'eliorer le code, les cas-tests
et les capacit\'es de d\'eveloppement de chacun autour de la librairie \code{xfem}. 
Aujourd'hui, ce but n'est pas encore atteint, mais il appartient \`a chacun de faire \'evoluer 
ce document (voir chapitre~\ref{doclatex}), lui m\'eme en d\'eveloppement collaboratif.
%\code{/Xfem/Xfem/Xfem/src_doc})

\section*{Organisation du document}
Ce guide est d\'ecompos\'e en plusieurs parties :
\begin{itemize}
\item[-] la partie~\ref{GettingStarted} fournit les premiers principes de fonctionnement. Pour le d\'ebutant, elle fournit les instructions pour installer et compiler les librairies en expliquant les outils utilis\'es (SVN, BuildUtil,...). Puis le lancement d'un cas test est d\'ecrit afin de v\'erifier que tout est bien installer. 

\item[-] la partie~\ref{tools_and_lib} fournit une aide sur les utiliatires \code{emacs} et \code{gmsh}. Concernant \code{emacs}, le d\'ebutant trouvera les quelques commandes \`a retenir pour commencer \`a d\'evelopper avec cet \'editeurs (\textit{couper, copier, coller, chercher, \guil{undo}, sauver} ainsi que les touches de raccourcis pour \textit{compiler, chercher l'erreur suivante, etc...}), ainsi que l'explication pour transformer cet utilitaires en un v\'eritable environnement de d\'eveloppement C++~;

\item[-] la partie~\ref{TheoryManual} fournit la base th\'eorique des m\'ethodes d\'evelopp\'ees dans \code{xfem}~; 

\item[-] enfin, la partie~\ref{UserManual} donnera les notions \`a la base des classes \index{Documentation!classes} d'objets sp\'ecifiques \`a \code{xfem} (en court d'\'ecriture)~;

\item[-] une annexes (partie~\ref{annexes}) rassemble des informations compl\'ementaires comme l'administration du serveur CVS, un m\'emento du langage C++ et une description de la mani\'ere dont ce document est d\'evelopp\'e. 
\end{itemize}




\part{Installation et compilation \texttt{xfem}\label{GettingStarted}}
   \markboth{Installation et compilation}{Installation et compilation}
    \chapter{Installation  \\ \textnormal{(contribution de Y. Guetari)}}\label{InstallationXfem}
        


\section{Pre-required}

You first need an account on the SVN server. Ask to your supervisor if you haven't any.

Once provided with a login account on the network and on the SVN
server, you will need a repository to work with (see
section~\ref{CVS-principe}). Files in repository\index{repository}
should be downloaded on your own disk workspace in order to modify them or to develop new application.

At ECN, a disk workspace is shared on most of the servers. This workspace is call \code{/glouton} and the working copy must be done in it. Every user has its
own working copy, and all of them are organised the same way.
Your working copy is in a directory labelled by your group, your name and the sub-directory (\code{develop/}) , which is your development directory :
(\code{/glouton/<groupname>/<usename>/develop/}).

A working copy of all the library you need to modify must be made in
sub-directories of  \code{develop/}:for instance, the library
\code{xfem} in the directory
\code{/glouton/<groupname> /<usename> /develop/Xfem/}. The working copy \code{Xfem/}
contains your local version of the whole common library \code{xfem}.

In general you will do development for your application in a sub-directory located in \\ 
 \code{/glouton/<groupname>/<usename>/develop/Applis/}. This working copy will be maintened in the Applis repository. For other part of the library (mandatory to compile your application) there is no reason to dowload a working copy of all module if you don't intend to modify them. In most case 
you will just create a symbolic links (shortcuts) to the directories of softserver (\code{/glouton/struct/softserver/LATEST/}  which contains the lastest version of the whole librairies.


\subsection{Softserver}

In ECN, a virtual user called \code{softserver}\index{softserver}
has been created to download and to check automatically the libraries. Every day or
every week, tasks (scripts) are launched for maintenance.
\begin{itemize}
\item Updates are made from the SVN server to download the latest version of the libraries managed by the SVN server of ECN
\item updates are made from  external SVN server to download the latest version of associated libraries (MeshAdapt, ...)
\item copies of the external libraries  are made on directory \code{/glouton/struct/softserver/LATEST}  
\item source files are compiled on different platforms 
\item test-cases are run to verify compatibility of the latest version with specified tests
\item ...
\end{itemize}

Thus, all the libraries used at ECN are up-to-date in the 
directory of \code{softserver}. For this reason symbolic links can
be made by every one in  ECN on the sub-directories of \\
\code{/glouton/struct/softserver/LATEST}. Links for common module may point to this
directory (recommended).


\subsection{The all libraires}

Here is the list of all the librairies you may need for your developments. Most of them are just to be accessible with symbolic links. This list is the content of the directory\\ \code{/glouton/struct/softserver/LATEST}.\\


Internal librairies (developed at ECN):
\begin{center}
% use packages: array
\begin{tabular}{p{3.75cm} p{11.75cm}}
\code{Applis/}  &   is the repository of all the user applications developed in ECN.   \\
\code{MeshMachine/}  &  is a specific application developed by Eric Bechet \\
\code{SolverInterfaces/}   &   is the general interface for any solver libraries 
located in \code{/glouton/struct/softserver/LATEST/Solver}. \\
\code{Xfem/}  &   is the basic classes and functions of \code{xfem}.\\
\code{Xext/}  &   is  a local (ECN) extension  of \code{xfem}.\\
\code{Xcrack/}  &   is a local (ECN) extension  of \code{xfem} containing development on crack propagation. \\
\code{Util/}  &   are utilities to   build \code{Makefile}, ...   You probably won't modify this library and a symbolic  link can be made to   \code{/glouton/struct/softserver/LATEST/Util}. for more information, see chapter~\ref{Util}\\
\end{tabular}
\end{center}
External librairies (developed at ECN):
\begin{center}
% use packages: array
\begin{tabular}{p{3.75cm} p{11.75cm}}
\code{Itk/}  &     is a library to manage medical digitally sampled representation (scanned data).\\
\code{MeshAdapt/}  &    provides  capabilities for  doing mesh adaptation.\\
\code{Solver/}   &  contains different solvers for linear system (superLU, PetscSeq, Taucs, Mtl, Itl ...) used by \code{xfem}.  A symbolic link must be made to \code{/glouton/struct/softserver/LATEST/Solver}. \\
\end{tabular}
\end{center}


In ECN, projects are deposit in the repository \code{Applis/}   where
special inherited classes are generated from  basic classes of
\code{xfem}.



\section{Installation \index{installation} \index{SVN!installation of librairies} }\label{first_installation}

The simplest way to install the development toolkit of \code{xfem}
is to follow the steps below:

\begin{enumerate}
\item Create your own subdirectory in the \code{/glouton} tree if not exists:

    \begin{verbatim}
    cd /glouton/<groupname>/<usename>
    mkdir develop
    \end{verbatim}
(\code{develop} is the standard name of the development directory in
ECN.)

\item Modify the \code{.bashrc} file in your home (\code{cd \$HOME}) directory by adding the following lines:
    \begin{verbatim}
    export DISTROOT=/glouton/struct/softserver/LATEST
    export DEVROOT=/glouton/struct/$USER/develop
    if [ -f $DISTROOT/Util/ecnUtil/bashrc.partial ]; then
        . $DISTROOT/Util/ecnUtil/bashrc.partial
    fi
    \end{verbatim}
    Warning : on the location of your own scratch directory (here \code{/glouton} ), the upper path may be different. It could
    be \code{export DEVROOT=/glouton/struct/\$USER/develop}, depending of your group name \code{<groupname>} ans the machine you use.
    It might be standardized at time.\index{\$DEVROOT} \index{\$DISTROOT} \index{export}


\item Execute your \code{.bashrc} file with the command: \code{source .bashrc}. \index{.bashrc}



\item checkout the library you need:
    \begin{itemize}
        \item Go in your develop directory :
        \begin{verbatim}
   cd $DEVROOT
        \end{verbatim} 
	A general common \code{alias} is defined in the file \code{bashrc.partial} so that this latest command can be resume with :
        \begin{verbatim}
   cdd
        \end{verbatim} 
	
        \item if you need to work with   \code{Xfem} library:
        \begin{verbatim}
   svn checkout https://svn.ec-nantes.fr/public/Xfem/trunk  Xfem
        \end{verbatim}         You  need to give your SVN-password once, but after the first connection, the password will not be necessary. \\

	\item for \code{SolverInterfaces/}:
        \begin{verbatim}
   svn checkout https://svn.ec-nantes.fr/public/SolverInterfaces/trunk  
	                                                       SolverInterfaces
        \end{verbatim}          	
	\item for \code{Util/}:
        \begin{verbatim}
   svn checkout https://svn.ec-nantes.fr/public/Util/trunk  Util
        \end{verbatim}        
	\item for \code{Meshmachine/}:
        \begin{verbatim}
   svn checkout  https://svn/ecn-nantes.fr/public/MeshMachine/trunk   
                                                            MeshMachine
        \end{verbatim}     	
	
	\item for \code{Applis/}:
        \begin{verbatim}
   svn checkout https://svn.ec-nantes.fr/ecn/Applis/trunk  Applis
        \end{verbatim}	If you want to checkout only one application (a part of the module \code{Applis}) on your local directory, you have to create you local directory with the command \code{mkdir}, and then to download the repository you need.  For instance, for the application \code{Adapt} :
        \begin{verbatim}
   cd $DEVROOT
   mkdir Applis
   cd Applis
   mkdir Adapt
   cd $DEVROOT
   svn checkout https://svn.ec-nantes.fr/ecn/Applis/trunk/Adapt 
                                                            Applis/Adapt
        \end{verbatim} To define a symblic link to this application, see the point 5 below .\\
	
	
		
        \item for \code{Xext/}:
        \begin{verbatim}
   svn checkout https://svn.ec-nantes.fr/ecn/Xext/trunk  Xext   
       
        \end{verbatim} 

        \item for \code{Xcrack/}:
        \begin{verbatim}
   svn checkout https://svn.ec-nantes.fr/ecn/Xcrack/trunk   Xcrack    
    
        \end{verbatim} 
    \end{itemize}
 
  
    
\item for \code{Trellis} and \code{Solver} or for libraries that you won't modify but whose you just have to use the up-to-date version, make symbolic links (shortcuts) to
the \code{softserver} \index{softserver}  folders: \index{symbolic links} ,\index{ln -s}
        \begin{verbatim}
        cd $DEVROOT
        ln -s $DISTROOT/Solver Solver
        ln -s $DISTROOT/Trellis Trellis
        ln -s $DISTROOT/Itk Itk
        \end{verbatim}
        and for libraries that you haven't checked out, for instance :
        \begin{verbatim}
        ln -s $DISTROOT/Util Util
        ln -s $DISTROOT/Xfem Xfem
        ln -s $DISTROOT/Xext Xext
        ln -s $DISTROOT/Applis Applis
        ln -s $DISTROOT/Xcrack Xcrack
        \end{verbatim}
\end{enumerate}

If you need to get a link for a particular application of \code{Applis}, for instance \code{Applis/Adapt}, you just define a link for this application by :
        \begin{verbatim}
        cd $DEVROOT/Applis
        ln -s $DISTROOT/Applis/Adapt Adapt
        \end{verbatim}




\paragraph*{Parallele development:}

The upper steps avoid to make anything special for parallele
developments. All is automatized as a function of the computer
architecture.
  \markboth{Installation}{Installation}
%    \chapter{CVS \\ \textnormal{(contribution de E. Bechet et Y. Guetari)}}\label{CVS}
%        
\section{Le principe de fonctionnement \`a l'ECN}\label{CVS-principe}

Cette partie d\'etaille les fonctionnalit\'es de l'utilitaire CVS.

\subsection{CVS \emph{(Concurrent Versions System)}}


CVS \index{CVS}  est un syst\`eme de contr�le de versions
client-serveur permettant \`a plusieurs personnes de travailler
simultan\'ement sur un m�me ensemble de fichiers. Les gros projets de
d\'eveloppement  s'appuient g\'en\'eralement sur ce type de syst\`eme afin
de permettre \`a un grand nombre de d\'eveloppeurs de travailler sur un
m�me projet. CVS permet, comme son nom l'indique, de g\'erer les acc\`es
concurrents, c'est-\`a-dire qu'il est capable de d\'etecter les conflits
de version lorsque deux personnes travaillent simultan\'ement sur le
m�me fichier.

Le fonctionnement de CVS s'appuie sur une base centralis\'ee appel\'ee �
repository �\index{repository} (d\'ep�t), h\'eberg\'ee sur un serveur (le
serveur CVS), contenant l'historique de l'ensemble des versions
successives de chaque fichier. Le repository stocke les diff\'erences
entre les versions successives, les dates de mise \`a jour, le nom de
l'auteur de la mise \`a jour et un commentaire \'eventuel, ce qui permet
un r\'eel suivi des modifications, tout en optimisant l'espace de
stockage d\'edi\'e au projet.

Chaque personne travaillant sur le projet poss\`ede un � r\'epertoire de
travail � (en anglais � working copy � ou � sandbox �, traduisez �
bac \`a sable �), c'est-\`a-dire un r\'epertoire contenant une copie de la
base CVS (repository). A l'ECN, il s'agit g\'en\'eralement des
repertoires de travail sur les disques
\code{/scratch/<username>/develop}

\subsection{Checkout \index{checkout}}

A l'aide d'un client CVS, chaque utilisateur souhaitant travailler
sur le projet (pour modifier des fichiers ou simplement pour voir la
derni\`ere version des fichiers dans la base) r\'ecup\`ere une copie de
travail gr�ce \`a une op\'eration appel\'ee � \code{checkout} �.



\subsection{Commit \index{commit}}

Lorsque l'utilisateur a termin\'e de modifier les fichiers, il peut
transmettre les modifications \`a la base. Cette op\'eration est appel\'ee
� \code{commit} �. Ainsi plusieurs d\'eveloppeurs peuvent travailler
simultan\'ement sur une copie du repository et transmettre leurs
modifications.



\subsection{Update \index{update}}

S'il arrive qu'un utilisateur tente de transmettre ses modifications
alors qu'un autre utilisateur a lui-m�me modifi\'e ce fichier
pr\'ec\'edemment, CVS va d\'etecter une incompatibilit\'e. Si les
modifications portent sur des parties diff\'erentes du fichier, le
syst\`eme CVS peut proposer une fusion des modifications, gr�ce \`a une
op\'eration appel\'ee \code{diff}, sinon CVS va demander \`a l'utilisateur
de fusionner manuellement les modifications. Il est \`a noter que les
fusions ne peuvent s'appliquer qu'aux fichiers textes. CVS peut
toutefois g\'erer des fichiers binaires dans sa base, mais il n'a pas
\'et\'e pr\'evu dans ce but. Les modifications apport\'ees par les autres
utilisateurs ne sont pas automatiquement r\'epercut\'ees par CVS sur la
copie locale, il est donc n\'ecessaire, avant chaque modification de
fichier, de mettre \`a jour sa copie de travail gr�ce \`a une op\'eration
appel\'ee � \code{update} �, afin de limiter les risques de conflits.




\section{Using CVS (\emph{in english})}

For more d\'etails and information, see
\web{http://ximbiot.com/cvs/manual/}.

CVS (Concurrent Versions System), is the tool used to manage the
sources and to operate on them without creating conflictual problems
between versions. CVS allows every developer to (commonly used
operations):

\begin{itemize}
\item add, remove and modify files or directories (irreversible)
\item get repercuted on his local versions the pertinent modifications made on the common ones
\item get repercuted on the common versions the pertinent modifications made on the local ones
\item create a history of modifications, allowing access to an old version of source (...)
\end{itemize}

A more exhaustive list of possibilities can be found in CVS web
page. Access to almost all cvs commands is available on the
\code{emacs} interface (\code{menu Tools, tab PCL-CVS}). It is also
possible to do them directly from a terminal.

\subsection{Changing password to ECN's cvs server:}

A  password is given to all users, use this command  to change it:
\begin{verbatim}
ssh <user>@cvs.ec-nantes.fr password
\end{verbatim}



\subsubsection{Acc\`es sans mot de passe}\label{acces_sans_passwd}
Il est parfois fastidieux de rentrer un mot de passe pour tout acc\`es
au serveur CVS. Il existe un moyen de mettre \`a jour le fichier de
clefs permettant une authentification sans mot de passe. Il faut
d'abord cr\'eer la clef sur la machine cliente; ceci se fait par la
commande suivante :

\begin{verbatim}
ssh-keygen -t dsa
\end{verbatim}


Par l'interm\'ediaire de l'utilitaire authorize, il est donc possible
de transmettre cette clef, ceci se fait de la mani\`ere suivante:
\begin{verbatim}
ssh <user>@cvs.ec-nantes.fr authorize < ~/.ssh/id_dsa.pub
\end{verbatim}

Une fois cette action effectu\'ee, plus aucun mot de passe ne sera
demand\'e (en fait, uniquement pour les clients sur lesquels l'action
a \'et\'e faite). Pour les usagers \'etendus, il faudra en plus mettre les
bons droits d'acc\`es du fichier
\code{$\thicksim$/.ssh/authorized\_keys2} :

\begin{verbatim}
ssh <user>@cvs.ec-nantes.fr "chmod 600 ~/.ssh/authorized_keys2"
\end{verbatim}

Pour effacer les clefs d\'ej\`a entr\'ees (et revenir \`a une
authentification par mot de passe pour tous les clients), il suffit
de faire :
\begin{verbatim}
ssh <user>@cvs.ec-nantes.fr authorize delete
\end{verbatim}

Modification de mot de passe La modification de mot de passe se fait
avec la commande password :
\begin{verbatim}
ssh <user>@cvs.ec-nantes.fr password
\end{verbatim}

Attention, le mot de passe entr\'e s'affiche en clair (il est
toutefois transmis sous forme crypt\'ee) Quelques commandes usager
utiles pour initialiser un projet Ces commandes se font cot\'e usager
et permettent de manipuler un projet. Ce sont des commandes standard
que toute installation cvs reconna�t. Pour plus d'information, voir
le site : \web{http://www.cvshome.org/}.






\subsection{Commonly used cvs commands}

\begin{center}
\begin{tabular}{p{6cm} p{9.5cm}}
\code{cvs checkout MODULE\_NAME}\index{CVS!checkout} &  Imports the module into the local account. \\
\code{cvs update -d REPOSITORY\_NAME}\index{CVS!update} &   Checks whether the local version is up-to-date with the CVS-server version \\
\code{cvs add FILE\_NAME} &     Declare a new file to the cvs repository. \\
\code{cvs commit REPOSITORY\_NAME}\index{CVS!commit}  or \code{cvs
commit FILE\_NAME}& Commits the local modifications (on the
repository or the file) into the repository.
\end{tabular}
\end{center}

Details on the recommended use of these commands can be read in
section~\ref{common_commit}.


\subsection{Syntax of the CVS commands \index{CVS!commands}}

Usage:
\begin{verbatim}
          cvs [cvs-options] command [command-options] [files...]
\end{verbatim}
where \code{cvs-options} are:
\begin{center}
\begin{longtable}{p{3.75cm} p{11.75cm}}
     \code{   -H          }  &   Displays Usage information for CVS command \\
     \code{   -Q          }  &   Cause CVS to be really quiet. \\
     \code{   -q          }  &   Cause CVS to be somewhat quiet. \\
     \code{   -r          }  &   Make checked-out files read-only \\
     \code{   -w          }  &   Make checked-out files read-write (default) \\
     \code{   -l          }  &   Turn History logging off \\
     \code{   -n          }  &   Do not execute anything that will change the disk \\
    \code{    -t          }  &   Show trace of program execution -- Try with -n \\
    \code{    -v          }  &   CVS version and copyright \\
    \code{    -b bindir   }  &   Find RCS programs in \code{bindir} \\
    \code{    -e editor   }  &   Use \code{editor} for editing log information \\
    \code{    -d CVS\_root }  &   Overrides \code{\$CVSROOT} as the root of the CVS tree
\end{longtable}
\end{center}
and where \code{command} is:
\begin{center}
\begin{longtable}{p{3.75cm} p{11.75cm}}
        \code{add  }  &        Adds a new file/directory to the repository \\
        \code{admin }    &  Administration front end for rcs \\
        \code{annotate}  &   Show last revision where each line was modified  \\
        \code{checkout }  &    Checkout sources for editing (download from CVS server)\\
    \code{commit }  &   Check changes into the repository. \\
        \code{diff    }  &   Show differences between revisions (or local version and CVS server version)\\
        \code{edit     } &   Get ready to edit a watched file \\
        \code{editors  } &   See who is editing a watched file \\
        \code{export  }  &   Export sources from CVS, similar to checkout \\
        \code{history  } &   Show repository access history \\
        \code{import  }  &   Import sources into CVS, using vendor branches \\
        \code{init    }  &   Create a CVS repository if it doesn't exist \\
        \code{kserver }  &   Kerberos server mode \\
        \code{log    }   &   Print out history information for files \\
        \code{login  }   &   Prompt for password for authenticating server \\
        \code{logout }   &   Removes entry in .cvspass for remote repository \\
        \code{pserver}   &   Password server mode \\
        \code{rannotate} &   Show last revision where each line of module was modified \\
        \code{rdiff  }   &   Create 'patch' format diffs between releases \\
        \code{release }  &   Indicate that a Module is no longer in use (check the difference and delete local repository)\\
        \code{remove }   &   Remove an file from the repository (definitively remove for further revision)\\
        \code{rlog   }   &   Print out history information for a module \\
        \code{rtag  }    &   Add a symbolic tag to a module \\
        \code{server}    &   Server mode \\
        \code{status }   &   Display status information on checked out files \\
        \code{tag    }   &   Add a symbolic tag to checked out version of files \\
        \code{unedit }   &   Undo an edit command \\
        \code{update     }  &  Brings work tree in sync with repository (download from CVS server the last modification by including the differences or let the use choose which lines must be kept)\\
        \code{version}   &   Show current CVS version(s) \\
        \code{watch  }   &   Set watches \\
        \code{watchers}  &    See who is watching a file
\end{longtable}
\end{center}








\subsection{Other sources of information}


\web{http://www.nongnu.org/cvs/}


\web{http://ximbiot.com/cvs/cvshome/}

    \chapter{SVN }\label{CVS}
          
\section{Le principe de fonctionnement �  l'ECN}\label{CVS-principe}

Cette partie d\'etaille les fonctionnalit\'es de l'utilitaire SVN.

Pour l'utilisation de SVN lors de l'installation des librairies xfem, voir le chapitre~\ref{first_installation}

\subsubsection{CVS et SVN}

CVS\index{CVS} et SVN\index{SVN}  sont des  syst\`emes de contr�le de versions
client-serveur permettant \`a� plusieurs personnes de travailler
simultan\'ement sur un m\^eme ensemble de fichiers. Les gros projets de
d\'eveloppement s'appuient g\'en\'eralement sur ce type de syst\`eme afin
de permettre \`a� un grand nombre de d\'eveloppeurs de travailler sur un
m\^eme projet. 

CVS (Concurrent Versions System) permet, comme son nom l'indique, de g\'erer les acc\`es
concurrents, c'est-\`a-dire qu'il est capable de d\'etecter les conflits
de version lorsque deux personnes travaillent simultan\'ement sur le
m\^eme fichier.

Le fonctionnement de CVS  s'appuie sur une base centralis\'ee appel\'ee \guil{
repository}\index{repository} (d\'ep�t), h\'eberg\'ee sur un serveur (le
serveur CVS), contenant l'historique de l'ensemble des versions
successives de chaque fichier. Le repository stocke les diff\'erences
entre les versions successives, les dates de mise \`a jour, le nom de
l'auteur de la mise \`a� jour et un commentaire \'eventuel, ce qui permet
un r\'eel suivi des modifications, tout en optimisant l'espace de
stockage d\'edi\'e au projet.

Chaque personne travaillant sur le projet poss\`ede un r\'epertoire de
travail (en anglais \guil{working copy } ou \guil{ sandbox}, traduisez \guil{
bac \`a� sable }), c'est-\`a-dire un r\'epertoire contenant une copie de la
base CVS (repository). A l'ECN, il s'agit g\'en\'eralement des
r\'epertoires de travail sur les disques
\code{/scratch/<username>/develop}

\subsection{SVN \emph{(Concurrent Versions System)}}

Subversion (en abr\'eg\'e svn) est un syst\`eme de gestion de versions, distribu\'e sous licence Apache et BSD. Il a \'et\'e con�u pour remplacer CVS. Ses auteurs s'appuient volontairement sur les m\^emes concepts (notamment sur le principe du d\'ep�t centralis\'e et unique) et consid\`erent que le mod\`ele de CVS est le bon, et que seule son impl\'ementation est en cause. Le projet a \'et\'e lanc\'e en f\'evrier 2000 par CollabNet, avec l'embauche par Jim Blandy de Karl Fogel, qui travaillait d\'ej\`a sur un nouveau gestionnaire de version.



Subversion a \'et\'e \'ecrit afin de combler certains manques de CVS. Voici les principaux apports :
\begin{itemize}
\item[-] Les commits, ou publications des modifications sont atomiques. Un serveur Subversion utilise de fa�on sous-jacente une base de donn\'ees capable de g\'erer les transactions atomiques (le plus souvent Berkeley DB) ;
\item[-] Subversion permet le renommage et le d\'eplacement de fichiers ou de r\'epertoires sans en perdre l'historique. ;
\item[-] les m\'eta-donn\'ees sont versionn\'ees : on peut attacher des propri\'et\'es, comme les permissions, \`a  un fichier, par exemple.
\end{itemize}

Du point de vue du simple utilisateur, les principaux changements lors du passage \`a  Subversion, sont :
\begin{itemize}
\item[-] Les num\'eros de r\'evision sont d\'esormais globaux (pour l'ensemble du d\'ep�t) et non plus par fichier : chaque patch a un num\'ero de r\'evision unique, quels que soient les fichiers touch\'es. Il devient simple de se souvenir d'une version particuli\`ere d'un projet, en ne retenant qu'un seul num\'ero ;
\item[-] svn rename (ou svn move) permet de renommer (ou d\'eplacer) un fichier ;
\item[-] Les r\'epertoires et m\'eta-donn\'ees sont versionn\'es.
\end{itemize}
 
Une des particularit\'es de Subversion est qu'il ne fait aucune distinction entre un label, une branche et un r\'epertoire. C'est une simple convention de nommage pour ses utilisateurs. Il devient ainsi tr\`es facile de comparer un label et une branche ou autre croisement.


\section{Les principales commandes de Subversion}


\subsection{Checkout \index{SVN!checkout}}\label{svn_checkout}

A l'aide d'un client SVN, chaque utilisateur souhaitant travailler
sur le projet (pour modifier des fichiers ou simplement pour voir la
derni\`ere version des fichiers dans la base) r\'ecup\`ere une copie de
travail gr�ce \`a  une op\'eration appel\'ee \guil{\code{checkout}} .

Pour la premi\`ere installation des librairies xfem \`a l'ECN, ce r\'ef\'erer au 
chapitre~\ref{first_installation}

\subsection{Commit \index{SVN!commit}}

Lorsque l'utilisateur a termin\'e de modifier les fichiers, il peut
transmettre les modifications \`a  la base. Cette op\'eration est appel\'ee
\guil{\code{commit}}. Ainsi plusieurs d\'eveloppeurs peuvent travailler
simultan\'ement sur une copie du repository et transmettre leurs
modifications.



\subsection{Update \index{SVN!update} }

Les modifications apport\'ees par les autres
utilisateurs ne sont pas automatiquement r\'epercut\'ees par SVN sur la
copie locale, il est donc n\'ecessaire, avant chaque modification de
fichier, de mettre \`a  jour sa copie de travail gr �ce \`a  une op\'eration
appel\'ee  \guil{ \code{update}  }, afin de limiter les risques de conflits.


S'il arrive qu'un utilisateur tente de transmettre ses modifications
alors qu'un autre utilisateur a lui-m\^eme modifi\'e ce fichier
pr\'ec\'edemment, SVN va d\'etecter une incompatibilit\'e. Si les
modifications portent sur des parties diff\'erentes du fichier, le
syst\`eme SVN peut proposer une fusion des modifications, gr�ce \`a  une
op\'eration appel\'ee \code{diff}, sinon SVN va demander \`a  l'utilisateur
de fusionner manuellement les modifications en signalant un conflit.

Il conviendra de signaler que le conflit est r\'esolu avant de faire un nouveau \guil{ \code{commit} }, par la command : 

\code{svn resolved} \code{<nom\_du\_fichier>} 

Si la commande signale que le r\'epertoire est \guil{omis}, c'est que le d\'epos local ne correspond pas \`a un \guil{checkout} du serveur SVN (voir la section~\ref{svn_checkout}).

\newpage

\subsubsection{Tableaux des commandes usuels}

\begin{center}
\begin{longtable}{p{3.75cm} p{11.75cm}}
\textbf{Commande} & \textbf{Signification} \\
\hline\\
\code{  add  }  &  D\'eclare l'ajout d'une nouvelle ressource pour le prochain commit. Lorsqu'un fichier est ajout\'e au d\'epos\code{NewApps} :

\begin{verbatim}
        cd NewApps
        svn add MyNewFile
        svn commit MyNewFile
\end{verbatim} 
 \\
\code{  blame  }  & Permet de savoir quel contributeur a soumis les lignes d'un fichier.\\
\code{  checkout (co)  }  & R\'ecup\`ere en local une r\'evision ainsi que ses m\'eta-donn\'ees depuis le d\'ep�t.\\
\code{  cleanup  }  & Nettoie la copie locale pour la remettre dans un \'etat stable.\\
\code{  commit (ci)  }  & Enregistre les modifications locales dans le d\'ep�t cr\'eant ainsi une nouvelle r\'evision.\\
\code{  copy  }  & Copie des ressources \`a  un autre emplacement (localement ou dans le d\'ep�t).\\
\code{  delete }  & D\'eclare la suppression d'une ressource existante pour le prochain commit.\\
\code{  diff  }  & Calcule la diff\'erence entre deux r\'evisions (permet de cr\'eer un patch \`a  appliquer sur une copie locale).\\
\code{  export  }  & R\'ecup\`ere une version sans m\'eta-donn\'ees depuis le d\'ep�t ou la copie locale.\\
\code{  import  }  & Envoie une arborescence locale vers le d\'ep�t.\\
\code{  info  }  & Donne les informations sur l'origine de la copie locale.\\
\code{  lock  }  & Verrouille un fichier.\\
\code{  log  }  & Donne les messages de commit d'une ressource.\\
\code{mkdir} &  Lors de la cr\'eation de nouvelle apllication dans le d\'epos Applis (\`a l'ECN), il convient de cr\'eer un r\'epertoire nouveau dans le d\'epos des applications. Ceci ce fait avec la commande \code{mkdir} :

\begin{verbatim}
        svn mkdir NewApps
        svn commit NewApps
\end{verbatim}    
\\
\code{  merge  }  & Calcule la diff\'erence entre deux versions et applique cette diff\'erence \`a  la copie locale.\\
\code{  move  }  & D\'eclare le d\'eplacement d'une ressource.\\
\code{  propdel  }  & Enl\`eve la propri\'et\'e du fichier.\\
\code{  propedit  }  & \'edit la valeur d 'une propri\'et\'e.\\
\code{  propget  }  & Retourne la valeur d'une propri\'et\'e.\\
\code{  proplist  }  & Donne une liste des propri\'et\'es.\\
\code{  propset  }  & Ajoute une propri\'et\'e.\\
\code{  resolved  }  & Permet de d\'eclarer un conflit de modifications comme r\'esolu.\\
\code{  revert }  & Revient \`a  une r\'evision donn\'ee d'une ressource. Les modifications locales sont \'ecras\'ees.\\
\code{  status (st)  }  & Indique les changements qui ont \'et\'e effectu\'es:
\begin{itemize}
\item A : fichier Ajout\'e
\item D :  fichier D\'etruit
\item M :   fichier Modifi\'e
\end{itemize}
\\
\code{  switch  }  & Met \`a  jour la copie du d\'ep�t.\\
\code{  update (up)  }  & Met \`a  jour la copie locale existante depuis la derni\`ere r\'evision disponible sur le d\'ep�t.\\
\code{  unlock  }  & Retire un verrou.
\end{longtable}
\end{center}




For more d\'etails and information, see \web{http://svnbook.red-bean.com/}.




\section{SVN en pratique}


Le serveur subversion est visible \`a l'adresse suivante : \web{https://svn.ec-nantes.fr}. Il est accessible en lecture \`a partir d'un nagigateur web (ie, firefox, ...)

Pour y acc\'eder, il faut accepter le certificat du serveur. Si vous acceptez ce certificat d\'efinitivement, la question ne sera plus pos\'ee.

\section{english documentation}

For an english documentation, refere to pdf book  \web{http://svnbook.red-bean.com/nightly/en/svn-book.pdf}.

    \chapter{Compilation }\label{Compilation}
        \section{The library  \code{Util}\index{Library!Util}}\label{Util}

\textnormal{(contribution de Y. Guetari)} 

The library \code{Util/} contains:
\begin{itemize}
\item The library \code{buildUtil/} contains utilities for easier library and executable building.
\item specific utility scripts for ECN (\code{ecnUtil/}\index{Library!ecnUtil}).
\end{itemize}




\subsection{Use of \code{buildUtil} \index{Library!buildUtil}: Basic
operations}

The library \code{Util} provides utilities for developpers. These
utilities are accessible with the \code{make} command. This command
\code{make}  compile sources files and links them  to create a executable. In order to help developpers, new commands have been
added:\index{Compilation!standard}

\begin{center}
% use packages: array
\begin{longtable}{p{7.75cm}  p{7.75cm}}
\code{make distclean NODEP=1}   & To fully clean a distribution. Compulsory if a \code{.h} file is added or removed \\
& \\
\code{make setup NODEP=1}   & To create links before compiling \\
& \\
\code{make}             & To actually compile and create the library \\
& \\
\code{make VERS=opt}        & Same as above but with optimization flags \\
& \\
\code{make check}       & To check all test cases. See results in \code{verif\_result.txt} \\
& \\
\code{make check VERS=opt}      & To check all test cases with the optimized code version \\
& \\
\code{make checkO}      & equivalent as \code{make check VERS=opt}    \\
& \\
\code{make checkone DIR=<directory>}   &    To check only the <directory> problem, depending on its location. See results in \code{verif\_result.txt} \\
& \\
\code{make checkone VERS=opt DIR=<directory>}   &    same as above in the optimized version\\
& \\
\code{make checkone\_update DIR=[directory]}   &    To check only the [directory] problem,  and update the reference file with the result file  \\
& \\
\code{make checkone DIR=test/toto VERS=opt}     &   Same as above but with the optimized code version \\
& \\
\code{make doc}     &   create the Doxygen documentation of the module (library).\\
& \\
\code{make latex}     &   compile this LaTeX documentation (in \code{\$DEVROOT/Xfem/Xfem/Xfem/src\_doc}).
\end{longtable}
\end{center}


You must add the option \code{METIS=1} for compilation
\index{Compilation!cluster} on the cluster \code{Pegase}.


\subsection{Utility scripts of \code{ecnUtil}}\label{ecnUtil} \index{Library!\code{ecnUtil}}


Specific tools are developed at ECN to automate some tasks related
to directories management. They are located in
\code{/scratch/softserver/LATEST/Util/ecnUtil/} and can be called
like the following example:
\begin{verbatim}
cd /scratch/<username>
/scratch/softserver/LATEST/Util/ecnUtil/<script>
\end{verbatim}

where \code{<script>} is one of the scripts discribed in the
following table:

\begin{center}
% use packages: array
\begin{longtable}{p{4cm} p{5.2cm} p{5.4cm}}
\textbf{Script name} & \textbf{Descriptio}n &   \textbf{Use} \\
& & \\
 \code{create-devel}  & Creates an executable test-case under
the subdirectory \code{devel/} of the current module. The new
test-case will contain an empty skeleton of a test-case with:
\begin{itemize} \item an empty directory \code{data/}
\item two minimal files \code{main.cc} and \code{main.h}
\item an empty directory \code{reference/}
\item the link directory \code{SVN/} (not to be touched)
\end{itemize} & \begin{enumerate} \item Go to your subdirectory \code{devel/} : \code{cd devel/}
\item  Type out: \code{create-devel [name]}
\item The system will ask you for your password for every new item to be created  \end{enumerate} \\
& & \\
\code{create-module} & Creates a new module (\code{appli}) in the
existant repository (e.g. \code{My\_First\_Module/} in the
repository \code{Applis/}) The new module will contain the minimal
components of the new appli  &
\begin{enumerate}
\item Go to the subdirectory \code{develop/}
\item Type out: \code{create-module Applis [name]}
\item  The system will ask you for your password for every new item to be created
\end{enumerate} \\
& & \\
\code{move-devel-to-test} & Moves a test-case directory from \code{devel/} to \code{test/} so that it is automatically executed when a \code{make check} is made & TO BE TESTED\\
& & \\
\code{update-all-reference} & Replaces each files in \code{reference} by the corresponding file in \code{results}.  &  TO BE TESTED \\
& & \\
\code{update-reference} & Same as above but only for the current
directory \code{reference/} &  TO BE TESTED
\end{longtable}
\end{center}


To easily use the development environment, you need to call options
defined in bashrc.partial. 
A canevas of the configuration file
\code{.bashrc} is available in the home directory. There's no need to get the
whole file bashrc.partial, but the path to the development directory
must (at-least!) be specified in it (\code{export
DEVROOT=<development directory>}, for example: \code{export
DEVROOT=/scratch/\$USER/develop}). Otherwise, compilation wouldn't
be possible in this directory (see the whole instructions in
installation).





\section{Compiling and executing a test-case}




To compile \index{Compilation!standard} and execute a program
located in a test-case directory (for instance
\code{Xfem/Xfem/Xfem/test/} for the \code{xfem} library). 
Go to the
directory containing the \code{Makefile} file and   perform the
following sequence of instructions:

On standard server (\code{mosix1, mosix2, ...}):
\begin{itemize}
\item First compile the library (upper level: the module):
    \begin{itemize}
    \item[1.] \code{make distclean NODEP=1}
    \item[2.] \code{make setup NODEP=1}
    \item[3.] \code{make}
    \end{itemize}
\item Then compile your test-case:
    \begin{itemize}
    \item[4.] \code{make checkone DIR=test/<test-name>}
\end{itemize}
\end{itemize}
(Here you did use the \code{Util}  library (see~\ref{Util}) by using
specific keywords with the \code{make} command.)\\


On a HPC cluster  \index{Compilation!cluster}
(\code{pegase}(\code{master0})):
\begin{itemize}
\item First compile the library (upper level: the module):
    \begin{itemize}
    \item[1.] \code{make distclean NODEP=1}
    \item[2.] \code{make setup  METIS=1 NODEP=1}
    \item[3.] \code{make METIS=1 }
    \end{itemize}
\item Then compile your test-case:
    \begin{itemize}
    \item[4.] \code{make checkone  METIS=1 DIR=test/<test-name>}
\end{itemize}
\end{itemize}
You can also use the option \code{make -j$n$} (with no space) to
execute the \code{make} command on $n$  processors. This make a
parallele compilation on several nodes of the cluster
(\code{profile} file needed) or a much faster compilation even on other servers (mosix1, mosix2, ...).





\subsection{Application example} \index{installation!test}

A typical example of development is presented here. It describes a
test-case existing in \code{Xfem/Xfem/Xfem/test/} \index{Xfem!test}
. In this example, a boundary condition (L2 projection) is imposed
on a border of a 2D domain. Watch here the files organisation. It is
the typical structure of the directories you may create  in your
\code{devel/} directory.\\

\begin{description}
\item[] mechanics\_bc\_l2
   \begin{description}
	\item[$\llcorner$]main.cc  : C++ file containing the main source code
	\item[$\llcorner$]main.h : file containing the declaration of the C++ classe
	\item[$\llcorner$]main : executable file of the application
	\item[$\llcorner$]data
		\begin{description}
		\item[$\llcorner$] law.mat : material parameters
		\item[$\llcorner$] main.dat : data describing the problem to treat
		\item[$\llcorner$] square\_5.msh : the mesh file (gmsh format)
		\item[$\llcorner$] .svn : automatically created by SVN (don't modify or copy it)
		\end{description}  
	\item[$\llcorner$]reference:
		\begin{description}
		\item[$\llcorner$] DISPLACEMENT.pos : result filechosen as a reference
		\item[$\llcorner$] .svn
		\end{description}  
	\item[$\llcorner$]results:
		\begin{description}
		\item[$\llcorner$] DISPLACEMENT.pos : result file of the local version of the application
		\item[$\llcorner$] dcl.dbg : debugging file created by main.cc
		\item[$\llcorner$] ess.dbg : same as above 
		\item[$\llcorner$] rcond.txt  : same as above 
		\item[$\llcorner$] sym.dbg : same as above 
		\item[$\llcorner$] res.dbg : same as above 
		\end{description}  
   \end{description}
\end{description}







\subsection{Create or modify an application}

The best way (recommended) is to create a new reposity in the SVN
module \code{Applis/}. Make a copy of an existing application which
will be the starting point of your developments, or modify an
existing application you want to improve.



\begin{enumerate}
\item Start by creating your test subdirectory. Choose your location (directory \code{devel/} or 		\code{test/}) and use
	the script \code{create-devel} to start with (details in
	section~\ref{ecnUtil}).
\item Open \code{emacs} (see~\ref{sec:emacs}) and create/modify your \code{.geo} file, save it in 		\code{data/}.
\item Now open a terminal and go to your \code{data/} subdirectory. Launch \code{gmsh}   (by command: 		\code{gmsh \&}, see~\ref{gmsh})
\item In \code{gmsh}, open your \code{.geo} file and create the corresponding mesh file. Choose \guil{mesh} 	mode and generate a \guil{2D} mesh. Save the resulting file. You won't have to specify a name or 	saving location for it. It is automatically saved in the subdirectory from which you opened your 	\code{.geo} and will have the same name.
\item Go back to the test directory and create \code{main.cc}. Make all required includes. Save.
\item Run this program (first compile dependent libraries), for that use the following command :
\begin{verbatim}
      make -C <path_of_the_Makefile_is> setup
      make -C <path_of_the_Makefile_is> 
\end{verbatim}
For a better visibility of the compilation error, use emacs and execute the command in the dialog box of emacs (or command \touche{F8}, make sure that the command above is writen and \touche{return} : see~\ref{DevelopEnvironTool} for more details (depending on the  file\code{.emacs})). 
\item  A subdirectory \code{results/} is generated in the  test-directory. Open it to see the results you asked for.
\item Choose among these results files the ones you would like to use as reference, copy them into subdirectory \code{reference/}.
\end{enumerate}




\subsection{Committing a modification of a library  to the Community}\label{common_commit}

As many users develop individually at the same time, one has to get
sure his modifications are functional \guil{harmless} before
committing them into the basic source.


A special sequence of operations is needed to be performed before
making SVN commit\index{CVS!commit}. This sequence ensures that the
manipulation will not be unsafe for the common version.

\begin{enumerate}
\item  Update the local version with svn-server's one, this will merge the local modifications with the external committed ones:
\begin{verbatim}
   svn update  <library>
\end{verbatim}
This operation \guil{patches} the modifications of the SVN version
to your own version. These patches are possible where you didn't
modify at time the same part of the source files than the ones you
patch. Else, a conflict is detected and the patches is added with
\code{+} and \code{-} character to indicate the different
modifications, yours against the SVN one. 
You are unlucky ! You have to resolve the conflict before committing and indicat to SVN that the problem is resolved by:
\begin{verbatim}
   svn resolved  <file>
\end{verbatim}
If the \code{update} was successful :
\item  Compile and run test-cases: (description)\index{run}
\begin{verbatim}
   make distclean NODEP=1
   make setup NODEP =1 
   make
\end{verbatim}

If no problem is encountered doing this, do:

\begin{verbatim}
   make check
\end{verbatim}

The last command compiles and executes all test-cases existing in
the subdirectory \code{test/}.

\item  \code{verif\_result.txt} is a file that records a comparison between the results files this execution has
generated and the reference \index{test!reference}
results files stocked in the directory \code{reference/}.\index{test!results} \index{file!reference}

Check (carefully!) the result of this execution in
\code{verif\_result.txt}. If this comparison is conclusive, you can
commit to the svn-server's files:\index{file!result}

\item  Commit the common files:
\begin{verbatim}
   cdd
   svn commit <library>
\end{verbatim}
where \code{<library>} is \code{Xfem}, \code{Xext}, \code{Xcrack} or \code{Applis}.
\end{enumerate}

The repository gets then the committed files. They will be compiled and
executed automatically by night by softserver.


	
	
\part{Les utilitaires et librairies de d\'eveloppement}\label{tools_and_lib}
    \markboth{Utilitaires et librairies}{Utilitaires et librairies}
    \chapter{Emacs et Gmsh  }\label{utilities}
        

\section{Emacs   \emph{(en fran\,cais)}  \\ \textnormal{(contribution de Christophe Pallier)} }\label{sec:emacs}


 \emph{ (Extrait de  http://www.pallier.org/ressources/intro\_emacs/intro\_emacs.html de Christophe Pallier) }

\code{emacs}\index{Emacs} est sans doute le  programme le plus
puissant sous linux pour les d\'eveloppeurs, mais aussi le plus \'esot\'eriques. Le but de cette partie est
de faciliter la prise de contact avec Emacs. Il s'adresse donc \`a
ceux qui voudraient bien l'essayer mais se sentent un peu
\guil{perdus} face \'e l'aust\'erit\'e de son \'ecran d'accueil.



\subsection{Pourquoi utiliser Emacs ?}


Par rapport \'e des \'editeurs plus modernes (\code{nedit}, \code{gedit}, ...), Emacs, est d'un abord
r\'ebarbatif. L'un des principaux int\'er\'ets d'Emacs r\'eside dans les
modes qui sp\'ecialisent son comportement en fonction du type de
fichiers sur lequel on travaille. Cela se r\'ev\'ele tr\'es appr\'eciable
lorsque on r\'edige un programme en C ou en Perl.


Un autre aspect int\'eressant d'Emacs est qu'il peut servir
d'environnement au sein duquel on commande certains programmes
interactifs. Cela permet d'utiliser divers programmes \`a l'int\'erieur
d'un environnement unique. On peut \'egalement   \guil{explorer} son
syst\'eme de fichiers avec le mode \guil{dired}.

Le principal \'ecueil du d\'ebutant, sans nul doute, ce sont les
s\'equences de touches \'e m\'emoriser. En r\'ealit\'e, pourtant, seuls quelques
 s\'equences de touches  suffisent pour r\'ealiser 99\% des t\'eches. Apr\'es quelques heures d'utilisation elles deviennent
automatiques.

\subsection{Cr\'eer, ouvrir et fermer des fichiers}

Sur la ligne de commande d'un terminal, tapez \code{emacs \& }.

Quand emacs appara\'et, appuyez sur la touche
\touche{Ctrl}\index{Emacs!CRTL} et, tout en la maintenant enfonc\'ee,
appuyez sur \code{x} puis \code{f} (combinaison de touches qu'on
notera \code{C-x C-f} ; notez qu'il n'est pas n\'ecessaire de rel\'echer
la touche \touche{Ctrl}). Le message \code{Find file:} s'affiche sur
la derni\'ere ligne (appel\'ee \code{minibuffer} selon la terminologie
emacs). Entrez \code{test1} et appuyez sur \touche{ENTER},
\code{test1} est le nom du fichier cr\'e\'e. Entrez quelques lignes en
frappant normalement au clavier (par exemple \code{Ceci est un
essai} puis  \touche{ENTER} d'Emacs). Puis appuyer sur \code{C-x
C-s} pour sauver votre fichier, puis \code{C-x C-c} pour sortir
d'Emacs. Le tableau suivant d\'ecrit pas \'e pas la suite d'actions que
vous venez d'effectuer.

\begin{center}
\begin{longtable}{l l}
\code{C-x}  \code{C-f} test1 \touche{ENTER}  &  ouvre un nouveau fichier au nom de test1 \\
\code{Ceci est un essai} \touche{ENTER}   &     \'ecrit sur la premi\'ere ligne\\
\code{d'Emacs}   &      \'ecrit sur la deuxi\'eme ligne\\
 \code{C-x}  \code{C-s}   &     sauve le fichier\\
 \code{C-x}  \code{C-c}   &     quitte emacs
\end{longtable}
\end{center}


Vous pouvez v\'erifier le contenu du fichier que vous venez de cr\'eer
en entrant la commande \code{cat test1} sur une ligne de commande du
shell . Vous devriez alors voir s'afficher les deux lignes
\code{Ceci est un essai} et \code{d'Emacs}.

Relancez Emacs et ouvrez votre fichier avec \code{C-x C-f test1}
\touche{ENTER}. Fermez-le imm\'ediatement en tapant  \code{C-x k}
\touche{ENTER}  (rel\'echer la touche \touche{Ctrl} avant de frapper
\touche{K}). Ouvrez-le \'e nouveau, faites quelques modifications et
tapez  \code{C-x k}  \touche{ENTER} . Emacs vous demande
confirmation avant de d\'etruire (\code{kill}) le buffer \code{test1}.
Si vous r\'epondez \code{yes}, vos modifications seront perdues ; si
vous r\'epondez \code{no}, l'op\'eration est abandonn\'ee et vous aurez
l'opportunit\'e de sauver votre fichier avec  \code{C-x}  \code{C-s}.
Sachez \'egalement que si vous appuyez sur  \code{C-x}  \code{C-c}
trop rapidement et qu'emacs vous demande si vous voulez ou non
sauver des fichiers, vous pouvez renoncer \'e sortir en tapant
\code{C-g}. Plus g\'en\'eralement  \code{C-g} permet d'abandonner
l'action courante. C'est une combinaison de touches tr\'es utile !
Effectivement, si Emacs ne r\'epond plus, ou si vous ne comprenez plus
o\'e vous \'etes, tapez sur  \code{C-g}, \'eventuellement \'e plusieurs
reprises ; il y aura alors toutes les chances que vous vous
retrouviez dans une situation mieux ma\'etris\'ee.

Il y a d'autres fa\'eons d'ouvrir le fichier \code{test1}. Tout
d'abord vous auriez pu \'ecrire \code{emacs test1} sur la ligne de
commande. Encore une autre fa\'eon, de l'int\'erieur d'Emacs, est
d'utiliser  \code{C-x}  \code{C-f} puis la touche \touche{TAB} pour
obtenir la compl\'etion automatique de nom de fichier: Essayez
\code{C-x C-f te} \touche{TAB}\index{Emacs!TAB} ; si \code{test} est
le seul nom de fichier qui commence par \code{te}, il s'affichera
directement. Si par contre il y a d'autres fichiers dont le nom
commence par \code{te}, alors leur liste s'affichera dans une
fen\'etre. Vous pouvez d\'eplacer le curseur dans cette fen\'etre avec la
s\'equence  \code{C-x  o} (tapez cela plusieurs fois, et observez
comment le curseur se d\'eplace de fen\'etre en fen\'etre); utilisez les
fl\'eches pour d\'eplacer le curseur sur le nom du fichier qui vous
int\'eresse et tapez \touche{ENTER}.

Emacs permet bien s\'er d'ouvrir plusieurs fichiers simultan\'ement ;
pour cela, utilisez  \code{C-x C-f} plusieurs fois. Chaque fichier
s'ouvre dans un nouveau buffer. Pour passer d'un buffer \'e l'autre,
on peut utiliser  \code{C-x b} et taper le nom du fichier
destination (sans oublier que la compl\'etion du nom par \touche{TAB}
fonctionne). Une autre mani\'ere consiste \'e afficher la liste des
buffers avec  \code{C-x C-b}; on peut alors se d\'eplacer vers cette
liste avec un  \code{C-x o} suivi de d\'eplacements avec les fl\'eches.
Finalement,  \code{C-x 1} supprime les fen\'etres suppl\'ementaires et
permet de retrouver le buffer courant dans une seule fen\'etre dans
l'\'ecran d'Emacs. Fa\'etes l'exercice suivant:
\begin{center}
\begin{tabular}{p{7cm} p{7cm}}
 \code{C-x}  \code{C-f test1} \touche{ENTER}  & ouvre le fichier \code{test1} \Large{ }\\
 \code{C-h t}  & ouvre le tutorial d'emacs dans un nouveau buffer \\
 \code{C-x b} \touche{ENTER}  &  retourne a \code{test1} \\
 \code{C-x C-b C-x o C-n} \touche{ENTER}  \code{C-x o C-x 1}  & se promener
\end{tabular}
\end{center}

Finalement, voici un r\'ecapitulatif de commandes utiles pour naviguer
d'un fichier ou d'une fen\'etre \'e l'autre. Dans la derni\'ere colonne,
nous avons indiqu\'e le nom de la fonction Emacs associ\'ee \'e la
s\'equence de touche. Cela est utile si vous d\'esirez changer les
fonctions des touches. (En fait, Emacs poss\'ede de nombreuses
fonctions dont seule une minorit\'e sont associ\'ees \'e des s\'equences de
touches. On peut cependant ex\'ecuter n'importe quelle fonction en
tapant \touche{Esc}\code{-x} suivit du nom de la fonction).

\begin{center}
\begin{longtable}{p{2.5cm} p{7.5cm} p{5cm} }
 \code{C-x  C-c}     &  quitte emacs             & \code{save-buffers-kill-emacs} \\
 \code{C-g}          &  abandonne l'action courante      & \code{keyboard-quit} \\
 \code{C-x   C-f}    &  ouvre ou cr\'ee un fichier     & \code{find-file} \\
 \code{C-x   C-s}    &  sauve le buffer courant      & \code{save-buffer} \\
 \code{C-x   C-w}    &  sauve dans un nouveau fichier    &  \code{write-file} \\
 \code{C-x k}    &  ferme le buffer courant          & \code{kill-buffer } \\
 \code{C-x b}    & change de buffer      & \code{switch-to-buffer} \\
 \code{C-x  C-b}  &     affiche la liste des buffers     &  \code{list-buffers} \\
 \code{C-x  1}   &  ferme toutes les fen\'etres sauf la courante   & \code{delete-other-windows} \\
 \code{C-x  o}   &  d\'eplace le curseur vers une autre fen\'etre    & \code{other-window} \\
 \code{C-x  5 f}  &     ouvre un fichier dans une autre fen\'etre X    & \code{find-file-other-frame} \\
 \code{C-x  5 0}  &     d\'etruit la fen\'etre X (frame)             & \code{delete-frame} \\
\end{longtable}
\end{center}

\subsection{Se d\'eplacer et \'editer un fichier}

Le tableau suivant liste les touches de d\'eplacement les plus
utilis\'ees. Sous la plupart des syst\'emes Linux, la touche
\touche{Echap} fait office de touche \code{META}\index{Emacs!META}
(dans le langage de la documentation d'Emacs) donc \code{M-$>$}
signifie appuyer sur \touche{Echap} et \touche{$>$} simultan\'ement.
Notez que ces combinaisons de touches fonctionnent de la m\'eme fa\'eon
sur les lignes de commande des shells \code{bash} et \code{tcsh},
ainsi que dans de nombreux programmes interactifs tels que
\code{gnuplot}, \code{octave}... Ainsi votre apprentissage sera vite
rentabilis\'e.

 \begin{center}
\begin{longtable}{l l l }
 \code{C-v} ou \code{ M-v }  &  avancer vers le bas ou vers le haut  &  \code{scroll-up}, \code{scroll-down} \\
 \code{M-$>$ }  &   aller en fin de fichier      &  \code{end-of-buffer}\\
 \code{M-$<$ }  &   aller en d\'ebut de fichier    &  \code{beginning-of-buffer} \\
 \code{C-a}      &  aller en d\'ebut ligne     &  \code{beginning-of-line} \\
 \code{C-e }     &  aller en fin de ligne    &  \code{end-of-line} \\
 \code{C-p }     &  line pr\'ec\'edente      &  \code{previous-line} \\
 \code{C-n }     &  ligne suivante       &  \code{next-line} \\
 \code{C-d }     &  effacer le caract\'ere sous le curseur  &     \code{delete-char} \\
 \code{C-k }     &  effacer jusqu'\'e la fin de la ligne  &   \code{kill-line} \\
 \code{C-x u}    &  annuler l'action pr\'ec\'edente      &  \code{undo}
\end{longtable}
\end{center}

\subsection{Couper/Copier/Coller}

Ouvrez un fichier ou bien cr\'eez-le avec Emacs. D\'eplacez le curseur
au d\'ebut d'un passage que vous souhaitez couper ou coller. Appuyez
sur  \code{C-}\touche{SPC}: cela pose une \guil{marque} au d\'ebut du
passage. D\'eplacez vous jusqu'\'e la fin du passage, puis tapez
\code{C-w} pour couper le texte (ou \code{M-w} pour le copier).
Allez l\'e o\'e vous voulez le coller, puis tapez  \code{C-y}
(\code{yank}). Vous pouvez coller le texte (avec  \code{C-y}) \'e
plusieurs reprises.

Une propri\'et\'e int\'eressante du \guil{presse-papier} d'Emacs est qu'il
se souvient non seulement du dernier passage copi\'e (ou coup\'e), mais
\'egalement des pr\'ec\'edents: si vous avez coup\'e plusieurs passages,
presser  \code{C-y} affiche le dernier, mais si vous appuyez ensuite
sur \code{M-y}, l'avant-dernier est ins\'er\'e. On peux appuyer
plusieurs fois sur \code{M-y} pour remonter plus en arri\'ere dans
l'historique des passages coup\'es. Cela peut s'av\'erer tr\'es utile si
l'on a effac\'e par inadvertance un long passage, et beaucoup
travaill\'e depuis de fa\'eon que la commande undo ( \code{C-x u}) n'est
plus efficace.

On a souvent besoin de couper le texte jusqu'\'e la fin de la ligne
courante; pour cela on peut utiliser  \code{C-k} ; Plusieurs
\code{C-k} successifs effacent autant de lignes.

Vous remarquerez que, contrairement \'e d'autres \'editeurs de texte, le
texte plac\'e entre la marque et le curseur (i.e. le texte
\guil{s\'electionn\'e}) n'appara\'et pas en video-inverse (en fait cela
est vrai pour un Emacs d'origine, votre configuration peut varier).
Cela peut para\'etre g\'enant au premier abord, mais il y a une bonne
raison \'e cela (on peut poser plusieurs marques, accessible par
\code{C-h i m emacs} \touche{ENTER} \code{m mark} \touche{ENTER}).
Pour savoir o\'e vous avez pos\'e la derni\'ere marque (i.e. le d\'ebut de
la r\'egion \guil{s\'electionn\'ee}), vous pouvez utiliser l'une des deux
commandes:
\begin{enumerate}
\item  \code{C-x C-x}, qui \'echange la position de la marque et du curseur;
\item  \code{C-u C-}\touche{SPACE}, qui am\'ene le curseur \'e la position de la marque.
\end{enumerate}


Si vous tenez absolument \'e voir le texte entre la marque et le
curseur en vid\'eo inverse, ex\'ecutez la commande
\code{transient-mark-mode} (tapez \code{M-x}, puis \code{transi},
puis \touche{TAB}, vous devez voir s'afficher
\code{transient-mark-mode}); Appuyez alors sur la touche
\touche{ENTER}. Dans ce mode, le texte s\'electionn\'e change de
couleur. Quand vous voudrez sortir du mode \code{selection},
utilisez  \code{C-g}.

Quelques remarques \'e propos des marques sont int\'eressantes.
Premi\'erement, on peut poser plusieurs marques dans un texte, et
revenir successivement de l'une \'e l'autre par  \code{C-u}
\code{C-\touche{SPC}}. Deuxi\'emement, certaines commandes posent
implicitement des marques. Par exemple allez au milieu d'un fichier
texte, puis tapez \code{M-$>$} pour aller en fin de fichier. Si vous
tapez ensuite  \code{C-u C-\touche{SPC}}, vous revenez
automatiquement au point de d\'epart. Une marque est \'egalement pos\'ee
implicitement quand vous effectuez une recherche de cha\'ene de
caract\'eres.

Il se peut que votre syst\'eme permette de copier/coller avec la
souris. Par exemple, sous X-Window: cliquer \'e gauche et glissez pour
s\'electionner du texte (si vous avez rel\'ech\'e le bouton avant d'avoir
atteint la fin du texte \'e s\'electionner, allez en cette position,
puis cliquez sur le bouton de droite pour \'etendre la s\'election);
pour coller, cliquez sur le bouton du milieu. Ce m\'ecanisme est
\'egalement utile pour copier du texte entre Emacs et une autre
application X, par exemple une fen\'etre terminal.

\subsection{Chercher/Remplacer}

Pour chercher un mot, tapez  \code{C-s} puis les premi\'eres lettres
de ce mot. Observez bien le comportement d'Emacs : \'e chaque fois que
vous tapez une lettre, il avance dans le texte de fa\'eon
incr\'ementale. Utilisez  \code{C-g} pour interrompre la recherche.
Une fois le mot localis\'e, si vous d\'esirez trouver les occurrences
suivantes, tapez \'e nouveau  \code{C-s}. Quand vous serez habitu\'e \'e
cette recherche incr\'ementale, vous aurez du mal \'e vous en passer
sous d'autres \'editeurs. Si vous d\'esirez revenir \'e l'endroit d'o\'e
vous avez d\'emarr\'e la recherche, tapez  \code{C-u C-}\touche{SPC}
(Emacs a pos\'e une marque implicitement \'e l'endroit o\'e vous avez
commenc\'e la recherche).  \code{C-s} cherche en avant dans le texte;
pour chercher en arri\'ere, utilisez  \code{C-r} ( \code{C-r} est bien
utile sous \code{bash}: il permet de rechercher une ligne dans
l'historique des commandes!).

Pour chercher et remplacer des cha\'enes de caract\'eres, utilisez
\code{M-\%}. Notez une possibilit\'e int\'eressante: il est possible
d'entrer des retours chariots dans la cha\'ene recherch\'ee avec
\code{C-q C-j}. Par exemple, pour ajouter syst\'ematiquement un signe
\$ \'e la fin de chaque ligne, tapez \code{M-\%  C-q C-j}
\touche{ENTER} \code{ \$   C-q C-j} \touche{ENTER} .

Il est \'egalement possible de chercher et de remplacer en utilisant
des expressions r\'eguli\'eres (grosso modo des \guil{wildcards}). Pour
chercher une expression r\'eguli\'ere, utilisez \code{M-C-s}. Supposons
par exemple que vous vouliez rechercher toutes les initiales de noms
(une lettre majuscule suivie d'un point) dans votre texte; il vous
suffit de taper  \code{C-M-s} \touche{SPC}  \code{[A-Z]}. Si voulez
chercher et remplacer utilisez la command \code{M-x
query-replace-regexp}. La description d\'etaill\'ee des expressions
r\'eguli\'eres n\'ecessiterait plusieurs pages. Pour bien comprendre
comment elles fonctionnent, il vous faudra lire la documentation
en-ligne d'Emacs, ce que le paragraphe suivant va nous apprendre \'e
faire.


 \begin{center}
\begin{longtable}{l l }
  \code{C-}\touche{SPC}   &     pose une marque (p.ex. d\'ebut de s\'election) \\
 \code{C-u  C-}\touche{SPC}   &     va \'e la derni\'ere marque \\
 \code{C-x C-x}  &  \'echange le curseur et la marque \\
 \code{C-w}      &  coupe le texte entre la marque et le curseur \\
 \code{M-w }     & copie le texte entre la marque et le curseur \\
 \code{C-y}  &  colle le presse-papier \\
 \code{M-y}  &  remonte dans l'historique des textes coup\'es \\
 \code{C-s} ,  \code{C-r}  &    recherche incr\'ementale
\end{longtable}
\end{center}


\subsection{Utiliser l'aide}

Il est tr\'es vivement recommand\'e d'apprendre \'e se servir des
diff\'erents syst\'emes d'aide en-ligne d'Emacs, auxquels on acc\'ede par
\code{C-h}.

Par exemple, tapez  \code{C-h t}. Vous devez voir s'afficher un
\guil{tutorial}, c'est \'e dire une documentation interactive pour
d\'ebutant. Parcourez la rapidement, en utilisant les touches
\code{C-v} et \code{M-v} pour descendre ou monter dans le texte.
Quand vous en aurez assez, fermez-le avec  \code{C-x k}.

La s\'equence de touche  \code{C-h i m emacs} vous emm\'ene dans la
documentation d'Emacs, au format \code{info}. Par exemple, tapez
\code{m regexps} pour lire les quelques pages d\'ecrivant la syntaxe
des expressions r\'eguli\'eres (d\'eplacez vous en appuyant sur la touche
espace, remontez en arri\'ere avec la touche \touche{U}) Tapez
\code{C-x k} pour fermer cette documentation. Plus g\'en\'eralement
\code{C-h i} permet d'acc\'eder \'e toutes les documentations au format
\code{info}, notamment aux utilitaires GNU, et \'e la documentation de
la librairie C ( \code{C-h i m libc}), ce qui est quasiment
indispensable pour programmer en C sous Linux.  \code{C-h i h} vous
pr\'esentera des explications plus compl\'etes pour utiliser
efficacement le mode \code{info}.

Les fonctions suivantes sont \'egalement tr\'es utiles:

\begin{center}
\begin{longtable}{p{2cm} p{12cm} }
  \code{C-h k}  &  d\'ecrit la fonction associ\'ee \'e une touche; par exemple  \code{C-h k C-k} affiche la description de
  la commande \code{kill-line}. \\
 \code{C-h f}  &  d\'ecrit une fonction. Essayez, par exemple,  \code{C-h f iso-accents-mode}. \\
 \code{C-h a}   &  liste les fonctions dont le noms contient un mot cl\'e. Par exemple,  \code{C-h a spell} vous listera
 les fonctions li\'ees \'e la d\'etections des fautes d'orthographe. \\
 \code{C-h p} &  liste les modules d'extensions Emacs (fichiers \code{.el} ou \code{.elc}). Pour charger un module, on doit
 utiliser la commande \code{load-library} et fournir le nom du module. Par exemple si vous chargez le module \code{autorevert}
 puis tapez \code{M-x auto-revert-mode}: emacs d\'etectera alors toute modification sur le disque du fichier associ\'e au buffer
 courant.
\end{longtable}
\end{center}

\subsection{Les modes\index{Emacs!mode}}

On n'attend pas le m\'eme comportement d'un \'editeur selon qu'on
travaille sur un texte ou un programme en C. Emacs poss\'ede des modes
distincts pour travailler plus efficacement sur les diff\'erents type
de fichiers. Le mode est affich\'e au milieu de la ligne d'\'etat en bas
de l'\'ecran. Bien qu'Emacs t\'eche de d\'etecter le type de fichier \'e
l'ouverture, il faut n\'eanmoins savoir changer manuellement de mode,
par exemple lorsqu'on cr\'ee un nouveau fichier. Comme toujours sous
Emacs, il suffit d'ex\'ecuter la commande ad\'equate, par exemple:
\code{M-x text-mode}, \code{M-x latex-mode}, \code{M-x perl-mode}...
Pour obtenir la liste des modes accessibles, tapez  \code{C-h a
mode}. Pour avoir de l'aide sur le mode courant, tapez  \code{C-h
m}.

Emacs poss\'ede \'egalement la notion de \code{mode mineur}, qui
sp\'ecifie, en quelque sorte, des sous-modes. Par exemple
\code{auto-fill-mode} sp\'ecifie que emacs doit revenir
automatiquement \'e la ligne quand une certaine colonne est atteinte.
La remise en forme d'un paragraphe se fait manuellement avec
\code{M-q}.

Un des modules d'extension les plus remarquables d'Emacs fournit des
modes pour \TeX~et \LaTeX. Nomm\'ee AUCTex, il rend l'\'edition de
documents \LaTeX~tr\'es confortable : les mots cl\'es \LaTeX~sont
color\'es, le reformatage des paragraphes tient compte des commandes
\LaTeX~(par exemple, un environnement verbatim ne sera pas reformat\'e
accidentellement!),  \code{C-c-c} lance la compilation par \LaTeX~ou
le visualiseur \code{xdvi}..., etc.



\subsection{Quelques \guil{trucs} en vrac}

Quelques fonctions int\'eressantes:

\begin{center}
\begin{longtable}{ p{5cm} p{9cm} }
\code{M-x ispell-buffer}  &     appelle le correcteur orthographique \\
\code{M-x iso-accents-mode} &   permet de taper des caract\'eres accentu\'es avec un clavier am\'ericain \\
\code{M-! }  & ex\'ecute une commande shell \\
\code{M-| } &   ex\'ecute une commande shell sur la r\'egion courante \\
 \code{C-x r k} et  \code{C-x r y}  &   Couper/coller des colonnes \\
 \code{C-u 72 * } &     entrer 72 fois le caract\'ere \guil{*} \\
 \code{C-u 3 M-x sort-numeric-fields}  &    trier un tableau sur la troisi\'eme colonne \\
\code{M-x global-set-key }   & permet d'associer une fonction \'e une touche \\
\code{M-x compile}   &  execute \code{make -k}
\end{longtable}
\end{center}

%



\section{Environnement de d\'eveloppement C++}\label{DevelopEnvironTool}

Le plus simple, a terme, est de profiter des possibilit\'es
\code{emacs} (voir~\ref{sec:emacs}). Il s'agit de l'outil standard
de d\'eveloppement C++ sous linux.
\subsection{Editeur Emacs}

 Nous rappelons ici les quelques
commandes d'\'edition utiles pour le d\'ebutant :
\begin{description}
 \item[]C- correspond \'e la touche \touche{Ctrl}
 \item[]M- correspond \'e la touche \touche{Echap} (appel\'ee parfois META)
\end{description}

Premi\'eres commandes :
\begin{center}
\begin{longtable}{ p{5cm} p{9cm} }
 \code{C-x   C-f} &  ouvrir    \\
 \code{C-x   C-s} &  sauver     \\
 \code{C-x  C-c}  &  quitter \\
 \code{C-x  u}    &  \guil{undo} \\
 \code{C-x  s}    &  chercher \\
 \code{M-\%}      &  remplacer    \\
 \code{C-w}       &  couper    \\
 \code{M-w}       &  copier    \\
 \code{C-y}       &  coller    \\
 \code{C-g}       &  abandonne l'action en cours (\guil{\'echappe})   \\
 \code{C-k}       &  efface la ligne courante   \\
 \code{C-2}       &  divise la fen\'etre en 2  \\
 \code{C-1}       &  repasse en mode \'e 1 seule fen\'etre
\end{longtable}
\end{center}

Par ailleurs, le mode \code{windows} offre des raccourcis dans la
barre de menu ( dont\code{copier, couper, coller, sauver}, ...).

\subsection{Personnaliser Emacs}

Emacs peut \^etre personnalis\'e par l'interm\'ediaire des touches de
fonctions par exemple. A l'ECN, il est usuel d'utiliser les
raccourcis suivant :
\begin{center}
\begin{longtable}{ p{5cm} p{9cm} }
 \code{F1}  &  tags-search     \\
 \code{F8}  &  compile     \\
 \code{F3}  &  goto next error     \\
 \code{F2}  &  recompile (last command line) \\
 \code{F12} &   undo
\end{longtable}
\end{center}
Ils sont d\'efinis dans le fichiers \code{.emacs}. Pour travailler
avec les m\'emes raccourcis qu'un autre  d\'eveloppeur  de l'ECN, il est
pr\'ef\'erable de recopier son fichier \code{.emacs} dans son r\'epertoire
\code{/home}.


Il est possible \'egalement d'activer une barre de navigation \'e partir
de la barre de menus : menu \code{tools - Display Speedbar}. Elle
peut \^etre lancer au d\'emarrage en ajoutant la commande suivante dasn
le fichier \code{.emacs} :
\begin{verbatim}
    (when window-system   (speedbar t))
\end{verbatim}
L'inconvenient, de la commande ci-dessus est qu'emacs place la speedbar au milieu de l'\'ecran. 
Un autre moyen d'afficher facilement la speedbar est de red\'efinir une touche (\touche{F4} par exemple) par la commande :
\begin{verbatim}
    (global-set-key [f4] 'speedbar-frame-mode)
\end{verbatim} 
Elle n'apparait pas au d\'emarrage, mais la touche \touche{F4} place alors la speedbar sur le c\'et\'e gauche de la fen\'etre d'\'edition.

Il est assez facile de d\'efinir les fichiers visible par d\'efault dans la speedbar, ainsi que les dossiers \'e cacher. Ceci ce fait par l'interm\'ediaire des commandes suivantes :

\begin{verbatim}
    (setq speedbar-supported-extension-expressions
      (quote (".[cc]\\(\\+\\+\\|\hh\|c\\|h\\|xx\\)?" ".tex\\(i\\(nfo\\)?\\)?"
       ".h" ".cc" ".c" ".d" ".dbg" ".emacs" ".msh" ".dat" ".inp" ".odb" ".f\\(90\\|77\\|or\\)?"
       ".log" "msg" "sta" "txt" "jpg" "gif" "pdf" "res" "com" "all" "cae" ".s?html"
       "[Mm]akefile\\(\\.in\\)?" ".ml" ".mli" ".nw")))

    (setq speedbar-directory-unshown-regexp "^\\.") 
\end{verbatim} 


Avec la \guil{speedbar}, la gestion des \code{Tags} et leur
pr\'e-d\'efinition (voir~\ref{tags}) et les touches de raccourcis, les d\'ebutants habitu\'es aux outils en mode
\code{windows}  pourrons trouver un  environnement  de d\'eveloppement
relativement convivial.

\onefigure{16cm}{emacs-DTK}{environnement de d\'eveloppement emacs}


\subsection{Rechercher : les \emph{Tags}} \label{tags}

Les \code{Tags} sont des indexations de fichiers permettant des
recherches rapides de nom de classe dans une librairie, comme dans
un moteur de recherche. Pour utiliser cette indexation, \code{emacs}
doit conna\'etre l'endroit o\'e sont index\'es les fichiers (fichier
\code{tags/TAGS}). Si ce fichier n'existe pas, allez dans le
r\'epertoire de la librairie contenant le fichier \code{Makefile} et
tapez \code{make tags}.

Pour remettre \'e z\'ero la table des tags (d'indexation) dans
\code{emacs}, il faut utiliser la commande \code{emacs} suivante :
\begin{verbatim}
        ESC x tags-reset-tags-tables <chemin>
\end{verbatim}
Elle d\'efinit le fichier tags que vous souhaitez utiliser (c'est \'e
dire la librairie dans laquelle vous voulez chercher).


Alors, dans \code{emacs}, la recherche sur un mot cl\'e se fait en
pla\'eant \textbf{ le curseur sur le mot recherch\'e} et d'effectuer la
s\'equence (M-.) :
\begin{description}
 \item[] \touche{ESC}~\touche{.}
 \end{description}

\code{Emacs} affiche alors  directement le fichier o\'e est d\'efini la
classe en question.Pour refermer la fen\'etre contenant le r\'esultat
effectuer la s\'equence (M-*) :

\begin{description}
 \item[]  \touche{ESC}~\touche{*}
 \end{description}

Si vous voulez rechercher dans plusieurs registres de \code{Tags}
simultan\'ement, il est pr\'ef\'erable d'utiliser la commande :
\begin{verbatim}
        ESC x visit-tags-table <chemin>
\end{verbatim}

Enfin, l'ensemble des registres de \code{Tags} peut \^etre charg\'e au
d\'emarrage en d\'efinissant une liste de registre dans le fichier
\code{.emacs} :
\begin{verbatim}
 (setq tags-table-list
       '(
                "/glouton/softserver/LATEST/Xfem/Xfem/tags"
                "/glouton/softserver/LATEST/Xext/Xext/tags"
                "/glouton/softserver/LATEST/Xcrack/Xcrack/tags"
                "/glouton/softserver/LATEST/SolverInterfaces/SolverBase/tags"
                "/glouton/softserver/LATEST/SolverInterfaces/SuperLu/tags"
                "/glouton/softserver/LATEST/SolverInterfaces/SparsKit/tags"
         )
)
\end{verbatim}


\subsection{Autre mode de recherche}

Si vous chercher un mot cl\'e quelconque,   utiliser   la touche
\touche{F1}. Cette commande cherche une cha\'ene de caract\'ere dans les
fichiers \code{.h} et \code{.cc}. Pour chercher l'occurence
suivante, taper (M-,) :
\begin{description}
 \item[]  \touche{ESC}~\touche{,}
 \end{description}





\subsection{Compiler}

Il est recommand\'e de profiter des outils de d\'eveloppement (voir
\ref{DevTools})propos\'es par la librairie \code{BuildUtil}. Cette
derni\'ere fournit des configurations permettant de compiler et de
g\'en\'erer les liens de mani\'ere automatique. Pour cela, vous devez
d\'evelopper dans un repertoire de travail   \code{/devel}  et
utiliser la commande  \code{gmake}.

Dans emacs, vous pouvez utiliser le raccourci \touche{F8} qui
propose une commande de compilation de la forme :
\begin{verbatim}
      make -k -C $DEVROOT/Xfem/Xfem/Xfem  -j 2
\end{verbatim}
A vous pouver modifier cette expression pour votre cas de figure. La
commande \code{make} (ou \code{gmake}) doit  \^etre execut\'ee depuis le
r\'epertoire o\'e est situ\'e le fichier \code{Makefile}, sinon il
convient de lui indiquer o\'u ce trouve ce fichier (option \code{-C
}).



En premier lieu, avant de faire une premi\'ere compilation, il
convient de nettoyer les r\'epertoires et d'initialiser la compilation
:
\begin{verbatim}
      make  -k -C $DEVROOT/Xfem/Xfem/Xfem  distclean  NODEP=1
      make  -k -C $DEVROOT/Xfem/Xfem/Xfem  setup      NODEP=1
\end{verbatim}
ou :
\begin{verbatim}
      make  distclean  NODEP=1
      make  setup      NODEP=1
\end{verbatim}
(ajouter l'option \code{METIS=1} \'e chaque ligne si vous travaillez
sur le cluster \code{pegase}). Cette fonction \code{setup}
construira les d\'ependances entre fichiers et cr\'eera le r\'epertoire
\code{Include} contenant les liens vers les fichiers sources.


Compiler les librairies avec :
\begin{verbatim}
      make  -k -C $DEVROOT/Xfem/Xfem/Xfem
\end{verbatim}

Pour compiler et executer une application d\'evelopp\'ee dans le
repertoire \code{/devel/<exemple>}:
\begin{verbatim}
      make  -k -C $DEVROOT/Xfem/Xfem/Xfem  checkone DIR=/devel/<exemple>
\end{verbatim}





\subsection{D\'ebugger \'e la compilation}

Le d\'ebuggage \'e la compilation peut ce faire en compilant \'e partir
d'\code{emacs} (en lan\'eant la compilation avec \touche{F8}).
Celui-ci permet de remonter directement au ligne pr\'esentant des
erreurs.

Sinon, la technique du Write est couramment utiliser.

\subsection{D\'ebugger \'e l'execution}

Utiliser le d\'ebuggeur \code{ddd}.

Pour utiliser un d\'ebuggeur, il convient d'ouvrir le fichier
\code{Makefile} et d'ajouter l'option \code{-g} sur la ligne de
compilation (\code{CXXFLAGS}. Puis lancer la commande : \code{ddd
-}<nom de application>. Puis faire un \code{RUN}.

\textit{(\'e faire...)}

\subsection{Temps d'execution}

Pour tester l'impl\'ementation d'une routine, il est parfois utile de
connaitre le temps d'execution. Dans le fichier \code{Makefile}, il
faut alors ajouter l'option \code{-pg} sur la ligne de compilation
(\code{CXXFLAGS}. Il est alors pr\'ef\'erable de compiler en version
optimis\'ee en ajoutant :
\begin{verbatim}
      VERS=opt
\end{verbatim}
sur la ligne de commande \code{gmake}

Puis executer le code par la commande :
\begin{verbatim}
        gprof <nom_de_l_executable> >& listing
\end{verbatim}
le fichier \code{listing} contiendra les informations recherch\'ees.





\section{gmsh \index{gmsh} : meshing and post-processing}\label{gmsh}


\code{gmsh} is an automatic 3D finite element mesh generator with
build-in pre- and post-processing facilities.

See gmsh's web site : \web{http://www.geuz.org/gmsh/}


\subsection{.geo file format}\index{gmsh!.geo} \label{gmsh_geo}

see \web{http://www.geuz.org/gmsh/doc/texinfo/gmsh_4.html} for details.



Gmsh's geometry module provides a simple CAD engine, using a bottom-up (boundary representation) approach: you need to first define points (using the Point command: see below), then lines (using Line, Circle, Spline, ..., commands or by extruding points), then surfaces (using for example the Plane Surface or Ruled Surface commands, or by extruding lines), and finally volumes (using the Volume command or by extruding surfaces).

These geometrical entities are called "elementary" in Gmsh's jargon, and are assigned identification numbers when they are created:

\begin{enumerate}
  \item  each elementary point must possess a unique identification number;
  \item  each elementary line must possess a unique identification number;
  \item  each elementary surface must possess a unique identification number;
  \item  each elementary volume must possess a unique identification number. 
\end{enumerate}

Elementary geometrical entities can then be manipulated in various ways, for example using the Translate, Rotate, Scale or Symmetry commands.

Compound groups of elementary geometrical entities can also be defined and are called "physical" entities. These physical entities cannot be modified by geometry commands: their only purpose is to assemble elementary entities into larger groups, possibly modifying their orientation, so that they can be referred to by the mesh module as single entities. As is the case with elementary entities, each physical point, physical line, physical surface or physical volume must be assigned a unique identification number. See 4. Mesh module, for more information about how physical entities affect the way meshes are saved.

 

The next subsections describe all the available geometry commands. 
 
\subsubsection{Points}


\begin{verbatim}
      Point ( expression ) = { expression, expression, expression, expression };

      Physical Point ( expression | char-expression ) = { expression-list };
\end{verbatim}
 
\subsubsection{Lines}

\begin{verbatim}
      Bezier ( expression ) = { expression-list };

      BSpline ( expression ) = { expression-list };
 
      Circle ( expression ) = { expression, expression, expression };

      CatmullRom ( expression ) = { expression-list }; 

      Ellipse ( expression ) = { expression, expression, expression, expression };

      Line ( expression ) = { expression, expression };

      Spline ( expression ) = { expression-list };

      Line Loop ( expression ) = { expression-list };

      Physical Line ( expression | char-expression ) = 
                                                    { expression-list };
\end{verbatim}

\subsubsection{Surfaces}

\begin{verbatim}
      Plane Surface ( expression ) = { expression-list };

      Ruled Surface ( expression ) = { expression-list };

      Surface Loop ( expression ) = { expression-list };

      Physical Surface ( expression | char-expression ) = { expression-list };
\end{verbatim}


\subsubsection{Volumes}

\begin{verbatim}
      Volume ( expression ) = { expression-list };

      Physical Volume ( expression | char-expression ) = { expression-list };
\end{verbatim}
 

\subsubsection{Extrusions}

Lines, surfaces and volumes can also be created through extrusion of points, lines and surfaces, respectively. Here is the syntax of the geometrical extrusion commands :

extrude:

\begin{verbatim}
      Extrude { expression-list } { extrude-list }

      Extrude { { expression-list }, { expression-list }, expression } 
                                                            { extrude-list }
  
      Extrude { { expression-list }, { expression-list }, { expression-list },
                                                expression } { extrude-list } 
\end{verbatim}
  

with
\begin{verbatim}
extrude-list: 
  Point | Line | Surface { expression-list }; ...
\end{verbatim}


\subsubsection{Transformations}

Geometrical transformations can be applied to elementary entities, or to copies of elementary entities (using the Duplicata command: see below). The syntax of the transformation commands is:

transform:

\begin{verbatim}
      Dilate { { expression-list }, expression } { transform-list }

      Rotate { { expression-list }, { expression-list }, expression } 
                                                         { transform-list }

      Symmetry { expression-list } { transform-list }

      Translate { expression-list } { transform-list }
\end{verbatim}
 
   
    
with
\begin{verbatim}
transform-list: 
  Point | Line | Surface { expression-list }; ... |
  Duplicata { Point | Line | Surface { expression-list }; ... } |
  transform 
\end{verbatim}

\subsubsection{Example}
\begin{verbatim}
      r = 3;
      L = 3.;
      l = L/r;
      ta = 1/((r+1)*l);
	      xM= L/2.;
	      yM = L/2.;
	      zM = 0.;
	      xm = l/2.;
	      ym = l/2.;
	      zm = 0.;
	      Point(1) = {-xM, -xM, zM,ta };
	      Point(2) = {xM, -yM, zM,  ta};
	      Point(3) = {xM, yM, zM,ta };
	      Point(4) = {-xM, yM, zM,ta };
	      Point(5) = {-xm, -ym, zm,ta };
	      Point(6) = {xm, -ym, zm,ta };
	      Point(7) = {xm, ym, zm,ta };
	      Point(8) = {-xm, ym, zm, ta};
	      Line (9) = {1, 2};
	      Line (10) = {2, 3};
	      Line (11) = {4, 3};
	      Line (12) = {1, 4};
	      Line (121) = {5, 6};
	      Line (122) = {6, 7};
	      Line (123) = {7, 8};
	      Line (124) = {8, 5};
	
	      /* carr\'e exterieur moins carr\'e interieur */
	      Line Loop (2000025) = {-12, 9, 10, -11, -122, -121, -124, -123};
	      Plane Surface (1000025) = {2000025};
	      /* carre interieur  */
	      Line Loop (2000026) = {121, 122, 123, 124};
	      Plane Surface (1000026) = {2000026};
	      Transfinite Line {9,11,10,12} = ((3*r)+1) Using Power 1.0; 
	      /*Transfinite Line {121,122,123,124} = (r+1) Using Power 1.0; */
	      Transfinite Surface {1000025} ={1, 2, 3, 4};
	      Transfinite Surface {1000026} ={5, 6, 7, 8}; 
	      Physical Point (101) = {1};
	      Physical Point (102) = {2};
	      Physical Line (201) = {9};
	      Physical Line (202) = {10};
	      Physical Line (203) = {11};
	      Physical Line (204) = {12};
	      Physical Line (1000) = {121,122,123,124};
	      Physical Surface (301) = {1000025,1000026};
\end{verbatim}




\subsection{.msh file format}\index{gmsh!.msh}  \label{gmsh_msh}

see \web{http://www.geuz.org/gmsh/doc/texinfo/gmsh_10.html} for details.






\subsubsection{Version 1.0}

The `.msh' file format, version 1.0, is Gmsh's old native mesh file format, now superseded by the format described in 9.1.2 Version 2.0 ASCII.

In the `.msh' file format, version 1.0, the file is divided in two sections, defining the nodes (\$NOD-\$ENDNOD) and the elements (\$ELM-\$ENDELM) in the mesh:

 	
\begin{verbatim}
$NOD
number-of-nodes
node-number x-coord y-coord z-coord
...
$ENDNOD
$ELM
number-of-elements
elm-number elm-type reg-phys reg-elem number-of-nodes node-number-list
...
$ENDELM
\end{verbatim}


where

\begin{longtable}{p{4cm} p{11cm}}
\code{number-of-nodes} &
    is the number of nodes in the mesh.\\
\\
\code{node-number} &
    is the number (index) of the n-th node in the mesh. \\
\code{x-coord y-coord z-coord} &
    are the floating point values giving the X, Y and Z coordinates of the n-th node.\\
\\
\code{number-of-elements} &
    is the number of elements in the mesh.\\
\\
\code{elm-number} &
    is the number (index) of the n-th element in the mesh. \\
\\
\code{elm-type} &
    defines the geometrical type of the n-th element (see~\ref{code_elem_type}).\\
    

\code{reg-phys} &
    is the number of the physical entity to which the element belongs.\\

\code{reg-elem} &
    is the number of the elementary entity to which the element belongs.\\

\code{number-of-nodes} &
    is the number of nodes for the n-th element. \\

\code{node-number-list} &
    is the list of the number-of-nodes node numbers of the n-th element.  \\
\end{longtable}


\subsubsection{Version 2.0 ASCII}

The version 2.0 of the `.msh' file format is Gmsh's new native mesh file format. It is very similar to the old one, but is more general: it contains information about itself and allows to associate an arbitrary number of integer tags with each element. It also exists in both ASCII and binary form.

The `.msh' file format, version 2.0, is divided in three main sections, defining the file format (\$MeshFormat-\$EndMeshFormat), the nodes (\$Nodes-\$EndNodes) and the elements (\$Elements-\$EndElements) in the mesh:

 	
\begin{verbatim}
$MeshFormat
2.0 file-type data-size
$EndMeshFormat
$Nodes
number-of-nodes
node-number x-coord y-coord z-coord
...
$EndNodes
$Elements
number-of-elements
elm-number elm-type number-of-tags < tag > ... node-number-list
...
$EndElements
\end{verbatim}


where

\begin{longtable}{p{4cm} p{11cm}}
\code{file-type} &
    is an integer equal to 0 in the ASCII file format.\\

\code{data-size} &
    is an integer equal to the size of the floating point numbers used in the file (currently only data-size = sizeof(double) is supported).\\

\code{number-of-nodes} &
    is the number of nodes in the mesh.\\

\code{node-number} &
    is the number (index) of the n-th node in the mesh.\\

\code{x-coord y-coord z-coord} &
    are the floating point values giving the X, Y and Z coordinates of the n-th node.\\

\code{number-of-elements} &
    is the number of elements in the mesh.\\

\code{elm-number} &
    is the number (index) of the n-th element in the mesh.\\

\code{elm-type} &
    defines the geometrical type of the n-th element (see~\ref{code_elem_type}).\\


\code{number-of-tags} &
    gives the number of tags for the n-th element. \\

\code{tag} &
    is an integer tag associated with the n-th element. \\

\code{node-number-list}  &
    is the list of the node numbers of the n-th element. 
\end{longtable}

Additional sections can be present in the file, which an external parser can simply ignore. One such additional section (\$PhysicalNames-\$EndPhysicalNames) associates names with physical entity identification numbers:
 	
\begin{verbatim}
$PhysicalNames
number-of-names
phyical-number "physical-name"
$EndPhysicalNames 
\end{verbatim}



    \chapter{Associated tools and libraries \\ \textnormal{(contribution de Y. Guetari)}}\label{DevTools}
        



\section{Libraries}

This chapter gives a bref description of the libaries  \code{/AOMD},
\code{/Solver}, \code{/STL} and  \code{/Treillis}.


\subsection{\code{Treillis/AOMD/}: Algorithm Oriented Mesh Database library \index{Library!AOMD}\index{Library!Trellis} }

Contains the meshing library used to work with \code{xfem} .

\paragraph*{Description}

 The aim of the Algorithm Oriented Mesh Database (AOMD) is to be a mesh management library (or database) that is able
  to provide a variety of services for mesh users. The optimal form of the mesh representation is application dependant
  with different applications requiring different sets of mesh adjacencies.

 AOMD supports hybrid meshes (Triangles, Quads, Hexes, Tets, Prisms and Pyramids). The parallel paradigm used is
 the Rensselaer  Partitioning Method (RPM).

AOMD is written in C++ and it uses STL as well for containers,
iterators and algorithms. Parallel communications are made using the
Message Passing Interface (MPI). Optimal message packing is made
using the autopack library that was developed at Argonne National
Labs by Ray Loy.

AOMD provides advanced services like automatic mesh refinement and
coarsening. Mesh refinement introduces load unbalance in partitions.
This unbalance is not acceptable if one wants to achieve scalable
parallel software. The solution to this unbalance is to dynamically
re-partition the mesh. Some load balancing libraries are available
on the web. The library Zoltan from Sandia is a package which
includes four load balancing libraries based on both graph and
octree partitioning. Classically, the balancer takes as input a
representation of the parallel mesh (octree or partitioned graph)
and provides as output a partition vector telling on which partition
a given mesh entity has to be in order to restore the load balance.
The completion of dynamic re-partitioning consists of dynamically
moving the appropriate entities from one partition to another.

For DOXYGEN documentation:
\web{http://www.scorec.rpi.edu/AOMD/Doc/html/index.html}


\subsection{\code{Solver/} \index{Library!Solver}}

Contains:

\begin{itemize}
\item  The Iterative Template Library (ITL):   \web{http://www.osl.iu.edu/research/itl/}
\item  The Matrix Template Library (MTL):   \web{http://www.osl.iu.edu/research/mtl/}
\item the Portable, Extensible Toolkit for  Scientific Computation (PETSc):
 \web{http://www-unix.mcs.anl.gov/petsc/petsc-as/index.html}
\item The  SuperLU library (SuperLU): \web{http://crd.lbl.gov/~xiaoye/SuperLU/}
\item The  Taucs: a Library of Sparse Linear Solvers \web{http://www.tau.ac.il/~stoledo/taucs/}
\end{itemize}

 the solver's library Sparskit used for \code{xfem}:

\web{http://www-users.cs.umn.edu/~saad/software/SPARSKIT/sparskit.html}

\subsection{\code{Xext/} \index{Library!Xext}}

This library contains general developments for \code{xfem} which are
only available for ECN users.

\subsection{\code{Xcrack/} \index{Library!Xcrack}}

This library contains  developments for crack propagation and is
only available for ECN users.



\part{Manuel Th\'eorique\label{TheoryManual}}
    \markboth{Manuel Th\'eorique}{Manuel Th\'eorique}
    \chapter{L'approche X-FEM}\label{elements_finis_etendus}
        
%%%%%%%%%%%%%%%%%%%%
\section{Introduction}
Ce chapitre se veut une introduction
\'e la m\'ethode des \'el\'ements finis \'etendu \index{m\'ethode des \'el\'ements finis \'etendus}(commun\'ement appel\'e X-FEM pour \guil{eXtended Finite Element Method}).
                             
Cette extension \'e la M\'ethode des El\'ements Finis est partie du constat que beaucoup
d'applications industrielles restent encore aujourd'hui
hors de port\'ee de la
m\'ethode des \'el\'ements finis classiques pour des raisons
de gestion de maillage, ou plus pr\'ecis\'ement par le fait
que le maillage doive respecter toutes
les surfaces physiques du probl\'eme
(fissures, fronts de solidification, interfaces entre
mat\'eriaux, ...). Ces surfaces peuvent \^etre de forme
complexe et/ou \'evoluer dans le temps.

Les surfaces physiques doivent \^etre  maill\'ees  car
l'interpolation \'el\'ements finis est classiquement
tr\'es (trop) r\'eguli\'ere.
L'approche X-FEM permet d'enrichir les \'el\'ements finis avec des modes
moins r\'eguliers (discontinus ou \'e d\'eriv\'ees discontinues)
pour qu'ils puissent \^etre travers\'es par des discontinuit\'es.
Ainsi, les surfaces physiques peuvent \^etre positionn\'ees de mani\'ere
quelconque par rapport au maillage et ce dernier peut \^etre conserv\'e
lors de leur \'evolution. La technique centrale dans cet
enrichissement est la partition de l'unit\'e \index{partition de l'unit\'e} \cite{Babuska:PUFEM}.

Les surfaces ne sont donc plus maill\'ees et, pour les localiser sur le
maillage, la notion de fonction de niveau est tr\'es appropri\'ee.
A chaque noeud dans le voisinage de la surface on associe la distance
sign\'ee \'e cette surface (compt\'ee positivement d'un c\'et\'e et
n\'egativement de l'autre).
Cette fonction distance peut \^etre interpol\'ee sur chaque \'el\'ement
avec les fonctions \'el\'ements finis classique de premier ordre.
L'iso-z\'ero de la fonction localise compl\'etement l'interface.
En clair, dans le contexte X-FEM, les surfaces sont stock\'ees par
un champ \'el\'ements finis d\'efini dans le voisinage de la surface.
Ces champs ``g\'eom\'etriques'' participent au calcul au m\'eme
titre que les champs physiques (d\'eplacement, temp\'erature, ...).
Il se peut que ces surfaces \'evoluent dans le temps sous
un champ de vitesse.
Pour prendre en compte cette \'evolution, la m\'ethode des
\guil{level sets}\index{level sets} \cite{Sethian:book} est particuli\'erement
adapt\'ee. Nous utiliserons aussi bien l'expression
\guil{fonction de niveau} que \guil{level set}.


Dans la prochaine section, la partition de l'unit\'e sera d\'etaill\'ee et
l'on s'int\'eressera \'e l'historique de son av\'enement et ses relations
avec les m\'ethodes sans maillage. Dans la section
\ref{sec:model-de-disc}, cette technique sera utilis\'ee pour
introduire des surfaces de discontinuit\'e dans un maillage. Dans
cette section, on ne s'inqui\'etera pas de l'\'evolution possible de ces
surfaces ni de leur repr\'esentation ``informatique''. Ces deux
questions seront \'etudi\'ees dans la section \ref{sec:local-et-evol}.

La section \ref{sec:applications-et-mise}
discute certains points de la mise
en oeuvre de l'approche X-FEM et d\'ecrit deux
domaines d'applications~:
la m\'ecanique de la rupture et l'homog\'en\'eisation.

Ce document se veut une introduction de la m\'ethode X-FEM
et un condens\'e des outils de base.
Dans les applications (chapitre~\ref{Applications_de_XFEM}), nous nous sommes pench\'es sur des
mod\'eles \'elastiques lin\'eaires en statique.
D'autres travaux r\'ecents ou en cours, \'etendent l'approche
au non lin\'eaire ou \'e la dynamique.
Ces travaux sont cit\'es \'e la section~\ref{sec:conclusion-et-autres}.





\section{De Rayleigh-Ritz \'e la partition de l'unit\'e}
\label{sec:la-partition-de}

La m\'ethode X-FEM se distingue de la m\'ethode des \'el\'ements finis
classique par le rajout dans l'approximation \index{approximation} de fonctions
suppl\'ementaires \'e l'aide d'une technique appel\'ee \index{partition de l'unit\'e} partition de
l'unit\'e~\cite{Babuska:PUFEM}.  Avant de d\'ecrire cette technique
et l'usage qui en est fait dans X-FEM, nous discutons diff\'erentes
m\'ethodes classiques d'approximation du principe variationnel en
d\'eplacement pour l'\'elasticit\'e~:
l'approximation de Rayleigh-Ritz, la
m\'ethode des \'el\'ements finis et les m\'ethodes sans maillage
(\guil{meshfree methods}).

\subsection{Formulation \index{formulation du probl\'eme} du probl\'eme de r\'ef\'erence et notations}\label{probleme_de_reference}

Le milieu solide \'etudi\'e occupe un domaine $\Ome$ de fronti\'ere
$\Gam$. Cette fronti\'ere se compose des l\'evres de la fissure $\Gam_{c^+}$ et $\Gam_{c^-}$  suppos\'ees libres
 de tractions, d'une partie $\Gam_u$ \'e d\'eplacements impos\'es not\'es   $\overline{\bfu}$ et
d'une partie $\Gam_t$ \'e tractions impos\'ees not\'ees   $\overline{\bft}$.
Les contraintes, d\'eformations et d\'eplacements sont not\'es
respectivement $\bfsig$, $\bfeps$ et $\bfu$. L'hypoth\'ese des petites
perturbations est utilis\'ee.
En l'absence de forces de volume, les \'equations d'\'equilibre sont~:
\begin{eqnarray}
\label{eq:equil}
\bfnabla \cdot \bfsig & = & \bfzer \text{\;sur\;} \Ome  \\
\bfsig \cdot \bfn & = & \overline{\bft} \text{\;sur\;} \Gam_t            \\
\bfsig \cdot \bfn & = &  0  \text{\;sur\;} \Gam_{c^+}        \\
\bfsig \cdot \bfn & = & 0  \text{\;sur\;} \Gam_{c^-}
\end{eqnarray}
o\'e $\bfn$ est la normale ext\'erieure \'e la fronti\'ere consid\'er\'ee. Les
\'equations cin\'ematiques s'\'ecrivent~:
\begin{eqnarray}
    \bfeps & = & \bfeps(\bfu) = \bfnabla_{\rms} \bfu  \quad\mbox\;sur\;\quad
    \Ome \\
    \bfu   & = & \overline{\bfu}  \quad\mbox\;sur\;\quad \Gam_u
\end{eqnarray}
o\'e $\bfnabla_{\rms}$ est la partie sym\'etrique de l'op\'erateur
gradient\index{Op\'erateur!gradient}.  Finalement, la loi de comportement consid\'er\'ee est
\'elastique~:
\begin{equation}
    \bfsig = \bfC:\bfeps
\end{equation}
o\'e $\bfC$ est le tenseur de Hooke.
Les espaces des champs de d\'eplacement admissibles, $\cu$, et
admissibles \'e z\'ero, $\cu_0$, sont d\'efinis par~:
\begin{eqnarray}
    \cu & = & \{ \bfv \in \cv: \bfv = \overline{\bfu} \text\;sur\; \Gam_u \} \\
    \cu_0 & = & \{ \bfv \in \cv: \bfv = \bfzer \text\;sur\; \Gam_u \}
\end{eqnarray}
o\'e l'espace $\cv$ est reli\'e \'e la r\'egularit\'e de la solution
et est d\'etaill\'e par exemple dans \cite{Babuska:Corners} et \cite{Grisvard:ellip}.  Cet espace
contient des champs de d\'eplacement discontinus au passage de la fronti\'ere $\Gam_c$.
La forme faible des \'equations d'\'equilibre s'\'ecrit~:
\begin{equation}
    \int_{\Ome}    \bfsig:\bfeps(\bfv) \dint \Ome =
    \int_{\Gam_t} \ov{\bft}\cdot\bfv  \dint \Gam
    \quad \forall \bfv \in \cu_0
    \label{eq:equil2}
\end{equation}
Notons que la fronti\'ere $\Gam_c$ ne contribue pas \'e la forme
faible car libre de traction.
En combinant~(\ref{eq:equil2}) avec la loi constitutive\index{loi constitutive}
et les \'equations cin\'ematiques, le
probl\'eme en d\'eplacement est obtenu~:
\begin{equation}
    \label{eq:varia}
    \text{trouver} \; \bfu\in\cu  \; \vert \;
    \int_{\Ome}   \bfeps(\bfu):\bfC:\bfeps(\bfv) \dint \Ome =
    \int_{\Gam_t} \ov{\bft}\cdot\bfv  \dint \Gam
    \quad \forall \bfv \in \cu_0
\end{equation}

\subsection{L'approximation \index{Approximation} de Rayleigh-Ritz}


Dans la m\'ethode de Rayleigh-Ritz, l'approximation s'\'ecrit comme
une combinaison lin\'eaire de modes de d\'eplacement $\bfphi_i(\bfx), i = 1,
\dots, N$ d\'efinis sur l'ensemble du domaine~:
\begin{equation}
    \bfu(\bfx) = \sum_i^N a_i \bfphi_i(\bfx)
\end{equation}
Ces modes doivent satisfaire \emph{a priori} les conditions
aux limites
essentielles (les d\'eplacements impos\'es sont consid\'er\'es nuls
pour simplifier la pr\'esentation).





L'introduction de cette
approximation dans le principe variationnel\index{principe variationnel} (\ref{eq:varia}) conduit au
syst\'eme d'\'equations
\begin{equation}
\label{eq:global_systeme}
K_{ij} a_j = f_i, \quad i = 1, \ldots, N
\end{equation}
o\'e la sommation sur l'indice $j$ est d'application et o\'e
\begin{eqnarray}
K_{ij} & = & \int_{\Ome}   \bfeps(\bfphi_i):\bfC:\bfeps(\bfphi_j) \dint
\Ome \\
f_i & = & \int_{\Gam_t} \ov{\bft}\cdot\bfphi_i  \dint \Gam
\end{eqnarray}

La m\'ethode de Rayleigh-Ritz offre une grande
libert\'e dans le choix des modes.  Ces modes peuvent par exemple
\^etre choisis de mani\'ere \'e satisfaire les \'equations
int\'erieures du domaine.  Cette m\'ethode a cependant
l'inconv\'enient de conduire \'e un syst\'eme lin\'eaire \'e matrice
dense.



\subsection{La m\'ethode des \'el\'ements finis\index{m\'ethode des \'el\'ements finis}}

Dans la m\'ethode des \'el\'ements finis, le domaine \'etudi\'e,
$\Ome$, est d\'ecompos\'e en sous-domaines g\'eom\'etriques de forme
simple $\Ome_e, e = 1, \ldots, N_e$ appel\'es \'el\'ements~:
\begin{equation}
    \Ome = \cup_{e=1}^{N_e} \Ome_e
\end{equation}
L'ensemble des \'el\'ements constituent le maillage\index{Maillage}. Sur
chaque \'el\'ement, le champ inconnu est \guil{approxim\'e} \'e l'aide de
fonctions de base simples, de type polyn\'emial, appel\'ees fonctions
d'approximation et de coefficients inconnus appel\'es degr\'es de
libert\'e. Les degr\'es de libert\'e\index{degr\'es de libert\'e} ont en
g\'en\'eral une signification m\'ecanique simple. Pour des \'el\'ements de
premier degr\'e, il s'agit de la valeur du d\'eplacement selon $x$, $y$
ou $z$ aux sommets (noeuds) de chaque \'el\'ement. D\'esignons par
$u_i^\alpha$ le d\'eplacement au noeud $i$ dans la direction $\alpha$
($\alpha$ d\'esigne la direction $x$,   $y$  ou $z$) et par
$\bfphi_i^\alpha$ la fonction d'approximation correspondante.


Dans le plan $xy$, l'approximation \'el\'ements finis sur l'\'el\'ement $\Ome_e$
s'\'ecrit~:
\begin{equation}
  \label{eq:approx_ef}
  \bfu(\bfx) \mid_{\Ome_e} = \sum_{i \in N_n} \sum_{\alpha}
  a_i^\alpha \bfphi_i^\alpha(\bfx)
\end{equation}
o\'e $N_n$ est l'ensemble  des noeuds de l'\'el\'ement $\Ome_e$.
Par exemple pour un triangle, les six fonctions
d'approximation sont donn\'ees par~:
\begin{equation}
  \{ \bfphi_i^\alpha \} = \{ \phi_1 \bfe_x, \phi_2 \bfe_x,
  \phi_3 \bfe_x, \phi_1 \bfe_y, \phi_2 \bfe_y, \phi_3 \bfe_y \}
\end{equation}
o\'e les fonctions de forme\index{Fonction!fonctions de forme} scalaires $\phi_1$, $\phi_2$ et $\phi_3$
sont lin\'eaires sur l'\'el\'ement et valent 0 ou 1 aux noeuds du triangle.
L'approximation~(\ref{eq:approx_ef}) permet
de repr\'esenter tous
les modes rigides ou de d\'eformations constantes sur l'\'el\'ement.
Cette condition doit \^etre remplie par l'approximation
pour tous les types d'\'el\'ements.
La continuit\'e du champ d'approximation sur le domaine
est obtenue en imposant que
les degr\'es de libert\'e d\'efinis en un noeud ont la m\^eme
valeur pour tous les \'el\'ements connect\'es en ce noeud.
La matrice de raideur, $K^e_{ij}$,
et le vecteur force, $f^e_i$, sont donn\'es pour un \'el\'ement fini par~:
\begin{eqnarray}
  \label{eq:local_raideur}
  K^e_{i\alpha, j\beta} & = & \int_{\Ome^e}
          \bfeps(\bfphi_i^\alpha):\bfC:\bfeps(\bfphi_j^\beta)
  \dint \Ome \\
  \label{eq:local_force}
  f^e_{i\alpha} & = & \int_{\Gam_t \cap \partial \Ome^e}   \ov{\bft}\cdot\bfphi_i^\alpha  \dint \Gam
\end{eqnarray}

Le syst\'eme global d'\'equations est obtenu en assemblant les
matrices\index{Matrice} et forces \'el\'ementaires dans une matrice de
raideur globale et un vecteur de force global. Dans la phase
d'assemblage, les \'equations associ\'ees aux degr\'es de libert\'e fix\'es
par des conditions aux limites de type Dirichlet ne sont pas
construites.

Contrairement \'e l'approximation de Rayleigh-Ritz,
le caract\'ere local de l'approximation\index{Approximation} \'el\'ement fini
conduit \'e des matrices creuses.
De plus, l'\'el\'ement fini a une interpr\'etation m\'ecanique
forte~: la cin\'ematique est d\'ecrite par des d\'eplacements
nodaux auxquels sont associ\'ees par dualit\'e\index{Dualit\'e} des forces nodales.
Le comportement de l'\'el\'ement est caract\'eris\'e par
la matrice de raideur \'el\'ementaire
qui relie les forces et d\'eplacements nodaux.
Le syst\'eme global \'e r\'esoudre impose
l'\'equilibre de la structure~:
la somme des forces nodales en chaque noeud doit
\^etre nulle.
Enfin, la m\'ethode des \'el\'ements finis s'est r\'ev\'el\'ee
tr\'es robuste pour de nombreuses applications dans l'industrie
ce qui en fait un outil de choix pour la simulation
num\'erique.
Cependant, l'utilisation de la m\'ethode des \'el\'ements finis
pour des probl\'emes \'e g\'eom\'etrie complexe
ou \'e \'evolution de surfaces internes
est actuellement g\^en\'ee  par l'aspect maillage.
Une de motivations derri\'ere les m\'ethodes sans maillage
est de s'affranchir des
difficult\'es.

\subsection{Les m\'ethodes sans maillage}\label{methode_sans_maillage}\index{m\'ethodes sans maillage}

De nombreuses recherches ont \'et\'e conduites
ces dix derni\'eres ann\'ees pour d\'evelopper
des m\'ethodes o\`u l'approximation ne repose pas
sur un maillage mais plut\'et sur un ensemble de points.
Diff\'erentes m\'ethodes existent \'e ce jour~:
\'el\'ements diffus~\cite{Touzot:diffus},
Element Free Galerkin method (EFG)~\cite{TB:EFG},
Reproducing Kernel Particle Method (RKPM)~\cite{Liu:RKPM93},
$h-p$ cloud method~\cite{DuarteOdenCloud96}.
Chaque point a un domaine d'influence
de forme simple (cercle ou rectangle par exemple en 2D) sur lequel
des fonctions d'approximation sont construites.
Ces fonctions sont nulles sur le bord et en dehors
du domaine d'influence. En clair, ces fonctions sont
 \'e support compact.
Par abus de langage, nous parlerons
du support $i$ pour le support associ\'e au point $i$.
Les fonctions d'interpolation d\'efinies sur le support $i$ sont
not\'ees $\bfphi_i^\alpha$, $\alpha = 1, \ldots, N_f(i)$
o\'e  $N_f(i)$ est le nombre de fonctions d'approximation
d\'efinies sur le support $i$.
Les degr\'es de libert\'e correspondants  sont not\'es
$a_i^\alpha$. L'approximation en un point quelconque
$\bfx$ s'\'ecrit~:
\begin{equation}
  \label{eq:approx_efg}
  \bfu(\bfx) = \sum_{i \in N_s(\bfx)} \sum_{\alpha = 1}^{N_f(i)} a_i^\alpha \bfphi_i^\alpha(\bfx)
\end{equation}
o\'e $N_s(\bfx)$ est l'ensemble des points $i$ dont le support
contient le point $\bfx$. La figure~\ref{fig:efg_support} montre par exemple un point
$\bfx$ couvert par trois supports\index{Support}.

\onefigure{4cm}{efg_support}{Trois supports couvrant un point $\bfx$.}


Les fonctions d'approximation sont construites de
mani\'ere \'e ce que l'approximation~(\ref{eq:approx_efg})
puisse repr\'esenter tous les modes rigides
et de d\'eformation constante sur le domaine.
Ces conditions sont n\'ecessaires pour prouver la convergence
de la m\'ethode. Les diff\'erentes approches
(\'el\'ement diffus, EFG, RKPM, \ldots)
se distinguent, entre autres, par les techniques
utilis\'ees pour la construction de ces fonctions d'interpolation.


Une fois les fonctions d'interpolation
construites, il est possible d'en rajouter
par enrichissement\index{Enrichissement}.
Diverses mani\'eres d'enrichir existent et nous d\'ecrirons
un  enrichissement qualifi\'e d'externe
par Belytschko et Fleming (1999)~\cite{Fleming:enrichment}.
L'enrichissement de l'approximation permet
de repr\'esenter
un mode de d\'eplacement donn\'e, par exemple
$F(\bfx)\bfe_x$ sur un sous-domaine donn\'e
$\Ome_F\subset\Ome$.
Notons $N_F$ l'ensemble des supports qui ont une intersection
avec le sous-domaine $\Ome_F$. L'approximation enrichie s'\'ecrit~:
\begin{equation}
  \label{eq:approx_efg_enrichi}
  \bfu(\bfx) = \sum_{i \in N_s(\bfx)}  \sum_{\alpha = 1}^{N_f(i)}
                                 a_i^\alpha \bfphi_i^\alpha(\bfx) +
               \sum_{i \in N_s(\bfx) \cap N_F}
                                 \sum_{\alpha = 1}^{N_f(i)}
                                 b_i^\alpha \bfphi_i^\alpha(\bfx) F(\bfx)
\end{equation}
o\'e les nouveaux degr\'es de libert\'e\index{degr\'es de libert\'e}, $b_i^\alpha$, multiplient les
fonctions d'interpolation enrichies $\bfphi_i^\alpha(\bfx) F(\bfx)$.
Montrons maintenant que la fonction $F(\bfx)\bfe_x$
peut-\^etre repr\'esent\'ee sur le sous-domaine $\Ome_F$.
En fixant \'e z\'ero tous les degr\'e de libert\'e $a_i^\alpha$
et en sortant la fonction $F(\bfx)$ du signe somme,
l'approximation en un point $\bfx \in \Ome_F$
s'\'ecrit~:
\begin{equation}
  \bfu(\bfx) =    \left( \sum_{i \in N_s(\bfx) \cap N_F}
                   \sum_{\alpha = 1}^{N_f(i)}
                   b_i^\alpha \bfphi_i^\alpha(\bfx) \right)
                   F(\bfx)
\end{equation}

Les degr\'es de libert\'e $b_i^\alpha$ peuvent \^etre
choisis de mani\'ere \'e ce que le facteur devant $F(\bfx)$
soit le mode rigide $\bfe_x$. Cela est possible
puisque les fonctions d'approximation $\bfphi_i^\alpha$
sont capables de repr\'esenter
n'importe quel mode rigide. En conclusion,
l'approximation~(\ref{eq:approx_efg_enrichi}) peut repr\'esenter
le d\'eplacement $F(\bfx)\bfe_x$ sur le sous-domaine $\Ome_F$.

L'enrichissement a permis
dans le cadre de l'Element Free Galerkin Method
de r\'esoudre des probl\'emes de propagation de
fissures en deux et trois dimensions sans remaillage~\cite{Krysl:3D}:
la fissure se propage a travers un nuage de points et
est mod\'elis\'ee par enrichissement de l'approximation
avec des fonctions $F(\bfx)$ discontinues sur la fissure
ou repr\'esentant la singularit\'e en fond de fissure.

La grande flexibilit\'e dans l'\'ecriture de l'approximation et
de son enrichissement ainsi que la possibilit\'e de cr\'eer des champs
d'approximation tr\'es r\'eguliers sont deux atouts importants des
m\'ethodes sans maillage et de l'approche EFG en particulier.
L'utilisation des m\'ethodes sans maillage pr\'esente cependant un
certain nombre de difficult\'es par rapport \'e la m\'ethode des
\'el\'ements finis~:
\begin{itemize}
\item Dans la m\'ethode des \'el\'ements finis l'assemblage de la
  matrice de raideur globale peut se faire en assemblant les
  contributions de chaque \'el\'ement. Dans les m\'ethodes sans
  maillage, l'assemblage se fait plut\'et en couvrant le domaine de
  points d'int\'egration et en ajoutant la contribution de chacun
  d'eux.  Le choix de la position et du nombre des points
  d'int\'egration n'est pas trivial pour un nuage arbitraire de points
  d'approximation ;
\item Les conditions aux limites de type Dirichlet sont d\'elicates \'e
  imposer ;
\item Les fonctions de forme sont \'e construire et ne sont pas
  explicites ;
\item La taille des domaines d'influence est un param\'etre dans la
  m\'ethode que l'utilisateur doit choisir avec soin.
\end{itemize}




\subsection{La partition de l'unit\'e}\label{partition_de_l_unite}\index{partition de l'unit\'e}

Melenk et Babu\v{s}ka (1996)~\nocite{Babuska:PUFEM} ont montr\'e
que la base \'el\'ement fini classique pouvait \^etre enrichie
(et donc qu'il n'\'etait pas n\'ecessaire de recourir
\'e des m\'ethodes sans maillage)
de mani\'ere \'e repr\'esenter une fonction donn\'ee sur un domaine
donn\'e. Leur point de vue peut \^etre r\'esum\'e comme suit.
Rappelons que l'approximation \'el\'ements finis s'\'ecrit sur un \'el\'ement $e$ d\'ecrivant le domaine $\Ome_e$~:
\begin{equation}
  \label{eq:approx_ef2}
  \bfu(\bfx) \mid_{\Ome_e} = \sum_{i \in N_n} \sum_{\alpha} a_i^\alpha
  \bfphi_i^\alpha(\bfx)
\end{equation}
o\'e $N_n$ est le nombre de noeuds de l'\'el\'ement $e$.
Comme les degr\'es de libert\'e\index{degr\'es de libert\'e} d\'efinis en un noeud ont la m\^eme
valeur pour tous les \'el\'ements connect\'es en ce noeud.
Les approximations sur chaque \'el\'ement
peuvent \^etre ``assembl\'ees''
pour donner
une approximation\index{Approximation} valable en tout point $\bfx$ du domaine~:
\begin{equation}
  \label{eq:approx_ef_glo}
  \bfu(\bfx)  = \sum_{i \in N_n(\bfx)} \sum_{\alpha} a_i^\alpha \bfphi_i^\alpha(\bfx)
\end{equation}
o\'e $N_n(\bfx)$ est l'ensemble des noeuds de l'\'el\'ement contenant
le point $\bfx$. Le domaine d'influence (support) de la fonction
d'interpolation
 $\bfphi_i^\alpha$ est l'ensemble des \'el\'ements connect\'es
au noeud $i$. L'ensemble $N_n(\bfx)$ est donc \'egalement l'ensemble
des noeuds dont le support couvre le point $\bfx$.
L'approximation \'el\'ements finis~(\ref{eq:approx_ef_glo}) peut ainsi \^etre
interpr\'et\'ee comme une particularisation
de l'approximation~(\ref{eq:approx_efg}) utilis\'ee dans les m\'ethodes
sans maillage~:
\begin{itemize}
\item le nuage de points est l'ensemble des noeuds du maillage;
\item le domaine d'influence de chaque noeud est
l'ensemble des \'el\'ements connect\'es \'e ce noeud.
\end{itemize}
Il est donc possible d'enrichir l'approximation \'el\'ements
finis par les m\^emes techniques que celles utilis\'ees dans les m\'ethodes
sans maillage. Voici l'approximation \'el\'ements finis enrichie
qui permet de repr\'esenter la fonction $F(\bfx) \bfe_x$ sur le
domaine $\Ome_F$~:
\begin{equation}
  \label{eq:approx_ef_enrichi}
  \bfu(\bfx)  = \sum_{i \in N_n(\bfx)} \sum_{\alpha} \bfphi_i^\alpha a_i^\alpha   +
  \sum_{i \in N_n(\bfx) \cap N_F}
  \sum_{\alpha} b_i^\alpha \bfphi_i^\alpha(\bfx) F(\bfx)
\end{equation}
o\'e $N_F$ est l'ensemble des noeuds dont le support\index{Support} a une
intersection avec le domaine $\Ome_F$.
La preuve est obtenue en fixant \'e z\'ero
les coefficients $a_i^\alpha$ et en prenant en compte
le fait que les fonctions d'approximation \'el\'ements finis
sont capables de repr\'esenter tous les modes rigides et
donc le mode $\bfe_x$.
Nous passons maintenant \'e l'utilisation concr\'ete
de la partition de l'unit\'e pour
la mod\'elisation de discontinuit\'es dans X-FEM.





\section{Mod\'elisation de discontinuit\'es\index{Discontinuit\'es} avec X-FEM}
\label{sec:model-de-disc}
L'extension X-FEM de la m\'ethode des \'el\'ements finis,
donne une grande
flexibilit\'e dans la g\'en\'eration des maillages  4~:
il n'est plus demand\'e au maillage de se conformer \'e des surfaces\index{Surfaces}
qu'elles soient ext\'erieures ou internes. Pour arriver \'e cette
flexibilit\'e,
il faut que l'approximation soit capable de mod\'eliser des vides ou
des discontinuit\'es au sein m\^eme des \'el\'ements. La m\'ethode la
partition de l'unit\'e d\'evelopp\'ee par Melenk et Babu\v{s}ka (1996)
est utilis\'ee \'e cette fin. Nous donnons dans cette section ces
diff\'erents enrichissements. Ensuite, nous nous int\'eressons \'e la
localisation (et l'\'evolution) de ces surfaces.



\subsection{Revisite du double noeud~: la fonction Heaviside\index{Fonction!Heavyside}}

L'id\'ee de l'utilisation de la partition de l'unit\'e pour repr\'esenter une
discontinuit\'e a d\'ebut\'e par une revisite du
concept de double noeud et sa g\'en\'eralisation~\cite{Moes:discont}.

\twofigures{5cm}{crack}{5cm}{grid}{Un maillage de quatre \'el\'ements avec un double noeud (a); sans double noeud (b)}{fig:dbl_noeud}


Avec des \'el\'ements finis conformes,
la repr\'esentation d'un champ discontinu sur un maillage
ne peut se faire qu'en bordure des \'el\'ements par l'utilisation
de doubles noeuds. La figure~\ref{fig:crack} montre par exemple
un maillage de 4 \'el\'ements dans lequel une discontinuit\'e
a \'et\'e introduite par le biais d'un double noeud (noeuds 9 et 10).
L'approximation \'el\'ements finis correspondante s'\'ecrit
\begin{equation}
\bfu^h = \sum_{i=1}^{10} \bfu_i \phi_i
\label{eq:simdisp}
\end{equation}
o\'e $\bfu_i$ est le d\'eplacement (vecteur) au noeud  $i$ et $\phi_i$
est la fonction de forme bi-lin\'eaire scalaire associ\'ee au noeud $i$.
Chaque fonction d'approximation $\phi_i$ a un support  compact
$\omega_i$ donn\'e par l'union des \'el\'ements connect\'es au noeud $i$.
R\'e\'ecrivons (\ref{eq:simdisp}) afin de faire appara\'etre une
approximation \'el\'ements finis correspondant \'e un maillage sans double
noeud, figure~\ref{fig:grid}, et un terme suppl\'ementaire.
D\'efinissons le d\'eplacement moyen $\bfa$ et le saut $\bfb$
\begin{equation}
\bfa = \frac{\bfu_{9} + \bfu_{10}}{2} \quad
\bfb = \frac{\bfu_{9} - \bfu_{10}}{2}
\label{eq:cfem}
\end{equation}
En inversant ce syst\'eme, il vient
\begin{equation}
\bfu_{9} = \bfa + \bfb \quad \bfu_{10} = \bfa - \bfb
\end{equation}
L'approximation \'el\'ements finis~(\ref{eq:simdisp}) peut alors
se r\'e\'ecrire
\begin{equation}
\bfu^h = \sum_{i=1}^{8} \bfu_i \phi_i + \bfa (\phi_{9} + \phi_{10}) +
\bfb (\phi_{9} + \phi_{10}) H(\bfx)
\label{eq:last}
\end{equation}
o\'e $H(\bfx)$ est une fonction discontinue donn\'ee par
\begin{equation}
H(x,y) = \begin{cases}
1 & \mbox{for}\quad y > 0 \\
-1 & \mbox{for}\quad  y<0
\end{cases}
\label{hdef}
\end{equation}
La somme des fonctions d'approximation associ\'ees aux noeuds 9 et 10
dans le maillage figure~\ref{fig:crack} correspond \'e la fonction
d'approximation localis\'ee sur le noeud 11 dans le maillage figure~\ref{fig:grid}.
On peut donc r\'e\'ecrire~(\ref{eq:last}) comme
\begin{equation}
\bfu^h = \sum_{i=0}^{8} \bfu_i \phi_i + \bfu_{11} \phi_{11}
+ \bfb \phi_{11} H(\bfx)
\label{eq:last2}
\end{equation}
Les deux premiers termes du membre de droite correspondent
\'e une approximation \'el\'ement fini classique sur le maillage
figure~\ref{fig:grid}. La fonction d'approximation
intervenant dans le
troisi\'eme terme est le produit d'une fonction d'approximation
classique
$\phi_{11}$ par la fonction discontinue $H(\bfx)$.
Ce troisi\'eme terme
peut s'interpr\'eter comme un enrichissement de la base
\'el\'ements finis par une technique de type partition de l'unit\'e.
La d\'erivation que nous venons de mener sur un petit maillage de
quatre \'el\'ements peut \^etre r\'eit\'er\'ee sur n'importe
quel  maillage 1D, 2D ou 3D contenant une discontinuit\'e
mod\'elis\'ee par doubles noeuds. Cette d\'erivation m\'enera au m\^eme
constat~: la mod\'elisation d'une discontinuit\'e\index{Discontinuit\'e} par double noeud
est \'equivalente \'e la mod\'elisation \'el\'ements finis classique
si on lui ajoute des termes correspondant \'e l'enrichissement
par la partition de l'unit\'e
des noeuds situ\'es sur le parcours de la discontinuit\'e.
Notons que les noeuds qui sont enrichis (ex-doubles noeuds)
sont caract\'eris\'es par le
fait que leur support est coup\'e en deux par la discontinuit\'e.

\subsection{G\'en\'eralisation du double noeud}
Supposons maintenant que l'on souhaite mod\'eliser une discontinuit\'e
qui ne suit pas le bord des \'el\'ements. Nous
proposons d'enrichir
tous les noeuds dont le support est (compl\'etement) coup\'e en deux
par la  discontinuit\'e~\cite{Moes:discont}.
En ces noeuds, nous ajoutons un degr\'e
de libert\'e (vectoriel) agissant sur la fonction d'approximation classique
au noeud multipli\'ee par une fonction discontinue $H(\bfx)$ valant 1 d'un
c\'et\'e de la fissure et -1 de l'autre.
Par exemple, sur la figure~\ref{fig:uniform-heavyside}, les noeuds encercl\'es seront enrichis.

% \twofigures{6cm}{uniform-heavyside}{6cm}{general-heavyside}{Surface de discontinuit\'e
% plac\'ee sur un maillage uniforme (a) et non uniforme (b).
% Les noeuds marqu\'es d'un point noir sont enrichis par la fonction Heaviside.}{fig:both_disc}
\onefigure{6cm}{uniform-heavyside}{Surface de discontinuit\'e
plac\'ee sur un maillage uniforme. Les noeuds marqu\'es d'un point noir sont enrichis par la fonction Heaviside.}


\subsection{Mod\'elisation d'une fissure\index{Fissure}}

Passons maintenant \'e une ligne de discontinuit\'e qui ne s\'epare pas le
domaine en deux zones distinctes, c'est-\'e-dire une fissure. Un noeud
dont le support n'est pas totalement coup\'e par la discontinuit\'e
\index{Discontinuit\'e} ne peut \^etre enrichi par la fonction
$H$\index{Fonction!Heavyside} car cela conduirait \'e allonger
artificiellement la discontinuit\'e. Par exemple, si pour le maillage
de la figure~\ref{fig:tip}, les noeuds $C$ et $D$ sont enrichis par
la discontinuit\'e, la fissure sera active jusqu'au point $q$ (le
champ de d\'eplacement sera discontinu jusqu'au point $q$). D'un autre
c\'et\'e, si seuls les noeuds $A$ et $B$ sont enrichis par la
discontinuit\'e, le champ de d\'eplacement n'est discontinu que jusqu'au
point $p$ et la fissure appara\'et malheureusement plus courte.

Afin de repr\'esenter la fissure sur sa ``bonne longueur'', les noeuds
dont le support contient la pointe de fissure\index{pointe de
fissure} (noeuds avec carr\'e figure~\ref{fig:tip}) sont enrichis avec
des fonctions discontinues jusqu'au point $t$ mais pas au-del\'e. De
telles fonctions sont fournies par les modes de d\'eplacement
asymptotiques\index{Asymptotique} (\'elastiques si le calcul est
\'elastique) en pointe de fissure. Cet
enrichissement\index{Enrichissement}, d\'ej\'e utilis\'e
par~\cite{TBlack:minim} et \cite{Stroub:GFEM1}, permet en outre des
calculs pr\'ecis puisque les caract\'eristiques asymptotiques du champ
de d\'eplacement sont incorpor\'ees dans le calcul.

\onefigure{6cm}{tip}{Les noeuds entour\'es d'un carr\'e sont
                            enrichis par les modes asymptotiques en
                            pointe de fissure.}


Nous sommes maintenant en mesure de d\'etailler la mod\'elisation par X-FEM
d'une fissure compl\'ete dont la position est quelconque par rapport au maillage
(voir illustration figure~\ref{fig:both}). L'approximation \'el\'ements finis enrichie s'\'ecrit~:

\begin{eqnarray*}
\bfu^h(\bfx) & = & \sum_{i\in I} \bfu_i \phi_i(\bfx) +
\sum_{i\in L} \bfa_i \phi_i(\bfx) H(\bfx) \\ \nonumber
& + & \sum_{i\in K_1} \phi_i(\bfx) (\sum_{l=1}^{4} \bfb_{i,1}^{l}  F^l_{1}(\bfx)) +
\sum_{i\in K_2} \phi_i(\bfx) (\sum_{l=1}^{4} \bfb_{i,2}^{l}  F^l_{2}(\bfx)) \nonumber
\label{eq:appgen}
\end{eqnarray*}
o\'e~:
\begin{itemize}
\item $I$ est l'ensemble des noeuds du maillage ;
\item $\bfu_i$ est le degr\'e de libert\'e (vectoriel) classique au noeud $i$ ;
\item $\phi_i$ est la fonction de forme (scalaire) associ\'ee au noeud $i$ ;
\item $L\subset I$ est l'ensemble des noeuds enrichis par
la discontinuit\'e et  les coefficients $\bfa_i$ sont les
degr\'es de libert\'e (vectoriels) correspondants.
Un noeud appartient \'e $L$ si son support est coup\'e par
la fissure mais ne contient aucune de ses pointes (ces
noeuds sont encercl\'es sur la figure~\ref{fig:both} dans
le cas d'un maillage uniforme et non uniforme) ;

\item $K_1\subset I$ et $K_2\subset I$ sont les ensembles
des noeuds \'e enrichir pour mod\'eliser les fonds de fissures
1 et 2, respectivement.
Les degr\'es de libert\'e correspondants sont
$\bfb_{i,1}^{l}$ et $\bfb_{i,2}^{l}$, $l=1,\ldots,4$.
Un noeud appartient \'e $K_1$
($K_2$)  si son support contient la premi\'ere (seconde)
pointe de fissure. Ces noeuds sont entour\'es d'un carr\'e
figure~\ref{fig:both}.
\end{itemize}

\twofigures{5cm}{uniform}{5cm}{general}{Fissure plac\'ee sur un maillage uniforme (a)
et non uniforme (b). Les noeuds encercl\'es sont enrichis par la fonction $H(\bfx)$
et les noeuds entour\'es d'un carr\'e sont enrichis par les modes asymptotiques \index{Asympotique!modes asymptotiques} en fond
de fissure.}{fig:both}

Les fonctions $F^l_1(\bfx), l=1,\ldots,4$ mod\'elisant le fond de fissure
sont donn\'ees en \'elasticit\'e par~:
\begin{eqnarray}
\label{enrich_fun}
\{F_1^l(\bfx)\} & \equiv \left\{ \sqrt{r}
\mbox{sin}(\frac{\theta}{2}), \sqrt{r} \mbox{cos}(\frac{\theta}{2}),
\sqrt{r}\mbox{sin}(\frac{\theta}{2})\mbox{sin}(\theta),
\sqrt{r}\mbox{cos}(\frac{\theta}{2})\mbox{sin}(\theta) \right\}
\end{eqnarray}
o\'e $(r,\theta)$ sont les coordonn\'ees polaires
dans les axes locaux en fond de fissure
(``tip'' 1 sur la  figure~\ref{fig:local}).
\onefigure{6cm}{local}{Axes locaux pour les coordonn\'ees polaires en pointe de fissure.}

Les figures \ref{fig:f1-1}, \ref{fig:f1-2}, \ref{fig:f1-3} et \ref{fig:f1-4} illustrent respectivement les fonctions
 $\{F_1^1(\bfx)\}$, $\{F_1^2(\bfx)\}$ , $\{F_1^3(\bfx)\}$ et  $\{F_1^4(\bfx)\}$.

\fourfigures{5cm}{f1-1}{5cm}{f1-2}{5cm}{f1-3}{5cm}{f1-4}{visualisation des fonctions $\{F_1^1(\bfx)\}$ (a) ; $\{F_1^2(\bfx)\}$ (b)~; $\{F_1^3(\bfx)\}$ (c)~; $\{F_1^4(\bfx)\}$ (d)~;}

Il est \'e noter que la deuxi\'eme fonction est discontinue sur la fissure ($cos(\frac{2 8pi}{2} = - cos(0) $).
De mani\'ere similaire les fonctions $F^l_2(\bfx)$ sont \'egalement
donn\'ees par~(\ref{enrich_fun}); le rep\'ere de coordonn\'ees
polaires \'etant maintenant d\'efini pour la seconde pointe de fissure.









La fonction $H(\bfx)$ est discontinue sur la fissure
et de valeur constante de part et d'autre de celle-ci~:
+1.0 d'un c\'et\'e et -1.0 de l'autre.
La d\'etermination des fonctions d'enrichissement F et H en
un point quelconque s'exprime ais\'ement si l'on repr\'esente
la fissure par level sets. Ceci sera expos\'e plus loin dans
cette section.

\subsection{Extension au 3D, plaques fissur\'ees et fissures \'e branches\index{Fissure!branches}}
L'extension au cas tridimensionnel de la mod\'elisation de fissures
par X-FEM a \'et\'e r\'ealis\'ee par~\cite{Sukumar:3D}. Cette extension est
assez directe. Tout comme dans le cas bi-dimensionnel, le fait qu'un
noeud soit enrichi ou non et le type d'enrichissement d\'epend de la
position relative du support associ\'e au noeud par rapport au front
de la fissure et \'e son int\'erieur. Le support d'un noeud est un
volume, le front de fissure une courbe (ou plusieurs courbes
disjointes) et l'int\'erieur de la fissure est une surface.

L'extension aux plaques pour le mod\'ele de
Reissner-Mindlin a \'et\'e r\'ealis\'ee dans
~\cite{Dolbow:melosh} et \cite{Dolbow:plate2}.
La formulation \'el\'ements finis de d\'epart
est celle du MITC4 d\'evelopp\'ee par
Bathe et al. (1990)\nocite{Bathe:MITC}
qui est connue pour ses bonnes propri\'et\'es
vis-\'e-vis du bloquage en cisaillement.
Les champs de rotation et de d\'eplacement transversal
sont enrichis pour mod\'eliser
la fissure. L'enrichissement en fond de fissure
est bas\'e sur les champs asymptotiques
obtenus par Knowles et Wang (1960). \nocite{Knowles:plate}

Lorsque plus de deux segments aboutissent
en un point de la fissure,
(fissures en Y ou en croix),
un soin particulier doit \^etre apport\'e \'e
l'enrichissement. Cette extension a \'et\'e r\'ealis\'ee
dans \cite{Daux:holes}.
En  un point o\`u plus de deux branches
aboutissent, l'enrichissement fait intervenir une fonction
dite de jonction.
Dans le cas o\`u le maillage se conforme \'e la fissure
\'e branches, l'enrichissement conduit \'e la m\^eme mod\'elisation
que la mod\'elisation classique par doubles noeuds
(et triples au noeud A).
Les diff\'erents
types d'enrichissement pour une fissure en Y sont illustr\'es
figure~\ref{fig:enrich_mesh2}.

       \onefigure{7cm}{enrich_mesh2}{Strat\'egie pour la mod\'elisation\
       d'une fissure en Y. Les diff\'erents types d'enrichissement sont
       indiqu\'es.}


Dans le cadre de X-FEM,
la mod\'elisation de la cin\'ematique
de bandes de cisaillement est un
cas particulier de la mod\'elisation de fissures~:
une bande de cisaillement est une fissure sans fond
\'e discontinuit\'e purement tangentielle.
Tout noeud dont le support est coup\'e par la bande
est enrichi \'e l'aide d'une fonction vectorielle
parall\'ele \'e la bande et dont la direction s'inverse
au passage de la bande~\cite{Belytschko:circle}.
La mod\'elisation de discontinuit\'e de type
rotation (cr\'eation du mode rigide
de rotation pour un disque trac\'e sur le maillage)
est \'egalement propos\'ee dans~\cite{Belytschko:circle}.



\subsection{Mod\'elisation de trous}
\label{sec:trou_un}
La mod\'elisation classique
de trous par la m\'ethode des \'el\'ements finis
impose au maillage de se conformer \'e la fronti\'ere
de ces trous. La m\'ethode X-FEM permet de se d\'ebarrasser
de cette contrainte.
Pour un noeud dont le support coupe la fronti\'ere
du trou, la fonction d'approximation classique
est multipli\'ee par une fonction valant 1.0 dans
la mati\'ere et 0.0 dans les trous.
Un noeud dont le support est compl\'etement
\'e l'int\'erieur du trou ne donne pas
lieu \'e la cr\'eation de degr\'es de libert\'e.
Il est \'e noter qu'un noeud peut \^etre actif m\^eme si il
est situ\'e dans un trou. Ce qui importe est le fait que
son support couvre de la mati\'ere ou non.
La figure~\ref{fig:mesh_holes}
illustre la s\'election des noeuds
\'e \'eliminer et les noeuds pour lesquels
les fonctions d'approximation  doivent
\^etre modifi\'ees dans le cas de deux trous
plac\'es sur un maillage.

    \onefigure{7cm}{mesh_holes}{S\'election des noeuds pour le traitement des trous dans X-FEM.}


Au niveau de l'impl\'ementation, la m\'ethode
propos\'ee ci-dessus revient pour les
\'el\'ements coup\'es par le trous \'e
restreindre l'int\'egration
\'e la ``fraction mati\'ere de l'\'el\'ement''\cite{Daux:holes}.


\section{Localisation et \'evolution des surfaces par level sets\index{level sets}}
\label{sec:local-et-evol}
Dans la section pr\'ec\'edente, les fronti\'eres des trous\index{Trous}
et des fissures \'etaient d\'ecrites par des entit\'es g\'eom\'etriques
(segments de droite).
Cette description peut-\^etre qualifi\'ee d'explicite
(ou ``lagrangienne'') puisque
la fronti\'ere est d\'ecrite par l'ensemble des points appartenant
\'e cette fronti\'ere.
Une autre description, implicite (``eul\'erienne''), est possible.

\subsection{Description d'une surface par level set\index{Surface!level set}}

Cette description consiste \'e transformer la repr\'esentation explicite
de la fronti\'ere en un champ de valeur d\'efinie dans un voisinage de
cette fronti\'ere. La vision eul\'erienne d'une surface et les
algorithmes de propagation qui en d\'ecoulent sont \'e la base de la
m\'ethode des level sets dont les pionniers sont Sethian et
Osher~\cite{osher-sethian,Sethian:book}. Le champ dont nous parlons
est appel\'e ``level set'' que nous traduirons par fonction de niveau.
La fonction de niveau associ\'ee \'e une fronti\'ere donne en chaque noeud
du maillage la distance de ce point \'e la fronti\'ere. Le signe plac\'e
devant la distance indique si le point se trouve d'un c\'et\'e ou de
l'autre de la fronti\'ere. Pour un trou, la fonction de niveau est
choisie arbitrairement n\'egative \'e l'int\'erieur du trou et positive \'e
l'ext\'erieur du trou avec une valeur donnant la plus courte distance
\'e la fronti\'ere du trou. L'iso-z\'ero de la fonction de niveau donne la
position du bord du trou. En pratique, la fonction de niveau est
calcul\'ee aux noeuds et interpol\'ee entre les noeuds par les fonctions
de base \'el\'ements finis classiques. Pour des \'el\'ements triangulaires,
l'iso z\'ero est donc un segment de droite et pour un t\'etra\'edre un
morceau de plan. Une fois cette fonction de niveau cr\'e\'ee, elle
permet de g\'erer les op\'erations g\'eom\'etriques n\'ecessaires \'e la
mod\'elisation de trous par X-FEM, \'e savoir~: trouver les \'el\'ements
coup\'es par le bord d'un trou, \'eliminer les noeuds dont le support
est totalement \'e l'int\'erieur d'un trou et modifier la fonction
d'approximation associ\'ee aux noeuds dont le support est coup\'e par le
bord d'un trou.

Une fissure peut \'egalement \^etre rep\'er\'ee par level sets.
Alors que pour un trou une seule fonction suffit, il en faut deux
pour une fissure. En effet, une fissure ne s\'epare pas un domaine
en deux zones distinctes.
La premi\'ere level set, baptis\'ee $ls_\mathrm{n}$,
donne la position de la surface sur laquelle est pos\'ee la fissure.
La seconde, baptis\'ee $ls_\mathrm{t}$, permet de localiser le front
de la fissure sur cette surface. La figure~\ref{fig:lsn_lst}
illustre ces deux level sets pour la
localisation d'une fissure en
2D et  en 3D.


    \twofigures{7cm}{lset1_color}{7cm}{lset3d}{Localisation d'un fissure par deux
    level sets en 2D (a) et 3D (b).}{fig:lsn_lst}


Les level sets $ls_\mathrm{n}$ et $ls_\mathrm{t}$ donnent respectivement les distances normale et
tangentielle \'e la fissure.
En un point proche du front de fissure, le calcul des fonctions
d'enrichissement~(\ref{enrich_fun}) n\'ecessite le calcul des param\'etres
$r$ et $\theta$ dont la signification g\'eom\'etrique est donn\'ee sur la
figure~\ref{fig:local_axis}.
Ces deux param\'etres ont une expression simple en terme des deux
level sets~:
\begin{equation}
  \label{eq:rtheta}
  r = \sqrt{ls_\mathrm{n}^2 + ls_\mathrm{t}^2}, \quad \theta = \tan^{-1}(\frac{ls_\mathrm{n}}{ls_\mathrm{t}})
\end{equation}



      \onefigure{6cm}{local_axis}{Signification des param\'etres $r$ et $\theta$ en front de fissure.}

\subsection{Evolution d'une level set sous un champ de vitesse \index{Level sets!\'evolution}}

Nous consid\'erons ici une fonction level set, $\phi$,
soumise \'e un champ de
vitesse, $\bfV$ (pour fixer les id\'ees, une cavit\'e qui croit dans un
mat\'eriau).
La mise \'e jour de la level set pour conna\'etre la nouvelle
position de la cavit\'e apr\'es un pas de temps $\Delta t$ se d\'ecompose en
trois \'etapes \cite{Sethian:book}~:
\begin{itemize}
\item \underline{Phase d'extension de la vitesse~:} la vitesse n'est en
    g\'en\'eral connue que sur la surface physique consid\'er\'ee
    (cavit\'e par exemple). Or, pour mettre \'e jour
    la level set (voir \'equation~(\ref{eq:1})), il
    faut disposer de cette vitesse partout. La premi\'ere phase
    consiste donc en une ``extension'' du champ des vitesses
    depuis la surface vers tout le domaine.
    Dans cette phase, on cherche \'e r\'esoudre  l'\'equation~:
    \begin{equation}
    \nabla\phi.\nabla V_{\phi}=0
    \label{eq:3}
    \end{equation}
    qui impose \'e la vitesse normale de s'\'etendre de mani\'ere
    constante selon les lignes de plus grandes pentes.

\item \underline{Phase de propagation~:\index{Propagation}} elle permet de
    calculer la nouvelle level set. L'\'evolution de la level set
    est donn\'ee par l'\'equation
    \begin{equation}
        \frac{\partial\phi}{\partial t}+\bfV \cdot  \nabla\phi  = 0
        \label{eq:1}
    \end{equation}
    Cette \'equation est une \'equation de transport sous le champ
    de vitesse $\bfV$.
    Comme le gradient de la level set repr\'esente la normale au iso-surface,
    il est aussi courant d'\'ecrire~(\ref{eq:1}) sous la forme
    \begin{equation}
        \frac{\partial\phi}{\partial t}+V_{\phi}\left|\nabla\phi\right|=0
        \label{eq:2}
    \end{equation}
    o\`u $V_{\phi}$ est la vitesse normale de la surface.
    L'\'equation~(\ref{eq:2}) est discr\'etis\'ee sur un pas
    de temps pour donner la nouvelle valeur
    de la level set \'e l'issue de ce pas de temps.
\item \underline{Phase de r\'e-initialisation~:} la nouvelle valeur de la
    level set
    n'est plus n\'ecessairement rigoureusement
    une distance sign\'ee apr\'es propagation.
    Il est alors bon de la
    r\'e\-initia\-liser en r\'esolvant l'\'equation~(\ref{eq:4})~:
    \begin{equation}
        \left|\nabla\phi\right|=1
        \label{eq:4}
    \end{equation}
\end{itemize}

\ajouter{peut-\'etre faut-il d\'etailler plus d'o\'e viennent ces \'equations...pour moi ce n'est pas clair.De m\'eme,
 l'\'equivalence avec \ref{eq:hamilton38} et \ref{eq:hamilton39} ne saute pas aux yeux  }


Les \'equations (\ref{eq:3}) et (\ref{eq:4}) se pr\'etent mal
\'e une r\'esolution directe, on leur pr\'ef\'ere la recherche
de la solution stationnaire des probl\'emes suivants~:
\begin{equation}
    \frac{\partial
    V_{\phi}}{\partial\tau}+\textrm{sign}(\phi)\frac{\nabla\phi}{\left|\nabla\phi\right|}.\nabla V_{\phi}=0
    \label{eq:hamilton38}
\end{equation}
\begin{equation}
    \frac{\partial\phi}{\partial\tau}+\textrm{sign}\left(\phi\right)\left(\left|\nabla\phi\right|-1\right)=0
    \label{eq:hamilton39}
\end{equation}
Il ne faut pas confondre le param\'etre $\tau$ intervenant dans ces
deux \'equations (temps virtuel vers
une solution stationnaire) et le temps physique $t$ intervenant dans
(\ref{eq:2}).

On constate que les \'equations \'e r\'esoudre pour
la propagation d'une level set sont toutes
de la m\'eme forme (dite de Hamilton-Jacobi)~:
\begin{equation}
\frac{\partial\ldots}{\partial t^*}+F \left|\nabla\ldots\right|=f
\end{equation}
o\'e $t^*$ est le temps r\'eel ou un temps virtuel.
La signification des op\'erateurs $F$ et $f$ est donn\'ee dans le
tableau ci-dessous.
\begin{table}[htbp]
\begin{center}\begin{tabular}{|c|p{6cm}|c|c|c|}
\hline
&
Phase&
$\ldots$&
$F$&
$f$\tabularnewline
\hline
$(1)$&
Extension des vitesses&
$V_{\phi}$&
$\textrm{sign}(\phi)\frac{\nabla\phi}{\left|\nabla\phi\right|}\cdot\frac{\nabla V_{\phi}}{\left|\nabla V_{\phi}\right|}$&
$0$\tabularnewline
\hline
$(2)$&
Propagation des level sets &
$\phi$&
$V_{\phi}$&
$0$\tabularnewline
\hline
$(3)$&
R\'einitialisation&
$\phi$&
$\textrm{sign}(\phi)$&
$\textrm{sign}(\phi)$ \tabularnewline
\hline
\end{tabular}\end{center}
\end{table}

La r\'esolution de cette \'equation de Hamilton-Jacobi \index{Hamilton-Jacobi}
sur un maillage \'el\'ements finis quelconque peut se faire
par exemple \'e l'aide d'une m\'ethode de Runge et Kutta du second ordre en temps
et avec une formulation explicite de mise \'e jour des
noeuds \cite{barth-sethian,OsherFedkiw02}.
Il est \'e noter que la fonction \'e propager
est tr\'es  r\'eguli\'ere
(contrairement \'e une approche de type Volume Of Fluid).
Enfin, une condition de stabilit\'e classique (CFL)
p\'ese sur le pas de temps
dont la valeur d\'epend de la taille des \'el\'ements et de
l'op\'erateur $F$.

Les fissures \'etant repr\'esent\'ees par deux level sets, ces derni\'eres
doivent \^etre propag\'ees. L'ordre des op\'erations doit faire l'objet
d'un soin particulier \cite{Moes3DGrowthII}. Outre la
r\'e-initialisation, une r\'e-orthogonalisation est effectu\'ee pour
s'assurer que les gradients des deux level sets restent orthogonaux.




\subsection{Mod\'elisation de discontinuit\'es dans la d\'eriv\'ee du champ}

Les surfaces de discontinuit\'e
dans la d\'eriv\'ee du champ sont importantes  pour mod\'eliser les interfaces entre mat\'eriaux.


Consid\'erons un domaine $\Ome$ compos\'e
de deux mat\'eriaux occupant les r\'egions
$\Ome_1$ et $\Ome_2$, figure~\ref{fig:1dbar}.
Pour mod\'eliser l'interface
mat\'eriau situ\'ee sur
$\Gam = \overline{\Ome_1} \cap  \overline{\Ome_2}$,
il faut que le champ d'approximation en d\'eplacement
contienne des modes discontinus de d\'eformation
au passage de l'interface, cette discontinuit\'e
\'etant une caract\'eristique importante
de la solution du probl\'eme.



    \onefigure{5cm}{1dbar}{Un probl\'eme bi-mat\'eriaux avec une interface
  non maill\'ee.}


Lorsque le maillage respecte l'interface,
la d\'eformation \'el\'ements finis est discontinue au travers
de l'interface par le fait
de la faible r\'egularit\'e ($C^0$)
de l'approximation \'el\'ements finis.
Lorsque le maillage ne respecte pas l'interface,
la base \'el\'ements finis doit \^etre enrichie \'e l'aide de
fonctions \'e d\'eriv\'ee discontinue sur l'interface.
L'approximation enrichie s'\'ecrit
de mani\'ere g\'en\'erale~:
\begin{equation}
\bfu^h(\bfx)  =  \sum_{i\in I} \bfu_i \phi_i(\bfx) +
\sum_{i\in D} \bfa_i \phi_i(\bfx) F(\bfx)
\end{equation}
o\'e $D$ est l'ensemble des noeuds dont au moins un
des \'el\'ements du support
est coup\'e par l'interface mat\'eriau $\Gam$ et
$F(\bfx)$ est une fonction continue mais
\'e d\'eriv\'ee discontinue sur l'interface.
La fonction de niveau $ls(\bfx)$ associ\'ee
\'e l'interface donne la position de cette
interface (contour sur lequel $ls = 0$) et
sa valeur absolue repr\'esente en tout point de $\Ome$
la distance \'e l'interface.
La fonction $|ls(\bfx)|$ est donc continue
mais \'e d\'eriv\'ee discontinue sur l'interface.
C'est un choix possible pour la fonction d'enrichissement
$F$ \cite{Sukumar:inclusion}.


       \onefigure{9cm}{fct_enrich_bw}{Diff\'erents choix pour la fonction
  d'enrichissement associ\'ee \'e une interface entre deux mat\'eriaux.}



Afin de tester ce choix,
prenons le probl\'eme d'un carr\'e compos\'e de deux mat\'eriaux,
figure~\ref{fig:1dbar}. La fonction
$F^1 = |ls(\bfx)|$ est repr\'esent\'ee
figure~\ref{fig:fct_enrich_bw} pour une coupe selon $y=0$.
Les coefficients de Poisson
des deux mat\'eriaux sont pris \'egaux \'e z\'ero pour
rendre le probl\'eme unidimensionnel selon $x$ mais
les modules de Young sont diff\'erents~:
$E_1 = 1$ et $E_2 = 10$. Le bord gauche du carr\'e
est fix\'e alors que le bord droit est soumis \'e un
d\'eplacement de valeur 1.0 selon $x$.
La solution en d\'eplacement de ce probl\'eme est lin\'eaire
de part et d'autre de l'interface et \'e
d\'eriv\'ee normale discontinue
sur l'interface.



Si le maillage se conforme \'e l'interface, la solution \'el\'ements finis
classique co\'encide avec la solution exacte. Si par contre, le
maillage ne se conforme pas \'e l'interface, l'erreur est diff\'erente
de z\'ero. L'erreur relative en \'energie pour le calcul \'el\'ements finis
classique est donn\'ee dans la seconde colonne du tableau~\ref{tab:1D}
pour un maillage de $10 \times 10$ \'el\'ements en fonction du d\'efaut
d'alignement, $\delta$, du maillage par rapport \'e l'interface. Afin
de r\'eduire cette erreur, enrichissons les noeuds dont le support
coupe l'interface \'e l'aide de la fonction $F^1$. Les erreurs sont
donn\'ees dans la troisi\'eme colonne du tableau~\ref{tab:1D}. Notons
que lorsque $\delta=0$ aucun noeud n'est enrichi car cela conduirait
\'e une matrice de raideur singuli\'ere; la fonction $F^1$ \'etant d\'ej\'e
contenue dans la base \'el\'ements finis classique. Les erreurs obtenues
avec X-FEM sont plus faibles que pour le calcul \'el\'ements finis
classique mais encore \'elev\'ees en regard de la simplicit\'e de la
solution exacte.


Consid\'erons une autre fonction
d'enrichissement\index{Enrichissement} $F^1(\bfx)$ r\'egularis\'ee
repr\'esent\'ee figure~\ref{fig:fct_enrich_bw}. Cette fonction co\'encide
avec $|ls(\bfx)|$ sur les \'el\'ements coup\'es par l'interface et est
constante en dehors de ces \'el\'ements. L'erreur \index{Erreur}
obtenue, colonne~4 du tableau~\ref{tab:1D}, est maintenant z\'ero
num\'eriquement. On peut d'ailleurs montrer que cet enrichissement
permet \'e l'approximation de repr\'esenter parfaitement la solution
exacte.

\begin{table}[htb]
\begin{center}

\begin{tabular}{c|c|cc} \hline
%& & \multicolumn{2}{c}{} \\
      & FEM &\multicolumn{2}{c}{X-FEM} \\
$\delta$ & & $F^1$ & $F^1$ + r\'egularisation \\ \hline
%% & & \\ \hline
0.00 & $3.0 \times 10^{-8}$  & $3.0 \times 10^{-8}$   &  $3.0 \times 10^{-8}$ \\ \hline
0.01 & $2.6 \times 10^{-2}$ &  $8.3 \times 10^{-2}$ & $3.0 \times 10^{-8}$  \\
0.05 & $4.9 \times 10^{-2}$ &  $1.6 \times 10^{-1}$ & $2.8 \times 10^{-8}$  \\
0.10 & $5.6 \times 10^{-2}$ &  $1.8 \times 10^{-1}$ & $2.1 \times 10^{-8}$  \\
0.15 & $5.1 \times 10^{-2}$   &  $1.8 \times 10^{-1}$ & $3.8 \times 10^{-8}$  \\
0.19 & $2.2 \times 10^{-2}$   &  $1.6 \times 10^{-1}$ & $3.6 \times 10^{-8}$  \\ \hline
\end{tabular}
\vspace*{0.1in}
\caption{Erreur relative en \'energie en fonction de la position
de l'interface (probl\'eme figure~\ref{fig:fct_enrich_bw})
pour le calcul \'el\'ements finis classique (FEM) et le calcul par
X-FEM pour deux types de fonction d'enrichissement.}
\label{tab:1D}
\end{center}
\end{table}


Consid\'erons maintenant un exemple o\`u l'interface mat\'eriau n'est pas
rectiligne. La figure~\ref{fig:bimaterial} montre le probl\'eme d'une
inclusion \'elastique circulaire. Le chargement en d\'eplacement est
normal \'e la surface ext\'erieure de rayon $b$
\cite{Sukumar:inclusion}. Dans le mod\'ele num\'erique, on consid\'ere un
carr\'e ($L \times L$, $L = 2$) avec une inclusion circulaire en son
centre de rayon $a=0.4$. Sur le bord de ce carr\'e, les tractions
exactes correspondant au probl\'eme figure~\ref{fig:bimaterial} (avec
$a = 0.4$ et $b = 2.0$) sont impos\'ees.


\onefigure{6cm}{bimaterial}{Le probl\'eme d'une inclusion circulaire \'elastique.}



Des maillages quasi-uniformes d'\'el\'ements
triangulaires \'e trois noeuds et de plus en plus raffin\'es sont
utilis\'es pour \'etudier la convergence de l'erreur
en \'energie. La figure \ref{fig:graph2Dinclusion} donne la
convergence \index{Convergence} en \'energie dans diff\'erents cas de figure.
La courbe ``FEM'' correspond \'e une approche \'el\'ements finis
classique dans laquelle le maillage respecte l'interface.
La convergence est d'ordre 1 conform\'ement \'e la pr\'evision
th\'eorique. La courbe ``FEM non conforming'' correspond \'e un
calcul \'el\'ements finis standard avec un maillage ne respectant
pas l'interface. La convergence est tr\'es m\'ediocre et explique la
n\'ecessit\'e dans l'approche \'el\'ements finis de mailler les
interfaces
mat\'eriaux.
La courbe ``XFEM 1 + smoothing'' correspond \'e l'enrichissement
avec $F^1$ r\'egularis\'e \index{Enrichissement!r\'egularis\'e}.
La fonction  $F^1$ r\'egularis\'ee
est \'egale \'e  la valeur absolue de la level set sur les
\'el\'ements coup\'es par l'interface mat\'eriau et est
``la plus constante possible'' en dehors de ces
\'el\'ements \cite{Sukumar:inclusion}.
Enfin, la courbe ``XFEM 2'' donne la convergence pour un nouvel
enrichissement propos\'e dans \cite{MoesCloirec02}.
La fonction d'enrichissement s'\'ecrit
\begin{equation}
F^2(\bfx) = \sum_i  |ls_i| N_i(\bfx)  - | \sum_i ls_i N_i(\bfx) |
\label{eq:5}
\end{equation}
Il s'agit de la diff\'erence entre l'interpol\'ee des valeurs
absolues nodales de la level set et la valeur absolue de la level
set.
Ce dernier enrichissement donne une convergence qui semble optimale.
Cette fonction d'enrichissement n'est non nulle que sur les
\'el\'ements coup\'es par l'interface. Elle est illustr\'ee sur la
figure~\ref{fig:fct_enrich_bw} (fonction $F^2$).



 \onefigure{13cm}{graph2Dinclusion_3}{Convergence de l'erreur
pour une inclusion circulaire. La valeur de $\alpha$ indique le taux de
convergence en fonction de la taille moyenne des \'el\'ements
(calcul\'ee par $h=\sqrt{2 A/N}$ o\`u $A$ est l'aire du domaine et N le
nombre d'\'el\'ements).}



La m\'eme \'etude de convergence a \'et\'e r\'ealis\'ee en 3D pour une
inclusion sph\'erique.
Les convergences observ\'ees sont donn\'ees sur la
figure~\ref{fig:graph3Dinclusion}. On note \'e nouveau que
l'enrichissement (\ref{eq:5}) semble donner une convergence optimale.


\onefigure{13cm}{graph3Dinclusion_4}{Convergence de l'erreur
pour une inclusion sph\'erique. La valeur de $\alpha$ indique le taux de
convergence en fonction de la taille moyenne des \'el\'ements
(calcul\'ee par $h=(6 V/N)^{1/3}$ o\`u $V$ est le volume du domaine
 du domaine et N le nombre d'\'el\'ements).}

    \chapter{Quelques exemples d'application}\label{Applications_de_XFEM}
        

\section{Applications et Mise en oeuvre}
\label{sec:applications-et-mise}

L'impl\'ementation de l'approche X-FEM dans un code \'el\'ements
finis standard requiert un certain nombre d'extensions
du code cible. Ces extensions concernent au minimum les points
suivants :
\begin{itemize}
\item La gestion des level sets ;
\item L'int\'egration des matrices et des vecteurs \'el\'ementaires ;
\item La s\'election des degr\'es de libert\'e \`a enrichir ;
\item Le stockage de ces degr\'es de libert\'e ;
\item Le post-traitement.
\end{itemize}
Chacun de ces points est explicit\'e ci-dessous.
Bien entendu, le travail sp\'ecifique \`a r\'ealiser dans un code
d\'epend de son architecture et il est d\'elicat de tirer
des r\`egles g\'en\'erales.



Dans l'approche X-FEM, certaines surfaces physiques
(voire toutes) vont \^etre localis\'ees sur le maillage \`a l'aide
d'un champ \'el\'ements finis. Le code doit donc \^etre capable de
g\'erer \`a la fois des champs physiques (vitesse, pression, ...)
et des champs dont l'iso-z\'ero localise une surface
physique. La possibilit\'e d'extraire ces iso-z\'eros est
importante
si l'on veut \^etre capable d'imposer par exemple
une pression dans un trou ou
sur les l\`evres d'une fissure (dans une canalisation par exemple).

La level set permet ensuite par son changement de signe de d\'efinir
les supports \index{Support} qui doivent \^etre enrichis. Le nombre et
la nature des degr\'es de libert\'e en un noeud est variable et d\'epend
de la position relative du support de ce noeud par rapport \`a la
fissure (ou l'interface mat\'eriau). Il faut \^etre capable de g\'erer
sur le plan informatique cette diversit\'e dans les degr\'es de libert\'e.
Un \'el\'ement peut n'avoir que quelques noeuds enrichis. Le type
d'enrichissement peut aussi \'evoluer dans le temps (propagation d'une
fissure par exemple).

L'int\'egration des matrices de raideur et des forces ext\'erieures
fait intervenir des fonctions discontinues voire singuli\`eres sur les
\'el\'ements coup\'es par la fissure. De m\^eme, si un \'el\'ement
est coup\'e par le bord d'un trou, il faut limiter l'int\'egration \`a
la partie mati\`ere de l'\'el\'ement. Un soin particulier est donc \`a
apporter dans l'int\'egration des matrices de raideur \'el\'ementaires.
Notre exp\'erience sur le sujet nous a amen\'e \`a suivre l'approche
suivante. Dans un premier temps, il faut d\'ecouper l'\'el\'ement en
sous-domaines d'int\'egration sur lesquels les fonctions \`a int\'egrer
sont continues. Ensuite, sur chaque sous-domaine une r\`egle
appropri\'ee d'int\'egration doit \^etre choisie. Pour la gestion des
trous, interface mat\'eriaux ou partie Heaviside de la mod\'elisation
d'une fissure, une int\'egration par points de Gauss classique
convient. Pour les fonctions singuli\`eres de front de fissure,
l'int\'egration de Gauss converge tr\`es mal. Il faut privil\'egier une
int\'egration ad hoc \cite{Bechet05}.



Enfin, au niveau du post-traitement, il faut \^etre capable
de visualiser des champs discontinus sur les \'el\'ements pour
effectivement voir les \'el\'ements coup\'es par une fissure se
scinder.



\subsection{Application en m\'ecanique de la rupture \index{Rupture}}

Cinq exemples sont trait\'es dans ce chapitre. Les trois premiers
concernent le calcul des facteurs d'intensit\'e de contrainte pour
diff\'erentes configurations: une fissure inclin\'ee sous tension dans
une membrane, une fissure elliptique inclin\'ee dans un cube sous
tension et enfin une fissure en forme de Y (fissure \`a branches). Le
quatri\`eme exemple concerne la propagation d'une fissure en forme de
lentille de contact et le dernier la mod\'elisation de bandes de
cisaillement dans un massif perc\'e d'un tunnel.

\subsubsection{Le probl\`eme de Griffith: une fissure inclin\'ee}
Les facteurs d'intensit\'e de contrainte d'une fissure dans une plaque
infinie sous tension sont donn\'es par la solution de Griffith :
\begin{subequations}
\begin{align}
K_I &= \sigma\sqrt{\pi a}~\mbox{cos}^2(\beta) \\
K_{II} &= \sigma\sqrt{\pi a}~\mbox{sin}(\beta)~\mbox{cos}(\beta)
\end{align}
\end{subequations}
o\`u $a$ et la demi-longueur de la fissure, $\sigma$ l'intensit\'e du
chargement et $\beta$ l'angle entre le chargement et la fissure. Ce
probl\`eme est mod\'elis\'e  figure~\ref{fig:plate2} dans une plaque
carr\'ee suffisamment grande par rapport \`a la taille de la fissure.
Le maillage utilis\'e est uniforme et contient $40\times 40$
\'el\'ements. Les facteurs d'intensit\'e de contrainte obtenus avec
X-FEM pour diff\'erents angles $\beta$ sont compar\'es avec les
facteurs de r\'ef\'erence
figure~\ref{fig:rotk1k2}~\cite{Moes:discont}. L'accord est
excellent. Notons que le m\^eme maillage ($40\times 40$
\'el\'ements) est utilis\'e pour \emph{tous} les angles $\beta$. Seul
la position relative de la fissure par rapport au maillage change.



\twofigures{5cm}{plate2}{9cm}{rotk1k2}{Le probl\`eme de la plaque
fissur\'ee sous
       tension: $W = 10 in.$, $a = 0.5 in.$ ; (a), Comparatif entre les facteurs
       d'intensit\'e de
       contrainte num\'eriques (X-FEM) et de r\'ef\'erence
       en fonction de l'angle de chargement  $\beta$ pour le
       probl\`eme de la plaque fissur\'ee ; (b) }{fig:plaque_fissuree}



\subsubsection{Fissure elliptique inclin\'ee sous tension}

Le second exemple est celui d'une fissure elliptique inclin\'ee sous
tension \cite{Moes3DGrowthI}. La fissure est plane mais la courbure
du front n'est pas constante. La solution exacte de ce probl\`eme
(pour un probl\`eme infini) est donn\'ee dans \cite{KassirSih66}. Dans
le calcul, le cube \`a un cot\'e de taille $h$ et est charg\'e selon l'axe
$Z$, voir figure~\ref{fig:ellipse}. L'axe majeur de l'ellipse est
selon l'axe $X$ et de taille $a$. L'axe mineur de taille $b=a/2$ est
orient\'e selon la bissectrice du quadrant $XY$ (la fissure est donc
inclin\'ee \`a 45 degr\'es du chargement). Les dimensions sont choisies de
telle sorte que l'hypoth\`ese de milieu infini soit applicable
($h/a=10$). Le coefficient de Poisson est 0.3 et le module de Young
\'egal \`a 1. La Figure~\ref{fig:Kellipse} montre une  comparaison entre
les solutions exactes et num\'eriques obtenues avec X-FEM. Le d\'etail
du calcul des facteurs d'intensit\'e de contraintes est donn\'e
dans~\cite{Moes3DGrowthI}. La signification de l'abscisse $\theta$
(en degr\'es) est donn\'ee sur la figure~\ref{fig:angle}. Le nombre de
degr\'es de libert\'e est de 130.000 et la taille caract\'eristique des
\'el\'ements pr\`es du front de la fissure est de l'ordre du dixi\`eme de
l'axe majeur de l'ellipse.


\twofigures{7cm}{ellipse}{6cm}{angle}{Une fissure elliptique inclin\'ee dans
       un cube sous tension ; (a), d\'efinition de la position angulaire le long du
  front de la fissure ; (b) }{fig:fissure_elliptique}

\onefigure{10cm}{Kellipse}{Les facteurs d'intensit\'e de
     contraintes exacts (ligne continue) et calcul\'es (symboles)
     pour la fissure elliptique en fonction de la position
     angulaire sur le front.}



\subsubsection{Une fissure en forme de Y}

Cette exemple, tir\'e de \cite{Daux:holes}, permet de montrer la
possibilit\'e qu'offre X-FEM de g\'erer des fissures \`a branches. On
consid\`ere une fissure \`a branches sym\'etriques dans une plaque de
largeur $2w$ et de hauteur $2H$ soumise \`a une tension de valeur
$\sigma$ perpendiculaire \`a la fissure principale,
figure~\ref{fig:mesh_infinite}. La taille de la plaque est grande
par rapport \`a la taille de la fissure ($w=20$, $H=16$ et $a=1.$)
de mani\`ere \`a pouvoir utiliser la solution de r\'ef\'erence pour
une plaque infinie donn\'ee par Chen et Hasebe (1995).
\nocite{Chen:branching} Notons que dans le calcul, la fissure n'est
pas repr\'esent\'ee \`a l'aide de level sets mais de segments de droite.

Les facteurs d'intensit\'e de contrainte normalis\'es en pointes de
fissures A et B sont d\'efinis par
\begin{equation}
F_{I}^{A} = K_{I}^{A}/\sigma \sqrt{\pi c} \mbox{ }, \quad
F_{I}^{B} = K_{I}^{B}/\sigma \sqrt{\pi c} \mbox{ }, \quad
F_{II}^{B} = K_{II}^{B}/\sigma \sqrt{\pi c} \nonumber
\end{equation}
Les r\'esultats obtenus avec X-FEM pour $F_{I}^{A}$, $F_{I}^{B}$ et
$F_{II}^{B}$ sont compar\'es aux valeurs de r\'ef\'erence pour
diff\'erents rapports $b/a$ et angles $\theta$ dans la
Table~\ref{tab:infinite_br}. Le maillage utilis\'e est donn\'e
figure~\ref{fig:mesh_infinite}.

Dans le cas  $b/a = 1$ et $\theta = \pi/4$, l'influence du
raffinement est montr\'ee dans la Table~\ref{tab:robust}. La taille
moyenne des \'el\'ements en fond de fissure est $h$. On observe que
les r\'esultats num\'eriques convergent vers les r\'esultats de
r\'ef\'erence avec le raffinement du maillage. Les r\'esultats sont
d\'ej\`a excellents pour un rapport $h/a$ de 0.1


\twofigures{6cm}{mesh_infinite}{6cm}{mesh_infinite_zoom}{Le maillage
utilis\'e pour le probl\`eme de la fissure
  \`a branches (1218 noeuds, $h/a$ = 0.12) ; (a), un zoom dans la
  r\'egion de la fissure ; (b).}{fig:mesh_infinite}

\begin{table}[!p]
\begin{center}
\begin{tabular}{cccccccc} \hline
&$\theta$ & \multicolumn{2}{c}{$15^\circ$} & \multicolumn{2}{c}{$45^\circ$} &
\multicolumn{2}{c}{$75^\circ$} \\
\hline
$b/a$ & & X-FEM & $\star$ & X-FEM & $\star$ & X-FEM & $\star$ \\
\hline
&$F_{I}^{A}$  & 1.016 & 1.018 & 1.045 & 1.044 & 1.118 & 1.117 \\
1.0 & $F_{I}^{B}$ & 0.750 & 0.737 & 0.493 &
0.495 & 0.061 & 0.061 \\
& $F_{II}^{B}$ & 0.123 & 0.114 & 0.504 & 0.506 & 0.535 & 0.541 \\
\hline
& $F_{I}^{A}$ & 1.015 & 1.016 & 1.036 & 1.036 & 1.086 & 1.087 \\
0.8 & $F_{I}^{B}$ & 0.736 & 0.735 & 0.494 &
0.495 & 0.057 & 0.056 \\
& $F_{II}^{B}$ & 0.107 & 0.107 & 0.497 & 0.498 & 0.546 & 0.551 \\
\hline
& $F_{I}^{A}$ & 1.012 & 1.011 & 1.022 & 1.023 & 1.035 & 1.037 \\
0.4 & $F_{I}^{B}$ & 0.752 & 0.729 & 0.502 & 0.504 & 0.067 & 0.066 \\
& $F_{II}^{B}$ & 0.096 & 0.078 & 0.459 & 0.460 & 0.542 & 0.542 \\
\hline
\multicolumn{8}{l}{($\star$) Solution de r\'ef\'erence de Chen et Hasebe (1995)}
\end{tabular}
\end{center}
\caption{Facteurs d'intensit\'e de contrainte normalis\'es pour
         diff\'erents rapports $b/a$ et angles $\theta$ pour le
         probl\`eme de la fissure en forme de Y.}
\label{tab:infinite_br}
\end{table}

\begin{table} [htbp]
\begin{center}
\begin{tabular}{ccccccccc} \hline
\mbox{ } $h/a$ \mbox{ }& 0.40 & 0.30 & 0.22 & 0.14 & 0.12 & 0.10 &
0.05 & $\star$ \\
\hline
$F_{I}^{A}$ & 0.963 & 1.009 & 1.027 & 1.042 & 1.045 &
1.045 & 1.044 & 1.044 \\
$F_{I}^{B}$ & 0.460 & 0.468 & 0.498 & 0.494 & 0.493 &
0.495 & 0.496 & 0.495 \\
$F_{II}^{B}$ & 0.458 & 0.464 & 0.501 & 0.505 & 0.504 &
0.507 & 0.508 & 0.506 \\
\hline
\multicolumn{9}{l}{($\star$) Solution de r\'ef\'erence de Chen et Hasebe (1995)}
\end{tabular}
\end{center}
\caption{Facteurs d'intensit\'e de contrainte num\'eriques et de
  r\'ef\'erence  normalis\'es pour diff\'erents rapports $h/a$
  (probl\`eme de la fissure en forme de Y avec $b/a=1.$ et
   $\theta=45^\circ$).}
\label{tab:robust}
\end{table}


\subsubsection{Propagation d'une fissure en forme de lentille}

Dans cette exemple,  on consid\`ere une fissure en forme de lentille
plac\'ee dans un cube soumis \`a de la tension hydrostatique
\cite{Moes3DGrowthII}. La g\'eom\'etrie de la fissure,
figure~\ref{fig:im12}, est caract\'eris\'ee par un rayon $R=0.005$ et un
angle azimuthal $\alpha=45^o$. Le cube a  un cot\'e de 0.05. Le
chargement est de type fatigue et on consid\`ere une loi de
propagation de Paris. La direction de propagation est donn\'ee par la
direction dans laquelle la contrainte circonf\'erentielle
($\sigma_{\theta \theta}$) est maximale.

Le maillage utilis\'e est non structur\'e et compos\'e de 8895 tetra\`edres
  et 1767 noeuds. Ce maillage ne respecte pas la position de
la fissure. La figure~\ref{fig:im13} donne la direction et amplitude
de la vitesse initiale de la fissure (avanc\'ee par nombre de cycle de
chargement). On observe une sym\'etrie de cette distribution.

La figure~\ref{fig:im14} montre la position de la fissure apr\`es 15
pas de propagation. Il faut noter que le front de la fissure est
maintenant compos\'e de quatre fronts ind\'ependants (les quatre
faces du cube ayant \'et\'e coup\'ees). Ce changement de topologie
du front ne demande aucune attention particuli\`ere
dans l'algorithme de propagation
par level sets. On note que la convexit\'e du front a \'egalement
chang\'e.

\onefigure{9cm}{im12}{Forme initiale de la fissure lentille dans
  le cube.}

\twofigures{6cm}{im13}{6cm}{im14}{Distribution initiale de la vitesse de propagation
 sur le front de la fissure en forme de lentille ; (a),
 position de la fissure apr\`es 15 pas de propagation ; (b)}{fig:fissure3D}

\subsubsection{Bande de cisaillement autour d'un tunnel}
Comme dernier exemple de ``m\'ecanique de la rupture'',
consid\'erons le probl\`eme
d'un tunnel creus\'e dans une roche avec joints, figure \ref{fig:tunnel}.
Ce probl\`eme met en
jeu \`a la fois des surfaces de discontinuit\'e et des bords libres (tunnel)
et a \'et\'e trait\'e dans \cite{Belytschko:circle}.
Le maillage utilis\'e, tr\`es simple,
est montr\'e figure \ref{fig:tunnelmesh}. Il ne respecte ni la position
du tunnel, ni la position des joints. L'enrichissement au niveau des
joints ne fait intervenir qu'une discontinuit\'e tangentielle. La figure
\ref{fig:tunnel_disp_x} montre le d\'eplacement horizontal obtenu.
On peut noter la discontinuit\'e de ce champ au passage des joints.
Cet exemple met en \'evidence la possibilit\'e avec X-FEM de
n'enrichir et ne rendre discontinu que la composante tangentielle
du d\'eplacement pr\`es des joints.
Ces bandes de glissement se superposent sans probl\`eme, chacune
ajoutant
son enrichissement propre dans l'approximation.

\twofigures{7.75cm}{tunnel}{5cm}{tunnelmesh}{Repr\'esentation sch\'ematique du \guil{jointed
       rocks problem}; (a),  maillage utilis\'e avec la position
       des joints et du tunnel ; (b)}{fig:tunnel}

\onefigure{11cm}{tunnel_disp_x}{D\'eplacement horizontal discontinu au niveau des joints.}





\subsection{Application \`a la th\'eorie de l'homog\'en\'eisation \index{Homog\'en\'eisation}}
La possibilit\'e de repr\'esenter des interfaces mat\'eriaux
complexes sans devoir les mailler est
particuli\`erement attractive par exemple pour homog\'en\'eiser
des milieux p\'eriodiques.
Peu de simulations de cellules composites
ou al\'eatoires peuvent \^etre r\'ealis\'ees \`a l'heure actuelle
\`a cause de la difficult\'e dans le r\'ealisation du maillage.
Cette difficult\'e est d'autant plus aigu� si l'on souhaite
que le maillage soit p\'eriodique pour imposer facilement
la condition de p\'eriodicit\'e sur le champ micro inconnu.

On traitera deux exemples dans cette section :
un exemple de validation (bi-couche) et
un exemple sur l'homog\'en\'eisation de milieux al\'eatoires
(X-FEM est tr\`es attrayant car un seul maillage peut \^etre
consid\'er\'e pour toutes les configurations).
D'autres exemples peuvent \^etre trouv\'es dans~\cite{MoesCloirec02}
et \cite{Cartraud04}. En particulier, dans ce dernier papier,
un exemple d'homog\'en\'eisation de c�ble est consid\'er\'e.

\subsubsection{Bi-couche p\'eriodique}
Le cas d'un bi-couche est un probl\`eme simple dont
la solution est lin\'eaire par morceau.
L'objectif est, ici, de  v\'erifier
que l'espace X-FEM enrichi est capable de repr\'esenter exactement
la solution m\^eme si l'interface
n'est pas maill\'ee.
La p\'eriode consid\'er\'ee
est de forme cubique. Le maillage est compos\'e de
162 t\'etra\`edres (27 cellules cubiques chacune compos\'ee de
6 t\'etra\`edres). Une coupe du maillage est montr\'ee
figure~\ref{fig:periodic} avec une indication de la nature
des diff\'erents noeuds.
La
d\'eform\'ee correspondant \`a une d\'eformation macroscopique de
cisaillement dans le plan de l'interface est pr\'esent\'ee
figure~\ref{fig:bi_layer_disp}. On
v\'erifie la bonne prise en compte de la discontinuit\'e,
avec une d\'eform\'ee lin\'eaire par morceaux dans l'\'el\'ement fini o� les
deux mat\'eriaux sont pr\'esents. Enfin, on v\'erifie que la solution
num\'erique coincide avec la solution analytique du probl\`eme
donn\'ee dans \cite{Dumontet90}.

\onefigure{12cm}{periodic}{Vue d'une coupe du maillage du bi-couche
    p\'eriodique avec l'indication du type de noeud.}

%
\onefigure{7cm}{bi_layer_disp}{D\'eform\'ee du bi-couche
    p\'eriodique dans un mode de cisaillement. On peut distinguer
    les \'el\'ements finis et les sous-cellules utilis\'ees pour
    l'int\'egration de la matrice de raideur sur les \'el\'ements
    tranch\'es par l'interface.}


\twofigures{6cm}{spheres}{6cm}{spheres-bis}{Position de 32 sph\`eres dans une cellule p\'eriodique pour
  deux ``tirs'' al\'eatoires diff\'erents. Les surfaces visualis\'ees
  correspondent dans chaque cas \`a l'iso-z\'ero de la level set.}{fig:spheress}



\subsubsection{Mat\'eriau \`a inclusions sph\'eriques}
%
Nous traitons \`a pr\'esent un mat\'eriau \'elastique \`a inclusions
sph\'eriques, r\'eparties de fa�on al\'eatoire dans la matrice, cet
exemple est tir\'e de \cite{Michel99}. Nous \'etudions donc plusieurs
cellules, avec dans chacune une r\'epartition al\'eatoire (mais
p\'eriodique) des inclusions.

La m\'ethode X-FEM s'av\`ere ici particuli\`erement int\'eressante,
puisque le m\^eme maillage peut \^etre utilis\'e pour ces diff\'erentes
cellules. Il s'agit d'un maillage r\'egulier, avec
$32 \times 32 \times 32 $ cellules cubiques chacune compos\'ee
de 6 \'el\'ements t\'etra\`edriques. Pour deux configurations de
cellules
diff\'erentes, la figure~\ref{fig:spheress} donne la position
de l'iso-z\'ero de la level set.

Les caract\'eristiques m\'ecaniques de la particule et de la matrice
sont respectivement $E_p = 70$ GPa, $\nu_p = 0.2$ et $E_m = 3$
GPa, $\nu_m = 0.35$, avec un taux d'inclusion de 26,78\%. Pour la
raideur homog\'en\'eis\'ee $a^{hom}_{1111}$, en prenant la valeur
moyenne sur diff\'erentes cellules, on obtient $7,611$ GPa pour 8
particules et 7,711 pour 32 particules, soit des r\'esultats tr\`es
proches de ceux donn\'es dans \cite{Michel99}.
D'autres exemples d'homog\'en\'eisation p\'eriodique sont trait\'es dans
\cite{MoesCloirec02}.
%



\section{Conclusion et autres lectures}
\label{sec:conclusion-et-autres}

Ce document a r\'esum\'e succinctement la technologie X-FEM,
coupl\'ee \`a la puissance de la m\'ethode des level sets, pour
mod\'eliser des surfaces de discontinuit\'e dans un champ
ou sa d\'eriv\'ee (ou dans la mati\`ere : pr\'esence de trous).
Ces surfaces pouvant \^etre fixes ou mobiles.

Deux types d'applications ont \'et\'e privil\'egi\'es : la
m\'ecanique de la rupture et l'homog\'en\'eisation.

Il ne nous a pas \'et\'e permis dans ce document de balayer
tout l'apport que repr\'esente la partition de l'unit\'e
et les level sets pour la mod\'elisation de fissures.
Cet apport  s'\'etend rapidement compte tenu du nombre
grandissant d'\'equipes \`a travers le monde qui
utilisent maintenant ces techniques.
Sans \^etre exhaustif, voici quelques domaines de d\'eveloppement
important actuel en m\'ecanique num\'erique de la rupture :
\begin{itemize}
\item Passage de la m\'ecanique de l'endommagement \`a la
  m\'ecanique de la rupture : la probl\'ematique est ici
la transition de la micro-fissuration \`a une fissure macroscopique
\'etablie~\cite{patzak03,deborst04}.
 L'approche X-FEM permet de faire cette transition
  num\'erique sur un m\^eme maillage. Il est important de noter que
m\^eme si X-FEM offre des avanc\'ees \emph{num\'eriques}, le
  passage en terme de \emph{mod\`ele} reste un probl\`eme d\'elicat.
\item M\'ecanique de la rupture non lin\'eaire :
la fissuration en milieu subissant de grande transformation
(caoutchouc par exemple) a \'et\'e \'etudi\'ee
dans~\cite{dolbow04}
et~\cite{Legrain05}. La
prise en compte du contact sur les l\`evres de la fissure
a \'et\'e consid\'er\'e dans \cite{Dolbow:contact} et la
pr\'esence d'une zone coh\'esive dans~\cite{WellsSluys01}
et \cite{MoesCohesive}.
\item La multi-fissuration et le ph\'enom\`ene de percolation a
\'et\'e consid\`ere en 2D dans~\cite{BudynMulti}.
\item La fissuration dynamique est un domaine important mais difficile
  car le nombre de degr\'es de libert\'e \'evolue dans le temps et
  la stabilit\'e des sch\'emas num\'eriques standard doit \^etre
  r\'eexamin\'ee. Des avanc\'ees importantes sur ce sujet ont
  \'et\'e obtenues par  \cite{BelytschkoDyn03} et \cite{Rethore05}.
\item La possibilit\'e de la rencontre de plusieurs fissures
  coplanaires
a \'et\'e \'etudi\'ee dans \cite{chopp03}.
\end{itemize}

De m\^eme, nous n'avons pu d\'ecrire plus en d\'etail d'autres domaines
d'applications comme l'optimisation de forme \cite{BelytschkoOpti03}
ou le suivi de front de solidification
\cite{ChessaPhase2002,DolbowPhase2002}.


	
\part{Guide du d\'eveloppeur\label{UserManual}}
    \markboth{Guide du d\'eveloppeur}{Guide du d\'eveloppeur}
   % \chapter{La lecture des donn\'ees}
       % 
\section{la lecture des donn\'ees}

\subsection{Pr\'e-processeur}

La g\'en\'eration d'un maillage peut se faire \`a partir de n'importe quel outil \`a condition que l'interpr\'eteur soit disponible ou d\'evelopp\'e par l'utilisateur. 1
Un d\'ebut d'interpr\'eteur pour les fichiers \code{.inp} d'Abaqus existe, mais nous consid\'erons dans cette section que le d\'eveloppeur utilise \code{gmsh} (voir la section~\ref{gmsh}).

\subsection{La classe \code{xData}}\label{xData}\index{xData}

\paragraph*{Motivation :}

Un fichier \code{.geo} ne contient qu'un maillage et la d\'efinition de zone (voir~\ref{gmsh_msh}).  La d\'efinition des conditions de calcul (condiions aux limites, mat\'eriau, ...) se fait par les fichiers  \code{data/main.dat}. Le r�le de la classe \code{xData} est de permettre l'interpr\'etation de ce fichier.


\paragraph*{Impl\'ementation :}

Peu d'attention n'a encore \'et\'e port\'es \`a cette classe et des am\'eliorations sont encore possible pour g\'en\'eraliser son utilisation. 






\begin{verbatim}
/*  
    xfem : C++ Finite Element Library 
    developed under the GNU Lesser General Public License
    See the NOTICE & LICENSE files for conditions.
*/
#include <cstdio>
#include <string>
#include <cassert>
#include "mAOMD.h"
#include "xData.h"
#include "xBoundary.h"
#include "xFiniteElement.h"
#include "xEnv.h"
#include "xMesh.h"

namespace xfem 
{

char xData::keyword_list[][NB_CHAR_MAX] = {
"PROCEDURE", "ASS",
"COUPLE_X", "COUPLE_Y",  
"CRACK", "CRACK_DIS", 
"DISPLACEMENT", "DISPLACEMENT_X", "DISPLACEMENT_Y", "DISPLACEMENT_Z",
"VECTOR_POTENTIAL_X", "VECTOR_POTENTIAL_Y", "VECTOR_POTENTIAL_Z",
"VELOCITY_X", "VELOCITY_Y", "VELOCITY_Z", "VELOCITY",
"ACCELERATION_X", "ACCELERATION_Y", "ACCELERATION_Z",
"DISPLACEMENT_T","DO_EIGEN_ANALYSIS",
"DO_ENRICH_CHANNELS", "DO_ENRICH_CHANNEL_PRESSURE",
"DO_ENRICH_MANUAL", "DO_ENRICH_AUTOMATIC", "DO_POSTPRO_MANUAL", "DO_POSTPRO_AUTOMATIC",
"DOF_GROUP_SCA_1",
"ELASTIC_SUPPORT", "ENDIF",
"FIX", "FORMULATION_ONE", "FORMULATION_TWO",
"GEOM_FILE", "GEOM_TYPE", "INTEGRATION_TYPE", 
"BC_VOLUME", 
"BC_SURFACE", 
"BC_LINE", 
"BC_POINT", 
"GROUP_ON_CRACK",
"HOLE",
"PARTICLE","PARTICLE_VEL",
"PARTICLE_PRESS",
"PARTICLE_WALL_GAP_VEL","PARTICLE_WALL_GAP_PRESS", 
"IF",
"LINE",
"MATERIAL_PAIR", 
"MAT_CLASS", "MAT_PARAM", "MESH_FILE_TYPE", "MESH_FILE", 
"NEAR_TIP", "NEAR_REG_TIP", 
"NEAR_JUNCTION", 
"FORMULATION_PARAM_FILE", 
"PROCEDURE_PARAM_FILE", 
"PRESSURE", "POTENTIAL", "POTENTIAL_WEAK", "NORMAL_VELOCITY",
"PARTITION_FILE", "POINT", 
"RESULT_FILE", "ROTATION_X", "ROTATION_Y",
"SOLUTION_ONE_FILE",  "SOLUTION_TWO_FILE", "SOLVER_NAME", "SOLVER_PARAM_FILE", "SURFACIC_HEAT_FLUX",
"TEMPERATURE", "TEMPERATURE_MAC", "TRACTION_X",  "TRACTION_Y", "TRACTION_Z", 
"ZONE", "ZONE_ENRICH",  "ZONE_ENV", "ZONE_POSTPRO",
"ZONE_NAME",
"REFERENCE_SOLUTION_FILE", "STORE_SOLUTION_FILE", "STRESS", 
"COMPUTE_EXACT_ERROR", "EXACT_SOLUTION_NAME",
"NB_EVO", "EVO", 
"SHEARBAND", "SHEARBAND_DIS","SHEARBAND_DIS_RIGID",
"TANGENT_STIFFNESS","NORMAL_STIFFNESS","FRICTION", "WALL", "DENSITY", "EXPORT_FORMAT", "ELECTRIC_DISPLACEMENT", "CUSTOM"
};

int xData::keyword_sorted  = 0;

char xData::procedure_list[][NB_CHAR_MAX] = {
"crack_growth", "cohesive_crack_growth"
};
int xData::procedure_sorted = 0;
//
char xData::exact_solution_list[][NB_CHAR_MAX] = {
"patch_test","contact2d","hole", "inclusion", "bimaterial1d", "stokescyl", "crack_mode_I",
"bimaterial3d","spherical_inclusion","spherical_cavity"};
int xData::exact_solution_sorted = 0;

static int cmp(const void *vp, const void *vq)
{
  return strcmp((const char *) vp, (const char *) vq);
};




// DEFAULT CONSTRUCTOR
xData::xData(void) : MaterialManager(xMaterialManagerSingleton::instance()) { 

  
  Dimension = 0;
  ComputeExactError = false;

  time = 1.0; 
  dt   = 1.0;

  PhysicalEnv = new xPhysicalEnv;

  strcpy(procedure, "");
  strcpy(mesh_file_type, "");
  strcpy(mesh_file, "");
  export_format = "ascii";
  strcpy(formulation_param_file, "");
  strcpy(solver_param_file, "");
  strcpy(solver_name, "");
  strcpy(partition_file, "");

  strcpy(reference_solution_file, "");
  strcpy(store_solution_file, "");

  if (!keyword_sorted) {keyword_sorted = 1; 
			qsort(keyword_list, sizeof(keyword_list)/NB_CHAR_MAX, NB_CHAR_MAX, cmp);}


  if (!exact_solution_sorted) {exact_solution_sorted = 1; 
			qsort(exact_solution_list, sizeof(exact_solution_list)/NB_CHAR_MAX, 
			      NB_CHAR_MAX, cmp);}

  if (!procedure_sorted) {procedure_sorted = 1; 
			qsort(procedure_list, sizeof(procedure_list)/NB_CHAR_MAX, NB_CHAR_MAX, cmp);}




  xMaterialManagerSingleton::instance().registerMaterial("elastic", 
							 xMaterialCreator<xElastic>() );



return;
}


// DESTRUCTOR
xData::~xData(void) { 

//a bug needs to be fix in aomd because
//the destructor of mesh makes aomd crash  
//delete mesh;

  delete PhysicalEnv;

}

char * xData::GetMeshFile(void) 
{ 
  return mesh_file;
}


char * xData::GetSolverName(void) 
{ 
  return solver_name;
}
char * xData::GetSolverParamFile(void) 
{ 
  return solver_param_file;
}

// char * xData::Getanalysis(void) 
// { 
//   return analysis;
// }

char * xData::GetProcedure(void) 
{ 
  return procedure;
}



void xData::ReadMesh(void)
{
  if (mesh_file == "") return;
  ifstream f(mesh_file);
  if (f.good())
  {
    f.close();
    mesh = new xMesh(mesh_file);
  }
  else
  {
    f.close();
    cerr<< "no mesh file !" << endl;
    throw "no mesh file !";
  }
}

void xData::ReadZones(void)
{
  for (int_zones_t::const_iterator it = int_zones.begin(); it != int_zones.end(); ++it) 
    {
      MaterialManager.createZone(it->first, it->second.first, it->second.second);
    }
  for (string_zones_t::const_iterator it = string_zones.begin(); it != string_zones.end(); ++it) 
    {
      MaterialManager.createZone(it->first, it->second.first, it->second.second);
    }
}

inline void check_active(int active, char* s)
{
  if (active != 0) 
  { 
    printf( "Error in DAT file: you forgot to close a brace for the %s info\n", s);
    assert( 1 == 0);
  }
  return;
}
void   xData::SetMeshFile(const char* _mesh_file_name)
{
  strcpy(mesh_file, _mesh_file_name);
}

void xData::ReadInfo(const char *filename)
{
  FILE *fp = fopen(filename, "r");
  if (fp == 0) {fprintf(stderr, "The data file %s cannot be opened\n", filename);
		assert(1 == 0);}
  char key[256];
  char c;
  int i;
  int geom, entity, val_ass;
  string phys;
  double val_fix;
  xEnv env;
  xPhysicalEnv * InfoEnv;

  xBoundary crvboundary;
  PhysicalEnv->clear(); 

  char mat_class[NB_CHAR_MAX], mat_param[NB_CHAR_MAX], zone_name[NB_CHAR_MAX];
//for the load evolution
  int nb_evo;
  double factor, t;
  std::map<double, double> evo;




const int NOTHING_ACTIVE       = 0;
const int BC_VOLUME_ACTIVE         = 2;
const int BC_SURFACE_ACTIVE         = 3;
const int BC_LINE_ACTIVE         = 4;
const int BC_POINT_ACTIVE         = 5;
const int ZONE_ENV_ACTIVE         = 6;
const int ZONE_ACTIVE          = 10;
const int ZONE_NAME_ACTIVE          = 11;



  int active = NOTHING_ACTIVE;


  while( (i = fscanf(fp, "%[1234567890A-Z_]",  key)) != EOF) {
    if (i == 0)
      {
	c=fgetc(fp);
        //printf("c: %c\n", c);

	if ( c != '#' && c !='\n' && c != ' ' && c != '\t' && c != '}')
	{
	  fprintf(stderr, "The following character is not known : %c\n", c); 
	  assert(1 == 0);
	}
	if (c=='#') fscanf(fp, "%*[^\n] \n");

	if (c=='}') {	
	  if      ( active == BC_LINE_ACTIVE   ) { }
	  else if ( active == BC_SURFACE_ACTIVE  ) { }
	  else if ( active == BC_VOLUME_ACTIVE  ) { }
	  else if ( active == BC_POINT_ACTIVE  ) { }
	  else if ( active == ZONE_ENV_ACTIVE  ) { }
	  else if ( active == ZONE_ACTIVE   ) {
	    std::string mat_class_s(mat_class);
	    std::string mat_param_s(mat_param);
	    int_zones.insert(std::make_pair(entity, std::make_pair(mat_class_s, mat_param_s)));
            //MaterialManager.createZone(entity, mat_class, mat_param);
	  }
	  else if ( active == ZONE_NAME_ACTIVE   ) {
            //MaterialManager.createZone(zone_name, mat_class, mat_param);
	    std::string mat_class_s(mat_class);
	    std::string mat_param_s(mat_param);
            std::string zone_name_s(zone_name);
	    string_zones.insert(std::make_pair(zone_name_s, std::make_pair(mat_class_s, mat_param_s)));
	  }
	  else if ( active == NOTHING_ACTIVE ){
	    printf( "Error in DAT file :you closed a braced without opening one\n");
	    assert(0);}
	  else { assert(0); }
          //we clear things before reading the new FOO = 3 { .... }
	  active = NOTHING_ACTIVE;
	  entity = -1;
	}
      }
    else
      {
	//
	// Check if the key is valid
	//
        if (bsearch((const void *) key, keyword_list,
		       sizeof(xData::keyword_list)/NB_CHAR_MAX, NB_CHAR_MAX, cmp) == 0)
	  {fprintf(stderr, "The keyword %s is not known\n", key); assert(1 == 0);}
        
	if (strcmp(key, "ENDIF") == 0); //nothing to do

	//
	// action depending on the key
	//
	if (strcmp(key, "PROCEDURE") == 0) 
	  { 
	    fscanf(fp, "%*[\n= ] %s[a-z]", procedure);	
	    if (bsearch((const void *) procedure, procedure_list,
			sizeof(xData::procedure_list)/NB_CHAR_MAX, NB_CHAR_MAX, cmp) == 0)
	      {printf("The type of procedure %s is not known\n", procedure); assert(1 == 0);}
	  }
	if (strcmp(key, "MESH_FILE_TYPE") == 0) 
	  {
	    fscanf(fp, "%*[\n= ] %s[a-z]", mesh_file_type);		
	  }
	if (strcmp(key, "MESH_FILE") == 0) fscanf(fp, "%*[\n= ] %s[a-z]", mesh_file);		
	if (strcmp(key, "EXPORT_FORMAT") == 0) 
	  {
	    char tmp[NB_CHAR_MAX];
	    fscanf(fp, "%*[\n= ] %s[a-z]", tmp);
	    export_format = tmp;
	    if ( ( export_format != "binary" ) && (export_format != "ascii") )
	      {
                fprintf(stderr, "EXPORT_FOMAT is %s\n",  export_format.c_str());
		fprintf(stderr, "EXPORT_FORMAT should be ascii or binary\n"); assert(0);
	      }
	  }
	if (strcmp(key, "GEOM_FILE") == 0) fscanf(fp, "%*[\n= ] %s[a-z]", geom_file);
	if (strcmp(key, "GEOM_TYPE") == 0) fscanf(fp, "%*[\n= ] %s[a-z]", geom_type);
	if (strcmp(key, "INTEGRATION_TYPE") == 0) fscanf(fp, "%*[\n= ] %s[a-z]", integration_type);
	if (strcmp(key, "FORMULATION_PARAM_FILE") == 0) fscanf(fp, "%*[\n= ] %s[a-z]", formulation_param_file);	
	if (strcmp(key, "PROCEDURE_PARAM_FILE") == 0) fscanf(fp, "%*[\n= ] %s[a-z]",    procedure_param_file);	
	if (strcmp(key, "SOLVER_PARAM_FILE") == 0) fscanf(fp, "%*[\n= ] %s[a-z]", solver_param_file);	
	if (strcmp(key, "SOLVER_NAME") == 0) fscanf(fp, "%*[\n= ] %s[a-z]", solver_name);	
	
	if (strcmp(key, "PARTITION_FILE") == 0) fscanf(fp, "%*[\n= ] %s[a-z]", partition_file);		
	if (strcmp(key, "RESULT_FILE") == 0) fscanf(fp, "%*[\n= ] %s[a-z]", result_file);		

	if (strcmp(key, "COMPUTE_EXACT_ERROR") == 0) { ComputeExactError = true; }


	// STRESS needed 
	if (strcmp(key, "COMPUTE_EXACT_ERROR") == 0) ComputeExactError = true;		

	if (strcmp(key, "REFERENCE_SOLUTION_FILE") == 0) 
	  fscanf(fp, "%*[\n= ] %s[a-z]", reference_solution_file);	
	if (strcmp(key, "STORE_SOLUTION_FILE") == 0) 
	  fscanf(fp, "%*[\n= ] %s[a-z]", store_solution_file);	

	if (strcmp(key, "SOLUTION_ONE_FILE") == 0) fscanf(fp, "%*[\n= ] %s[a-z]", solution_one_file);
	if (strcmp(key, "SOLUTION_TWO_FILE") == 0) fscanf(fp, "%*[\n= ] %s[a-z]", solution_two_file);

	if (strcmp(key, "ZONE") == 0)
	  {
	    fscanf(fp, "%d%*[\n= {]", &entity); 
	    check_active(active, "ZONE");
	    active = ZONE_ACTIVE;
	  }

	if (strcmp(key, "ZONE_NAME") == 0)
	  {
	    fscanf(fp, "%s%*[\n= {]", zone_name); 
	    check_active(active, "ZONE_NAME");
	    active = ZONE_NAME_ACTIVE;
	  }


	if (strcmp(key, "MAT_CLASS") == 0)  fscanf(fp, "%*[\n= ] %[a-zA-Z0123456789_]", mat_class);
	if (strcmp(key, "MAT_PARAM") == 0) { fscanf(fp, "%*[\n= ] %[a-zA-Z0123456789_./*]", mat_param); }
	


	if (strcmp(key, "BC_VOLUME") == 0)  {fscanf(fp, "%d%*[\n= {]", &entity); 
					   InfoEnv = PhysicalEnv;
					   check_active(active, "BC_VOLUME");
					   geom    = BC_VOLUME;
					   active  = BC_VOLUME_ACTIVE;}
	if (strcmp(key, "BC_SURFACE") == 0)  {fscanf(fp, "%d%*[\n= {]", &entity); 
					   InfoEnv = PhysicalEnv;
					   check_active(active, "BC_SURFACE");
					   geom    = BC_SURFACE;
					   active  = BC_SURFACE_ACTIVE;}
	if (strcmp(key, "BC_LINE") == 0)  {fscanf(fp, "%d%*[\n= {]", &entity); 
					   InfoEnv = PhysicalEnv;
					   check_active(active, "BC_LINE");
					   geom    = BC_LINE;
					   active  = BC_LINE_ACTIVE;}
	if (strcmp(key, "BC_POINT") == 0)  {fscanf(fp, "%d%*[\n= {]", &entity); 
					   InfoEnv = PhysicalEnv;
					   check_active(active, "BC_POINT");
					   geom    = BC_POINT;
					   active  = BC_POINT_ACTIVE;}

//JF NEW
	if (strcmp(key, "ZONE_ENV") == 0)  {fscanf(fp, "%d%*[\n= {]", &entity); 
					    InfoEnv = PhysicalEnv;
					    geom = ZONE_ENV;
					    check_active(active, "ZONE_ENV");
	                                    active = ZONE_ENV_ACTIVE;}


//JF END


	if (strcmp(key, "DISPLACEMENT_X") == 0)     phys = key;
	if (strcmp(key, "DISPLACEMENT_Y") == 0)     phys = key;
	if (strcmp(key, "DISPLACEMENT_Z") == 0)     phys = key; 
	if (strcmp(key, "VECTOR_POTENTIAL_X") == 0)     phys = key;
	if (strcmp(key, "VECTOR_POTENTIAL_Y") == 0)     phys = key;
	if (strcmp(key, "VECTOR_POTENTIAL_Z") == 0)     phys = key; 

	if (strcmp(key, "ROTATION_X") == 0)         phys = key; 
	if (strcmp(key, "ROTATION_Y") == 0)         phys = key; 
	if (strcmp(key, "COUPLE_X") == 0)           phys = key; 
	if (strcmp(key, "COUPLE_Y") == 0)           phys = key; 
	if (strcmp(key, "TRACTION_X") == 0)         phys = key; 
	if (strcmp(key, "TRACTION_Y") == 0)         phys = key; 
	if (strcmp(key, "TRACTION_Z") == 0)         phys = key; 

	if (strcmp(key, "STRESS") == 0)             phys = key;

	if (strcmp(key, "DISPLACEMENT") == 0)       phys = key; 
	if (strcmp(key, "PRESSURE") == 0)           phys = key; 
	if (strcmp(key, "POTENTIAL") == 0)          phys = key; 
	if (strcmp(key, "POTENTIAL_WEAK") == 0)          phys = key; 
        if (strcmp(key, "ELECTRIC_DISPLACEMENT") == 0)  phys = key; 
	if (strcmp(key, "TEMPERATURE") == 0)        phys = key; 
	if (strcmp(key, "TEMPERATURE_MAC") == 0)    phys = key; 
	if (strcmp(key, "SURFACIC_HEAT_FLUX") == 0) phys = key; 
	if (strcmp(key, "NORMAL_VELOCITY") == 0)    phys = key; 
	if (strcmp(key, "ELASTIC_SUPPORT") == 0)    phys = key; 

	if (strcmp(key, "VELOCITY_X") == 0)     phys = key; 
	if (strcmp(key, "VELOCITY_Y") == 0)     phys = key; 
	if (strcmp(key, "VELOCITY_Z") == 0)     phys = key; 
	if (strcmp(key, "VELOCITY") == 0)       phys =  key;

	if (strcmp(key, "ACCELERATION_X") == 0)     phys = key; 
	if (strcmp(key, "ACCELERATION_Y") == 0)     phys = key; 
	if (strcmp(key, "ACCELERATION_Z") == 0)     phys = key; 
	if (strcmp(key, "CUSTOM") == 0)     phys = key; 

	if (strcmp(key, "FIX") == 0)  
	  {
	    fscanf(fp, "%*[\n= ] %lf", &val_fix);  
	    env.defineFixed(phys, geom, entity, val_fix);  
            InfoEnv->add(env);
            env.clear();
	  }
	if (strcmp(key, "ASS") == 0)  
	  {
	    fscanf(fp, "%*[\n= ] %d", &val_ass);  
	    env.defineAssociate(phys, geom, entity, val_ass);
            InfoEnv->add(env);
            env.clear();
	  }
        if (strcmp(key, "NB_EVO") == 0)  
	  {
            fscanf(fp, "%*[\n= ] %d", &nb_evo);
	  }
        if (strcmp(key, "EVO") == 0)  
	  {
	    fscanf(fp, "%*[\n= ]");
            evo.clear();
            for (i = 0; i < nb_evo; i++) {
	      fscanf(fp, "%*[( ] %lf %*[ ,] %lf %*[ )]", &t, &factor);
              if (evo.find(t) == evo.end()) {
		evo[t] = factor;
	      }
	      else {
		fprintf(stderr, "You cannot have two factor for the same time\n"
			"i.e. the loading evolution must be continuous\n"
			"the time %12.5e is there twice\n", t);
	      }
	    }
            env.setEvolution(evo);
	  }


      }
  }

  fclose(fp);


  return;
}

}  


\end{verbatim}


















    \chapter{Les notions de bases de la librairie \code{xfem}}
        
\section{la notions de valeurs}

\subsection{La classe \code{xValue}}

\paragraph*{Motivation :}
La m\'ethode des \'el\'ements finis repose sur une r\'eduction de certains
champs physiques sur un domaine en une approximation \`a un ensemble
fini de valeurs. La notions de \guil{valeurs} est donc une notions
importante et une classe paticuli\'ere lui est consacr\'ee dans la
librairie \code{xfem}. Il s'agit de la classe \doxygen{xValue}.

\paragraph*{Impl\'ementation :}
C'est une
\webify{http://fr.wikipedia.org/wiki/Classe_abstraite}{classe
abstraite} \index{Classe!abstraite} qui est, de plus
  \textit{Templatis\'ee} (param\'etr\'ee par un type) et DOIT \^etre d\'eriv\'ee en classe concr\'ete pour \^etre utilis\'ee, en pr\'ecisant le type des valeurs que l'on souhaite utiliser (r\'eels, entiers, ...).
La classe d\'eriv\'ee la plus utilis\'ee est la classe
\doxygen{xValueDouble}  qui est une classe de \code{xValue} de r\'eels
double pr\'ecision (\code{xValue<double>}). Il existe \'egalement la
classes d\'eriv\'ee,  \code{xTensor}. Par ailleurs, \code{xValueDouble}
est elle-m\'eme d\'eriv\'ee en \code{xValueLinearCombination} et
\code{xValueOldAndCurrent} (d\'efini dans la librairie \code{Xext}).

\inheritgraph{5cm}{xValueDouble}


Consid\'erons l'approximation \'el\'ements\index{Approximation} finis
Eq.~(\ref{eq:approx_ef2_2}) (voir la
section~\ref{partition_de_l_unite})~:
\begin{equation}
  \label{eq:approx_ef2_2}
  \bfu(\bfx) \mid_{\Ome_e} = \sum_{i \in N_n} \sum_{\alpha} a_i^\alpha
  \bfphi_i^\alpha(\bfx)
\end{equation}
Les coefficients scalaires\index{Scalaire} $a_i^\alpha$ sont
repr\'esent\'es par des objets de cette classe.





De m\'eme, dans le cas d'\'el\'ements finis enrichis, l'approximation
\'el\'ements finis enrichie
 permettant de repr\'esenter une fonction d'enrichissement $F(\bfx) \bfe_x$ sur le
domaine $\Ome_F$ s'\'ecrit~:
\begin{equation}
  \label{eq:approx_ef_enrichi_2}
  \bfu(\bfx)  = \sum_{i \in N_n(\bfx)} \sum_{\alpha} \bfphi_i^\alpha a_i^\alpha   +
  \sum_{i \in N_n(\bfx) \cap N_F}
  \sum_{\alpha} b_i^\alpha \bfphi_i^\alpha(\bfx) F(\bfx)
\end{equation}
o\'e $N_F$ est l'ensemble des noeuds dont le support a une
intersection avec le domaine $\Ome_F$. Les coefficients $b_i^\alpha$
sont l\'e aussi repr\'esent\'es par des objets de classe
\code{xValueDouble}.


On peut sch\'ematiser en disant que ces objets sont les degr\'es de
libert\'e du mod\'eles \'el\'ements finis avant la prise en compte des
conditions aux limites.









\subsection{Notions de valeurs li\'ees}

\paragraph*{motivation : }
Dans un mod\'ele \'el\'ements finis, il peut \^etre interessant de
contraindre une valeur d'\'etre li\'ee \`a d'autres valeurs. C'est le cas,
par exemple, lorsqu'on impose des conditions de periodicit\'e d'un
bord \`a l'autre d'un Volume El\'ementaire Repr\'esentatif (voir exemple
1), ou lorsqu'on impose au champ de d\'eplacement d'\'etre lin\'eaire sur
le bord d'un domaine (voir exemple 2).

De mani\'ere g\'en\'erale, on peut d\'efinir cette valeur comme \'etant une
combinaison lin\'eaire d'autres valeurs. Ce n'est alors plus un degr\'e
de libert\'e du mod\'ele final, puisque la valeur est fix\'ee par les
valeurs dont elle d\'epend.



\paragraph*{Impl\'ementation :}
Pour imposer que cette combinaison lin\'eaire \index{combinaison
lin\'eaire} soit vraie \`a tout moment dans le calcul, sans avoir \'e
recalculer cette combinaison lin\'eaire,  une classe d\'eriv\'ee de la
classe \code{xValue} a \'et\'e cr\'e\'ee. Il s'agit de la classe
\doxygen{xValueLinearCombination}. Elle s'applique aux classes
d\'eriv\'ees de \code{xValueDouble}.

Quelques exemples d'utilisation de la classe
\doxygen{xValueLinearCombination} sont visibles dans le programme
\code{main.cc} du repertoire \code{Xfem/devel/value}. Ces exemples
sont les suivants:

\paragraph{Exemple 1:}

L'exemple le plus simple est celui o\`u une valeur est impos\'ee \'egale \'e
une autre. De fa\'eon formel, on \'ecrira, par exemple :
\begin{equation}
    v_2 =  v_1
\end{equation}
ceci se traduira en d\'eclarant $v_2$ de la mani\'ere suivante :
\begin{verbatim}
  xValueDouble v1;
  xValueLinearCombination v2(1.0, &v1);
\end{verbatim}
On utilise ici le pointeur vers la variable \code{v1} et non pas sa
valeur, qui peu \'evoluer au cours du calcul. On notera que
l'utilisation d'une classe permet la d\'eclaration de \code{v2} en
m\'eme temps que sa d\'efinition.

\paragraph{Exemple 2:}

Si on souhaite imposer une combinaison lin\'eaire de la forme :
\begin{equation}
    v_3 = 2~v_1 + 3
\end{equation}
alors, on \'ecrira :
\begin{verbatim}
  xValueLinearCombination v3(2.0, &v1, 3.);
\end{verbatim}

\paragraph{Exemple 3:}

Si cette combinaison lin\'eaire fait intervenir plus de deux
variables, on utilisera  un vecteur de coefficients (vecteur
\code{coeffs} ci-dessous) ainsi qu'un vecteur de \code{xValue}
(vecteur \code{vals} ci-dessous). La relation suivante :
\begin{equation}
    v_4 = v_1 + 2~v_2 + 3~v_3 + 5
\end{equation}
se d\'eclarera de la mani\'ere suivante :
\begin{verbatim}
  std::vector<double> coeffs;
  coeffs.push_back(1.0);
  coeffs.push_back(2.0);
  coeffs.push_back(3.0);

  std::vector<xValue<double>*> vals;
  vals.push_back(&v1);
  vals.push_back(&v2);
  vals.push_back(&v3);

  xValueLinearCombination v4(coeffs, vals, 5.0);
\end{verbatim}
\remarque{La fonction \code{push\_back()} est une fonction de la
librairie \code{STL} permettant de remplir un vecteur \guil{par le
bas} sans avoir \`a introduire les indices.}



\section{La notion d'\'etat des valeurs}\label{section_xStateOfValue}

\paragraph*{Motivation :}
Les degr\'es de libert\'e du mod\'eles \'el\'ements finis peuvent intervenir
dans le membre de gauche ou de droite du syst\'eme matricielle final.
La distinction d\'epend de leur status. Pour pouvoir assembler
correctement le syst\'eme matriciel, l'algorithme d'assemblage doit
pouvoir acc\'eder aux informations sur l'\'etat des valeurs du probl\'eme
m\'ecanique.

\paragraph*{Impl\'ementation :}
L'\'etat d'une valeur est d\'efini par un objet de  classe
\doxygen{xStateOfValue}. Chaque membre de la classe  \code{xValue}
contient un pointeur vers un membre de la classe
\doxygen{xStateOfValue}. La classe \code{xStateOfValue} est
elle-m\'eme une classe abstraite. Les trois classes d\'eriv\'ees utilis\'ees
r\'eguli\'erement sont   \doxygen{xStateOfValueDof},
\doxygen{xStateOfValueFixed},
\doxygen{xStateOfValueLinearCombination}. Cette derni\'ere est l'\'etat
d'une valeur combinaison lin\'eaire d'autres valeurs. L'\'etat d'une
combinaison lin\'eaire n'est pas modifiable par l'utilisateur et est
g\'en\'er\'e automatiquement lors de l'appel de la fonction
\code{getState}.




\section{Les cl\'es des valeurs : \code{xValkey}}\label{section_xValkey}

\paragraph*{Motivation :}

Dans un mod\'ele \'el\'ements finis, la notions de valeurs peut
s'appliquer \`a diff\'erents grandeurs physiques. D'autres part, elles
peuvent \^etre attach\'ees \`a des noeuds, des \'el\'ements ou d'autres
entit\'es g\'eom\'etriques, enfin, elles peuvent \^etre des coefficients de
fonctions d'enrichissement  ou de  fonctions d'interpolation
classiques.\index{cl\'e de valeurs,xValkey}

\paragraph*{Impl\'ementation :}
Afin de retrouver les informations attach\'ees \`a une valeurs, la
classe \doxygen{xValkey} \index{cl\'es} a \'et\'e cr\'e\'ee. Cette classe
permet de g\'erer les informations sur~:
\begin{center}
% use packages: array
\begin{tabular}{p{9cm}p{.25cm}p{5.25cm}}
& & \\
-- la signification physique de la valeur (temp\'erature, d\'eplacement suivant x, d\'eplacement suivant y, ...). Elle est
d\'efinie par une cha\'ene de caract\'eres cod\'ee par un entier (short) &  & \code{getPhys , setPhys} (short)  \\
& & \\
-- la signification g\'eom\'etrique de la valeur (coefficients de fonctions nodales, de fonctions d'enrichissement, ...).
 Elle est \'egalement d\'efinie par une cha\'ene de caract\'eres cod\'ee par un entier  &  & \code{getGeom, setGeom} (short)  \\
& & \\
-- un pointeur vers l'entit\'e g\'eom\'etrique associ\'ee &  & \code{getEnti, setEnti} (pointeur vers un \code{AOMD::mEntity})\\
& & \\
-- un label suppl\'ementaire pouvant servir \`a distinguer plusieurs enrichissement de m\'eme signification physique et g\'eom\'etrique,
et s'appliquant sur une m\'eme entit\'e g\'eom\'etrique & &   \\
& & \\
\end{tabular}
\end{center}

\paragraph*{cr\'eation :}

Les cl\'es de valeurs sont g\'en\'er\'e lors de l'utilisation de la fonction
\code{DeclareInterpolation} (ATTENTION : contrairement \`a ce que sont
nom indique, cette fonction d\'eclare une approximation, et non une
interpolation)


\section{Les cl\'es d'informations : \code{xKeyInfo}}\label{xKeyInfo}

\paragraph*{Motivation :}

La signification physique ou g\'eom\'etrique des valeurs est \'ecrite,
pour l'utilisateur, sous forme de cha\'enes de caract\'ere \index{cl\'e
d'information,xKeyInfo} , comme \guil{DEPLACEMENT\_X} ou
\guil{TEMPERATURE}. Mais d'un point de vue informatique, il est plus
simple de travailler avec des entiers (pour un probl\'eme de taille
m\'emoire en particulier).

\paragraph*{Impl\'ementation :}
Ces cha\'enes de caract\'eres sont  cod\'es de mani\'ere automatique et
\textit{\'e la vol\'ee} par la classe \doxygen{xKeyInfo}. Cette derni\'ere
classe de fonction \code{static}\index{static} g\'ere la
correspondance entre ces cha\'enes de caract\'ere et des  \textit{short}
(entier), servant de codage tout le long du calcul. Par leur nature
\code{static}, ce codage est utilisable \`a tout moment dans le
programme.



\section{Le gestionnaire de cl\'es et le valeurs: \code{xValManager}}\label{xValManager}

\paragraph*{Motivation :}
D'un c\'et\'e, les objets de classe \code{xValue} contiennent (stockent)
les valeurs utiles aux calculs. De l'autre cot\'e, les classes
\code{xValkey} contiennent (stockent) les cl\'es de ces valeurs. Il
est n\'ecessaire de maintenir un lien entre les valeurs et leur cl\'es.


\paragraph*{Impl\'ementation :}
Ce lien se fait par l'interm\'ediaire d'une  classe de service appel\'ee
\doxygen{xValManager}\index{xValManager}. Ce gestionnaire de cl\'es et
de valeurs est g\'en\'eral (abstrait) et peut g\'erer des objets de
classes d\'eriv\'ee de  \code{xValue<T>} au sens large (tenseur, r\'eel,
entier, ...). Une classe d\'edi\'ee \`a la gestion des r\'eels
(\code{double}) a \'et\'e d\'eriv\'ee sous le nom de
\doxygen{xDoubleManager}.

\remarque{Cette liste de relations bijectives entre objets de deux
classes est appel\'ee \guil{map}.}

        \clearpage
\section{la notions d'approximation}

\subsection{Les champs physiques}

\paragraph*{Remarque :}
Dans un langage orient\'e objet d\'edi\'e \`a la m\'ethode des  \'el\'ements
finis, on s'attend \`a trouver des classes \guil{\'el\'ements finis}. Une
telle classe existe (\code{xFiniteElement}), mais il ne s'agit en
aucun cas d'une biblioth\'eque d'\'el\'ements au sens des codes \'el\'ements
finis. L'utilisation de cette classe n'est g\'en\'eralement pas utiles
dans la plupart de cas.

\paragraph*{Motivation :}
D'un point de vue conceptuel, la m\'ethode des \'el\'ements finis repose
sur l'\'ecriture du principe variationnel \`a un ou plusieurs champ
\'ecrit sur un domaine d\'ecoup\'e en \'el\'ements g\'eom\'etriques sur lesquelles
certains champs (le d\'eplacement, pour une formulation en
d\'eplacement) sont  approch\'es. Pour la m\'ethode de Galerkin continu,
l'interpolation des champs est continu sur le domaine \'etudi\'e.

Prenons l'approximation du champ de temp\'erature $\bfT(\bfx)$ sur un
domaine $\Ome$~:
\begin{equation}
  \label{eq:approx_T}
  \bfT(\bfx) \mid_{\Ome^{e}} = \sum_{i}^{N^{e}}  a_i
  \bfphi_i^{e}(\bfx)
\end{equation}
L'\'equation~\ref{eq:approx_T} regroupe trois notions importantes
repr\'esent\'ees par des classes diff\'erentes. $\bfT(\bfx)$ est un
\textbf{champ} approch\'e   dont l'approximation est fonction des
\textbf{valeurs} nodales $a_i$ et des \textbf{fonctions de forme}
$\bfphi_i(\bfx)$ :

\paragraph*{Impl\'ementation :}
Ces trois notions sont introduite de la mani\'ere suivante :
\begin{longtable}{p{4.5cm}p{.01cm}p{10cm}}
& & \\
-- les coefficients nodaux $a_i$  &  & ils sont ici  des r\'eels (\code{Double}) est sont stock\'es dans des objets de classes \code{xValueDouble} (section~\ref{section_xValkey}) et sont g\'er\'es par un manager de double de classe \doxygen{xDoubleManager} ; \\
& & \\
-- Les fonctions $\bfphi_i(\bfx)$ &  & elles ne sont pas
explicit\'ees, mais sont d\'eclar\'ees par  l'espace d'approximation
d\'efini par la classe \doxygen{xSpace}. Cette classes permet de
d\'efinir le type de fonction d'approximation utilis\'ees. Elle est
g\'en\'erale et est d\'eriv\'ee en :
\begin{description}
    \item[--] \code{xSpaceLagrange},   espace  des fonctions de Lagrange ;
    \item[--] \code{xSpaceConstante}, espace  des fonctions constantes par \'el\'ements ;
    \item[--]\code{xSpaceXFEM},  espace  des fonctions d'enrichissement   d\'efinissant les fonctions Heaviside et les fonctions $F^l_i(\bfx)$  (Eq:~\ref{enrich_fun}). Lors d'enrichissement, les champs de fonctions classiques sont ajout\'es aux champs de fonctions d'enrichissement. On utilise alors la classe \doxygen{xSpaceComposite} pour cr\'eer  un espace compos\'e de deux sous-espaces. Ainsi, ceci permet de red\'efinir le nombre total de degr\'es de libert\'e par noeud.
\end{description}
    La classe \code{xSpace} permet de d\'efinir un lien entre les cl\'es de valeur (c'est \`a dire les coefficients $a_i$) et les fonctions de formes associ\'es $\bfphi_i$. Cette classe poss\'ede des fonctions qui, \`a partir d'un objet de classe \code{xGeom} (element $e$ par exemple),  donne un vecteur de cl\'es contenant les cl\'es de valeurs ($a_i$) et les fonctions associ\'ees \`a ces cl\'es ($\bfphi_i$).
    \\
& & \\
-- le champ approch\'e  $\bfT(\bfx)$ &  & il est, quant \`a lui, stock\'e dans un objet de classe \doxygen{xField}. Un objet de classe \code{xField} est d\'efini par la connaissance de l'espace d'approximation (objet de classe \code{xSpace}) et les coefficients  nodales  $a_i$ par l'interm\'ediaire du \code{xDoubleManager}. \\
& &
\end{longtable}

\inheritgraph{14cm}{xSpace}


\paragraph*{Utilisation :}
La d\'eclaration  d'objets \code{xValue}, \code{xSpace} et
\code{xField} ne suffit pas \`a g\'en\'erer compl\'etement l'approximation.
Pour que cette approximation soit effective, il est n\'ecessaire
d'utiliser une fonction \textbf{g\'en\'erique} d\'efinie dans le fichier
\code{xAlgorithm.h} et appel\'ee \code{DeclareInterpolation()}. Ainsi,
pour cr\'eer un champ de temp\'erature \code{temp} sur un maillage
\code{mesh}, il est n\'ecessaire d'effectuer les op\'erations suivantes
:
\begin{description}
\item[--] d\'eclarer un gestionnaire de cl\'es de valeurs r\'eelles ;
\item[--] d\'efinir un espace de fonction de Lagrange d\'edi\'e \`a la temp\'erature ;
\item[--] d\'efinir le champs de temp\'erature en associant le gestionnaire de cl\'e et l'espace des fonctions ;
\item[--] d\'efinir un cr\'eateur de r\'eels. Ce cr\'eateur d\'erive  de classe abstraite \doxygen{xValueCreator<T>}.
Celle classe d\'efinit le probl\'eme \`a r\'esoudre en interpr\'etant le jeu de donn\'ees et en cr\'eant les diff\'erents type
des valeurs (\code{xValueDouble, xValueLinearCombination, ...}) en fonction du probl\'eme. Cette partie est en g\'en\'eral
 \`a d\'evelopper \`a chaque type de probl\'eme \`a r\'esoudre. Elle prend une cl\'e du (\code{xValKey}) et retourne une valeur
  (???????????? a pr\'eciser) ;
\item[--] d\'eclarer l'interpolation : cette fonction g\'en\'erique permet d'allouer la m\'emoire pour le champs de temp\'erature
sur le maillage \code{mesh}. Il g\'en\'ere ainsi les cl\'es    de
\code{xValueDouble} pour chaque degr\'e de libert\'e cr\'e\'e pour le champs
\code{temp}, en balayant tout les \'el\'ements du maillage (\'el\'ements de
dimension 2 pour un maillage surfacique).
\end{description}

Ceci se traduit de la mani\'ere suivante :
\begin{verbatim}
    xDoubleManager vals;
    xSpaceLagrange temp_space("TEMPERATURE", SCALAR, DEGREE_ONE);
    xField         temp(&vals, temp_space);
    xValueCreator<xValueDouble>  creator_double;

    DeclareInterpolation(temp, creator_double, mesh.begin(2), mesh.end(2));

    xStateDofCreator<xValueDouble> snh(double_manager, "dofs");
    DeclareState(temp, snh, mesh.begin(2), mesh.end(2));
\end{verbatim}
\paragraph*{Commentaire :}
Notons que l'\'ecriture des lignes ci-dessus v\'erifie les r\'egles de
programmation standard (bien qu'elles concerne des classes
fournissant des services parfois abstraits ou informatifs). Ainsi,
on effectue~:
\begin{enumerate}
\item la d\'eclataion les variables du programmes (ici des objets) ;
\item les actions sur les variables (initialisation,  transformation, fonctions, subroutine, ...)
\end{enumerate}
En \code{C++} comme en Fortran, les variables permettent de
param\'etrer les actions. En revanche, en \code{C++}, les actions
peuvent elles-m\'emes \'etre des objets.

Les deux derni\'ere lignes, ci-dessus, permettent ainsi de d\'efinir un
"cr\'eateur d'\'etat" (qui est une variable qui d\'efinie les actions \'e
effectuer), et d'effectuer l'action proprement dite par la fonction
DeclareState() qui utilise le cr\'eateur pour g\'en\'erer les \'etats.




\subsection{La classe \code{xMesh}}

Pour tout calcul \'el\'ements finis se pose la question de la g\'en\'eration
du maillage. Il est possible de charger des maillages issus de
l'utilitaire \code{gmsh} (voir section \textbf{\textit{??}}), mais
il est possible de d\'efinir  un maillage directement dans
l'application. Pour cela la classe \doxygen{xMesh} dispose d'un
certain nombre de commandes.


Pour cr\'eer le maillage de la figure~\ref{maillage2elements}, la
classe \code{xMesh} peut \'etre utiliser de la mani\'ere suivante~:
\begin{figure}[!htb]
\centering
    \begin{picture}(250,120)(0,0)
    \put(050,25){\framebox(75,75)[cc]{\textbf{1}}}
    \put(125,25){\framebox(75,75)[cc]{\textbf{2}}}
    \put(50,12){0}
    \put(125,12){1}
    \put(200,12){2}
    \put(50,105){3}
    \put(125,105){4}
    \put(200,105){5}
%   \put(050,25){\graphpaper(0,0)(125,75)}
    \end{picture}
\caption{maillage de deux \'el\'ements} \label{maillage2elements}
\end{figure}

\begin{verbatim}
  xMesh mesh;

  mesh.createVertex(0, 0., 0., 0., 0);
  mesh.createVertex(1, 1., 0., 0., 0);
  mesh.createVertex(2, 2., 0., 0., 0);
  mesh.createVertex(3, 0., 1., 0., 0);
  mesh.createVertex(4, 1., 1., 0., 0);
  mesh.createVertex(5, 2., 1., 0., 0);

  mesh.createFaceWithVertices(0,1,4,3,mesh.getGEntity(100,2));
  mesh.createFaceWithVertices(1,2,5,4,mesh.getGEntity(100,2));
  AOMD::classifyUnclassifiedVerices(&mesh);

  mesh.exportGmsh("mesh.msh");
\end{verbatim}

\code{xMesh} Les commandes relatives \`a la cr\'eation de maillage,
comme \code{createVertex}, se trouvent dans la librairie
\webify{www.scorec.rpi.edu/AOMD}{AOMD}. \dependgraph{8cm}{xMesh}

La fonction \code{mesh.exportGmsh()} permet d'exporter le   maillage
dans un fichier destin\'e \`a l'utilitaire \code{gmsh}.

        \clearpage
\section{la notions de forme bilin\'eaire et lin\'eaire}

\subsection{Probl\`eme de r\'ef\'erence}

Reprenons le probl\`eme de r\'ef\'erence de la section~\ref{probleme_de_reference}. L'\'equation~\ref{eq:varia} se r\'esume
\`a trouver le champ $\bfu$ satisfaisant :
\begin{equation}
    \label{eq:varia_2}
    \int_{\Ome}   \bfeps(\bfu):\bfC:\bfeps(\bfv) \dint \Ome =
    \int_{\Gam_t} \ov{\bft}\cdot\bfv  \dint \Gam
\end{equation}
quelque soit le champ de d\'eplacement virtuel $\bfv$. Apr\`es lin\'earisation de l'op\'erateur $ \bfeps(.)$, le premier
 terme de l'\'equation~\ref{eq:varia_2} est une forme bilin\'eaire en  $\bfu$ et  $\bfv$. Le deuxi\`eme terme est une forme
 lin\'eaire en $\bfv$.


En discr\'etisant le domaine par des \'el\'ement finis et en \'ecrivant le  champ  sous la forme :
\begin{equation}
    \bfu(\bfx) = \sum_i^N a_i \bfphi_i(\bfx)
\end{equation}
on arrive \`a un syst\`eme d'\'equation de type :
\begin{equation}
\label{eq:global_systeme}
K_{ij} a_j = f_i, \quad i = 1, \ldots, N
\end{equation}
o� la sommation sur l'indice $j$ est d'application et o�
\begin{eqnarray}
K_{ij} & = & \int_{\Ome}   \bfeps(\bfphi_i):\bfC:\bfeps(\bfphi_j) \dint
\Ome \\
f_i & = & \int_{\Gam_t} \ov{\bft}\cdot\bfphi_i  \dint \Gam
\end{eqnarray}

\subsection{Projection L2}

De mani\`ere generale, l'\'ecriture du principe variationnel sur un mod\`ele \'el\'ements finis conduit \`a la construction de
 formes bi-lin\'eaires et lin\'eaires. Ces derni\`eres sont construites par int\'egration sur chaque \'el\'ement de fonction de forme
 ou de leur d\'eriv\'ees puis l'assemblage des matrices \'el\'ementaires dans une matrice globale.

Ces op\'erations peuvent varier d'un probl\`eme \`a l'autre et font appel \`a diff\'erentes m\'ethodes. Pour g\'erer ces m\'ethodes :

\begin{description}
\item[--] \code{xFormBilinear} est une classe g\'en\'erale d'object permettant de d\'efinir des formes bilin\'eaires. Cette classe
est d\'eriv\'ee en  formes classiques (voir graphe d'h\'eritage ci-dessous). La forme construite d\'epend de l'application d'une
 \guil{loi de comportement} au sens de la m\'ecanique, et de l'application des op\'erateurs \`a appliquer sur les fonctions de
 forme $\bfphi_i(\bfx)$ (op\'erateur gradient, op\'erateur identit\'e, ...) ;
\inheritgraph{15cm}{xFormBilinear}

\item[--] \code{xIntegrationRuleBasic}, permet de d\'efinir le type d'int\'egration sur les \'el\'ements.


\item[--] \code{xAssemblerBasic} permet de d\'efinir le type d'assemblage.
\item[--] \code{xFormLinear}  est une classe g\'en\'erale d'object permettant de d\'efinir des formes lin\'eaires.
\inheritgraph{15cm}{xFormLinear}

\end{description}

Prenons l'exemple d'une projection L2  d'une fonction $f(x)$. On recherche les valeurs nodales $u_i$ satisfaisant
 l'\'equation suivante :
\begin{equation}
    \label{eq:L2proj}
    \int_{\Ome}   \bfu(x)  \bfv(x) \dint \Ome =
    \int_{\Ome} f(x) \bfv(x)  \dint \Ome  \hspace{2cm} \forall \hspace{2mm} \bfv
\end{equation}
ce qui se traduit par :
\begin{equation}
    \label{eq:L2proj-disc}
    \int_{\Ome} u_i  \bfphi_i \bfphi_j \dint \Ome =
    \int_{\Gam} f(x) \bfphi_j \dint \Gam
\end{equation}

Cette \'equation conduit \`a une forme bilin\'eaire sans op\'erateur sur les fonctions $\bfphi_i$ (op\'erateur identit\'e).
Cet exemple est programm\'e dans le fichier \code{main.cc} du repertoire \code{devel/l2proj} de la librairie \code{Xfem}.
 L'\'equation \ref{eq:L2proj-disc} se programme de la mani\`ere suivante :

\begin{verbatim}
 //un exemple de projection L2
xDoubleManager vals;
xSpaceLagrange temp_space("TEMPERATURE", SCALAR, DEGREE_ONE);
xField         temp(&vals, temp_space);
xValueCreator<xValueDouble>  creator_double;

DeclareInterpolation(temp, creator_double, mesh.begin(2), mesh.end(2));

xStateDofCreator<> snh(double_manager, "dofs");
DeclareState(temp, snh, mesh.begin(2), matrix.end(2));

xCSRVector b(vals.size("dofs"));
xCSRVector sol(vals.size("dofs"));
xCSRMatrix A(vals.size("dofs"));
xAssemblerBasic<> assembler(A, b);

xIntegrationRuleBasic integration_rule(2);

xFormBilinearWithoutLaw<xValOperator<xIdentity<double> >,
                        xValOperator<xIdentity<double> > > bilin;

Assemble(bilin, assembler, integration_rule, temp, temp, mesh.begin(2), mesh.end(2));


xFormLinearWithLoad<xValOperator<xIdentity<double> >, xEval<double> >  lin(t_exact);

Assemble(lin, assembler, integration_rule, temp, mesh.begin(2), mesh.end(2));

system.Solve(b, sol);
Visit(xWriteSolutionVisitor(sol.begin()), vals.begin("dofs"), vals.end("dofs"));
\end{verbatim}

Les objets courant d'alg\`ebre lin\'eaire  (\code{xCSRVector, xCSRMatrix}) sont d\'efinis sur de la base de classe de la librairie
 MTL auquelles sont ajouter des notions d'it\'erateur.


\subsection{A d\'evelopper ....}

notions d'assembleur et d'assemblage :   xAssemblerBasic<> assembler(A, b);

type d'integrateur : xIntegrationRuleBasic integration\_rule(2);


Assemble(bilin, assembler, integration\_rule, temp, temp, mesh.begin(2), mesh.end(2));

que signifie :

system.Solve(b, sol);
Visit(xWriteSolutionVisitor(sol.begin()), vals.begin("dofs"), vals.end("dofs"));

    \chapter{Organisations g\'en\'erale de la librairie}
        

\section{Le role des classes}

Les \doxygenmenu{classes}{classes}  de la librairie peuvent \^etre
regroup�es d'un point de vue fonctionnel en plusieurs grandes
cat�gories. Nous proposons ici de les regarder suivant plusieurs
classifications afin d'avoir une vue d'ensemble des roles de
chacune.

Rappelons, encore une fois, qu'en \code{C++}, une classe d�finit
avant tout un service. Ce service peut \^etre :
\begin{itemize}
\item[-] de stocker des valeurs ou des informations ;
\item[-] de fournir des param\`etrages pour algorithmes
\end{itemize}
L'int�r\^et de ce  dernier type de classes n'est pas de stocker une
information proprement dite, mais de servir d'interface g�n�rale
b�n�ficiant de l'h�ritage, afin que les services demand�s soient
accessibles par la m\^eme fonction. Elles ne servent alors qu'\`a
param\`etrer des algorithmes d'action d�finis dans
\code{xAlgorithm.h}.


\section{Les grandes cat�gories}

(A COMPLETER)

    \chapter{Exemple Xfem}
        
\section{La plaque carr� en traction}

Nous d�crivons ici le probl�me trait� par l'application \code{Xfem/test/mechanics2d/main.cc}.

Nous nous interessons � une plaque carr� en d�formation plane soumise � de la tration suivant l'axe y.

\subsection{La g�om�trie}

La g�om�trie consid�r�e est donn�e par la figure ci-dessous.
\onefigure{5cm}{./test/mechanics2d/square2D}{g�om�trie}

Cette g�om�trie est d�finie dans le fichier "square.geo" (ci-dessous) d�stin� � Gmsh permettant de g�n�rer un maillage \code{square.msh}:

\begin{verbatim}
     nbele = 10;
     nbpt = nbele+1;
     ta = 2/nbele;
     /* Point      1 */
     Point(newp) = {-1.0,-1.0,0.0,ta};
     Point(newp) = {1.0,-1.0,0.0,ta};
     Point(newp) = {1.0,1.0,0.0,ta};
     Point(newp) = {-1.0,1.0,0.0,ta};
     Line(9)  = {1,2};
     Line(10) = {2,3};
     Line(11) = {4,3};
     Line(12) = {1,4};
     Line Loop(18) = {-12,9,10,-11};
     Ruled Surface(21) = {18};    
     Transfinite Line {9,11,10,12} = nbpt Using Power 1.0; 
     Transfinite Surface {21} = {1,2,3,4};
\end{verbatim}

La d�finition de la g�om�trie est faite par la definition :
\begin{itemize}
\item[-] de certaines variables : \code{nelem, nbpt} et \code{ta} ;
\item[-] des points avec les coordonn�es et la taille des �l�ments ;
\item[-] des lignes (line) reliant les points ;
\item[-] d'un contours (line loop) ;
\item[-] des surfaces : ici,  une surface r�gl�e (Ruled Surface) s'appuyant sur un countour ; 
\item[-] le type de maillage : ici, un maillage transfini. 
\end{itemize}

Le maillage g�n�r� par ce fichier est le suivant :
\onefigure{5cm}{./test/mechanics2d/square2D-mesh}{maillage correspondant}

Remarquons que, meme si un maillage transfini a �t� demand�, Gmsh g�n�re un maillage compos� de triangles. Pour obtenir effectivement des quadrangles, il faudrait ajouter la commande \code{Recombine Surface {21};} � la fin du fichier ".geo".

Afin de pouvoir attribuer des conditions aux limites  des �tiquettes Gmsh sont  attach�es � certaines entit�s g�om�triques. Pour les distingu�s, elles sont  qualifi�es de physiques (point, lignes,...) car elles seront utilis�es pour d�finir la physique du probl�me trait� :
\begin{verbatim}
     Physical Point   (101)  = {1} ;
     Physical Point   (102)  = {2} ;

     Physical Line    (109)  = {9};
     Physical Line    (110)  = {10};
     Physical Line    (111)  = {11};
     Physical Line    (112)  = {12};

     Physical Surface (121)  = {21} ;
\end{verbatim}
Ces �tiquettes seront utilis�es par la suite dans le fichier "main.dat" pour d�finir les donn�es du probl�me. Ainsi, la ligne \code{BC\_LINE  111} sur laquelle est d�finie une condition de Neumann correspond � la ligne physique d�clar�e pas \code{Physical Line    (111)  = {11};}, c'est � dire la ligne g�om�trique initialement num�rot�e 11.




\subsection{Le jeu de donn�e "main.dat"}

La g�om�trie et le maillage ayant �t� construits par Gmsh, il convient d�sormais de transmettre ce maillage et de d�finir le probl�me m�canique dans une application xfem. Ceci se fait par la classe \code{xData} et par l'interm�diaire d'un fichier "main.dat". 

Le fichier \code{main.dat} se pr�sente sous la forme d'une liste de mots cl� (\code{MESH\_FILE\_TYPE}, \code{BC\_POINT}, \code{MAT\_CLASS}, ...) permettant de d�finir les param�tres du probl�me. 

\begin{verbatim}
     # lines starting by a # are comment lines
     ##########
     # COMPUTER INFORMATION
     MESH_FILE_TYPE   = msh
     MESH_FILE        = data/square.msh
     ###################################################
     ZONE 121  = {MAT_CLASS = elastic MAT_PARAM = data/law.mat}
     ####################################################
     ##2D case
     BC_LINE  111 ={  TRACTION_Y      FIX =   1.0 }
     BC_LINE  109 ={  TRACTION_Y      FIX =  -1.0 }
     BC_POINT 101 ={  DISPLACEMENT_X  FIX = 0.0 
                      DISPLACEMENT_Y  FIX = 0.0 }              
     BC_POINT 102 ={  DISPLACEMENT_Y  FIX = 0.0 }
\end{verbatim}

Ces mots cl�s seront relus par la classe  \code{xData} qui stockera alors les informations dans diff�rents attributs membres  de classe \code{xMesh}, \code{xPhysicalEnv}, \code{xZoneContainer}, \code{xMaterialManager}, \code{xBoundaryContainer}, \code{xBoundary} ou encore \code{string}.


En particulier, l'objet de classe \code{xPhysicalEnv} contenu dans \code{xData} est un conteneur d'objet \code{xEnv}. Chaque objet \code{xEnv} est construit � la lecture du fichier "main.dat" (par la commande \code{xData.ReadInfo}). Il contient les informations concernant la condition limite lue : 

\begin{center}
\begin{tabular}{p{3cm}|p{2cm}|p{10cm}}
\hline
std::string & Phys & cha�ne de caract�res d�finssant le type de condition limite (par exemple, ici "\code{DISPLACEMENT\_X}"). C'est le nom de l'espace d'approximation concern� par la condition limite.\\
\hline
int & Geom & un entier d�terminant la dimension de l'�l�ment g�om�trique. Il est d�termin� automatiquement :
\begin{itemize}
\item[-] 0 pour un point (\code{BC\_POINT}) ;
\item[-] 1 pour une ligne (\code{BC\_LINE}),
\item[-] 2 pour une surface (\code{BC\_SURFACE})
\item[-] 3 pour un volume (\code{BC\_VOLUME}) 
\end{itemize} \\
\hline
int & Entity & est le num�ro de l'entit� physique d�fini par \code{Physical Point}, \code{Physical Line}, ... \\
\hline
double &  Val\_fix & la valeur � fixer comme condition limite\\
\hline
\end{tabular}
\end{center}

Une fois stock�es ces objets pourront �tre interpr�t�s par le programme principal ("main.cc"). C'est ce que font les fonctions membres \code{TreatmentOfEssEnv()} et \code{TreatmentOfNatEnv()}.
   
   
   



\subsection{La d�finition du mod�le EF}

Le programme principal commence par la lecture des donn�es. Ceci se fait � l'aide des fonctions  \code{ReadInfo()}, \code{ReadMesh()} et \code{ReadZones()}  de l'objet \code{data} (classe \code{xData}) :
\begin{verbatim}
int main(int argc, char *argv[])  
{  
   xData data;
   ...
   data.ReadInfo(pname);
   ...
}
   void Mechanics_c :: TreatmentOfFormulation (xData *data) {

       data->ReadMesh();
       data->ReadZones();
       ... 
     }
\end{verbatim}


Deux espaces d'approximation sont alors d�finis : il s'agit des espaces d'approximation  "DISPLACEMENT\_X",  "DISPLACEMENT\_Y" d�fini sur la base des fonctions d'approximation de Lagrange (\code{xSpaceLagrange})

Il sont construits de mani�re similaire et se distinguent par le type tensoriel d�fini par les param�tres \code{xSpace::VECTOR\_X}  et \code{xSpace::VECTOR\_Y}. Ce param�tre permet de donner la direction des champs d�finis sur ces espaces respectifs (\code{SCALAR, VECTOR\_X, VECTOR\_Y, VECTOR\_Z }).

Le degr� d'approximation est fix� ici � 1 (param�tre \code{xSpaceLagrange::DEGREE\_ONE}.

Les deux espaces sont alors unis pour former l'espace des d�placement suivant les deux directions:
\begin{verbatim}
     xSpaceLagrange lagx("DISPLACEMENT_X", xSpace::VECTOR_X, xSpaceLagrange::DEGREE_ONE);
     xSpaceLagrange lagy("DISPLACEMENT_Y", xSpace::VECTOR_Y, xSpaceLagrange::DEGREE_ONE);
     xSpaceComposite  lagrange(lagx, lagy);
\end{verbatim}

Cet espace de fonction �tant d�fini, le champ de d�placement recherch� est un �l�ments de cet espace. Il est d�finit par les



\begin{verbatim}

  xRegion all(data->mesh);


  xField disp_l(&double_manager, lagrange);

  xValueCreator<xValueDouble>  creator;
  DeclareInterpolation(disp_l, creator, all.begin(), all.end());

  TreatmentOfEssEnv(disp_l, data);
 
  xStateDofCreator<> snh(double_manager, "dofs");
  DeclareState(disp_l, snh, all.begin(), all.end());

  xCSRVector b(double_manager.size("dofs"));
  xCSRVector sol(double_manager.size("dofs"));
  xCSRMatrix A(double_manager.size("dofs"));

  xLinearSystemSolverLU solver;
  xLinearSystem system(&A, &solver);
  xAssemblerBasic<> assembler(A, b);
  xIntegrationRuleBasic integration_rule_env(3);
  xIntegrationRuleBasic integration_rule(3); //2 is the degree to integrate

  TreatmentOfNatEnv(disp_l, assembler, integration_rule_env, data, data->allGroups);

  xUniformMaterialSensitivity<xTensor4> hooke("strain");
  xFormBilinearWithLaw<xGradOperator<xIdentity<xTensor2> >, 
                         xUniformMaterialSensitivity<xTensor4>,
                         xGradOperator<xIdentity<xTensor2> > > diffusive(hooke);
  Assemble(diffusive, assembler, integration_rule, disp_l, disp_l, all.begin(), all.end()); 

  system.Solve(b, sol);
  Visit(xWriteSolutionVisitor(sol.begin()), double_manager.begin("dofs"), double_manager.end("dofs"));


  xExportGmshAscii  pexport;
  xExportGmshBinary  pbinexport;

  xEvalField<xIdentity<xVector> > eval_disp(disp_l);
  Export(eval_disp, pexport, "DISPLACEMENT", integration_rule, all.begin(), all.end());
 
  xEvalGradField<xSymmetrize> eval_strain(disp_l);
  xEvalBinary< xMult<xTensor4, xTensor2, xTensor2> > stress(hooke, eval_strain);
  Export(stress, pbinexport, "STRESS", integration_rule, all.begin(), all.end());

\end{verbatim}

\onefigure{5cm}{./test/mechanics2d/stresses}{g�om�trie}
\onefigure{8cm}{./test/mechanics2d/displacements}{g�om�trie}






\part{Annexes}   \label{annexes}
   \markboth{Annexes}{Annexes}
   \chapter{Annexe A :Documenter \texttt{xfem}}\label{DocGene}
        


\section{Documentation \index{Documentation}}

Le d\'eveloppement de librairies  ne peut se faire sans une
documentation pr\'ecise et syst\'ematique. La documentation de
\code{xfem} a \'et\'e choisie de deux types. D'une part une
documentation syst\'ematique construite \`a partir des les fichiers
sources \code{xObjet.h} et \code{xObjet.cc}, d'autre part une
documentation \'elargie regroupant les aspects informatiques et
th\'eoriques autour de la librairie \code{xfem}.

\subsubsection{Doxygen \index{Documentation!Doxygen}}\label{doxygen}

\webify{http://www.stack.nl/~dimitri/doxygen/}{Doxygen} est une
application permettant la g\'en\'eration automatique de documentation.
Le code C++ est interpr\'et\'e et les structures des classes sont mises
en forme par des sch\'emas de hi\'erarchie de classe et des
descriptions. Ceci permet de conna�tre l'impl\'ementation des classes
sans avoir \`a chercher dans les fichiers sources.
 
Cette documentation de bases peut-\^etre augment\'ee de commentaires
faits par les d\'eveloppeurs. Ces commentaires sont plac\'es dans le
code source lui m\^eme (fichier \code{.h}) par l'interm\'ediaire de
marqueurs particuliers ( voir~:
\webify{http://www.stack.nl/~dimitri/doxygen/docblocks.html}{Documenting
the code} sur le site de Doxygen).

Ces commentaires peuvent \^etre agr\'ement\'e de mise en forme
particuli\`ere par des commandes sp\'eciales (voir~:
\webify{http://www.stack.nl/~dimitri/doxygen/commands.html}{Special
Commands}  sur le site de Doxygen).

L'ex\'ecution de Doxygen s'effectue par la commande~: \code{make doc}.

Les formats de sorties possibles de Doxygen sont le HTML, le LATEX
et le RTF. La documentation Doxygen en html est lisible \`a
l'emplacement
\pdflaunch{../../../../doc/html/main.html}{\texttt{\$DEVROOT/Xfem/Xfem/doc/html/mainpage.html}}.


\subsubsection{La documentation Latex}\label{doclatex}

La version \LaTeX~de la documentation Doxygen n'a pas \'et\'e retenue
comme support pour la documentation \'elargie de la librairie
\code{xfem}. En effet, ce mode de fonctionnement alourdirait la
librairie et serait un frein \`a la lisibilit\'e des lignes de codes
pour les d\'eveloppeurs.

Une documentation \LaTeX~ind\'ependante a \'et\'e mise en place afin de
fournir aux nouveaux utilisateurs~:
\begin{itemize}
\item l'explication des principes de fonctionnement de la plate-forme de d\'eveloppement~;
\item la proc\'edure d'installation des librairies~;
\item des informations sur les diff\'erentes librairies associ\'ees~;
\item des informations sur les outils utilis\'es pour d\'evelopper des applications X-FEM~;
\item les commandes de bases \`a utiliser pour compiler~;
\item les bases th\'eoriques sur la m\'ethode de \'el\'ements finis \'etendus~;
\item les notions d\'ecrites par les principales classes  C++ de \code{xfem} et le lien avec la th\'eorie.
\item un support permettant d'avoir une vue d'ensemble de la librairie.
\end{itemize}

Cette documentation est l'objectif du pr\'esent guide.


\section{Comment est construit ce document}

Ce document est construit \`a partir de fichiers sources
\LaTeX~~situ\'es dans le r\'epertoire~:
 \code{\$DEVROOT/Xfem/Xfem/Xfem/src\_doc/}.


Le fichier principal est le fichier \code{XfemGuide.tex}. Il permet
de g\'en\'erer le document \code{XfemGuide.pdf} \`a partir de diff\'erentes
sources \code{.tex} \index{LaTeX!sources} ainsi que de figures
\index{LaTeX!figures} situ\'ees dans le sous-r\'epertoire
\code{scr\_doc/figures/}.

\onefigure{5cm}{XfemXfemRepository}{arborescence de la
documentation}


\subsection{g\'en\'eration du manuel}

La commande \code{make doc} d\'efinie par
\code{\$DEVROOT/Util/buildUtil/buildUtil/make.common} et n\'ecessaire
\`a la generation de la documentation Doxygen.

La commande \code{make latex}  g\'en\`ere  la pr\'esente documentation
\LaTeX.


% 
% \subsection{Utilisation de script Perl}
% 
% Une des particularit\'es de cette documentation est, \`a l'instar de
% Doxygen, de pouvoir s'appuyer sur une documentation au sein m\^eme des
% codes sources. Cette id\'ee est reprise des m\'ethodes de documentation
% d'ITK (\web{http://www.itk.org/}). Ceci se fait, encore une fois,
% par l'utilisation de commentaires sp\'eciaux.
% 
% 
% Ces commentaires sont interpr\'et\'es par l'utilisation d'un script en
% Perl (repris d'ITK) situ\'e dans le r\'epertoire
% \code{\$DEVROOT/Xfem/Xfem/Xfem} et nomm\'e
% \code{parser\_src\_to\_tex.pl}. Ce
% \guil{parser}\index{LaTeX!parser} g\'en\`ere des fichiers \code{.tex} \`a
% partir des fichiers \code{.h} et  \code{.cc} et les place dans les
% sous-r\'epertoires \code{tex\_from\_h} et \code{tex\_from\_cc}. Pour
% que cela puisse ce faire, les commandes \LaTeX~doivent \^etre
% directement plac\'ees entre les textes \code{Software Guide :
% BeginLatex} et \code{Software Guide : EndLatex} dans des
% commentaires des fichiers \code{.h} et \code{.cc}. Les commentaires
% des fichiers prennent alors la forme de l'exemple suivant~:
% \begin{verbatim}
% /*
%  Software Guide : BeginLatex
%     \section{la classe \code{xValue}}
%     \subsection{Notions}
%     xValue est une classe contenant les informations sur les coefficients 
%     scalaires d'une approximation.
%     \begin{equation}\label{eq:approx_ef2_2}
%     \bfu(\bfx) \mid_{\Ome_e} = \sum_{i \in N_n} \sum_{\alpha} a_i^\alpha
%     \bfphi_i^\alpha(\bfx)
%     \end{equation}
%     Les coefficients scalaires $a_i^\alpha$ sont repr\'esent\'es par des objets de
%     classe \code{xValue}.
%  Software Guide : EndLatex
% */
% \end{verbatim}
% 
% La lecture se fait au fil  du texte et des lignes de codes sources
% \index{LaTeX!code source} peuvent \^etre incorpor\'ees \`a la
% documentation \LaTeX~par l'utilisation de s\'eparateur \code{Software
% Guide : BeginCodeSnippet} et \code{Software Guide : EndCodeSnippet}.
% Le fichier source s'\'ecrit alors~:
% 
% \begin{verbatim}
% /*
%  Software Guide : BeginLatex
%     L'impl\'ementation se fait de la mani\`ere suivante~:
%     Software Guide : EndLatex
% */
% / Software Guide : BeginCodeSnippet
% xValueLinearCombination::xValueLinearCombination(const coeffs_t& c, const
%                                              values_t& v, double coeff_last)
%     : coeffs(c), values(v)
% {
%   std::cout << "creating value average with " << values.size() << " values "
%                                                                 << std::endl;
%   v_last = new xValueDouble;
%   v_last->setVal(1.0);
%   v_last->setState(new xStateOfValueFixed(v_last));
%   coeffs.push_back(coeff_last);
%   values.push_back(v_last);
% }
% // Software Guide : EndCodeSnippet
% /*
%  Software Guide : BeginLatex
%     o� \verb|v_last| est d\'efini par ...
%  Software Guide : EndLatex
% */
% \end{verbatim}
% 
% 
% 
% Les fichiers \code{.tex} ainsi cr\'e\'es peuvent \^etre inclus dans la
% documentation \LaTeX~par une simple commande. Le fichier issu de
% \code{xValue.h} est inclus dans la documentation avec la commande
% \code{{\textbackslash}texfromh\{xValue\}}. Cette commande sp\'eciale
% est l'une des nombreuses commandes d\'ecrite dans la section suivante.
% 
% ATTENTION : l'utilisation du script d\'ecrit ci-dessus est possible,
% mais a \'et\'e d\'esactiv\'ee (mis en commentaire dans le
% \code{make.common}) car inutilis\'e pour le moment.
% 

\subsection{commandes sp\'eciales}

Afin de faciliter la r\'edaction de la documentation, un certain
nombres de commandes ont \'et\'e d\'efinies dans le fichier
\code{XfemGuideConfiguration.tex} par la commande
\index{LaTeX!{\textbackslash}\code{newcommand}}
{\textbackslash}\code{newcommand}.

\subsubsection{commandes de liens hypertextes}

Les commandes permettant de cr\'eer des liens hypertextes d\'ependent du
types de liens et du type d'affichage d\'esir\'e.

\begin{center}
% use packages: array
\begin{longtable}{p{6cm} | p{9cm}}
 \textbf{commandes} & \textbf{description} \\
 \hline
\code{{\textbackslash}pdflaunch\{chemin\}\{texte\}} & Le lien \code{chemin} est cr\'e\'e sur le mot   \code{texte}. Ici,
le chemin peut-\^etre le chemin relatif (\code{../../../../doc/html/figure1.gif}).\\
\hline
\code{{\textbackslash}doxygen\{xClasse\}} &  Le mot   \code{xClasse} est \'ecrit dans le texte et renvoie vers la
documentation   Doxygen \index{Doxygen!classes} concernant cette classe. Pour la classe \code{xValue} il faut \'ecrire
le mot \code{xValue} \`a la place de \code{xClasse}. Le chemin relatif utilis\'e sera
\code{\{../../../../doc/html/classxfem\_1\_1xValue.html\}}      \\
\hline
\code{{\textbackslash}web\{url\}} &  Le lien vers l'adresse \code{url} est cr\'e\'e sur le texte  \code{url}  \\
 \hline
\code{{\textbackslash}webify\{url\}\{texte\}} &  Le lien vers
l'adresse \code{url} est cr\'e\'e sur le texte  \code{texte}. Cette
commande est similaire \`a \code{{\textbackslash}pdflaunch}
\end{longtable}
\end{center}






\subsubsection{Mise en forme de texte particulier}

Des raccourcis permettent de mettre en forme le texte :

\begin{center}
% use packages: array
\begin{longtable}{p{7cm} | p{8cm}}
 \textbf{commandes} & \textbf{description} \\
 \hline
\code{{\textbackslash}code\{cmd\}} & Le texte \guil{cmd} est ecrit : \code{cmd}. Il distingue les mots d\'esignant
des commandes du texte lui-m\^eme. \\
\hline
\code{{\textbackslash}touche\{CAR\}} & Le mot \guil{CAR} est ecrit : \touche{CAR}. Il permet de distinguer le texte de
la touche correspondante au texte. \\
\hline \code{{\textbackslash}guil\{this is written in english\}} &
Le texte \textit{this is written in english} est ecrit entre
guillements : \guil{this is written in english}.
\end{longtable}
\end{center}




\subsubsection{Le code \LaTeX~des fichiers \code{.h}}

 \code{{\textbackslash}texfromh\{xClasse\}} est equivalent \`a la
 commande~:\\ \code{{\textbackslash}input\{../../../../doc/tex\_from\_h/xClasse.tex\}}.



\subsubsection{Les figures}

Des raccourcis permettent de mettre en forme rapidement les figures
et les informations qui leur sont associ\'ees. Une seule ligne de
commande et n\'ecessaire pour introduire une figure dans la
documentation :



\begin{center}
% use packages: array
\begin{longtable}{p{8cm} | p{7cm}}
 \textbf{commandes} & \textbf{description} \\
 \hline
\code{{\textbackslash}onefigure\{larg\}\{dess\}\{leg\}} &  le fichier image \code{dess} est introduit comme image
de largeur \code{larg}. Le nom du fichier est introduit sans extension, et les extensions .pdf, .png, .jpg, .eps, .mps
sont successivement test\'ees pour chercher le fichier correspondant. La figure pourra alors \^etre appel\'ee par la r\'ef\'erence :
\code{{\textbackslash}ref\{fig$:$dess\}}. \code{leg} est la l\'egende qui sera appos\'ee sous la figure.\\
\hline
\code{{\textbackslash}twofigures\{l1\}\{fic1\}\{l2\}\{fic2\} \{leg\}\{lab\}} &  Cette commande permet de pr\'esenter une figure
compos\'ee de deux sub-figures \index{LaTeX!subfigures} provenant des fichiers \code{fic1} et \code{fic2}, et affich\'ees avec
 les largeurs respectives \code{l1} et \code{l2}.   Les noms de fichier sont introduits sans extension, et les extensions .pdf,
 .png, .jpg, .eps, .mps sont successivement test\'ees pour chercher les fichiers correspondants. La figure pourra alors \^etre
 appel\'ee par la reference : \code{{\textbackslash}ref\{lab\}} alors que chaque sub-figures aura pour r\'ef\'erence respectivement
  \code{{\textbackslash}ref\{fig$:$fic1\}}  et \code{{\textbackslash}ref\{fig$:$fic2\}} . \code{leg} est la l\'egende qui sera
   appos\'ee sous la figure.\\
\hline
\code{{\textbackslash}fourfigures\{l1\}\{fic1\}\{l2\}\{fic2\} \{l3\}\{fic3\}\{l4\}\{fic4\}\{leg\}} &  Cette commande permet de
pr\'esenter une figure compos\'ee de quatre sub-figures provenant des fichiers \code{fic1},  \code{fic2},  \code{fic3} et \code{fic4}
 de  largeurs respectives \code{l1}, \code{l2}, \code{l3}   et \code{l4}, de r\'ef\'erence
 respective \code{{\textbackslash}ref\{fig$:$fic1\}}, \code{{\textbackslash}ref\{fig$:$fic2\}},
  \code{{\textbackslash}ref\{fig$:$fic3\}} et \code{{\textbackslash}ref\{fig$:$fic4\}}.
     La figure globale n'a pas de r\'ef\'erencement.  La l\'egende  appos\'ee sous la figure globale sera  \code{leg} \\
\hline
\code{{\textbackslash}fourturnedfigures\{l1\}\{fic1\}\{l2\} \{fic2\}\{l3\}\{fic3\}\{l4\}\{fic4\}\{leg\}} &  Cette
 commande est identique \`a la pr\'ec\'edente mais chaque figure est tourn\'ee  de $90^\circ$.  \\
\hline
\code{{\textbackslash}inheritgraph\{larg\}\{xClass\}} &  le   diagramme d'h\'eritage de la classe  \code{xClass} est
introduit comme image de largeur \code{larg}. Une l\'egende automatique est appos\'ee sous la figure.\\
\hline
\code{{\textbackslash}collabgraph\{larg\}\{xClass\}} &  le   diagramme de collaboration de la classe  \code{xClass}
est introduit comme image de largeur \code{larg}. Une l\'egende automatique est appos\'ee sous la figure.\\
\hline
\code{{\textbackslash}dependgraph\{larg\}\{xClass\}} &  le   diagramme de dependance de la classe  \code{xClass} est
introduit comme image de largeur \code{larg}. Une l\'egende automatique est appos\'ee sous la figure.\\
\hline
\end{longtable}
\end{center}

Exemples :

\begin{verbatim}
\onefigure{4cm}{efg_support}{Trois supports couvrant un point $\bfx$.}

\twofigures{6cm}{uniform}{6cm}{general}{Surface de discontinuit\'e
plac\'ee sur un maillage uniforme (a) et non uniforme (b). Les noeuds
encercl\'es sont enrichis par la fonction Heaviside.}{fig:both_disc}

\fourfigures90{5cm}{f1-1}{5cm}{f1-2}{5cm}{f1-3}{5cm}{f1-4}
{visualisation des fonctions $\{F_1^l(\bfx)\}$}
\end{verbatim}




\subsubsection{Les \'equations \index{LaTeX!\'equations}}

Des raccourcis permettent \'egalement de mettre en forme les \'equations
en respectant les notations. La liste est longue. Nous invitons les
personnes int\'eress\'ees \`a aller voir le fichier
\code{XfemGuideConfiguration.tex}.

   \chapter{Annexe B : Le langage C++}
        

L'ensemble des lignes de codes de \code{xfem}, \code{xext} ou
\code{xfem} utilise des notions propre au C++, \`a la librairie STL ou
m\^eme la librairie boost. Pour les d\'ebutants en C++ qui ont des
notions de programmation dans un autre langage (en Fortran, par
exemple), cette partie a pour vocation d'introduire bri\`evement  ce
nouveau langage et ces librairies en d\'egageant se qu'il faut en
retenir.


Les habitu\'es du C++ peuvent passer cette section.

\section{Les grandes lignes}


\subsection{la synthaxe :}

Dans les exemples qui suivent, certains termes sont mis entre
crochets ( \code{< >}). Cette notation d\'esigne un param\`etre \`a
l'interieur d'une syntaxe. Par exemple, le type d'une variable not\'ee
\code{<type>} peut \^etre remplac\'ee par :
\begin{center}
\begin{tabular}{ p{3cm} p{9cm} }
\hline
Type      &      Description \\
\hline
\code{int} & entier   \\
\code{unsigned int} & entier positif   \\
\code{long} & entier double pr\'ecision  \\
\code{unsigned long} & entier double pr\'ecision positif   \\
\code{float} & r\'eel  \\
\code{double} & r\'eel double pr\'ecision \\
\code{char} & caract\`ere   \\
\code{string} & cha�ne de caract\`eres    \\
\code{bool}  & bool\'een    \\
\hline
\end{tabular}
\end{center}


\vspace{2em} le corps du texte s'\'ecrit en tenant compte des r\`egles
suivantes :
\begin{center}
\begin{tabular}{ p{3cm} p{9cm} }
\hline
Signe      &      Description \\
\hline
\code{;} & les lignes se terminent par un point-virgule \\
 \{ \hspace{1cm} \} & les fonctions sont \'ecrites entre accolades \\
\code{//} & les commentaires s'\'ecrivent apr\`es un double slash\\
\code{const} & devant une d\'eclaration, permet de d\'efinir une constante valable dans tout le programme  \\
\hline
\end{tabular}
\end{center}

\vspace{2em} Le programme principal s'\'ecrit comme une fonction :

\begin{verbatim}
                int main(int argc, char *argv[])
                {
                     ...
                }
\end{verbatim}



%\subsection{Les op\'erateurs d'affectation}
\vspace{2em} En plus des op\'erations classiques sur les variables, le
C++ poss\`edent quelques raccourcis syntaxique pour les  op\'erateurs
d'affectation :
\begin{center}
\begin{tabular}{ p{3cm} p{9cm} }
\hline
Op\'erateur      &      Action \\
\hline
a++      &      incr\'ementation de \code{a} \'equivalent \`a \code{a=a+1} \\
a\-\-             &      d\'ecr\'ementation de \code{a} \'equivalent \`a \code{a=a-1} \\
a += b             &      \'equivalent \`a \code{a=a+b} \\
b=a++      &       \code{b} re�oit la valeur \code{a} puis \code{a} est incr\'ement\'e \\
b=++a       &       \code{a} est incr\'ement\'e puis \code{b} re�oit la valeur \code{a}   \\
\hline
\end{tabular}
\end{center}



\vspace{2em}
%\subsection{Les op\'erateurs logiques}
les op\'erateurs logiques ressemblent  \`a ceux de nombreux codes :
\begin{center}
\begin{tabular}{ p{3cm} p{9cm} }
\hline
Op\'erateur      &      Signification \\
\hline
$==$             &      \'egal \`a\\
$!=$             &      diff\'erent de \\
$>$ \hspace{1cm} $<$      &      plus grand que, plus petit que       \\
$>=$ \hspace{.75cm} $<=$      &       plus grand ou \'egal que, plus petit ou \'egal que             \\
\&\&       &      ET logique (AND)\\
$\|\|$             & OU logique (OR)      \\
!             & contraire de {NOT}\\
\hline
\end{tabular}
\end{center}


\vspace{2em}
%\subsection{Les pointeurs}
Les pointeurs sont largement utilis\'es en C++. Leur utilisation   se
fait de la synthaxe suivante :
\begin{center}
\begin{tabular}{ p{3cm} p{9cm} }
\hline
Op\'erateur      &      Signification \\
\hline
\code{\&a}       &      adresse de  \code{a}) \\
\code{int* p}       &       d\'eclaration d'un pointeur de \code{int} nomm\'e \code{p}\\
\code{*p}      &       la chose point\'ee par  \code{p})\\
\code{<type>*}       &       d\'eclaration d'un pointeur de \code{<type>}\\
\code{*(...\&)}  &       = identit\'e. Le contenu de l'adresse d'une variable est la variable elle-m\^eme. \\
\hline
\end{tabular}
\end{center}


\vspace{2em}
%\subsection{Les controleurs}
les controleurs sont classiques. La syntaxe est la suivante :
\begin{center}
\begin{longtable}{ p{6cm} p{6cm} }
\hline
Contr�leur      &      Syntaxe \\
\hline
tant que... &  \code{while( \emph{ cond }) \{   }         \\
            &  \code{       \emph{action}   }            \\
            &  \code{                      \} }            \\
\\pour...   &  \code{for( int i=0 ; i < max ; i++) }       \\
            &  \code{       \emph{action}   }            \\
            &  \code{                      \} }            \\
\\ sortie de boucle pour   &  \code{break;}        \\
\\ contituation imm\'ediate de boucle pour   &  \code{continue;} \\
\\ selon... faire  &  \code{switch( nValeur)\{}            \\
                &  \code{    case \emph{val1}:     }            \\
                &  \code{                           ...; }            \\
                &  \code{                           break; }            \\
                &  \code{    case \emph{val2}:  }            \\
                &  \code{                           ...; }            \\
                &  \code{                           break; }            \\
                 &  \code{    defaut; ... }            \\
                &  \code{ \} }            \\
\hline
\end{longtable}
\end{center}



\subsection{Les fonctions :}

Le C++ ne conna�t que les fonctions. L'\'equivalent d'un
\emph{sous-programme} ou \emph{subroutine} serait une fonction de
type vide (\code{void}) dont  les param\`etres sont les variables de
passage. Une fonction s'\'ecrit sous la forme :
\begin{verbatim}
        <type_fcn> nom(<type_arg1> argument1,<type_arg2> argument2)
        {
            ...
        return <expression>
        }
\end{verbatim}
Une fonction peut \^etre utilis\'ee sans \^etre \'ecrite \emph{a priori} ;
le compilateur a alors besoin du \emph{prototype} de la fonction,
c'est \`a dire les lignes de d\'eclaration :
\begin{verbatim}
        <type_fcn> nom (<type_arg1> argument1,<type_arg2> argument2)
        {
            //  un petit commentaire peut \^etre le bienvenu tout de m\^eme...
        };
\end{verbatim}
Le \emph{prototype} doit \^etre inclus via la commande \code{\#include
"fonction"}   lorsqu'il est \'ecrit dans le fichier \code{fonction.h}.


La particularit\'e des fonctions en C++ est qu'elles ne modifient pas
la valeur des arguments, tout comme $y=f(x)$ fournit une valeur \`a
$y$ sans modifie $x$. C'est la valeur de $x$ au moment de l'appel
qui est transmise \`a $f()$. Pour pouvoir modifier une valeur il
convient de construire la fonction $f()$ en passant l'adresse de
$x$, par exemple, si \code{pVal} utilise l'adresse de \code{val}:
\begin{verbatim}
        void fcn(double* pVal)
        {
        };
\end{verbatim}
Ci-dessus, le param\`etre d\'eclar\'e \`a la fonction est un pointeur.

L'appel se fait par l'adresse de la valeur :
\begin{verbatim}
        double val=10.5
        fcn(val&)
\end{verbatim}


Une autre fa�on est la passage par \emph{r\'ef\'erence}, c'est \`a dire
par l'adresse (le pointeur). La r\'ef\'erence n'est pas \`a proprement
parl\'e un vrai type, mais une syntaxe particuli\`ere pour indiquer que
l'on peut modifier le contenu d'un param\`etre (argument) :
\begin{verbatim}
        void fonction(double& val)
        {
        }
\end{verbatim}
Ci-dessus, le param\`etre d\'eclar\'e \`a la fonction est une r\'ef\'erence \`a la
valeur.


L'appel se fait par la valeur elle-m\^eme :
\begin{verbatim}
        double val=10.5
        fcn(val)
\end{verbatim}


Il convient de faire attention \`a la port\'ee des variables internes
des fonctions:
\begin{itemize}
\item une variable locale d'une fonction est inconnue du programme principal ;
\item une variable locale \code{static}  est connue du programme principal ;
\item une variable locale d'une fonction d\'eclar\'ee par \code{new} est gard\'ee en m\'emoire est utilisable dans le programme principal (peut \^etre d\'etruite par la fonction \code{delete}).
\end{itemize}

\subsection{Les classes}

Il existe, comme en fortran, la notion de \code{structure}, mais est
et peu utilis\'ee car contenu dans la notion de \code{class}. Seule la
notion de \code{class} est donc \`a retenir. Les classes sont
largement utilis\'ees en C++. C'est la base m\^eme de la Programmation
Orient\'ee Object (POO).


La syntaxe de d\'eclaration d'une classe est la suivante :
\begin{verbatim}
        class <non_de_la_classe>
        {
            public:
            static <type_1> <nom_1> ;
            <type_...> <nom_...> ;
            ...
            <type_fcn1> <nom_fcn1>
            {
            }
            protected:
            <type_2> <nom_2> ;
            <type_...> <nom_...> ;
            ...
            <type_fcn2> <nom_fcn2>
            {
            }
            private:
            <type_3> <nom_3> ;
            <type_...> <nom_...> ;
            ...
            <type_fcn3> <nom_fcn3>
            {
            }
            ...
            ...
        };
\end{verbatim}
Les objets sont des instances d'une classe, tout comme une variable
est l'instance d'un r\'eel (par exemple), c'est \`a dire un \'el\'ement
d'un certain ensemble (type). Les objets peuvent donc \^etre point\'es
ou r\'ef\'erenc\'e pour une fonction comme n'importe quel autre type de
variable. On peut \'egalement affect\'e la m\'emoire sur le tas en
utilisant un \code{new} lors de la d\'eclaration \`a l'int\'erieur d'une
fonction.

\note{les membres de donn\'ees \code{static} sont des membres commun \`a
tous les objets de la classe (par exemple, un num\'ero qui peut \^etre
incr\'ementer \`a chaque cr\'eation d'objet). C'est un variable globale
qui doit \^etre initialiser dans le programme principal.}

L'utilisation d'une fonction \code{demarrer} ou variable publique
\code{longueur} d'un  objet \code{maVoiture} de classe
\code{vehicule} se fait  par l'interm\'ediaire d'un point \guil{.}
selon la synthaxe  \code{objet.fonction()}, par exemple:
\begin{verbatim}
        vehicule maVoiture;
        vmaVoiture.demarrer();
        maVoiture.longueur = 10.;
\end{verbatim}
Cependant, il est pr\'ef\'erable de travailler avec uniquement des
fonctions. L'id\'eal \'etant  d'avoir une fonction publique \code{fixeLongueur} (la variable \code{longueur} sera alors rendue \code{protected})
dans la classe du type \code{vehicule}  pour \'ecrire :
\begin{verbatim}
        vehicule maVoiture;
        maVoiture.demarrer();
        maVoiture.fixeLongueur(10.);
\end{verbatim}
les fonctions internes \`a la classe sont appel\'ees fonctions
\emph{membres}. L'ensemble des fonctions membres   d'une classe
d\'efinit l'\emph{interface} de la classe. Cette interface permet
l'utilisation de la classe, m\^eme si l'impl\'ementation des fonctions
\'evolue. Le prototype d'une classe peut \^etre (fortement conseill\'e) \'ecrit dans un fichier
inclus \code{*.h}. Ainsi, il est possible de d\'eclarer une fonction
membre sans la programmer. Elle est d\'eclar\'ee et utilisable. La
programmation pouvant se faire au fur et \`a mesure des cas \`a traiter
ou des d\'erivations de la classe. Le code source de la classe peut
\^etre \'ecrit dans un autre fichier \code{*.cc} comme n'importe quelle
autre fonction moyennant l'utilisation de sont nom complet  :
\begin{verbatim}        nomDeLaClassnom::DeLaFonction()  \end{verbatim}
(les fonctions membres peuvent \^etre surcharg\'ees si n\'ecessaire).


Lorsqu'on utilise le pointeur d'un objet, l'acc\`es aux fonctions
membres se fait par l'op\'erateur \code{->} :
\begin{verbatim}
        vehicule maVoiture;
        vehicule* ptmaVoiture = &maVoiture;

        ptmaVoiture->demarrer();        // est \'equivalent \`a  :
        maVoiture.demarrer();

        // alors que ceci est faut :
        *ptmaVoiture.demarrer();

        // mais ceci serait acceptable bien que peu employ\'e :
        (*ptmaVoiture).demarrer();
\end{verbatim}


Une fonction particuli\`ere appel\'ee \emph{constructeur} ou
\emph{op\'erateur parenth\`ese} permet d'initialiser un objet \`a �a
d\'eclaration.
\begin{verbatim}
        class vehicule
        {
            public:
            vehicule()
            {
                // en sortie d'usine : initialiser le compteur \`a z\'ero
                ...
            }
        }
\end{verbatim}
Il peut en exister plusieurs selon le nombre et le type d'argument pass\'es pour construire l'objet.


Une autre fonction particuli\`ere appel\'ee \emph{destructeur} ou
\emph{op\'erateur ~} permet de lib\'erer la m\'emoire contenant l'objet.
Elle doit g\'en\'eralement contenir la fonction \code{delete} si le
cr\'eateur contient  la commande \code{new}.


Encore une autre fonction particuli\`ere appel\'ee \emph{constructeur de
copie}  permet au C++ de construire une copie d'un object lorsqu'il
est pass\'e dans une fonction. Il permet, en fait, de construire la
r\'ef\'erence de l'objet et donc de faire une copie \guil{parfaite} de l'objet, en particulier lorsque la structure de l'object est complexe ou de taille variable, ce que ne ferait pas un constructeur par d\'efaut. Il s'\'ecrit :
\begin{verbatim}
        class vehicule
        {
            public:
            vehicule(vehicule& v)
            {
               ...
            }
        }
\end{verbatim}

La d\'eclaration des classes d\'eriv\'ees se fait de la mani\`ere suivante :
\begin{verbatim}
        class voiture : public vehicule
        {
           public:
            voiture(...) : ... , ... // ajouter les constructeurs
                                     // des classes composant
                                     // la classe voiture
            {...}
        }
\end{verbatim}


Pour beneficier du polymorphisme, les functions membres doivent \^etre
d\'eclar\'ee \code{virtuel}.
\begin{verbatim}
        class vehicule
        {
            public:
            vitual void FaireLePlein()
            ...
        };
\end{verbatim}
La fonction virtuelle sera alors d\'efinie au cas par cas en fonction
des classes d\'eriv\'ees (par exemple : vehiculeDiesel, vehiculeEssence,
vehiculeGPL,...) car la fa�on de FaireLePlein diff\`erent  en fonction
du type de v\'ehicule, alors que la fonction de FaireLePlein doit
toujours exister (en tout cas pour les v\'ehicules motoris\'es).


Les fonctions membres \code{statique}s et les constructeurs ne
peuvent pas \^etre d\'eclar\'ees virtuelle alors que les d\'estructeurs
devraient \^etre toujours virtuel.



Les classes \guil{abstraites} sont des classes qui ne peuvent \^etre
utilis\'ees telles quelles. Seule leurs classes d\'eriv\'ees peuvent \^etre
instanci\'ee. Ce sont des classes contenant une ou plusieurs
\guil{fonctions virtuelles pures}, c'est \`a dire ne contenant pas
d'impl\'ementation. Ce type de fonction est d\'eclar\'ees par :
\begin{verbatim}
        virtual void fonction_toto(...) = 0;
\end{verbatim}
Les classes \guil{abstraites}, m\^emes si elles ne peut pas \^etre
instanci\'ee, peuvent \^etre pass\'ee comme param\`etre de fonction car lors
de l'appel de cette fonction la classe r\'eellement pass\'ee sera une
sous-classes concr\`ete.


Les fichiers inclus d\'ecrivant les classes prennent la forme suivante
:
\begin{verbatim}
       #ifndef _MaClasse_
       #define _MaClasse_

       #include <...>
       #include "xToto.h"
       ...

       namespace xfem
       {
              class MaClasse {
              private:
                  ...
              public:
                  ...
              };
       }
       #endif
\end{verbatim}
Ici, le \code{namespace xfem} signifie que la classe est d\'efinie
dans la librairie \code{xfem}.

Le programme principale, quant \`a lui s'\'ecrira sous la forme suivante
:
\begin{verbatim}
       #include <...>
       #include "..."
       ...

       using namespace xfem;
       int main(int argc, char *argv[])
       {
             ...
       }
\end{verbatim}
Ici,  \code{using namespace xfem} pr\'ecise que les classes appel\'ee
sans pr\'ecision appartiennent \`a l'espace de nom \code{xfem}.

\section{Les \code{Templates}}

Les \code{Template} sont une fa�on de d\'eclar\'ee une classe (ou une fonction) de mani\`ere \guil{g\'en\'erique}, c'est \`a dire sans donner explicitement le type des variables ou classes utilis\'ees \`a l'int\'erieur.
Une classe \code{template <class T> class liste} permet de d\'efinir une classe de liste sans pr\'eciser le type du contenu de la liste. C'est un \guil{patron de classe}. Elle sera \guil{typ\'ee} lors de l'instanciation d'une liste particuli\`ere, par exemple :
\code{liste<int> ma\_liste\_d\_entier}.

La classe peut \^etre templatis\'ee plusieurs fois, si elle n\'ecessite l'utilisation de plusieurs types diff\'erents d'objets : \code{template <class T, class U> class A}.

(voir les livres sp\'ecialis\'es pour plus d'information)




\section{gcc \index{gcc}}

voir les site officiel de gcc :
\web{http://www.gnu.org/software/gcc/gcc.html}


\section{STL: Standard Template Library \index{Library!stl}}

The Standard Template Library (STL) is a software library included
in the C++ Standard Library. It provides \textbf{containers},
\textbf{iterators}, and \textbf{algorithms}. More specifically, the
C++ Standard Library is based on the STL published by SGI. Both
include some features not found in the other. SGI's STL is rigidly
specified as a set of headers, while ISO C++ does not specify header
content, and allows implementation either in the headers, or in a
true library.

The  STL provides many of the basic algorithms and data structures
of computer science. The STL is a generic  library, meaning that its
components are heavily parameterized: almost every component in the
STL is a template. You should make sure that you understand how
templates work in C++ before you use the STL. Containers and
algorithms

Like many class libraries, the STL includes container classes:
classes whose purpose is to contain other objects. The STL includes
the classes \textbf{vector}, \textbf{list}, \textbf{deque},
\textbf{set}, \textbf{multiset}, \textbf{map}, \textbf{multimap},
\textbf{hash\_set}, \textbf{hash\_multiset}, \textbf{hash\_map}, and
\textbf{hash\_multimap}. Each of these classes is a template, and
can be instantiated to contain any type of object.

SGI documentation can be found here:
\web{http://www.sgi.com/tech/stl/}

A large French documentation can be found here:
\web{http://www.moteurprog.com/Tutoriaux/VisiteTuto.php?ID_tuto=92}




\section{La librairie BGL: Boost Graphic Library}


\textit{(\`a faire...)}

  % \chapter{Annexe C :  Compl\'ements d'informations \\ \textnormal{(contribution de E. Bechet)}}
  %       
\section{External Installation of \code{xfem} with CVS}

\subsection{Installation on a linux machine}

Remote access to ECN's CVS server is possible using ssh. To get
\code{xfem} library installed and functional on an external machine,
some other (\guil{downloadable}) tools are needed:
\begin{itemize}
    \item \code{boost} (\web{www.boost.org})
    \item \code{mtl} (\web{www.osl.iu.edu/research/mtl})
    \item \code{Trellis} (\web{www.scorec.rpi.edu/Trellis})
\end{itemize}
\begin{enumerate}
\item  Create your development directory (exp. \code{develop/} ). Then set your environment variables:
\begin{itemize}
\item  For versions sh, bash,.. :
\begin{verbatim}
export CVS_RSH=ssh export CVSROOT=:ext:public@cvs.ec-nantes.fr:/cvs
\end{verbatim}
\index{CVS!CVS RSH}\index{CVS!CVSROOT} In your \code{.bashrc},
define path to your development directory (exp:  \code{export
DEVROOT /scratch/user/develop} )

\item For versions csh, tcsh,.. :
\begin{verbatim}
setenv CVS_RSH ssh setenv CVSROOT :ext:$public@cvs.ec-nantes.fr:/cvs
\end{verbatim}


In your \code{.bashrc}, define path to your development directory
(exp:  \code{setenv DEVROOT /scratch/user/develop} )
\end{itemize}
\item Check out the needed tools:  (see \code{Tools} for description)

\begin{verbatim}
cd $DEVROOT cvs checkout Util
\end{verbatim}



\item Check out the library \code{xfem}:

\begin{verbatim}
cd $DEVROOT cvs checkout Xfem
\end{verbatim}

\item You need now to get the tools required for pre-compilation. Start by \code{Util} (which you've just installed)

\begin{verbatim}
cd $DEVROOT cd Util/buildUtil/buildUtil
\end{verbatim}

\begin{verbatim}
make setup NODEP=1 make
\end{verbatim}

\item Compile Loki:

\begin{verbatim}
cd $DEVROOT cd Util/Loki/Loki
\end{verbatim}

\begin{verbatim}
make setup NODEP=1 make
\end{verbatim}

\item Download boost and \guil{unzip} it in \code{/usr/local/include/} :
\begin{verbatim}
cp your-boost-version.tar.gz /usr/local/include/ cd
/usr/local/include/ tar -xzvf your-boost-version.tar.gz
\end{verbatim}


Create the link that will allow xfem find boost's includes (\code{in
/usr/local/include/}):

\begin{verbatim}
ln -s your-boost-version/boost boost
\end{verbatim}

\item Download mtl and install it in a repository \code{Solver/} :

\begin{verbatim}
cd $DEVROOT mkdir Solver
\end{verbatim}

\begin{verbatim}
cd Solver/ tar -xzvf your-mtl-version.tar.gz
\end{verbatim}
Then compile it:

\begin{verbatim}
cd your-mtl-version ./configure make make install
\end{verbatim}

As for boost, create a link to the repository where mtl is
installed:
\begin{verbatim}
cd $DEVROOT/Solver ln -s your-mtl-version mtl
\end{verbatim}

\item  \code{C++} includes \code{hash\_fct}, \code{hash\_map} and \code{hash\_set} are available as \code{RPM} or \code{tar}
archives. Copy them in \code{/usr/local/include}

\item Download and unzip Trellis:

\begin{verbatim}
cd $DEVROOT tar -xzvf your-Trellis-version.tar.gz
\end{verbatim}

Compile the needed libraries (AOMD and Model)

\begin{verbatim}
cd Trellis/Trellis/model/model make setup NODEP=1 make
\end{verbatim}

\begin{verbatim}
cd Trellis/Trellis/AOMD/AOMD make setup NODEP=1 make
\end{verbatim}
\end{enumerate}



\subsection{Installation on a Windows machine:}
\begin{enumerate}
\item Install cygwin (www.redhat.com )
\item  In a cygwin shell, follow the previous commands
\end{enumerate}








\section{Le serveur CVS de l'ECN (pour administrateur)}

Le serveur CVS est situ\'e \`a l'adresse : \code{cvs.ec-nantes.fr}. Il
s'agit d'une machine d'architecture Intel avec un syst\`eme
d'exploitation standard Red Hat linux 9.

\subsection{Syst\`eme CVS}
Le syst\`eme CVS est utilis\'e pour le d\'eveloppement de code. Il est un
outil de collaboration pour les d\'eveloppeurs travaillant sur des
projets communs. De plus, il permet de garder une trace de toutes
les modifications effectu\'ees (stockage incr\'emental). Du point de vue
utilisateur, son interface est standard \`a partir du moment ou
l'authentification est faite, nous ne nous attarderons pas sur ce
point dans la suite. Un manuel d'utilisation est disponible \`a
l'adresse suivante :

\web{ http://www.cvshome.org/}.

L'authentification est faite par un acc\`es ssh, du point de vue
utilisateur (sur linux ou unix) il faut simplement d\'efinir les
variables d'environnement suivantes :

\begin{verbatim}
CVSROOT = :ext:<point d'acc\`es au serveur CVS>
\end{verbatim}


\begin{verbatim}
CVS_RSH=ssh
\end{verbatim}


Le point d'acc\`es au serveur CVS est d\'ependant du mode d'acc\`es et
sera d\'etermin\'e dans la suite. Lors de l'acc\`es au serveur, un mot de
passe sera \'eventuellement demand\'e pour toute commande, par exemple
\code{cvs checkout} ou encore \code{cvs commit} . Pour \'eviter la
demande syst\'ematique d'un mot de passe il faut implanter dans le
serveur CVS les clef de la machine locale. Ceci peut \^etre fait \`a
l'aide d'une commande sp\'eciale.

\subsection{Implantation du serveur CVS \`a l'ECN}

Le serveur CVS est dot\'e d'un certain nombre de scripts  (voir le
fichier \code{/usr/local/cvs/doc/README} sur le serveur CVS). Il
s'agit d'un serveur s\'ecuris\'e permettant un acc\`es anonyme sur
certains projets, et un acc\`es contr�l\'e sur d'autres. La protection
des donn\'ees et de l'int\'egrit\'e du serveur sont assur\'es de deux fa�ons
: d'une part par une authentification et des \'echanges de donn\'ees
crypt\'es (seul le protocole ssh est utilis\'e lors des requ\^etes
r\'eseau), et d'autre part par un masquage du syst\`eme de fichiers
racine de la machine ( commande \code{chroot} ). Les scripts d\'ecrits
dans la suite sont invoqu\'es pour les actions suivantes : cr\'eation
d'un nouveau projet, l'ajout d'utilisateurs aux droits restreints \`a
un seul projet, la cr\'eation d'un module dans un projet
(sous-projet). Cette derni\`ere action peut aussi \^etre effectu\'ee
directement par un utilisateur ayant des droits d'\'ecriture dans un
projet. Des utilisateurs privil\'egi\'es peuvent avoir un acc\`es \`a
plusieurs modules \`a la condition de ne pas faire de \code{chroot}.
Ceci sera explicit\'e dans la suite.

\subsection{Installation}

Dans le cas d'une r\'einstallation, il faut utiliser l'archive
(version 1.5) des scripts d'administration:
\code{chrooted-ssh-cvs.1.5.tar.gz} et le d\'ecompresser dans le
r\'epertoire \code{/usr/local} :
\begin{verbatim}
cd /usr/local tar -zxf <chemin>/chrooted-ssh-cvs.1.5.tar.gz
\end{verbatim}

Une documentation est fournie avec (\'equivalente \`a
\code{/usr/local/cvs/doc/README} sur le serveur CVS) :
\code{chrooted-ssh-cvs.1.5.README}


\subsection{Modes d'acc\`es}
Il existe 2 modes d'acc\`es, chacun permettant un acc\`es en lecture
uniquement ou en lecture et \'ecriture. Le choix est d\'etermin\'e \`a la
cr\'eation de l'utilisateur mais peut \^etre modifi\'e ult\'erieurement.
\begin{enumerate}
\item  \underline{mode restreint :}
Ce mode est destin\'e aux utilisateurs d'un seul projet et disposant
de droits de lecture (\'eventuellement d'\'ecriture) dans ce projet. La
connexion est accord\'ee apr\`es authentification par \code{ssh} sous
forme d'un acc\`es \`a certains outils dans un environnement restreint
(apr\`es \code{chroot}) : \code{cvs} (commande serveur), et quelques
outils de modification de mot de passe (\code{password}) et d'import
de clef cryptographique pour acc\`es sans mot de passe
(\code{authorize}). Il n'y a pas de shell accessible. Dans ce mode
d'acc\`es, le \code{<point d'acc\`es au serveur CVS>} prend la forme
suivante:

\begin{verbatim}
CVSROOT=:ext:<user>@cvs.ec-nantes.fr:/cvs
\end{verbatim}



Le projet est d\'etermin\'e par \code{<user>} (dans ce mode, l'usager
n'a acc\`es qu'a un seul projet). C'est le mode d'acc\`es le plus
s\'ecuritaire, il est en particulier possible de l'utiliser en lecture
seule pour un acc\`es anonyme aux projets de nature libre (au sens
logiciel libre)

\item \underline{mode \'etendu :}
Ce mode permet \`a un utilisateur d'acc\'eder \`a un ou plusieurs projets,
en lecture ou lecture et \'ecriture. De ce fait, cet utilisateur ne
passera pas par une \'etape de\code{chroot} mais sera tout de m\^eme
authentifi\'e par \code{ssh}. Cet acc\`es ne devrait jamais \^etre utilis\'e
pour les acc\`es anonymes. Si possible (et en particulier si un usager
ne d\'esire travailler que sur un projet), il est pr\'ef\'erable
d'utiliser le mode restreint. Dans le mode d'acc\`es \'etendu, le
\code{<point d'acc\`es au serveur CVS>} prend la forme suivante:


\begin{verbatim}
CVSROOT=:ext:<user>@cvs.ec-nantes.fr:/usr/local/cvs/<proj>/chrooted-cvs/cvs
\end{verbatim}


Ici l'usager \code{<user>} ne d\'etermine pas le projet, c'est la
variable \code{<proj>} qui permet de s\'electionner le projet dans
lequel il d\'esire travailler. Selon la configuration du serveur, les
usagers \'etendus peuvent avoir acc\`es \`a un shell et donc \`a toutes les
commandes unix. Les projets sont en principe accessibles en lecture,
mais l'acc\`es en \'ecriture n\'ecessite d'appartenir au groupe du projet
en question.
\end{enumerate}



\subsection{Administration}
Toutes les actions d'administration se font en tant que \code{root}
sur la machine cvs :

\begin{verbatim}
ssh -X root@cvs.ec-nantes.fr
\end{verbatim}



\subsubsection{Cr\'eation d'un nouveau projet}

La cr\'eation d'un nouveau projet se fait \`a l'aide du script
\code{make-project}, celui-ci se chargeant de faire toutes les
actions n\'ecessaires : cr\'eation de groupes, de r\'epertoires, et mise
en place de l'espaces restreint (pour \code{chroot}). La syntaxe de
la commande est la suivante pour la cr\'eation du projet \code{<proj>}
:
\begin{verbatim}
/usr/local/cvs/sbin/make-project [-a] <proj>
\end{verbatim}

L'option \code{-a} permet l'acc\`es anonyme sur ce projet. Il est
possible de modifier cela par la suite (droits de lecture des
fichiers par tout utilisateur). Il est pr\'ef\'erable de mettre cette
option par d\'efaut puisque l'authentification est faite
syst\'ematiquement. Pour une aide en ligne sur cette commande :

\begin{verbatim}
/usr/local/cvs/sbin/make-project -h
\end{verbatim}

Note : Certains noms de projet sont interdits pour des raisons de
conflits possibles avec des r\'epertoires existants, il s'agit de :

\begin{verbatim}
chrooted-cvs home sbin doc src
\end{verbatim}

\subsubsection{Destruction d'un projet existant}

La destruction du projet \code{<proj>} se fait en deux temps
\begin{enumerate}
\item  Destruction de tous les usager \`a acc\`es restreint sur ce projet (la liste est disponible en faisant la
commande : \code{cat /etc/passwd $\vert$ grep \guil{Project <proj>}}
) voir commande de destruction plus bas
\item  Destruction du projet
\begin{verbatim}
rm -rf /usr/local/cvs/<proj> groupdel <proj>
\end{verbatim}

\end{enumerate}

\subsubsection{Cr\'eation d'usagers \`a acc\`es restreint}
La cr\'eation d'un usager \`a acc\`es restreint se fait gr�ce au script
\code{make-user}. Un usager \`a acc\`es restreint ne peut acc\'eder que \`a
un projet \`a la fois. La syntaxe est la suivante:
\begin{verbatim}
/usr/local/cvs/sbin/make-user <proj> [-ro|-rw] <user>
\end{verbatim}
La variable \code{<proj>} est \`a remplacer par le nom du projet
(existant) et la variable \code{<user>} par le nom d'usager
(inexistant). Les options \code{-ro} et \code{-rw} servent
respectivement \`a sp\'ecifier un acc\`es en lecture seule ou en lecture
et \'ecriture. Ceci peut \^etre modifi\'e \`a la main par la suite en
pla�ant l'usager dans le groupe du projet ou non (commande
\code{usermod}), mais il est pr\'ef\'erable de d\'etruire le compte et de
le recr\'eer.


\subsubsection{Modification d'usagers \`a acc\`es restreint}
La fa�on la plus simple de modifier un usager \`a acc\`es restreint
(pour changer le projet sur lequel il a des droits par exemple)
consiste \`a le d\'etruire et \`a le recr\'eer avec les bons arguments. La
destruction d'un usager restreint est identique \`a la destruction
d'un usager \`a acc\`es \'etendu :
\begin{verbatim}
userdel -r <user>
\end{verbatim}

\subsubsection{Cr\'eation ou modification d'usagers \`a acc\`es \'etendu}
Ces usagers ne sont pas cr\'e\'es ou modifi\'es par un script mais doivent
l'\^etre � \`a la main � par des commandes unix standard. Le fait
d'appartenir \`a certains groupes conditionne l'acc\`es en lecture ou en
lecture et \'ecriture aux projets. La cr\'eation est faite avec la
commande \code{useradd}, et la modification avec \code{usermod}. Un
exemple vaut mieux que mille mots : Cr\'eation de l'usager \code{toto}
avec acc\`es en \'ecriture aux projets \code{public} et \code{fissure} :
\begin{verbatim}
useradd -G cvsdev,public,fissure toto
\end{verbatim}


Modification : toto a de plus acc\`es en \'ecriture au projet ecn :
\begin{verbatim}
usermod -G cvsdev,public,fissure,ecn toto
\end{verbatim}

Maintenant, toto n'a plus acc\`es a rien (mais on veut garder
l'usager):
\begin{verbatim}
usermod -G "" toto
\end{verbatim}

On redonne de nouveau l'acc\`es en lecture \`a tous les projets :
\begin{verbatim}
usermod -G cvsdev toto
\end{verbatim}

On d\'esire d\'etruire l'usager toto :
\begin{verbatim}
userdel -r toto
\end{verbatim}

% \subsubsection{Configuration des usagers}
% Ces actions se font en dehors du serveur cvs, par l'usager, mais elles sont sp\'ecifiques au serveur cvs
% de l'ECN.

\subsubsection{Acc\`es sans mot de passe}\label{acces_sans_passwd}
Il est parfois fastidieux de rentrer un mot de passe pour tout acc\`es
au serveur CVS. Il existe un moyen de mettre \`a jour le fichier de
clefs permettant une authentification sans mot de passe. Il faut
d'abord cr\'eer la clef sur la machine cliente; ceci se fait par la
commande suivante :

\begin{verbatim}
ssh-keygen -t -dsa
\end{verbatim}


Par l'interm\'ediaire de l'utilitaire authorize, il est donc possible
de transmettre cette clef, ceci se fait de la mani\`ere suivante:
\begin{verbatim}
ssh <user>@cvs.ec-nantes.fr authorize < ~/.ssh/id_dsa.pub
\end{verbatim}

Une fois cette action effectu\'ee, plus aucun mot de passe ne sera
demand\'e (en fait, uniquement pour les clients sur lesquels l'action
a \'et\'e faite). Pour les usagers \'etendus, il faudra en plus mettre les
bons droits d'acc\`es du fichier
\code{$\thicksim$/.ssh/authorized\_keys2} :

\begin{verbatim}
ssh <user>@cvs.ec-nantes.fr "chmod 600 ~/.ssh/authorized_keys2"
\end{verbatim}

Pour effacer les clefs d\'ej\`a entr\'ees (et revenir \`a une
authentification par mot de passe pour tous les clients), il suffit
de faire :
\begin{verbatim}
ssh <user>@cvs.ec-nantes.fr authorize delete
\end{verbatim}

Modification de mot de passe La modification de mot de passe se fait
avec la commande password :
\begin{verbatim}
ssh <user>@cvs.ec-nantes.fr password
\end{verbatim}

Attention, le mot de passe entr\'e s'affiche en clair (il est
toutefois transmis sous forme crypt\'ee) Quelques commandes usager
utiles pour initialiser un projet Ces commandes se font cot\'e usager
et permettent de manipuler un projet. Ce sont des commandes standard
que toute installation cvs reconna�t. Pour plus d'information, voir
le site : \web{http://www.cvshome.org/} Charger le contenu du
r\'epertoire courant dans le projet en cours, dans le sous-projet
nomm\'e \code{<sproj>} :
\begin{verbatim}
cvs import <sproj> vendor_tag release_tag
\end{verbatim}

R\'ecup\'erer le sous-projet \code{<sproj>} (pour un nouvel usager par
exemple) :
\begin{verbatim}
cvs checkout <sproj>
\end{verbatim}





%%%%%%%%%%%%%%%%%%%%%%%%%%%%%%%%%%%%%%%%%
%
%  Insert the bibliography using BibTeX
%
%%%%%%%%%%%%%%%%%%%%%%%%%%%%%%%%%%%%%%%%%
\part*{Ref\'erences}
\markboth{R\'ef\'erences Bibliographiques}{R\'ef\'erences Bibliographiques}
\addcontentsline{toc}{chapter}{R\'ef\'erences}
%\bibliographystyle{chicago}
\bibliographystyle{bibsty}
\bibliography{biblio}

%%%%%%%%%%%%%%%%%%%%%%%%%%%%%%%%%%%%%%%%%
%
%  Insert the Index file
%
%%%%%%%%%%%%%%%%%%%%%%%%%%%%%%%%%%%%%%%%%
\part*{Index}
\addcontentsline{toc}{chapter}{Index}
\markboth{Index}{Index}
\printindex
%\InputIfFileExists{XfemGuide.idx}


\end{document}
