\clearpage
\section{la notions de forme bilin\'eaire et lin\'eaire}

\subsection{Probl\`eme de r\'ef\'erence}

Reprenons le probl\`eme de r\'ef\'erence de la section~\ref{probleme_de_reference}. L'\'equation~\ref{eq:varia} se r\'esume
\`a trouver le champ $\bfu$ satisfaisant :
\begin{equation}
    \label{eq:varia_2}
    \int_{\Ome}   \bfeps(\bfu):\bfC:\bfeps(\bfv) \dint \Ome =
    \int_{\Gam_t} \ov{\bft}\cdot\bfv  \dint \Gam
\end{equation}
quelque soit le champ de d\'eplacement virtuel $\bfv$. Apr\`es lin\'earisation de l'op\'erateur $ \bfeps(.)$, le premier
 terme de l'\'equation~\ref{eq:varia_2} est une forme bilin\'eaire en  $\bfu$ et  $\bfv$. Le deuxi\`eme terme est une forme
 lin\'eaire en $\bfv$.


En discr\'etisant le domaine par des \'el\'ement finis et en \'ecrivant le  champ  sous la forme :
\begin{equation}
    \bfu(\bfx) = \sum_i^N a_i \bfphi_i(\bfx)
\end{equation}
on arrive \`a un syst\`eme d'\'equation de type :
\begin{equation}
\label{eq:global_systeme}
K_{ij} a_j = f_i, \quad i = 1, \ldots, N
\end{equation}
o� la sommation sur l'indice $j$ est d'application et o�
\begin{eqnarray}
K_{ij} & = & \int_{\Ome}   \bfeps(\bfphi_i):\bfC:\bfeps(\bfphi_j) \dint
\Ome \\
f_i & = & \int_{\Gam_t} \ov{\bft}\cdot\bfphi_i  \dint \Gam
\end{eqnarray}

\subsection{Projection L2}

De mani\`ere generale, l'\'ecriture du principe variationnel sur un mod\`ele \'el\'ements finis conduit \`a la construction de
 formes bi-lin\'eaires et lin\'eaires. Ces derni\`eres sont construites par int\'egration sur chaque \'el\'ement de fonction de forme
 ou de leur d\'eriv\'ees puis l'assemblage des matrices \'el\'ementaires dans une matrice globale.

Ces op\'erations peuvent varier d'un probl\`eme \`a l'autre et font appel \`a diff\'erentes m\'ethodes. Pour g\'erer ces m\'ethodes :

\begin{description}
\item[--] \code{xFormBilinear} est une classe g\'en\'erale d'object permettant de d\'efinir des formes bilin\'eaires. Cette classe
est d\'eriv\'ee en  formes classiques (voir graphe d'h\'eritage ci-dessous). La forme construite d\'epend de l'application d'une
 \guil{loi de comportement} au sens de la m\'ecanique, et de l'application des op\'erateurs \`a appliquer sur les fonctions de
 forme $\bfphi_i(\bfx)$ (op\'erateur gradient, op\'erateur identit\'e, ...) ;
\inheritgraph{15cm}{xFormBilinear}

\item[--] \code{xIntegrationRuleBasic}, permet de d\'efinir le type d'int\'egration sur les \'el\'ements.


\item[--] \code{xAssemblerBasic} permet de d\'efinir le type d'assemblage.
\item[--] \code{xFormLinear}  est une classe g\'en\'erale d'object permettant de d\'efinir des formes lin\'eaires.
\inheritgraph{15cm}{xFormLinear}

\end{description}

Prenons l'exemple d'une projection L2  d'une fonction $f(x)$. On recherche les valeurs nodales $u_i$ satisfaisant
 l'\'equation suivante :
\begin{equation}
    \label{eq:L2proj}
    \int_{\Ome}   \bfu(x)  \bfv(x) \dint \Ome =
    \int_{\Ome} f(x) \bfv(x)  \dint \Ome  \hspace{2cm} \forall \hspace{2mm} \bfv
\end{equation}
ce qui se traduit par :
\begin{equation}
    \label{eq:L2proj-disc}
    \int_{\Ome} u_i  \bfphi_i \bfphi_j \dint \Ome =
    \int_{\Gam} f(x) \bfphi_j \dint \Gam
\end{equation}

Cette \'equation conduit \`a une forme bilin\'eaire sans op\'erateur sur les fonctions $\bfphi_i$ (op\'erateur identit\'e).
Cet exemple est programm\'e dans le fichier \code{main.cc} du repertoire \code{devel/l2proj} de la librairie \code{Xfem}.
 L'\'equation \ref{eq:L2proj-disc} se programme de la mani\`ere suivante :

\begin{verbatim}
 //un exemple de projection L2
xDoubleManager vals;
xSpaceLagrange temp_space("TEMPERATURE", SCALAR, DEGREE_ONE);
xField         temp(&vals, temp_space);
xValueCreator<xValueDouble>  creator_double;

DeclareInterpolation(temp, creator_double, mesh.begin(2), mesh.end(2));

xStateDofCreator<> snh(double_manager, "dofs");
DeclareState(temp, snh, mesh.begin(2), matrix.end(2));

xCSRVector b(vals.size("dofs"));
xCSRVector sol(vals.size("dofs"));
xCSRMatrix A(vals.size("dofs"));
xAssemblerBasic<> assembler(A, b);

xIntegrationRuleBasic integration_rule(2);

xFormBilinearWithoutLaw<xValOperator<xIdentity<double> >,
                        xValOperator<xIdentity<double> > > bilin;

Assemble(bilin, assembler, integration_rule, temp, temp, mesh.begin(2), mesh.end(2));


xFormLinearWithLoad<xValOperator<xIdentity<double> >, xEval<double> >  lin(t_exact);

Assemble(lin, assembler, integration_rule, temp, mesh.begin(2), mesh.end(2));

system.Solve(b, sol);
Visit(xWriteSolutionVisitor(sol.begin()), vals.begin("dofs"), vals.end("dofs"));
\end{verbatim}

Les objets courant d'alg\`ebre lin\'eaire  (\code{xCSRVector, xCSRMatrix}) sont d\'efinis sur de la base de classe de la librairie
 MTL auquelles sont ajouter des notions d'it\'erateur.


\subsection{A d\'evelopper ....}

notions d'assembleur et d'assemblage :   xAssemblerBasic<> assembler(A, b);

type d'integrateur : xIntegrationRuleBasic integration\_rule(2);


Assemble(bilin, assembler, integration\_rule, temp, temp, mesh.begin(2), mesh.end(2));

que signifie :

system.Solve(b, sol);
Visit(xWriteSolutionVisitor(sol.begin()), vals.begin("dofs"), vals.end("dofs"));
