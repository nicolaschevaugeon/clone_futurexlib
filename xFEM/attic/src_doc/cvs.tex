
\section{Le principe de fonctionnement \`a l'ECN}\label{CVS-principe}

Cette partie d\'etaille les fonctionnalit\'es de l'utilitaire CVS.

\subsection{CVS \emph{(Concurrent Versions System)}}


CVS \index{CVS}  est un syst\`eme de contr�le de versions
client-serveur permettant \`a plusieurs personnes de travailler
simultan\'ement sur un m�me ensemble de fichiers. Les gros projets de
d\'eveloppement  s'appuient g\'en\'eralement sur ce type de syst\`eme afin
de permettre \`a un grand nombre de d\'eveloppeurs de travailler sur un
m�me projet. CVS permet, comme son nom l'indique, de g\'erer les acc\`es
concurrents, c'est-\`a-dire qu'il est capable de d\'etecter les conflits
de version lorsque deux personnes travaillent simultan\'ement sur le
m�me fichier.

Le fonctionnement de CVS s'appuie sur une base centralis\'ee appel\'ee �
repository �\index{repository} (d\'ep�t), h\'eberg\'ee sur un serveur (le
serveur CVS), contenant l'historique de l'ensemble des versions
successives de chaque fichier. Le repository stocke les diff\'erences
entre les versions successives, les dates de mise \`a jour, le nom de
l'auteur de la mise \`a jour et un commentaire \'eventuel, ce qui permet
un r\'eel suivi des modifications, tout en optimisant l'espace de
stockage d\'edi\'e au projet.

Chaque personne travaillant sur le projet poss\`ede un � r\'epertoire de
travail � (en anglais � working copy � ou � sandbox �, traduisez �
bac \`a sable �), c'est-\`a-dire un r\'epertoire contenant une copie de la
base CVS (repository). A l'ECN, il s'agit g\'en\'eralement des
repertoires de travail sur les disques
\code{/scratch/<username>/develop}

\subsection{Checkout \index{checkout}}

A l'aide d'un client CVS, chaque utilisateur souhaitant travailler
sur le projet (pour modifier des fichiers ou simplement pour voir la
derni\`ere version des fichiers dans la base) r\'ecup\`ere une copie de
travail gr�ce \`a une op\'eration appel\'ee � \code{checkout} �.



\subsection{Commit \index{commit}}

Lorsque l'utilisateur a termin\'e de modifier les fichiers, il peut
transmettre les modifications \`a la base. Cette op\'eration est appel\'ee
� \code{commit} �. Ainsi plusieurs d\'eveloppeurs peuvent travailler
simultan\'ement sur une copie du repository et transmettre leurs
modifications.



\subsection{Update \index{update}}

S'il arrive qu'un utilisateur tente de transmettre ses modifications
alors qu'un autre utilisateur a lui-m�me modifi\'e ce fichier
pr\'ec\'edemment, CVS va d\'etecter une incompatibilit\'e. Si les
modifications portent sur des parties diff\'erentes du fichier, le
syst\`eme CVS peut proposer une fusion des modifications, gr�ce \`a une
op\'eration appel\'ee \code{diff}, sinon CVS va demander \`a l'utilisateur
de fusionner manuellement les modifications. Il est \`a noter que les
fusions ne peuvent s'appliquer qu'aux fichiers textes. CVS peut
toutefois g\'erer des fichiers binaires dans sa base, mais il n'a pas
\'et\'e pr\'evu dans ce but. Les modifications apport\'ees par les autres
utilisateurs ne sont pas automatiquement r\'epercut\'ees par CVS sur la
copie locale, il est donc n\'ecessaire, avant chaque modification de
fichier, de mettre \`a jour sa copie de travail gr�ce \`a une op\'eration
appel\'ee � \code{update} �, afin de limiter les risques de conflits.




\section{Using CVS (\emph{in english})}

For more d\'etails and information, see
\web{http://ximbiot.com/cvs/manual/}.

CVS (Concurrent Versions System), is the tool used to manage the
sources and to operate on them without creating conflictual problems
between versions. CVS allows every developer to (commonly used
operations):

\begin{itemize}
\item add, remove and modify files or directories (irreversible)
\item get repercuted on his local versions the pertinent modifications made on the common ones
\item get repercuted on the common versions the pertinent modifications made on the local ones
\item create a history of modifications, allowing access to an old version of source (...)
\end{itemize}

A more exhaustive list of possibilities can be found in CVS web
page. Access to almost all cvs commands is available on the
\code{emacs} interface (\code{menu Tools, tab PCL-CVS}). It is also
possible to do them directly from a terminal.

\subsection{Changing password to ECN's cvs server:}

A  password is given to all users, use this command  to change it:
\begin{verbatim}
ssh <user>@cvs.ec-nantes.fr password
\end{verbatim}



\subsubsection{Acc\`es sans mot de passe}\label{acces_sans_passwd}
Il est parfois fastidieux de rentrer un mot de passe pour tout acc\`es
au serveur CVS. Il existe un moyen de mettre \`a jour le fichier de
clefs permettant une authentification sans mot de passe. Il faut
d'abord cr\'eer la clef sur la machine cliente; ceci se fait par la
commande suivante :

\begin{verbatim}
ssh-keygen -t dsa
\end{verbatim}


Par l'interm\'ediaire de l'utilitaire authorize, il est donc possible
de transmettre cette clef, ceci se fait de la mani\`ere suivante:
\begin{verbatim}
ssh <user>@cvs.ec-nantes.fr authorize < ~/.ssh/id_dsa.pub
\end{verbatim}

Une fois cette action effectu\'ee, plus aucun mot de passe ne sera
demand\'e (en fait, uniquement pour les clients sur lesquels l'action
a \'et\'e faite). Pour les usagers \'etendus, il faudra en plus mettre les
bons droits d'acc\`es du fichier
\code{$\thicksim$/.ssh/authorized\_keys2} :

\begin{verbatim}
ssh <user>@cvs.ec-nantes.fr "chmod 600 ~/.ssh/authorized_keys2"
\end{verbatim}

Pour effacer les clefs d\'ej\`a entr\'ees (et revenir \`a une
authentification par mot de passe pour tous les clients), il suffit
de faire :
\begin{verbatim}
ssh <user>@cvs.ec-nantes.fr authorize delete
\end{verbatim}

Modification de mot de passe La modification de mot de passe se fait
avec la commande password :
\begin{verbatim}
ssh <user>@cvs.ec-nantes.fr password
\end{verbatim}

Attention, le mot de passe entr\'e s'affiche en clair (il est
toutefois transmis sous forme crypt\'ee) Quelques commandes usager
utiles pour initialiser un projet Ces commandes se font cot\'e usager
et permettent de manipuler un projet. Ce sont des commandes standard
que toute installation cvs reconna�t. Pour plus d'information, voir
le site : \web{http://www.cvshome.org/}.






\subsection{Commonly used cvs commands}

\begin{center}
\begin{tabular}{p{6cm} p{9.5cm}}
\code{cvs checkout MODULE\_NAME}\index{CVS!checkout} &  Imports the module into the local account. \\
\code{cvs update -d REPOSITORY\_NAME}\index{CVS!update} &   Checks whether the local version is up-to-date with the CVS-server version \\
\code{cvs add FILE\_NAME} &     Declare a new file to the cvs repository. \\
\code{cvs commit REPOSITORY\_NAME}\index{CVS!commit}  or \code{cvs
commit FILE\_NAME}& Commits the local modifications (on the
repository or the file) into the repository.
\end{tabular}
\end{center}

Details on the recommended use of these commands can be read in
section~\ref{common_commit}.


\subsection{Syntax of the CVS commands \index{CVS!commands}}

Usage:
\begin{verbatim}
          cvs [cvs-options] command [command-options] [files...]
\end{verbatim}
where \code{cvs-options} are:
\begin{center}
\begin{longtable}{p{3.75cm} p{11.75cm}}
     \code{   -H          }  &   Displays Usage information for CVS command \\
     \code{   -Q          }  &   Cause CVS to be really quiet. \\
     \code{   -q          }  &   Cause CVS to be somewhat quiet. \\
     \code{   -r          }  &   Make checked-out files read-only \\
     \code{   -w          }  &   Make checked-out files read-write (default) \\
     \code{   -l          }  &   Turn History logging off \\
     \code{   -n          }  &   Do not execute anything that will change the disk \\
    \code{    -t          }  &   Show trace of program execution -- Try with -n \\
    \code{    -v          }  &   CVS version and copyright \\
    \code{    -b bindir   }  &   Find RCS programs in \code{bindir} \\
    \code{    -e editor   }  &   Use \code{editor} for editing log information \\
    \code{    -d CVS\_root }  &   Overrides \code{\$CVSROOT} as the root of the CVS tree
\end{longtable}
\end{center}
and where \code{command} is:
\begin{center}
\begin{longtable}{p{3.75cm} p{11.75cm}}
        \code{add  }  &        Adds a new file/directory to the repository \\
        \code{admin }    &  Administration front end for rcs \\
        \code{annotate}  &   Show last revision where each line was modified  \\
        \code{checkout }  &    Checkout sources for editing (download from CVS server)\\
    \code{commit }  &   Check changes into the repository. \\
        \code{diff    }  &   Show differences between revisions (or local version and CVS server version)\\
        \code{edit     } &   Get ready to edit a watched file \\
        \code{editors  } &   See who is editing a watched file \\
        \code{export  }  &   Export sources from CVS, similar to checkout \\
        \code{history  } &   Show repository access history \\
        \code{import  }  &   Import sources into CVS, using vendor branches \\
        \code{init    }  &   Create a CVS repository if it doesn't exist \\
        \code{kserver }  &   Kerberos server mode \\
        \code{log    }   &   Print out history information for files \\
        \code{login  }   &   Prompt for password for authenticating server \\
        \code{logout }   &   Removes entry in .cvspass for remote repository \\
        \code{pserver}   &   Password server mode \\
        \code{rannotate} &   Show last revision where each line of module was modified \\
        \code{rdiff  }   &   Create 'patch' format diffs between releases \\
        \code{release }  &   Indicate that a Module is no longer in use (check the difference and delete local repository)\\
        \code{remove }   &   Remove an file from the repository (definitively remove for further revision)\\
        \code{rlog   }   &   Print out history information for a module \\
        \code{rtag  }    &   Add a symbolic tag to a module \\
        \code{server}    &   Server mode \\
        \code{status }   &   Display status information on checked out files \\
        \code{tag    }   &   Add a symbolic tag to checked out version of files \\
        \code{unedit }   &   Undo an edit command \\
        \code{update     }  &  Brings work tree in sync with repository (download from CVS server the last modification by including the differences or let the use choose which lines must be kept)\\
        \code{version}   &   Show current CVS version(s) \\
        \code{watch  }   &   Set watches \\
        \code{watchers}  &    See who is watching a file
\end{longtable}
\end{center}








\subsection{Other sources of information}


\web{http://www.nongnu.org/cvs/}


\web{http://ximbiot.com/cvs/cvshome/}
