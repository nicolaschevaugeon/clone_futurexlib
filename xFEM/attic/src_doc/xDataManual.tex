
\section{la lecture des donn\'ees}

\subsection{Pr\'e-processeur}

La g\'en\'eration d'un maillage peut se faire \`a partir de n'importe quel outil \`a condition que l'interpr\'eteur soit disponible ou d\'evelopp\'e par l'utilisateur. 1
Un d\'ebut d'interpr\'eteur pour les fichiers \code{.inp} d'Abaqus existe, mais nous consid\'erons dans cette section que le d\'eveloppeur utilise \code{gmsh} (voir la section~\ref{gmsh}).

\subsection{La classe \code{xData}}\label{xData}\index{xData}

\paragraph*{Motivation :}

Un fichier \code{.geo} ne contient qu'un maillage et la d\'efinition de zone (voir~\ref{gmsh_msh}).  La d\'efinition des conditions de calcul (condiions aux limites, mat\'eriau, ...) se fait par les fichiers  \code{data/main.dat}. Le r�le de la classe \code{xData} est de permettre l'interpr\'etation de ce fichier.


\paragraph*{Impl\'ementation :}

Peu d'attention n'a encore \'et\'e port\'es \`a cette classe et des am\'eliorations sont encore possible pour g\'en\'eraliser son utilisation. 






\begin{verbatim}
/*  
    xfem : C++ Finite Element Library 
    developed under the GNU Lesser General Public License
    See the NOTICE & LICENSE files for conditions.
*/
#include <cstdio>
#include <string>
#include <cassert>
#include "mAOMD.h"
#include "xData.h"
#include "xBoundary.h"
#include "xFiniteElement.h"
#include "xEnv.h"
#include "xMesh.h"

namespace xfem 
{

char xData::keyword_list[][NB_CHAR_MAX] = {
"PROCEDURE", "ASS",
"COUPLE_X", "COUPLE_Y",  
"CRACK", "CRACK_DIS", 
"DISPLACEMENT", "DISPLACEMENT_X", "DISPLACEMENT_Y", "DISPLACEMENT_Z",
"VECTOR_POTENTIAL_X", "VECTOR_POTENTIAL_Y", "VECTOR_POTENTIAL_Z",
"VELOCITY_X", "VELOCITY_Y", "VELOCITY_Z", "VELOCITY",
"ACCELERATION_X", "ACCELERATION_Y", "ACCELERATION_Z",
"DISPLACEMENT_T","DO_EIGEN_ANALYSIS",
"DO_ENRICH_CHANNELS", "DO_ENRICH_CHANNEL_PRESSURE",
"DO_ENRICH_MANUAL", "DO_ENRICH_AUTOMATIC", "DO_POSTPRO_MANUAL", "DO_POSTPRO_AUTOMATIC",
"DOF_GROUP_SCA_1",
"ELASTIC_SUPPORT", "ENDIF",
"FIX", "FORMULATION_ONE", "FORMULATION_TWO",
"GEOM_FILE", "GEOM_TYPE", "INTEGRATION_TYPE", 
"BC_VOLUME", 
"BC_SURFACE", 
"BC_LINE", 
"BC_POINT", 
"GROUP_ON_CRACK",
"HOLE",
"PARTICLE","PARTICLE_VEL",
"PARTICLE_PRESS",
"PARTICLE_WALL_GAP_VEL","PARTICLE_WALL_GAP_PRESS", 
"IF",
"LINE",
"MATERIAL_PAIR", 
"MAT_CLASS", "MAT_PARAM", "MESH_FILE_TYPE", "MESH_FILE", 
"NEAR_TIP", "NEAR_REG_TIP", 
"NEAR_JUNCTION", 
"FORMULATION_PARAM_FILE", 
"PROCEDURE_PARAM_FILE", 
"PRESSURE", "POTENTIAL", "POTENTIAL_WEAK", "NORMAL_VELOCITY",
"PARTITION_FILE", "POINT", 
"RESULT_FILE", "ROTATION_X", "ROTATION_Y",
"SOLUTION_ONE_FILE",  "SOLUTION_TWO_FILE", "SOLVER_NAME", "SOLVER_PARAM_FILE", "SURFACIC_HEAT_FLUX",
"TEMPERATURE", "TEMPERATURE_MAC", "TRACTION_X",  "TRACTION_Y", "TRACTION_Z", 
"ZONE", "ZONE_ENRICH",  "ZONE_ENV", "ZONE_POSTPRO",
"ZONE_NAME",
"REFERENCE_SOLUTION_FILE", "STORE_SOLUTION_FILE", "STRESS", 
"COMPUTE_EXACT_ERROR", "EXACT_SOLUTION_NAME",
"NB_EVO", "EVO", 
"SHEARBAND", "SHEARBAND_DIS","SHEARBAND_DIS_RIGID",
"TANGENT_STIFFNESS","NORMAL_STIFFNESS","FRICTION", "WALL", "DENSITY", "EXPORT_FORMAT", "ELECTRIC_DISPLACEMENT", "CUSTOM"
};

int xData::keyword_sorted  = 0;

char xData::procedure_list[][NB_CHAR_MAX] = {
"crack_growth", "cohesive_crack_growth"
};
int xData::procedure_sorted = 0;
//
char xData::exact_solution_list[][NB_CHAR_MAX] = {
"patch_test","contact2d","hole", "inclusion", "bimaterial1d", "stokescyl", "crack_mode_I",
"bimaterial3d","spherical_inclusion","spherical_cavity"};
int xData::exact_solution_sorted = 0;

static int cmp(const void *vp, const void *vq)
{
  return strcmp((const char *) vp, (const char *) vq);
};




// DEFAULT CONSTRUCTOR
xData::xData(void) : MaterialManager(xMaterialManagerSingleton::instance()) { 

  
  Dimension = 0;
  ComputeExactError = false;

  time = 1.0; 
  dt   = 1.0;

  PhysicalEnv = new xPhysicalEnv;

  strcpy(procedure, "");
  strcpy(mesh_file_type, "");
  strcpy(mesh_file, "");
  export_format = "ascii";
  strcpy(formulation_param_file, "");
  strcpy(solver_param_file, "");
  strcpy(solver_name, "");
  strcpy(partition_file, "");

  strcpy(reference_solution_file, "");
  strcpy(store_solution_file, "");

  if (!keyword_sorted) {keyword_sorted = 1; 
			qsort(keyword_list, sizeof(keyword_list)/NB_CHAR_MAX, NB_CHAR_MAX, cmp);}


  if (!exact_solution_sorted) {exact_solution_sorted = 1; 
			qsort(exact_solution_list, sizeof(exact_solution_list)/NB_CHAR_MAX, 
			      NB_CHAR_MAX, cmp);}

  if (!procedure_sorted) {procedure_sorted = 1; 
			qsort(procedure_list, sizeof(procedure_list)/NB_CHAR_MAX, NB_CHAR_MAX, cmp);}




  xMaterialManagerSingleton::instance().registerMaterial("elastic", 
							 xMaterialCreator<xElastic>() );



return;
}


// DESTRUCTOR
xData::~xData(void) { 

//a bug needs to be fix in aomd because
//the destructor of mesh makes aomd crash  
//delete mesh;

  delete PhysicalEnv;

}

char * xData::GetMeshFile(void) 
{ 
  return mesh_file;
}


char * xData::GetSolverName(void) 
{ 
  return solver_name;
}
char * xData::GetSolverParamFile(void) 
{ 
  return solver_param_file;
}

// char * xData::Getanalysis(void) 
// { 
//   return analysis;
// }

char * xData::GetProcedure(void) 
{ 
  return procedure;
}



void xData::ReadMesh(void)
{
  if (mesh_file == "") return;
  ifstream f(mesh_file);
  if (f.good())
  {
    f.close();
    mesh = new xMesh(mesh_file);
  }
  else
  {
    f.close();
    cerr<< "no mesh file !" << endl;
    throw "no mesh file !";
  }
}

void xData::ReadZones(void)
{
  for (int_zones_t::const_iterator it = int_zones.begin(); it != int_zones.end(); ++it) 
    {
      MaterialManager.createZone(it->first, it->second.first, it->second.second);
    }
  for (string_zones_t::const_iterator it = string_zones.begin(); it != string_zones.end(); ++it) 
    {
      MaterialManager.createZone(it->first, it->second.first, it->second.second);
    }
}

inline void check_active(int active, char* s)
{
  if (active != 0) 
  { 
    printf( "Error in DAT file: you forgot to close a brace for the %s info\n", s);
    assert( 1 == 0);
  }
  return;
}
void   xData::SetMeshFile(const char* _mesh_file_name)
{
  strcpy(mesh_file, _mesh_file_name);
}

void xData::ReadInfo(const char *filename)
{
  FILE *fp = fopen(filename, "r");
  if (fp == 0) {fprintf(stderr, "The data file %s cannot be opened\n", filename);
		assert(1 == 0);}
  char key[256];
  char c;
  int i;
  int geom, entity, val_ass;
  string phys;
  double val_fix;
  xEnv env;
  xPhysicalEnv * InfoEnv;

  xBoundary crvboundary;
  PhysicalEnv->clear(); 

  char mat_class[NB_CHAR_MAX], mat_param[NB_CHAR_MAX], zone_name[NB_CHAR_MAX];
//for the load evolution
  int nb_evo;
  double factor, t;
  std::map<double, double> evo;




const int NOTHING_ACTIVE       = 0;
const int BC_VOLUME_ACTIVE         = 2;
const int BC_SURFACE_ACTIVE         = 3;
const int BC_LINE_ACTIVE         = 4;
const int BC_POINT_ACTIVE         = 5;
const int ZONE_ENV_ACTIVE         = 6;
const int ZONE_ACTIVE          = 10;
const int ZONE_NAME_ACTIVE          = 11;



  int active = NOTHING_ACTIVE;


  while( (i = fscanf(fp, "%[1234567890A-Z_]",  key)) != EOF) {
    if (i == 0)
      {
	c=fgetc(fp);
        //printf("c: %c\n", c);

	if ( c != '#' && c !='\n' && c != ' ' && c != '\t' && c != '}')
	{
	  fprintf(stderr, "The following character is not known : %c\n", c); 
	  assert(1 == 0);
	}
	if (c=='#') fscanf(fp, "%*[^\n] \n");

	if (c=='}') {	
	  if      ( active == BC_LINE_ACTIVE   ) { }
	  else if ( active == BC_SURFACE_ACTIVE  ) { }
	  else if ( active == BC_VOLUME_ACTIVE  ) { }
	  else if ( active == BC_POINT_ACTIVE  ) { }
	  else if ( active == ZONE_ENV_ACTIVE  ) { }
	  else if ( active == ZONE_ACTIVE   ) {
	    std::string mat_class_s(mat_class);
	    std::string mat_param_s(mat_param);
	    int_zones.insert(std::make_pair(entity, std::make_pair(mat_class_s, mat_param_s)));
            //MaterialManager.createZone(entity, mat_class, mat_param);
	  }
	  else if ( active == ZONE_NAME_ACTIVE   ) {
            //MaterialManager.createZone(zone_name, mat_class, mat_param);
	    std::string mat_class_s(mat_class);
	    std::string mat_param_s(mat_param);
            std::string zone_name_s(zone_name);
	    string_zones.insert(std::make_pair(zone_name_s, std::make_pair(mat_class_s, mat_param_s)));
	  }
	  else if ( active == NOTHING_ACTIVE ){
	    printf( "Error in DAT file :you closed a braced without opening one\n");
	    assert(0);}
	  else { assert(0); }
          //we clear things before reading the new FOO = 3 { .... }
	  active = NOTHING_ACTIVE;
	  entity = -1;
	}
      }
    else
      {
	//
	// Check if the key is valid
	//
        if (bsearch((const void *) key, keyword_list,
		       sizeof(xData::keyword_list)/NB_CHAR_MAX, NB_CHAR_MAX, cmp) == 0)
	  {fprintf(stderr, "The keyword %s is not known\n", key); assert(1 == 0);}
        
	if (strcmp(key, "ENDIF") == 0); //nothing to do

	//
	// action depending on the key
	//
	if (strcmp(key, "PROCEDURE") == 0) 
	  { 
	    fscanf(fp, "%*[\n= ] %s[a-z]", procedure);	
	    if (bsearch((const void *) procedure, procedure_list,
			sizeof(xData::procedure_list)/NB_CHAR_MAX, NB_CHAR_MAX, cmp) == 0)
	      {printf("The type of procedure %s is not known\n", procedure); assert(1 == 0);}
	  }
	if (strcmp(key, "MESH_FILE_TYPE") == 0) 
	  {
	    fscanf(fp, "%*[\n= ] %s[a-z]", mesh_file_type);		
	  }
	if (strcmp(key, "MESH_FILE") == 0) fscanf(fp, "%*[\n= ] %s[a-z]", mesh_file);		
	if (strcmp(key, "EXPORT_FORMAT") == 0) 
	  {
	    char tmp[NB_CHAR_MAX];
	    fscanf(fp, "%*[\n= ] %s[a-z]", tmp);
	    export_format = tmp;
	    if ( ( export_format != "binary" ) && (export_format != "ascii") )
	      {
                fprintf(stderr, "EXPORT_FOMAT is %s\n",  export_format.c_str());
		fprintf(stderr, "EXPORT_FORMAT should be ascii or binary\n"); assert(0);
	      }
	  }
	if (strcmp(key, "GEOM_FILE") == 0) fscanf(fp, "%*[\n= ] %s[a-z]", geom_file);
	if (strcmp(key, "GEOM_TYPE") == 0) fscanf(fp, "%*[\n= ] %s[a-z]", geom_type);
	if (strcmp(key, "INTEGRATION_TYPE") == 0) fscanf(fp, "%*[\n= ] %s[a-z]", integration_type);
	if (strcmp(key, "FORMULATION_PARAM_FILE") == 0) fscanf(fp, "%*[\n= ] %s[a-z]", formulation_param_file);	
	if (strcmp(key, "PROCEDURE_PARAM_FILE") == 0) fscanf(fp, "%*[\n= ] %s[a-z]",    procedure_param_file);	
	if (strcmp(key, "SOLVER_PARAM_FILE") == 0) fscanf(fp, "%*[\n= ] %s[a-z]", solver_param_file);	
	if (strcmp(key, "SOLVER_NAME") == 0) fscanf(fp, "%*[\n= ] %s[a-z]", solver_name);	
	
	if (strcmp(key, "PARTITION_FILE") == 0) fscanf(fp, "%*[\n= ] %s[a-z]", partition_file);		
	if (strcmp(key, "RESULT_FILE") == 0) fscanf(fp, "%*[\n= ] %s[a-z]", result_file);		

	if (strcmp(key, "COMPUTE_EXACT_ERROR") == 0) { ComputeExactError = true; }


	// STRESS needed 
	if (strcmp(key, "COMPUTE_EXACT_ERROR") == 0) ComputeExactError = true;		

	if (strcmp(key, "REFERENCE_SOLUTION_FILE") == 0) 
	  fscanf(fp, "%*[\n= ] %s[a-z]", reference_solution_file);	
	if (strcmp(key, "STORE_SOLUTION_FILE") == 0) 
	  fscanf(fp, "%*[\n= ] %s[a-z]", store_solution_file);	

	if (strcmp(key, "SOLUTION_ONE_FILE") == 0) fscanf(fp, "%*[\n= ] %s[a-z]", solution_one_file);
	if (strcmp(key, "SOLUTION_TWO_FILE") == 0) fscanf(fp, "%*[\n= ] %s[a-z]", solution_two_file);

	if (strcmp(key, "ZONE") == 0)
	  {
	    fscanf(fp, "%d%*[\n= {]", &entity); 
	    check_active(active, "ZONE");
	    active = ZONE_ACTIVE;
	  }

	if (strcmp(key, "ZONE_NAME") == 0)
	  {
	    fscanf(fp, "%s%*[\n= {]", zone_name); 
	    check_active(active, "ZONE_NAME");
	    active = ZONE_NAME_ACTIVE;
	  }


	if (strcmp(key, "MAT_CLASS") == 0)  fscanf(fp, "%*[\n= ] %[a-zA-Z0123456789_]", mat_class);
	if (strcmp(key, "MAT_PARAM") == 0) { fscanf(fp, "%*[\n= ] %[a-zA-Z0123456789_./*]", mat_param); }
	


	if (strcmp(key, "BC_VOLUME") == 0)  {fscanf(fp, "%d%*[\n= {]", &entity); 
					   InfoEnv = PhysicalEnv;
					   check_active(active, "BC_VOLUME");
					   geom    = BC_VOLUME;
					   active  = BC_VOLUME_ACTIVE;}
	if (strcmp(key, "BC_SURFACE") == 0)  {fscanf(fp, "%d%*[\n= {]", &entity); 
					   InfoEnv = PhysicalEnv;
					   check_active(active, "BC_SURFACE");
					   geom    = BC_SURFACE;
					   active  = BC_SURFACE_ACTIVE;}
	if (strcmp(key, "BC_LINE") == 0)  {fscanf(fp, "%d%*[\n= {]", &entity); 
					   InfoEnv = PhysicalEnv;
					   check_active(active, "BC_LINE");
					   geom    = BC_LINE;
					   active  = BC_LINE_ACTIVE;}
	if (strcmp(key, "BC_POINT") == 0)  {fscanf(fp, "%d%*[\n= {]", &entity); 
					   InfoEnv = PhysicalEnv;
					   check_active(active, "BC_POINT");
					   geom    = BC_POINT;
					   active  = BC_POINT_ACTIVE;}

//JF NEW
	if (strcmp(key, "ZONE_ENV") == 0)  {fscanf(fp, "%d%*[\n= {]", &entity); 
					    InfoEnv = PhysicalEnv;
					    geom = ZONE_ENV;
					    check_active(active, "ZONE_ENV");
	                                    active = ZONE_ENV_ACTIVE;}


//JF END


	if (strcmp(key, "DISPLACEMENT_X") == 0)     phys = key;
	if (strcmp(key, "DISPLACEMENT_Y") == 0)     phys = key;
	if (strcmp(key, "DISPLACEMENT_Z") == 0)     phys = key; 
	if (strcmp(key, "VECTOR_POTENTIAL_X") == 0)     phys = key;
	if (strcmp(key, "VECTOR_POTENTIAL_Y") == 0)     phys = key;
	if (strcmp(key, "VECTOR_POTENTIAL_Z") == 0)     phys = key; 

	if (strcmp(key, "ROTATION_X") == 0)         phys = key; 
	if (strcmp(key, "ROTATION_Y") == 0)         phys = key; 
	if (strcmp(key, "COUPLE_X") == 0)           phys = key; 
	if (strcmp(key, "COUPLE_Y") == 0)           phys = key; 
	if (strcmp(key, "TRACTION_X") == 0)         phys = key; 
	if (strcmp(key, "TRACTION_Y") == 0)         phys = key; 
	if (strcmp(key, "TRACTION_Z") == 0)         phys = key; 

	if (strcmp(key, "STRESS") == 0)             phys = key;

	if (strcmp(key, "DISPLACEMENT") == 0)       phys = key; 
	if (strcmp(key, "PRESSURE") == 0)           phys = key; 
	if (strcmp(key, "POTENTIAL") == 0)          phys = key; 
	if (strcmp(key, "POTENTIAL_WEAK") == 0)          phys = key; 
        if (strcmp(key, "ELECTRIC_DISPLACEMENT") == 0)  phys = key; 
	if (strcmp(key, "TEMPERATURE") == 0)        phys = key; 
	if (strcmp(key, "TEMPERATURE_MAC") == 0)    phys = key; 
	if (strcmp(key, "SURFACIC_HEAT_FLUX") == 0) phys = key; 
	if (strcmp(key, "NORMAL_VELOCITY") == 0)    phys = key; 
	if (strcmp(key, "ELASTIC_SUPPORT") == 0)    phys = key; 

	if (strcmp(key, "VELOCITY_X") == 0)     phys = key; 
	if (strcmp(key, "VELOCITY_Y") == 0)     phys = key; 
	if (strcmp(key, "VELOCITY_Z") == 0)     phys = key; 
	if (strcmp(key, "VELOCITY") == 0)       phys =  key;

	if (strcmp(key, "ACCELERATION_X") == 0)     phys = key; 
	if (strcmp(key, "ACCELERATION_Y") == 0)     phys = key; 
	if (strcmp(key, "ACCELERATION_Z") == 0)     phys = key; 
	if (strcmp(key, "CUSTOM") == 0)     phys = key; 

	if (strcmp(key, "FIX") == 0)  
	  {
	    fscanf(fp, "%*[\n= ] %lf", &val_fix);  
	    env.defineFixed(phys, geom, entity, val_fix);  
            InfoEnv->add(env);
            env.clear();
	  }
	if (strcmp(key, "ASS") == 0)  
	  {
	    fscanf(fp, "%*[\n= ] %d", &val_ass);  
	    env.defineAssociate(phys, geom, entity, val_ass);
            InfoEnv->add(env);
            env.clear();
	  }
        if (strcmp(key, "NB_EVO") == 0)  
	  {
            fscanf(fp, "%*[\n= ] %d", &nb_evo);
	  }
        if (strcmp(key, "EVO") == 0)  
	  {
	    fscanf(fp, "%*[\n= ]");
            evo.clear();
            for (i = 0; i < nb_evo; i++) {
	      fscanf(fp, "%*[( ] %lf %*[ ,] %lf %*[ )]", &t, &factor);
              if (evo.find(t) == evo.end()) {
		evo[t] = factor;
	      }
	      else {
		fprintf(stderr, "You cannot have two factor for the same time\n"
			"i.e. the loading evolution must be continuous\n"
			"the time %12.5e is there twice\n", t);
	      }
	    }
            env.setEvolution(evo);
	  }


      }
  }

  fclose(fp);


  return;
}

}  


\end{verbatim}

















