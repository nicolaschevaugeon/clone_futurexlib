


\section{Documentation \index{Documentation}}

Le d\'eveloppement de librairies  ne peut se faire sans une
documentation pr\'ecise et syst\'ematique. La documentation de
\code{xfem} a \'et\'e choisie de deux types. D'une part une
documentation syst\'ematique construite \`a partir des les fichiers
sources \code{xObjet.h} et \code{xObjet.cc}, d'autre part une
documentation \'elargie regroupant les aspects informatiques et
th\'eoriques autour de la librairie \code{xfem}.

\subsubsection{Doxygen \index{Documentation!Doxygen}}\label{doxygen}

\webify{http://www.stack.nl/~dimitri/doxygen/}{Doxygen} est une
application permettant la g\'en\'eration automatique de documentation.
Le code C++ est interpr\'et\'e et les structures des classes sont mises
en forme par des sch\'emas de hi\'erarchie de classe et des
descriptions. Ceci permet de conna�tre l'impl\'ementation des classes
sans avoir \`a chercher dans les fichiers sources.
 
Cette documentation de bases peut-\^etre augment\'ee de commentaires
faits par les d\'eveloppeurs. Ces commentaires sont plac\'es dans le
code source lui m\^eme (fichier \code{.h}) par l'interm\'ediaire de
marqueurs particuliers ( voir~:
\webify{http://www.stack.nl/~dimitri/doxygen/docblocks.html}{Documenting
the code} sur le site de Doxygen).

Ces commentaires peuvent \^etre agr\'ement\'e de mise en forme
particuli\`ere par des commandes sp\'eciales (voir~:
\webify{http://www.stack.nl/~dimitri/doxygen/commands.html}{Special
Commands}  sur le site de Doxygen).

L'ex\'ecution de Doxygen s'effectue par la commande~: \code{make doc}.

Les formats de sorties possibles de Doxygen sont le HTML, le LATEX
et le RTF. La documentation Doxygen en html est lisible \`a
l'emplacement
\pdflaunch{../../../../doc/html/main.html}{\texttt{\$DEVROOT/Xfem/Xfem/doc/html/mainpage.html}}.


\subsubsection{La documentation Latex}\label{doclatex}

La version \LaTeX~de la documentation Doxygen n'a pas \'et\'e retenue
comme support pour la documentation \'elargie de la librairie
\code{xfem}. En effet, ce mode de fonctionnement alourdirait la
librairie et serait un frein \`a la lisibilit\'e des lignes de codes
pour les d\'eveloppeurs.

Une documentation \LaTeX~ind\'ependante a \'et\'e mise en place afin de
fournir aux nouveaux utilisateurs~:
\begin{itemize}
\item l'explication des principes de fonctionnement de la plate-forme de d\'eveloppement~;
\item la proc\'edure d'installation des librairies~;
\item des informations sur les diff\'erentes librairies associ\'ees~;
\item des informations sur les outils utilis\'es pour d\'evelopper des applications X-FEM~;
\item les commandes de bases \`a utiliser pour compiler~;
\item les bases th\'eoriques sur la m\'ethode de \'el\'ements finis \'etendus~;
\item les notions d\'ecrites par les principales classes  C++ de \code{xfem} et le lien avec la th\'eorie.
\item un support permettant d'avoir une vue d'ensemble de la librairie.
\end{itemize}

Cette documentation est l'objectif du pr\'esent guide.


\section{Comment est construit ce document}

Ce document est construit \`a partir de fichiers sources
\LaTeX~~situ\'es dans le r\'epertoire~:
 \code{\$DEVROOT/Xfem/Xfem/Xfem/src\_doc/}.


Le fichier principal est le fichier \code{XfemGuide.tex}. Il permet
de g\'en\'erer le document \code{XfemGuide.pdf} \`a partir de diff\'erentes
sources \code{.tex} \index{LaTeX!sources} ainsi que de figures
\index{LaTeX!figures} situ\'ees dans le sous-r\'epertoire
\code{scr\_doc/figures/}.

\onefigure{5cm}{XfemXfemRepository}{arborescence de la
documentation}


\subsection{g\'en\'eration du manuel}

La commande \code{make doc} d\'efinie par
\code{\$DEVROOT/Util/buildUtil/buildUtil/make.common} et n\'ecessaire
\`a la generation de la documentation Doxygen.

La commande \code{make latex}  g\'en\`ere  la pr\'esente documentation
\LaTeX.


% 
% \subsection{Utilisation de script Perl}
% 
% Une des particularit\'es de cette documentation est, \`a l'instar de
% Doxygen, de pouvoir s'appuyer sur une documentation au sein m\^eme des
% codes sources. Cette id\'ee est reprise des m\'ethodes de documentation
% d'ITK (\web{http://www.itk.org/}). Ceci se fait, encore une fois,
% par l'utilisation de commentaires sp\'eciaux.
% 
% 
% Ces commentaires sont interpr\'et\'es par l'utilisation d'un script en
% Perl (repris d'ITK) situ\'e dans le r\'epertoire
% \code{\$DEVROOT/Xfem/Xfem/Xfem} et nomm\'e
% \code{parser\_src\_to\_tex.pl}. Ce
% \guil{parser}\index{LaTeX!parser} g\'en\`ere des fichiers \code{.tex} \`a
% partir des fichiers \code{.h} et  \code{.cc} et les place dans les
% sous-r\'epertoires \code{tex\_from\_h} et \code{tex\_from\_cc}. Pour
% que cela puisse ce faire, les commandes \LaTeX~doivent \^etre
% directement plac\'ees entre les textes \code{Software Guide :
% BeginLatex} et \code{Software Guide : EndLatex} dans des
% commentaires des fichiers \code{.h} et \code{.cc}. Les commentaires
% des fichiers prennent alors la forme de l'exemple suivant~:
% \begin{verbatim}
% /*
%  Software Guide : BeginLatex
%     \section{la classe \code{xValue}}
%     \subsection{Notions}
%     xValue est une classe contenant les informations sur les coefficients 
%     scalaires d'une approximation.
%     \begin{equation}\label{eq:approx_ef2_2}
%     \bfu(\bfx) \mid_{\Ome_e} = \sum_{i \in N_n} \sum_{\alpha} a_i^\alpha
%     \bfphi_i^\alpha(\bfx)
%     \end{equation}
%     Les coefficients scalaires $a_i^\alpha$ sont repr\'esent\'es par des objets de
%     classe \code{xValue}.
%  Software Guide : EndLatex
% */
% \end{verbatim}
% 
% La lecture se fait au fil  du texte et des lignes de codes sources
% \index{LaTeX!code source} peuvent \^etre incorpor\'ees \`a la
% documentation \LaTeX~par l'utilisation de s\'eparateur \code{Software
% Guide : BeginCodeSnippet} et \code{Software Guide : EndCodeSnippet}.
% Le fichier source s'\'ecrit alors~:
% 
% \begin{verbatim}
% /*
%  Software Guide : BeginLatex
%     L'impl\'ementation se fait de la mani\`ere suivante~:
%     Software Guide : EndLatex
% */
% / Software Guide : BeginCodeSnippet
% xValueLinearCombination::xValueLinearCombination(const coeffs_t& c, const
%                                              values_t& v, double coeff_last)
%     : coeffs(c), values(v)
% {
%   std::cout << "creating value average with " << values.size() << " values "
%                                                                 << std::endl;
%   v_last = new xValueDouble;
%   v_last->setVal(1.0);
%   v_last->setState(new xStateOfValueFixed(v_last));
%   coeffs.push_back(coeff_last);
%   values.push_back(v_last);
% }
% // Software Guide : EndCodeSnippet
% /*
%  Software Guide : BeginLatex
%     o� \verb|v_last| est d\'efini par ...
%  Software Guide : EndLatex
% */
% \end{verbatim}
% 
% 
% 
% Les fichiers \code{.tex} ainsi cr\'e\'es peuvent \^etre inclus dans la
% documentation \LaTeX~par une simple commande. Le fichier issu de
% \code{xValue.h} est inclus dans la documentation avec la commande
% \code{{\textbackslash}texfromh\{xValue\}}. Cette commande sp\'eciale
% est l'une des nombreuses commandes d\'ecrite dans la section suivante.
% 
% ATTENTION : l'utilisation du script d\'ecrit ci-dessus est possible,
% mais a \'et\'e d\'esactiv\'ee (mis en commentaire dans le
% \code{make.common}) car inutilis\'e pour le moment.
% 

\subsection{commandes sp\'eciales}

Afin de faciliter la r\'edaction de la documentation, un certain
nombres de commandes ont \'et\'e d\'efinies dans le fichier
\code{XfemGuideConfiguration.tex} par la commande
\index{LaTeX!{\textbackslash}\code{newcommand}}
{\textbackslash}\code{newcommand}.

\subsubsection{commandes de liens hypertextes}

Les commandes permettant de cr\'eer des liens hypertextes d\'ependent du
types de liens et du type d'affichage d\'esir\'e.

\begin{center}
% use packages: array
\begin{longtable}{p{6cm} | p{9cm}}
 \textbf{commandes} & \textbf{description} \\
 \hline
\code{{\textbackslash}pdflaunch\{chemin\}\{texte\}} & Le lien \code{chemin} est cr\'e\'e sur le mot   \code{texte}. Ici,
le chemin peut-\^etre le chemin relatif (\code{../../../../doc/html/figure1.gif}).\\
\hline
\code{{\textbackslash}doxygen\{xClasse\}} &  Le mot   \code{xClasse} est \'ecrit dans le texte et renvoie vers la
documentation   Doxygen \index{Doxygen!classes} concernant cette classe. Pour la classe \code{xValue} il faut \'ecrire
le mot \code{xValue} \`a la place de \code{xClasse}. Le chemin relatif utilis\'e sera
\code{\{../../../../doc/html/classxfem\_1\_1xValue.html\}}      \\
\hline
\code{{\textbackslash}web\{url\}} &  Le lien vers l'adresse \code{url} est cr\'e\'e sur le texte  \code{url}  \\
 \hline
\code{{\textbackslash}webify\{url\}\{texte\}} &  Le lien vers
l'adresse \code{url} est cr\'e\'e sur le texte  \code{texte}. Cette
commande est similaire \`a \code{{\textbackslash}pdflaunch}
\end{longtable}
\end{center}






\subsubsection{Mise en forme de texte particulier}

Des raccourcis permettent de mettre en forme le texte :

\begin{center}
% use packages: array
\begin{longtable}{p{7cm} | p{8cm}}
 \textbf{commandes} & \textbf{description} \\
 \hline
\code{{\textbackslash}code\{cmd\}} & Le texte \guil{cmd} est ecrit : \code{cmd}. Il distingue les mots d\'esignant
des commandes du texte lui-m\^eme. \\
\hline
\code{{\textbackslash}touche\{CAR\}} & Le mot \guil{CAR} est ecrit : \touche{CAR}. Il permet de distinguer le texte de
la touche correspondante au texte. \\
\hline \code{{\textbackslash}guil\{this is written in english\}} &
Le texte \textit{this is written in english} est ecrit entre
guillements : \guil{this is written in english}.
\end{longtable}
\end{center}




\subsubsection{Le code \LaTeX~des fichiers \code{.h}}

 \code{{\textbackslash}texfromh\{xClasse\}} est equivalent \`a la
 commande~:\\ \code{{\textbackslash}input\{../../../../doc/tex\_from\_h/xClasse.tex\}}.



\subsubsection{Les figures}

Des raccourcis permettent de mettre en forme rapidement les figures
et les informations qui leur sont associ\'ees. Une seule ligne de
commande et n\'ecessaire pour introduire une figure dans la
documentation :



\begin{center}
% use packages: array
\begin{longtable}{p{8cm} | p{7cm}}
 \textbf{commandes} & \textbf{description} \\
 \hline
\code{{\textbackslash}onefigure\{larg\}\{dess\}\{leg\}} &  le fichier image \code{dess} est introduit comme image
de largeur \code{larg}. Le nom du fichier est introduit sans extension, et les extensions .pdf, .png, .jpg, .eps, .mps
sont successivement test\'ees pour chercher le fichier correspondant. La figure pourra alors \^etre appel\'ee par la r\'ef\'erence :
\code{{\textbackslash}ref\{fig$:$dess\}}. \code{leg} est la l\'egende qui sera appos\'ee sous la figure.\\
\hline
\code{{\textbackslash}twofigures\{l1\}\{fic1\}\{l2\}\{fic2\} \{leg\}\{lab\}} &  Cette commande permet de pr\'esenter une figure
compos\'ee de deux sub-figures \index{LaTeX!subfigures} provenant des fichiers \code{fic1} et \code{fic2}, et affich\'ees avec
 les largeurs respectives \code{l1} et \code{l2}.   Les noms de fichier sont introduits sans extension, et les extensions .pdf,
 .png, .jpg, .eps, .mps sont successivement test\'ees pour chercher les fichiers correspondants. La figure pourra alors \^etre
 appel\'ee par la reference : \code{{\textbackslash}ref\{lab\}} alors que chaque sub-figures aura pour r\'ef\'erence respectivement
  \code{{\textbackslash}ref\{fig$:$fic1\}}  et \code{{\textbackslash}ref\{fig$:$fic2\}} . \code{leg} est la l\'egende qui sera
   appos\'ee sous la figure.\\
\hline
\code{{\textbackslash}fourfigures\{l1\}\{fic1\}\{l2\}\{fic2\} \{l3\}\{fic3\}\{l4\}\{fic4\}\{leg\}} &  Cette commande permet de
pr\'esenter une figure compos\'ee de quatre sub-figures provenant des fichiers \code{fic1},  \code{fic2},  \code{fic3} et \code{fic4}
 de  largeurs respectives \code{l1}, \code{l2}, \code{l3}   et \code{l4}, de r\'ef\'erence
 respective \code{{\textbackslash}ref\{fig$:$fic1\}}, \code{{\textbackslash}ref\{fig$:$fic2\}},
  \code{{\textbackslash}ref\{fig$:$fic3\}} et \code{{\textbackslash}ref\{fig$:$fic4\}}.
     La figure globale n'a pas de r\'ef\'erencement.  La l\'egende  appos\'ee sous la figure globale sera  \code{leg} \\
\hline
\code{{\textbackslash}fourturnedfigures\{l1\}\{fic1\}\{l2\} \{fic2\}\{l3\}\{fic3\}\{l4\}\{fic4\}\{leg\}} &  Cette
 commande est identique \`a la pr\'ec\'edente mais chaque figure est tourn\'ee  de $90^\circ$.  \\
\hline
\code{{\textbackslash}inheritgraph\{larg\}\{xClass\}} &  le   diagramme d'h\'eritage de la classe  \code{xClass} est
introduit comme image de largeur \code{larg}. Une l\'egende automatique est appos\'ee sous la figure.\\
\hline
\code{{\textbackslash}collabgraph\{larg\}\{xClass\}} &  le   diagramme de collaboration de la classe  \code{xClass}
est introduit comme image de largeur \code{larg}. Une l\'egende automatique est appos\'ee sous la figure.\\
\hline
\code{{\textbackslash}dependgraph\{larg\}\{xClass\}} &  le   diagramme de dependance de la classe  \code{xClass} est
introduit comme image de largeur \code{larg}. Une l\'egende automatique est appos\'ee sous la figure.\\
\hline
\end{longtable}
\end{center}

Exemples :

\begin{verbatim}
\onefigure{4cm}{efg_support}{Trois supports couvrant un point $\bfx$.}

\twofigures{6cm}{uniform}{6cm}{general}{Surface de discontinuit\'e
plac\'ee sur un maillage uniforme (a) et non uniforme (b). Les noeuds
encercl\'es sont enrichis par la fonction Heaviside.}{fig:both_disc}

\fourfigures90{5cm}{f1-1}{5cm}{f1-2}{5cm}{f1-3}{5cm}{f1-4}
{visualisation des fonctions $\{F_1^l(\bfx)\}$}
\end{verbatim}




\subsubsection{Les \'equations \index{LaTeX!\'equations}}

Des raccourcis permettent \'egalement de mettre en forme les \'equations
en respectant les notations. La liste est longue. Nous invitons les
personnes int\'eress\'ees \`a aller voir le fichier
\code{XfemGuideConfiguration.tex}.
