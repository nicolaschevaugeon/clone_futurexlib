
%%%%%%%%%%%%%%%%%%%%
\section{Introduction}
Ce chapitre se veut une introduction
\'e la m\'ethode des \'el\'ements finis \'etendu \index{m\'ethode des \'el\'ements finis \'etendus}(commun\'ement appel\'e X-FEM pour \guil{eXtended Finite Element Method}).
                             
Cette extension \'e la M\'ethode des El\'ements Finis est partie du constat que beaucoup
d'applications industrielles restent encore aujourd'hui
hors de port\'ee de la
m\'ethode des \'el\'ements finis classiques pour des raisons
de gestion de maillage, ou plus pr\'ecis\'ement par le fait
que le maillage doive respecter toutes
les surfaces physiques du probl\'eme
(fissures, fronts de solidification, interfaces entre
mat\'eriaux, ...). Ces surfaces peuvent \^etre de forme
complexe et/ou \'evoluer dans le temps.

Les surfaces physiques doivent \^etre  maill\'ees  car
l'interpolation \'el\'ements finis est classiquement
tr\'es (trop) r\'eguli\'ere.
L'approche X-FEM permet d'enrichir les \'el\'ements finis avec des modes
moins r\'eguliers (discontinus ou \'e d\'eriv\'ees discontinues)
pour qu'ils puissent \^etre travers\'es par des discontinuit\'es.
Ainsi, les surfaces physiques peuvent \^etre positionn\'ees de mani\'ere
quelconque par rapport au maillage et ce dernier peut \^etre conserv\'e
lors de leur \'evolution. La technique centrale dans cet
enrichissement est la partition de l'unit\'e \index{partition de l'unit\'e} \cite{Babuska:PUFEM}.

Les surfaces ne sont donc plus maill\'ees et, pour les localiser sur le
maillage, la notion de fonction de niveau est tr\'es appropri\'ee.
A chaque noeud dans le voisinage de la surface on associe la distance
sign\'ee \'e cette surface (compt\'ee positivement d'un c\'et\'e et
n\'egativement de l'autre).
Cette fonction distance peut \^etre interpol\'ee sur chaque \'el\'ement
avec les fonctions \'el\'ements finis classique de premier ordre.
L'iso-z\'ero de la fonction localise compl\'etement l'interface.
En clair, dans le contexte X-FEM, les surfaces sont stock\'ees par
un champ \'el\'ements finis d\'efini dans le voisinage de la surface.
Ces champs ``g\'eom\'etriques'' participent au calcul au m\'eme
titre que les champs physiques (d\'eplacement, temp\'erature, ...).
Il se peut que ces surfaces \'evoluent dans le temps sous
un champ de vitesse.
Pour prendre en compte cette \'evolution, la m\'ethode des
\guil{level sets}\index{level sets} \cite{Sethian:book} est particuli\'erement
adapt\'ee. Nous utiliserons aussi bien l'expression
\guil{fonction de niveau} que \guil{level set}.


Dans la prochaine section, la partition de l'unit\'e sera d\'etaill\'ee et
l'on s'int\'eressera \'e l'historique de son av\'enement et ses relations
avec les m\'ethodes sans maillage. Dans la section
\ref{sec:model-de-disc}, cette technique sera utilis\'ee pour
introduire des surfaces de discontinuit\'e dans un maillage. Dans
cette section, on ne s'inqui\'etera pas de l'\'evolution possible de ces
surfaces ni de leur repr\'esentation ``informatique''. Ces deux
questions seront \'etudi\'ees dans la section \ref{sec:local-et-evol}.

La section \ref{sec:applications-et-mise}
discute certains points de la mise
en oeuvre de l'approche X-FEM et d\'ecrit deux
domaines d'applications~:
la m\'ecanique de la rupture et l'homog\'en\'eisation.

Ce document se veut une introduction de la m\'ethode X-FEM
et un condens\'e des outils de base.
Dans les applications (chapitre~\ref{Applications_de_XFEM}), nous nous sommes pench\'es sur des
mod\'eles \'elastiques lin\'eaires en statique.
D'autres travaux r\'ecents ou en cours, \'etendent l'approche
au non lin\'eaire ou \'e la dynamique.
Ces travaux sont cit\'es \'e la section~\ref{sec:conclusion-et-autres}.





\section{De Rayleigh-Ritz \'e la partition de l'unit\'e}
\label{sec:la-partition-de}

La m\'ethode X-FEM se distingue de la m\'ethode des \'el\'ements finis
classique par le rajout dans l'approximation \index{approximation} de fonctions
suppl\'ementaires \'e l'aide d'une technique appel\'ee \index{partition de l'unit\'e} partition de
l'unit\'e~\cite{Babuska:PUFEM}.  Avant de d\'ecrire cette technique
et l'usage qui en est fait dans X-FEM, nous discutons diff\'erentes
m\'ethodes classiques d'approximation du principe variationnel en
d\'eplacement pour l'\'elasticit\'e~:
l'approximation de Rayleigh-Ritz, la
m\'ethode des \'el\'ements finis et les m\'ethodes sans maillage
(\guil{meshfree methods}).

\subsection{Formulation \index{formulation du probl\'eme} du probl\'eme de r\'ef\'erence et notations}\label{probleme_de_reference}

Le milieu solide \'etudi\'e occupe un domaine $\Ome$ de fronti\'ere
$\Gam$. Cette fronti\'ere se compose des l\'evres de la fissure $\Gam_{c^+}$ et $\Gam_{c^-}$  suppos\'ees libres
 de tractions, d'une partie $\Gam_u$ \'e d\'eplacements impos\'es not\'es   $\overline{\bfu}$ et
d'une partie $\Gam_t$ \'e tractions impos\'ees not\'ees   $\overline{\bft}$.
Les contraintes, d\'eformations et d\'eplacements sont not\'es
respectivement $\bfsig$, $\bfeps$ et $\bfu$. L'hypoth\'ese des petites
perturbations est utilis\'ee.
En l'absence de forces de volume, les \'equations d'\'equilibre sont~:
\begin{eqnarray}
\label{eq:equil}
\bfnabla \cdot \bfsig & = & \bfzer \text{\;sur\;} \Ome  \\
\bfsig \cdot \bfn & = & \overline{\bft} \text{\;sur\;} \Gam_t            \\
\bfsig \cdot \bfn & = &  0  \text{\;sur\;} \Gam_{c^+}        \\
\bfsig \cdot \bfn & = & 0  \text{\;sur\;} \Gam_{c^-}
\end{eqnarray}
o\'e $\bfn$ est la normale ext\'erieure \'e la fronti\'ere consid\'er\'ee. Les
\'equations cin\'ematiques s'\'ecrivent~:
\begin{eqnarray}
    \bfeps & = & \bfeps(\bfu) = \bfnabla_{\rms} \bfu  \quad\mbox\;sur\;\quad
    \Ome \\
    \bfu   & = & \overline{\bfu}  \quad\mbox\;sur\;\quad \Gam_u
\end{eqnarray}
o\'e $\bfnabla_{\rms}$ est la partie sym\'etrique de l'op\'erateur
gradient\index{Op\'erateur!gradient}.  Finalement, la loi de comportement consid\'er\'ee est
\'elastique~:
\begin{equation}
    \bfsig = \bfC:\bfeps
\end{equation}
o\'e $\bfC$ est le tenseur de Hooke.
Les espaces des champs de d\'eplacement admissibles, $\cu$, et
admissibles \'e z\'ero, $\cu_0$, sont d\'efinis par~:
\begin{eqnarray}
    \cu & = & \{ \bfv \in \cv: \bfv = \overline{\bfu} \text\;sur\; \Gam_u \} \\
    \cu_0 & = & \{ \bfv \in \cv: \bfv = \bfzer \text\;sur\; \Gam_u \}
\end{eqnarray}
o\'e l'espace $\cv$ est reli\'e \'e la r\'egularit\'e de la solution
et est d\'etaill\'e par exemple dans \cite{Babuska:Corners} et \cite{Grisvard:ellip}.  Cet espace
contient des champs de d\'eplacement discontinus au passage de la fronti\'ere $\Gam_c$.
La forme faible des \'equations d'\'equilibre s'\'ecrit~:
\begin{equation}
    \int_{\Ome}    \bfsig:\bfeps(\bfv) \dint \Ome =
    \int_{\Gam_t} \ov{\bft}\cdot\bfv  \dint \Gam
    \quad \forall \bfv \in \cu_0
    \label{eq:equil2}
\end{equation}
Notons que la fronti\'ere $\Gam_c$ ne contribue pas \'e la forme
faible car libre de traction.
En combinant~(\ref{eq:equil2}) avec la loi constitutive\index{loi constitutive}
et les \'equations cin\'ematiques, le
probl\'eme en d\'eplacement est obtenu~:
\begin{equation}
    \label{eq:varia}
    \text{trouver} \; \bfu\in\cu  \; \vert \;
    \int_{\Ome}   \bfeps(\bfu):\bfC:\bfeps(\bfv) \dint \Ome =
    \int_{\Gam_t} \ov{\bft}\cdot\bfv  \dint \Gam
    \quad \forall \bfv \in \cu_0
\end{equation}

\subsection{L'approximation \index{Approximation} de Rayleigh-Ritz}


Dans la m\'ethode de Rayleigh-Ritz, l'approximation s'\'ecrit comme
une combinaison lin\'eaire de modes de d\'eplacement $\bfphi_i(\bfx), i = 1,
\dots, N$ d\'efinis sur l'ensemble du domaine~:
\begin{equation}
    \bfu(\bfx) = \sum_i^N a_i \bfphi_i(\bfx)
\end{equation}
Ces modes doivent satisfaire \emph{a priori} les conditions
aux limites
essentielles (les d\'eplacements impos\'es sont consid\'er\'es nuls
pour simplifier la pr\'esentation).





L'introduction de cette
approximation dans le principe variationnel\index{principe variationnel} (\ref{eq:varia}) conduit au
syst\'eme d'\'equations
\begin{equation}
\label{eq:global_systeme}
K_{ij} a_j = f_i, \quad i = 1, \ldots, N
\end{equation}
o\'e la sommation sur l'indice $j$ est d'application et o\'e
\begin{eqnarray}
K_{ij} & = & \int_{\Ome}   \bfeps(\bfphi_i):\bfC:\bfeps(\bfphi_j) \dint
\Ome \\
f_i & = & \int_{\Gam_t} \ov{\bft}\cdot\bfphi_i  \dint \Gam
\end{eqnarray}

La m\'ethode de Rayleigh-Ritz offre une grande
libert\'e dans le choix des modes.  Ces modes peuvent par exemple
\^etre choisis de mani\'ere \'e satisfaire les \'equations
int\'erieures du domaine.  Cette m\'ethode a cependant
l'inconv\'enient de conduire \'e un syst\'eme lin\'eaire \'e matrice
dense.



\subsection{La m\'ethode des \'el\'ements finis\index{m\'ethode des \'el\'ements finis}}

Dans la m\'ethode des \'el\'ements finis, le domaine \'etudi\'e,
$\Ome$, est d\'ecompos\'e en sous-domaines g\'eom\'etriques de forme
simple $\Ome_e, e = 1, \ldots, N_e$ appel\'es \'el\'ements~:
\begin{equation}
    \Ome = \cup_{e=1}^{N_e} \Ome_e
\end{equation}
L'ensemble des \'el\'ements constituent le maillage\index{Maillage}. Sur
chaque \'el\'ement, le champ inconnu est \guil{approxim\'e} \'e l'aide de
fonctions de base simples, de type polyn\'emial, appel\'ees fonctions
d'approximation et de coefficients inconnus appel\'es degr\'es de
libert\'e. Les degr\'es de libert\'e\index{degr\'es de libert\'e} ont en
g\'en\'eral une signification m\'ecanique simple. Pour des \'el\'ements de
premier degr\'e, il s'agit de la valeur du d\'eplacement selon $x$, $y$
ou $z$ aux sommets (noeuds) de chaque \'el\'ement. D\'esignons par
$u_i^\alpha$ le d\'eplacement au noeud $i$ dans la direction $\alpha$
($\alpha$ d\'esigne la direction $x$,   $y$  ou $z$) et par
$\bfphi_i^\alpha$ la fonction d'approximation correspondante.


Dans le plan $xy$, l'approximation \'el\'ements finis sur l'\'el\'ement $\Ome_e$
s'\'ecrit~:
\begin{equation}
  \label{eq:approx_ef}
  \bfu(\bfx) \mid_{\Ome_e} = \sum_{i \in N_n} \sum_{\alpha}
  a_i^\alpha \bfphi_i^\alpha(\bfx)
\end{equation}
o\'e $N_n$ est l'ensemble  des noeuds de l'\'el\'ement $\Ome_e$.
Par exemple pour un triangle, les six fonctions
d'approximation sont donn\'ees par~:
\begin{equation}
  \{ \bfphi_i^\alpha \} = \{ \phi_1 \bfe_x, \phi_2 \bfe_x,
  \phi_3 \bfe_x, \phi_1 \bfe_y, \phi_2 \bfe_y, \phi_3 \bfe_y \}
\end{equation}
o\'e les fonctions de forme\index{Fonction!fonctions de forme} scalaires $\phi_1$, $\phi_2$ et $\phi_3$
sont lin\'eaires sur l'\'el\'ement et valent 0 ou 1 aux noeuds du triangle.
L'approximation~(\ref{eq:approx_ef}) permet
de repr\'esenter tous
les modes rigides ou de d\'eformations constantes sur l'\'el\'ement.
Cette condition doit \^etre remplie par l'approximation
pour tous les types d'\'el\'ements.
La continuit\'e du champ d'approximation sur le domaine
est obtenue en imposant que
les degr\'es de libert\'e d\'efinis en un noeud ont la m\^eme
valeur pour tous les \'el\'ements connect\'es en ce noeud.
La matrice de raideur, $K^e_{ij}$,
et le vecteur force, $f^e_i$, sont donn\'es pour un \'el\'ement fini par~:
\begin{eqnarray}
  \label{eq:local_raideur}
  K^e_{i\alpha, j\beta} & = & \int_{\Ome^e}
          \bfeps(\bfphi_i^\alpha):\bfC:\bfeps(\bfphi_j^\beta)
  \dint \Ome \\
  \label{eq:local_force}
  f^e_{i\alpha} & = & \int_{\Gam_t \cap \partial \Ome^e}   \ov{\bft}\cdot\bfphi_i^\alpha  \dint \Gam
\end{eqnarray}

Le syst\'eme global d'\'equations est obtenu en assemblant les
matrices\index{Matrice} et forces \'el\'ementaires dans une matrice de
raideur globale et un vecteur de force global. Dans la phase
d'assemblage, les \'equations associ\'ees aux degr\'es de libert\'e fix\'es
par des conditions aux limites de type Dirichlet ne sont pas
construites.

Contrairement \'e l'approximation de Rayleigh-Ritz,
le caract\'ere local de l'approximation\index{Approximation} \'el\'ement fini
conduit \'e des matrices creuses.
De plus, l'\'el\'ement fini a une interpr\'etation m\'ecanique
forte~: la cin\'ematique est d\'ecrite par des d\'eplacements
nodaux auxquels sont associ\'ees par dualit\'e\index{Dualit\'e} des forces nodales.
Le comportement de l'\'el\'ement est caract\'eris\'e par
la matrice de raideur \'el\'ementaire
qui relie les forces et d\'eplacements nodaux.
Le syst\'eme global \'e r\'esoudre impose
l'\'equilibre de la structure~:
la somme des forces nodales en chaque noeud doit
\^etre nulle.
Enfin, la m\'ethode des \'el\'ements finis s'est r\'ev\'el\'ee
tr\'es robuste pour de nombreuses applications dans l'industrie
ce qui en fait un outil de choix pour la simulation
num\'erique.
Cependant, l'utilisation de la m\'ethode des \'el\'ements finis
pour des probl\'emes \'e g\'eom\'etrie complexe
ou \'e \'evolution de surfaces internes
est actuellement g\^en\'ee  par l'aspect maillage.
Une de motivations derri\'ere les m\'ethodes sans maillage
est de s'affranchir des
difficult\'es.

\subsection{Les m\'ethodes sans maillage}\label{methode_sans_maillage}\index{m\'ethodes sans maillage}

De nombreuses recherches ont \'et\'e conduites
ces dix derni\'eres ann\'ees pour d\'evelopper
des m\'ethodes o\`u l'approximation ne repose pas
sur un maillage mais plut\'et sur un ensemble de points.
Diff\'erentes m\'ethodes existent \'e ce jour~:
\'el\'ements diffus~\cite{Touzot:diffus},
Element Free Galerkin method (EFG)~\cite{TB:EFG},
Reproducing Kernel Particle Method (RKPM)~\cite{Liu:RKPM93},
$h-p$ cloud method~\cite{DuarteOdenCloud96}.
Chaque point a un domaine d'influence
de forme simple (cercle ou rectangle par exemple en 2D) sur lequel
des fonctions d'approximation sont construites.
Ces fonctions sont nulles sur le bord et en dehors
du domaine d'influence. En clair, ces fonctions sont
 \'e support compact.
Par abus de langage, nous parlerons
du support $i$ pour le support associ\'e au point $i$.
Les fonctions d'interpolation d\'efinies sur le support $i$ sont
not\'ees $\bfphi_i^\alpha$, $\alpha = 1, \ldots, N_f(i)$
o\'e  $N_f(i)$ est le nombre de fonctions d'approximation
d\'efinies sur le support $i$.
Les degr\'es de libert\'e correspondants  sont not\'es
$a_i^\alpha$. L'approximation en un point quelconque
$\bfx$ s'\'ecrit~:
\begin{equation}
  \label{eq:approx_efg}
  \bfu(\bfx) = \sum_{i \in N_s(\bfx)} \sum_{\alpha = 1}^{N_f(i)} a_i^\alpha \bfphi_i^\alpha(\bfx)
\end{equation}
o\'e $N_s(\bfx)$ est l'ensemble des points $i$ dont le support
contient le point $\bfx$. La figure~\ref{fig:efg_support} montre par exemple un point
$\bfx$ couvert par trois supports\index{Support}.

\onefigure{4cm}{efg_support}{Trois supports couvrant un point $\bfx$.}


Les fonctions d'approximation sont construites de
mani\'ere \'e ce que l'approximation~(\ref{eq:approx_efg})
puisse repr\'esenter tous les modes rigides
et de d\'eformation constante sur le domaine.
Ces conditions sont n\'ecessaires pour prouver la convergence
de la m\'ethode. Les diff\'erentes approches
(\'el\'ement diffus, EFG, RKPM, \ldots)
se distinguent, entre autres, par les techniques
utilis\'ees pour la construction de ces fonctions d'interpolation.


Une fois les fonctions d'interpolation
construites, il est possible d'en rajouter
par enrichissement\index{Enrichissement}.
Diverses mani\'eres d'enrichir existent et nous d\'ecrirons
un  enrichissement qualifi\'e d'externe
par Belytschko et Fleming (1999)~\cite{Fleming:enrichment}.
L'enrichissement de l'approximation permet
de repr\'esenter
un mode de d\'eplacement donn\'e, par exemple
$F(\bfx)\bfe_x$ sur un sous-domaine donn\'e
$\Ome_F\subset\Ome$.
Notons $N_F$ l'ensemble des supports qui ont une intersection
avec le sous-domaine $\Ome_F$. L'approximation enrichie s'\'ecrit~:
\begin{equation}
  \label{eq:approx_efg_enrichi}
  \bfu(\bfx) = \sum_{i \in N_s(\bfx)}  \sum_{\alpha = 1}^{N_f(i)}
                                 a_i^\alpha \bfphi_i^\alpha(\bfx) +
               \sum_{i \in N_s(\bfx) \cap N_F}
                                 \sum_{\alpha = 1}^{N_f(i)}
                                 b_i^\alpha \bfphi_i^\alpha(\bfx) F(\bfx)
\end{equation}
o\'e les nouveaux degr\'es de libert\'e\index{degr\'es de libert\'e}, $b_i^\alpha$, multiplient les
fonctions d'interpolation enrichies $\bfphi_i^\alpha(\bfx) F(\bfx)$.
Montrons maintenant que la fonction $F(\bfx)\bfe_x$
peut-\^etre repr\'esent\'ee sur le sous-domaine $\Ome_F$.
En fixant \'e z\'ero tous les degr\'e de libert\'e $a_i^\alpha$
et en sortant la fonction $F(\bfx)$ du signe somme,
l'approximation en un point $\bfx \in \Ome_F$
s'\'ecrit~:
\begin{equation}
  \bfu(\bfx) =    \left( \sum_{i \in N_s(\bfx) \cap N_F}
                   \sum_{\alpha = 1}^{N_f(i)}
                   b_i^\alpha \bfphi_i^\alpha(\bfx) \right)
                   F(\bfx)
\end{equation}

Les degr\'es de libert\'e $b_i^\alpha$ peuvent \^etre
choisis de mani\'ere \'e ce que le facteur devant $F(\bfx)$
soit le mode rigide $\bfe_x$. Cela est possible
puisque les fonctions d'approximation $\bfphi_i^\alpha$
sont capables de repr\'esenter
n'importe quel mode rigide. En conclusion,
l'approximation~(\ref{eq:approx_efg_enrichi}) peut repr\'esenter
le d\'eplacement $F(\bfx)\bfe_x$ sur le sous-domaine $\Ome_F$.

L'enrichissement a permis
dans le cadre de l'Element Free Galerkin Method
de r\'esoudre des probl\'emes de propagation de
fissures en deux et trois dimensions sans remaillage~\cite{Krysl:3D}:
la fissure se propage a travers un nuage de points et
est mod\'elis\'ee par enrichissement de l'approximation
avec des fonctions $F(\bfx)$ discontinues sur la fissure
ou repr\'esentant la singularit\'e en fond de fissure.

La grande flexibilit\'e dans l'\'ecriture de l'approximation et
de son enrichissement ainsi que la possibilit\'e de cr\'eer des champs
d'approximation tr\'es r\'eguliers sont deux atouts importants des
m\'ethodes sans maillage et de l'approche EFG en particulier.
L'utilisation des m\'ethodes sans maillage pr\'esente cependant un
certain nombre de difficult\'es par rapport \'e la m\'ethode des
\'el\'ements finis~:
\begin{itemize}
\item Dans la m\'ethode des \'el\'ements finis l'assemblage de la
  matrice de raideur globale peut se faire en assemblant les
  contributions de chaque \'el\'ement. Dans les m\'ethodes sans
  maillage, l'assemblage se fait plut\'et en couvrant le domaine de
  points d'int\'egration et en ajoutant la contribution de chacun
  d'eux.  Le choix de la position et du nombre des points
  d'int\'egration n'est pas trivial pour un nuage arbitraire de points
  d'approximation ;
\item Les conditions aux limites de type Dirichlet sont d\'elicates \'e
  imposer ;
\item Les fonctions de forme sont \'e construire et ne sont pas
  explicites ;
\item La taille des domaines d'influence est un param\'etre dans la
  m\'ethode que l'utilisateur doit choisir avec soin.
\end{itemize}




\subsection{La partition de l'unit\'e}\label{partition_de_l_unite}\index{partition de l'unit\'e}

Melenk et Babu\v{s}ka (1996)~\nocite{Babuska:PUFEM} ont montr\'e
que la base \'el\'ement fini classique pouvait \^etre enrichie
(et donc qu'il n'\'etait pas n\'ecessaire de recourir
\'e des m\'ethodes sans maillage)
de mani\'ere \'e repr\'esenter une fonction donn\'ee sur un domaine
donn\'e. Leur point de vue peut \^etre r\'esum\'e comme suit.
Rappelons que l'approximation \'el\'ements finis s'\'ecrit sur un \'el\'ement $e$ d\'ecrivant le domaine $\Ome_e$~:
\begin{equation}
  \label{eq:approx_ef2}
  \bfu(\bfx) \mid_{\Ome_e} = \sum_{i \in N_n} \sum_{\alpha} a_i^\alpha
  \bfphi_i^\alpha(\bfx)
\end{equation}
o\'e $N_n$ est le nombre de noeuds de l'\'el\'ement $e$.
Comme les degr\'es de libert\'e\index{degr\'es de libert\'e} d\'efinis en un noeud ont la m\^eme
valeur pour tous les \'el\'ements connect\'es en ce noeud.
Les approximations sur chaque \'el\'ement
peuvent \^etre ``assembl\'ees''
pour donner
une approximation\index{Approximation} valable en tout point $\bfx$ du domaine~:
\begin{equation}
  \label{eq:approx_ef_glo}
  \bfu(\bfx)  = \sum_{i \in N_n(\bfx)} \sum_{\alpha} a_i^\alpha \bfphi_i^\alpha(\bfx)
\end{equation}
o\'e $N_n(\bfx)$ est l'ensemble des noeuds de l'\'el\'ement contenant
le point $\bfx$. Le domaine d'influence (support) de la fonction
d'interpolation
 $\bfphi_i^\alpha$ est l'ensemble des \'el\'ements connect\'es
au noeud $i$. L'ensemble $N_n(\bfx)$ est donc \'egalement l'ensemble
des noeuds dont le support couvre le point $\bfx$.
L'approximation \'el\'ements finis~(\ref{eq:approx_ef_glo}) peut ainsi \^etre
interpr\'et\'ee comme une particularisation
de l'approximation~(\ref{eq:approx_efg}) utilis\'ee dans les m\'ethodes
sans maillage~:
\begin{itemize}
\item le nuage de points est l'ensemble des noeuds du maillage;
\item le domaine d'influence de chaque noeud est
l'ensemble des \'el\'ements connect\'es \'e ce noeud.
\end{itemize}
Il est donc possible d'enrichir l'approximation \'el\'ements
finis par les m\^emes techniques que celles utilis\'ees dans les m\'ethodes
sans maillage. Voici l'approximation \'el\'ements finis enrichie
qui permet de repr\'esenter la fonction $F(\bfx) \bfe_x$ sur le
domaine $\Ome_F$~:
\begin{equation}
  \label{eq:approx_ef_enrichi}
  \bfu(\bfx)  = \sum_{i \in N_n(\bfx)} \sum_{\alpha} \bfphi_i^\alpha a_i^\alpha   +
  \sum_{i \in N_n(\bfx) \cap N_F}
  \sum_{\alpha} b_i^\alpha \bfphi_i^\alpha(\bfx) F(\bfx)
\end{equation}
o\'e $N_F$ est l'ensemble des noeuds dont le support\index{Support} a une
intersection avec le domaine $\Ome_F$.
La preuve est obtenue en fixant \'e z\'ero
les coefficients $a_i^\alpha$ et en prenant en compte
le fait que les fonctions d'approximation \'el\'ements finis
sont capables de repr\'esenter tous les modes rigides et
donc le mode $\bfe_x$.
Nous passons maintenant \'e l'utilisation concr\'ete
de la partition de l'unit\'e pour
la mod\'elisation de discontinuit\'es dans X-FEM.





\section{Mod\'elisation de discontinuit\'es\index{Discontinuit\'es} avec X-FEM}
\label{sec:model-de-disc}
L'extension X-FEM de la m\'ethode des \'el\'ements finis,
donne une grande
flexibilit\'e dans la g\'en\'eration des maillages  4~:
il n'est plus demand\'e au maillage de se conformer \'e des surfaces\index{Surfaces}
qu'elles soient ext\'erieures ou internes. Pour arriver \'e cette
flexibilit\'e,
il faut que l'approximation soit capable de mod\'eliser des vides ou
des discontinuit\'es au sein m\^eme des \'el\'ements. La m\'ethode la
partition de l'unit\'e d\'evelopp\'ee par Melenk et Babu\v{s}ka (1996)
est utilis\'ee \'e cette fin. Nous donnons dans cette section ces
diff\'erents enrichissements. Ensuite, nous nous int\'eressons \'e la
localisation (et l'\'evolution) de ces surfaces.



\subsection{Revisite du double noeud~: la fonction Heaviside\index{Fonction!Heavyside}}

L'id\'ee de l'utilisation de la partition de l'unit\'e pour repr\'esenter une
discontinuit\'e a d\'ebut\'e par une revisite du
concept de double noeud et sa g\'en\'eralisation~\cite{Moes:discont}.

\twofigures{5cm}{crack}{5cm}{grid}{Un maillage de quatre \'el\'ements avec un double noeud (a); sans double noeud (b)}{fig:dbl_noeud}


Avec des \'el\'ements finis conformes,
la repr\'esentation d'un champ discontinu sur un maillage
ne peut se faire qu'en bordure des \'el\'ements par l'utilisation
de doubles noeuds. La figure~\ref{fig:crack} montre par exemple
un maillage de 4 \'el\'ements dans lequel une discontinuit\'e
a \'et\'e introduite par le biais d'un double noeud (noeuds 9 et 10).
L'approximation \'el\'ements finis correspondante s'\'ecrit
\begin{equation}
\bfu^h = \sum_{i=1}^{10} \bfu_i \phi_i
\label{eq:simdisp}
\end{equation}
o\'e $\bfu_i$ est le d\'eplacement (vecteur) au noeud  $i$ et $\phi_i$
est la fonction de forme bi-lin\'eaire scalaire associ\'ee au noeud $i$.
Chaque fonction d'approximation $\phi_i$ a un support  compact
$\omega_i$ donn\'e par l'union des \'el\'ements connect\'es au noeud $i$.
R\'e\'ecrivons (\ref{eq:simdisp}) afin de faire appara\'etre une
approximation \'el\'ements finis correspondant \'e un maillage sans double
noeud, figure~\ref{fig:grid}, et un terme suppl\'ementaire.
D\'efinissons le d\'eplacement moyen $\bfa$ et le saut $\bfb$
\begin{equation}
\bfa = \frac{\bfu_{9} + \bfu_{10}}{2} \quad
\bfb = \frac{\bfu_{9} - \bfu_{10}}{2}
\label{eq:cfem}
\end{equation}
En inversant ce syst\'eme, il vient
\begin{equation}
\bfu_{9} = \bfa + \bfb \quad \bfu_{10} = \bfa - \bfb
\end{equation}
L'approximation \'el\'ements finis~(\ref{eq:simdisp}) peut alors
se r\'e\'ecrire
\begin{equation}
\bfu^h = \sum_{i=1}^{8} \bfu_i \phi_i + \bfa (\phi_{9} + \phi_{10}) +
\bfb (\phi_{9} + \phi_{10}) H(\bfx)
\label{eq:last}
\end{equation}
o\'e $H(\bfx)$ est une fonction discontinue donn\'ee par
\begin{equation}
H(x,y) = \begin{cases}
1 & \mbox{for}\quad y > 0 \\
-1 & \mbox{for}\quad  y<0
\end{cases}
\label{hdef}
\end{equation}
La somme des fonctions d'approximation associ\'ees aux noeuds 9 et 10
dans le maillage figure~\ref{fig:crack} correspond \'e la fonction
d'approximation localis\'ee sur le noeud 11 dans le maillage figure~\ref{fig:grid}.
On peut donc r\'e\'ecrire~(\ref{eq:last}) comme
\begin{equation}
\bfu^h = \sum_{i=0}^{8} \bfu_i \phi_i + \bfu_{11} \phi_{11}
+ \bfb \phi_{11} H(\bfx)
\label{eq:last2}
\end{equation}
Les deux premiers termes du membre de droite correspondent
\'e une approximation \'el\'ement fini classique sur le maillage
figure~\ref{fig:grid}. La fonction d'approximation
intervenant dans le
troisi\'eme terme est le produit d'une fonction d'approximation
classique
$\phi_{11}$ par la fonction discontinue $H(\bfx)$.
Ce troisi\'eme terme
peut s'interpr\'eter comme un enrichissement de la base
\'el\'ements finis par une technique de type partition de l'unit\'e.
La d\'erivation que nous venons de mener sur un petit maillage de
quatre \'el\'ements peut \^etre r\'eit\'er\'ee sur n'importe
quel  maillage 1D, 2D ou 3D contenant une discontinuit\'e
mod\'elis\'ee par doubles noeuds. Cette d\'erivation m\'enera au m\^eme
constat~: la mod\'elisation d'une discontinuit\'e\index{Discontinuit\'e} par double noeud
est \'equivalente \'e la mod\'elisation \'el\'ements finis classique
si on lui ajoute des termes correspondant \'e l'enrichissement
par la partition de l'unit\'e
des noeuds situ\'es sur le parcours de la discontinuit\'e.
Notons que les noeuds qui sont enrichis (ex-doubles noeuds)
sont caract\'eris\'es par le
fait que leur support est coup\'e en deux par la discontinuit\'e.

\subsection{G\'en\'eralisation du double noeud}
Supposons maintenant que l'on souhaite mod\'eliser une discontinuit\'e
qui ne suit pas le bord des \'el\'ements. Nous
proposons d'enrichir
tous les noeuds dont le support est (compl\'etement) coup\'e en deux
par la  discontinuit\'e~\cite{Moes:discont}.
En ces noeuds, nous ajoutons un degr\'e
de libert\'e (vectoriel) agissant sur la fonction d'approximation classique
au noeud multipli\'ee par une fonction discontinue $H(\bfx)$ valant 1 d'un
c\'et\'e de la fissure et -1 de l'autre.
Par exemple, sur la figure~\ref{fig:uniform-heavyside}, les noeuds encercl\'es seront enrichis.

% \twofigures{6cm}{uniform-heavyside}{6cm}{general-heavyside}{Surface de discontinuit\'e
% plac\'ee sur un maillage uniforme (a) et non uniforme (b).
% Les noeuds marqu\'es d'un point noir sont enrichis par la fonction Heaviside.}{fig:both_disc}
\onefigure{6cm}{uniform-heavyside}{Surface de discontinuit\'e
plac\'ee sur un maillage uniforme. Les noeuds marqu\'es d'un point noir sont enrichis par la fonction Heaviside.}


\subsection{Mod\'elisation d'une fissure\index{Fissure}}

Passons maintenant \'e une ligne de discontinuit\'e qui ne s\'epare pas le
domaine en deux zones distinctes, c'est-\'e-dire une fissure. Un noeud
dont le support n'est pas totalement coup\'e par la discontinuit\'e
\index{Discontinuit\'e} ne peut \^etre enrichi par la fonction
$H$\index{Fonction!Heavyside} car cela conduirait \'e allonger
artificiellement la discontinuit\'e. Par exemple, si pour le maillage
de la figure~\ref{fig:tip}, les noeuds $C$ et $D$ sont enrichis par
la discontinuit\'e, la fissure sera active jusqu'au point $q$ (le
champ de d\'eplacement sera discontinu jusqu'au point $q$). D'un autre
c\'et\'e, si seuls les noeuds $A$ et $B$ sont enrichis par la
discontinuit\'e, le champ de d\'eplacement n'est discontinu que jusqu'au
point $p$ et la fissure appara\'et malheureusement plus courte.

Afin de repr\'esenter la fissure sur sa ``bonne longueur'', les noeuds
dont le support contient la pointe de fissure\index{pointe de
fissure} (noeuds avec carr\'e figure~\ref{fig:tip}) sont enrichis avec
des fonctions discontinues jusqu'au point $t$ mais pas au-del\'e. De
telles fonctions sont fournies par les modes de d\'eplacement
asymptotiques\index{Asymptotique} (\'elastiques si le calcul est
\'elastique) en pointe de fissure. Cet
enrichissement\index{Enrichissement}, d\'ej\'e utilis\'e
par~\cite{TBlack:minim} et \cite{Stroub:GFEM1}, permet en outre des
calculs pr\'ecis puisque les caract\'eristiques asymptotiques du champ
de d\'eplacement sont incorpor\'ees dans le calcul.

\onefigure{6cm}{tip}{Les noeuds entour\'es d'un carr\'e sont
                            enrichis par les modes asymptotiques en
                            pointe de fissure.}


Nous sommes maintenant en mesure de d\'etailler la mod\'elisation par X-FEM
d'une fissure compl\'ete dont la position est quelconque par rapport au maillage
(voir illustration figure~\ref{fig:both}). L'approximation \'el\'ements finis enrichie s'\'ecrit~:

\begin{eqnarray*}
\bfu^h(\bfx) & = & \sum_{i\in I} \bfu_i \phi_i(\bfx) +
\sum_{i\in L} \bfa_i \phi_i(\bfx) H(\bfx) \\ \nonumber
& + & \sum_{i\in K_1} \phi_i(\bfx) (\sum_{l=1}^{4} \bfb_{i,1}^{l}  F^l_{1}(\bfx)) +
\sum_{i\in K_2} \phi_i(\bfx) (\sum_{l=1}^{4} \bfb_{i,2}^{l}  F^l_{2}(\bfx)) \nonumber
\label{eq:appgen}
\end{eqnarray*}
o\'e~:
\begin{itemize}
\item $I$ est l'ensemble des noeuds du maillage ;
\item $\bfu_i$ est le degr\'e de libert\'e (vectoriel) classique au noeud $i$ ;
\item $\phi_i$ est la fonction de forme (scalaire) associ\'ee au noeud $i$ ;
\item $L\subset I$ est l'ensemble des noeuds enrichis par
la discontinuit\'e et  les coefficients $\bfa_i$ sont les
degr\'es de libert\'e (vectoriels) correspondants.
Un noeud appartient \'e $L$ si son support est coup\'e par
la fissure mais ne contient aucune de ses pointes (ces
noeuds sont encercl\'es sur la figure~\ref{fig:both} dans
le cas d'un maillage uniforme et non uniforme) ;

\item $K_1\subset I$ et $K_2\subset I$ sont les ensembles
des noeuds \'e enrichir pour mod\'eliser les fonds de fissures
1 et 2, respectivement.
Les degr\'es de libert\'e correspondants sont
$\bfb_{i,1}^{l}$ et $\bfb_{i,2}^{l}$, $l=1,\ldots,4$.
Un noeud appartient \'e $K_1$
($K_2$)  si son support contient la premi\'ere (seconde)
pointe de fissure. Ces noeuds sont entour\'es d'un carr\'e
figure~\ref{fig:both}.
\end{itemize}

\twofigures{5cm}{uniform}{5cm}{general}{Fissure plac\'ee sur un maillage uniforme (a)
et non uniforme (b). Les noeuds encercl\'es sont enrichis par la fonction $H(\bfx)$
et les noeuds entour\'es d'un carr\'e sont enrichis par les modes asymptotiques \index{Asympotique!modes asymptotiques} en fond
de fissure.}{fig:both}

Les fonctions $F^l_1(\bfx), l=1,\ldots,4$ mod\'elisant le fond de fissure
sont donn\'ees en \'elasticit\'e par~:
\begin{eqnarray}
\label{enrich_fun}
\{F_1^l(\bfx)\} & \equiv \left\{ \sqrt{r}
\mbox{sin}(\frac{\theta}{2}), \sqrt{r} \mbox{cos}(\frac{\theta}{2}),
\sqrt{r}\mbox{sin}(\frac{\theta}{2})\mbox{sin}(\theta),
\sqrt{r}\mbox{cos}(\frac{\theta}{2})\mbox{sin}(\theta) \right\}
\end{eqnarray}
o\'e $(r,\theta)$ sont les coordonn\'ees polaires
dans les axes locaux en fond de fissure
(``tip'' 1 sur la  figure~\ref{fig:local}).
\onefigure{6cm}{local}{Axes locaux pour les coordonn\'ees polaires en pointe de fissure.}

Les figures \ref{fig:f1-1}, \ref{fig:f1-2}, \ref{fig:f1-3} et \ref{fig:f1-4} illustrent respectivement les fonctions
 $\{F_1^1(\bfx)\}$, $\{F_1^2(\bfx)\}$ , $\{F_1^3(\bfx)\}$ et  $\{F_1^4(\bfx)\}$.

\fourfigures{5cm}{f1-1}{5cm}{f1-2}{5cm}{f1-3}{5cm}{f1-4}{visualisation des fonctions $\{F_1^1(\bfx)\}$ (a) ; $\{F_1^2(\bfx)\}$ (b)~; $\{F_1^3(\bfx)\}$ (c)~; $\{F_1^4(\bfx)\}$ (d)~;}

Il est \'e noter que la deuxi\'eme fonction est discontinue sur la fissure ($cos(\frac{2 8pi}{2} = - cos(0) $).
De mani\'ere similaire les fonctions $F^l_2(\bfx)$ sont \'egalement
donn\'ees par~(\ref{enrich_fun}); le rep\'ere de coordonn\'ees
polaires \'etant maintenant d\'efini pour la seconde pointe de fissure.









La fonction $H(\bfx)$ est discontinue sur la fissure
et de valeur constante de part et d'autre de celle-ci~:
+1.0 d'un c\'et\'e et -1.0 de l'autre.
La d\'etermination des fonctions d'enrichissement F et H en
un point quelconque s'exprime ais\'ement si l'on repr\'esente
la fissure par level sets. Ceci sera expos\'e plus loin dans
cette section.

\subsection{Extension au 3D, plaques fissur\'ees et fissures \'e branches\index{Fissure!branches}}
L'extension au cas tridimensionnel de la mod\'elisation de fissures
par X-FEM a \'et\'e r\'ealis\'ee par~\cite{Sukumar:3D}. Cette extension est
assez directe. Tout comme dans le cas bi-dimensionnel, le fait qu'un
noeud soit enrichi ou non et le type d'enrichissement d\'epend de la
position relative du support associ\'e au noeud par rapport au front
de la fissure et \'e son int\'erieur. Le support d'un noeud est un
volume, le front de fissure une courbe (ou plusieurs courbes
disjointes) et l'int\'erieur de la fissure est une surface.

L'extension aux plaques pour le mod\'ele de
Reissner-Mindlin a \'et\'e r\'ealis\'ee dans
~\cite{Dolbow:melosh} et \cite{Dolbow:plate2}.
La formulation \'el\'ements finis de d\'epart
est celle du MITC4 d\'evelopp\'ee par
Bathe et al. (1990)\nocite{Bathe:MITC}
qui est connue pour ses bonnes propri\'et\'es
vis-\'e-vis du bloquage en cisaillement.
Les champs de rotation et de d\'eplacement transversal
sont enrichis pour mod\'eliser
la fissure. L'enrichissement en fond de fissure
est bas\'e sur les champs asymptotiques
obtenus par Knowles et Wang (1960). \nocite{Knowles:plate}

Lorsque plus de deux segments aboutissent
en un point de la fissure,
(fissures en Y ou en croix),
un soin particulier doit \^etre apport\'e \'e
l'enrichissement. Cette extension a \'et\'e r\'ealis\'ee
dans \cite{Daux:holes}.
En  un point o\`u plus de deux branches
aboutissent, l'enrichissement fait intervenir une fonction
dite de jonction.
Dans le cas o\`u le maillage se conforme \'e la fissure
\'e branches, l'enrichissement conduit \'e la m\^eme mod\'elisation
que la mod\'elisation classique par doubles noeuds
(et triples au noeud A).
Les diff\'erents
types d'enrichissement pour une fissure en Y sont illustr\'es
figure~\ref{fig:enrich_mesh2}.

       \onefigure{7cm}{enrich_mesh2}{Strat\'egie pour la mod\'elisation\
       d'une fissure en Y. Les diff\'erents types d'enrichissement sont
       indiqu\'es.}


Dans le cadre de X-FEM,
la mod\'elisation de la cin\'ematique
de bandes de cisaillement est un
cas particulier de la mod\'elisation de fissures~:
une bande de cisaillement est une fissure sans fond
\'e discontinuit\'e purement tangentielle.
Tout noeud dont le support est coup\'e par la bande
est enrichi \'e l'aide d'une fonction vectorielle
parall\'ele \'e la bande et dont la direction s'inverse
au passage de la bande~\cite{Belytschko:circle}.
La mod\'elisation de discontinuit\'e de type
rotation (cr\'eation du mode rigide
de rotation pour un disque trac\'e sur le maillage)
est \'egalement propos\'ee dans~\cite{Belytschko:circle}.



\subsection{Mod\'elisation de trous}
\label{sec:trou_un}
La mod\'elisation classique
de trous par la m\'ethode des \'el\'ements finis
impose au maillage de se conformer \'e la fronti\'ere
de ces trous. La m\'ethode X-FEM permet de se d\'ebarrasser
de cette contrainte.
Pour un noeud dont le support coupe la fronti\'ere
du trou, la fonction d'approximation classique
est multipli\'ee par une fonction valant 1.0 dans
la mati\'ere et 0.0 dans les trous.
Un noeud dont le support est compl\'etement
\'e l'int\'erieur du trou ne donne pas
lieu \'e la cr\'eation de degr\'es de libert\'e.
Il est \'e noter qu'un noeud peut \^etre actif m\^eme si il
est situ\'e dans un trou. Ce qui importe est le fait que
son support couvre de la mati\'ere ou non.
La figure~\ref{fig:mesh_holes}
illustre la s\'election des noeuds
\'e \'eliminer et les noeuds pour lesquels
les fonctions d'approximation  doivent
\^etre modifi\'ees dans le cas de deux trous
plac\'es sur un maillage.

    \onefigure{7cm}{mesh_holes}{S\'election des noeuds pour le traitement des trous dans X-FEM.}


Au niveau de l'impl\'ementation, la m\'ethode
propos\'ee ci-dessus revient pour les
\'el\'ements coup\'es par le trous \'e
restreindre l'int\'egration
\'e la ``fraction mati\'ere de l'\'el\'ement''\cite{Daux:holes}.


\section{Localisation et \'evolution des surfaces par level sets\index{level sets}}
\label{sec:local-et-evol}
Dans la section pr\'ec\'edente, les fronti\'eres des trous\index{Trous}
et des fissures \'etaient d\'ecrites par des entit\'es g\'eom\'etriques
(segments de droite).
Cette description peut-\^etre qualifi\'ee d'explicite
(ou ``lagrangienne'') puisque
la fronti\'ere est d\'ecrite par l'ensemble des points appartenant
\'e cette fronti\'ere.
Une autre description, implicite (``eul\'erienne''), est possible.

\subsection{Description d'une surface par level set\index{Surface!level set}}

Cette description consiste \'e transformer la repr\'esentation explicite
de la fronti\'ere en un champ de valeur d\'efinie dans un voisinage de
cette fronti\'ere. La vision eul\'erienne d'une surface et les
algorithmes de propagation qui en d\'ecoulent sont \'e la base de la
m\'ethode des level sets dont les pionniers sont Sethian et
Osher~\cite{osher-sethian,Sethian:book}. Le champ dont nous parlons
est appel\'e ``level set'' que nous traduirons par fonction de niveau.
La fonction de niveau associ\'ee \'e une fronti\'ere donne en chaque noeud
du maillage la distance de ce point \'e la fronti\'ere. Le signe plac\'e
devant la distance indique si le point se trouve d'un c\'et\'e ou de
l'autre de la fronti\'ere. Pour un trou, la fonction de niveau est
choisie arbitrairement n\'egative \'e l'int\'erieur du trou et positive \'e
l'ext\'erieur du trou avec une valeur donnant la plus courte distance
\'e la fronti\'ere du trou. L'iso-z\'ero de la fonction de niveau donne la
position du bord du trou. En pratique, la fonction de niveau est
calcul\'ee aux noeuds et interpol\'ee entre les noeuds par les fonctions
de base \'el\'ements finis classiques. Pour des \'el\'ements triangulaires,
l'iso z\'ero est donc un segment de droite et pour un t\'etra\'edre un
morceau de plan. Une fois cette fonction de niveau cr\'e\'ee, elle
permet de g\'erer les op\'erations g\'eom\'etriques n\'ecessaires \'e la
mod\'elisation de trous par X-FEM, \'e savoir~: trouver les \'el\'ements
coup\'es par le bord d'un trou, \'eliminer les noeuds dont le support
est totalement \'e l'int\'erieur d'un trou et modifier la fonction
d'approximation associ\'ee aux noeuds dont le support est coup\'e par le
bord d'un trou.

Une fissure peut \'egalement \^etre rep\'er\'ee par level sets.
Alors que pour un trou une seule fonction suffit, il en faut deux
pour une fissure. En effet, une fissure ne s\'epare pas un domaine
en deux zones distinctes.
La premi\'ere level set, baptis\'ee $ls_\mathrm{n}$,
donne la position de la surface sur laquelle est pos\'ee la fissure.
La seconde, baptis\'ee $ls_\mathrm{t}$, permet de localiser le front
de la fissure sur cette surface. La figure~\ref{fig:lsn_lst}
illustre ces deux level sets pour la
localisation d'une fissure en
2D et  en 3D.


    \twofigures{7cm}{lset1_color}{7cm}{lset3d}{Localisation d'un fissure par deux
    level sets en 2D (a) et 3D (b).}{fig:lsn_lst}


Les level sets $ls_\mathrm{n}$ et $ls_\mathrm{t}$ donnent respectivement les distances normale et
tangentielle \'e la fissure.
En un point proche du front de fissure, le calcul des fonctions
d'enrichissement~(\ref{enrich_fun}) n\'ecessite le calcul des param\'etres
$r$ et $\theta$ dont la signification g\'eom\'etrique est donn\'ee sur la
figure~\ref{fig:local_axis}.
Ces deux param\'etres ont une expression simple en terme des deux
level sets~:
\begin{equation}
  \label{eq:rtheta}
  r = \sqrt{ls_\mathrm{n}^2 + ls_\mathrm{t}^2}, \quad \theta = \tan^{-1}(\frac{ls_\mathrm{n}}{ls_\mathrm{t}})
\end{equation}



      \onefigure{6cm}{local_axis}{Signification des param\'etres $r$ et $\theta$ en front de fissure.}

\subsection{Evolution d'une level set sous un champ de vitesse \index{Level sets!\'evolution}}

Nous consid\'erons ici une fonction level set, $\phi$,
soumise \'e un champ de
vitesse, $\bfV$ (pour fixer les id\'ees, une cavit\'e qui croit dans un
mat\'eriau).
La mise \'e jour de la level set pour conna\'etre la nouvelle
position de la cavit\'e apr\'es un pas de temps $\Delta t$ se d\'ecompose en
trois \'etapes \cite{Sethian:book}~:
\begin{itemize}
\item \underline{Phase d'extension de la vitesse~:} la vitesse n'est en
    g\'en\'eral connue que sur la surface physique consid\'er\'ee
    (cavit\'e par exemple). Or, pour mettre \'e jour
    la level set (voir \'equation~(\ref{eq:1})), il
    faut disposer de cette vitesse partout. La premi\'ere phase
    consiste donc en une ``extension'' du champ des vitesses
    depuis la surface vers tout le domaine.
    Dans cette phase, on cherche \'e r\'esoudre  l'\'equation~:
    \begin{equation}
    \nabla\phi.\nabla V_{\phi}=0
    \label{eq:3}
    \end{equation}
    qui impose \'e la vitesse normale de s'\'etendre de mani\'ere
    constante selon les lignes de plus grandes pentes.

\item \underline{Phase de propagation~:\index{Propagation}} elle permet de
    calculer la nouvelle level set. L'\'evolution de la level set
    est donn\'ee par l'\'equation
    \begin{equation}
        \frac{\partial\phi}{\partial t}+\bfV \cdot  \nabla\phi  = 0
        \label{eq:1}
    \end{equation}
    Cette \'equation est une \'equation de transport sous le champ
    de vitesse $\bfV$.
    Comme le gradient de la level set repr\'esente la normale au iso-surface,
    il est aussi courant d'\'ecrire~(\ref{eq:1}) sous la forme
    \begin{equation}
        \frac{\partial\phi}{\partial t}+V_{\phi}\left|\nabla\phi\right|=0
        \label{eq:2}
    \end{equation}
    o\`u $V_{\phi}$ est la vitesse normale de la surface.
    L'\'equation~(\ref{eq:2}) est discr\'etis\'ee sur un pas
    de temps pour donner la nouvelle valeur
    de la level set \'e l'issue de ce pas de temps.
\item \underline{Phase de r\'e-initialisation~:} la nouvelle valeur de la
    level set
    n'est plus n\'ecessairement rigoureusement
    une distance sign\'ee apr\'es propagation.
    Il est alors bon de la
    r\'e\-initia\-liser en r\'esolvant l'\'equation~(\ref{eq:4})~:
    \begin{equation}
        \left|\nabla\phi\right|=1
        \label{eq:4}
    \end{equation}
\end{itemize}

\ajouter{peut-\'etre faut-il d\'etailler plus d'o\'e viennent ces \'equations...pour moi ce n'est pas clair.De m\'eme,
 l'\'equivalence avec \ref{eq:hamilton38} et \ref{eq:hamilton39} ne saute pas aux yeux  }


Les \'equations (\ref{eq:3}) et (\ref{eq:4}) se pr\'etent mal
\'e une r\'esolution directe, on leur pr\'ef\'ere la recherche
de la solution stationnaire des probl\'emes suivants~:
\begin{equation}
    \frac{\partial
    V_{\phi}}{\partial\tau}+\textrm{sign}(\phi)\frac{\nabla\phi}{\left|\nabla\phi\right|}.\nabla V_{\phi}=0
    \label{eq:hamilton38}
\end{equation}
\begin{equation}
    \frac{\partial\phi}{\partial\tau}+\textrm{sign}\left(\phi\right)\left(\left|\nabla\phi\right|-1\right)=0
    \label{eq:hamilton39}
\end{equation}
Il ne faut pas confondre le param\'etre $\tau$ intervenant dans ces
deux \'equations (temps virtuel vers
une solution stationnaire) et le temps physique $t$ intervenant dans
(\ref{eq:2}).

On constate que les \'equations \'e r\'esoudre pour
la propagation d'une level set sont toutes
de la m\'eme forme (dite de Hamilton-Jacobi)~:
\begin{equation}
\frac{\partial\ldots}{\partial t^*}+F \left|\nabla\ldots\right|=f
\end{equation}
o\'e $t^*$ est le temps r\'eel ou un temps virtuel.
La signification des op\'erateurs $F$ et $f$ est donn\'ee dans le
tableau ci-dessous.
\begin{table}[htbp]
\begin{center}\begin{tabular}{|c|p{6cm}|c|c|c|}
\hline
&
Phase&
$\ldots$&
$F$&
$f$\tabularnewline
\hline
$(1)$&
Extension des vitesses&
$V_{\phi}$&
$\textrm{sign}(\phi)\frac{\nabla\phi}{\left|\nabla\phi\right|}\cdot\frac{\nabla V_{\phi}}{\left|\nabla V_{\phi}\right|}$&
$0$\tabularnewline
\hline
$(2)$&
Propagation des level sets &
$\phi$&
$V_{\phi}$&
$0$\tabularnewline
\hline
$(3)$&
R\'einitialisation&
$\phi$&
$\textrm{sign}(\phi)$&
$\textrm{sign}(\phi)$ \tabularnewline
\hline
\end{tabular}\end{center}
\end{table}

La r\'esolution de cette \'equation de Hamilton-Jacobi \index{Hamilton-Jacobi}
sur un maillage \'el\'ements finis quelconque peut se faire
par exemple \'e l'aide d'une m\'ethode de Runge et Kutta du second ordre en temps
et avec une formulation explicite de mise \'e jour des
noeuds \cite{barth-sethian,OsherFedkiw02}.
Il est \'e noter que la fonction \'e propager
est tr\'es  r\'eguli\'ere
(contrairement \'e une approche de type Volume Of Fluid).
Enfin, une condition de stabilit\'e classique (CFL)
p\'ese sur le pas de temps
dont la valeur d\'epend de la taille des \'el\'ements et de
l'op\'erateur $F$.

Les fissures \'etant repr\'esent\'ees par deux level sets, ces derni\'eres
doivent \^etre propag\'ees. L'ordre des op\'erations doit faire l'objet
d'un soin particulier \cite{Moes3DGrowthII}. Outre la
r\'e-initialisation, une r\'e-orthogonalisation est effectu\'ee pour
s'assurer que les gradients des deux level sets restent orthogonaux.




\subsection{Mod\'elisation de discontinuit\'es dans la d\'eriv\'ee du champ}

Les surfaces de discontinuit\'e
dans la d\'eriv\'ee du champ sont importantes  pour mod\'eliser les interfaces entre mat\'eriaux.


Consid\'erons un domaine $\Ome$ compos\'e
de deux mat\'eriaux occupant les r\'egions
$\Ome_1$ et $\Ome_2$, figure~\ref{fig:1dbar}.
Pour mod\'eliser l'interface
mat\'eriau situ\'ee sur
$\Gam = \overline{\Ome_1} \cap  \overline{\Ome_2}$,
il faut que le champ d'approximation en d\'eplacement
contienne des modes discontinus de d\'eformation
au passage de l'interface, cette discontinuit\'e
\'etant une caract\'eristique importante
de la solution du probl\'eme.



    \onefigure{5cm}{1dbar}{Un probl\'eme bi-mat\'eriaux avec une interface
  non maill\'ee.}


Lorsque le maillage respecte l'interface,
la d\'eformation \'el\'ements finis est discontinue au travers
de l'interface par le fait
de la faible r\'egularit\'e ($C^0$)
de l'approximation \'el\'ements finis.
Lorsque le maillage ne respecte pas l'interface,
la base \'el\'ements finis doit \^etre enrichie \'e l'aide de
fonctions \'e d\'eriv\'ee discontinue sur l'interface.
L'approximation enrichie s'\'ecrit
de mani\'ere g\'en\'erale~:
\begin{equation}
\bfu^h(\bfx)  =  \sum_{i\in I} \bfu_i \phi_i(\bfx) +
\sum_{i\in D} \bfa_i \phi_i(\bfx) F(\bfx)
\end{equation}
o\'e $D$ est l'ensemble des noeuds dont au moins un
des \'el\'ements du support
est coup\'e par l'interface mat\'eriau $\Gam$ et
$F(\bfx)$ est une fonction continue mais
\'e d\'eriv\'ee discontinue sur l'interface.
La fonction de niveau $ls(\bfx)$ associ\'ee
\'e l'interface donne la position de cette
interface (contour sur lequel $ls = 0$) et
sa valeur absolue repr\'esente en tout point de $\Ome$
la distance \'e l'interface.
La fonction $|ls(\bfx)|$ est donc continue
mais \'e d\'eriv\'ee discontinue sur l'interface.
C'est un choix possible pour la fonction d'enrichissement
$F$ \cite{Sukumar:inclusion}.


       \onefigure{9cm}{fct_enrich_bw}{Diff\'erents choix pour la fonction
  d'enrichissement associ\'ee \'e une interface entre deux mat\'eriaux.}



Afin de tester ce choix,
prenons le probl\'eme d'un carr\'e compos\'e de deux mat\'eriaux,
figure~\ref{fig:1dbar}. La fonction
$F^1 = |ls(\bfx)|$ est repr\'esent\'ee
figure~\ref{fig:fct_enrich_bw} pour une coupe selon $y=0$.
Les coefficients de Poisson
des deux mat\'eriaux sont pris \'egaux \'e z\'ero pour
rendre le probl\'eme unidimensionnel selon $x$ mais
les modules de Young sont diff\'erents~:
$E_1 = 1$ et $E_2 = 10$. Le bord gauche du carr\'e
est fix\'e alors que le bord droit est soumis \'e un
d\'eplacement de valeur 1.0 selon $x$.
La solution en d\'eplacement de ce probl\'eme est lin\'eaire
de part et d'autre de l'interface et \'e
d\'eriv\'ee normale discontinue
sur l'interface.



Si le maillage se conforme \'e l'interface, la solution \'el\'ements finis
classique co\'encide avec la solution exacte. Si par contre, le
maillage ne se conforme pas \'e l'interface, l'erreur est diff\'erente
de z\'ero. L'erreur relative en \'energie pour le calcul \'el\'ements finis
classique est donn\'ee dans la seconde colonne du tableau~\ref{tab:1D}
pour un maillage de $10 \times 10$ \'el\'ements en fonction du d\'efaut
d'alignement, $\delta$, du maillage par rapport \'e l'interface. Afin
de r\'eduire cette erreur, enrichissons les noeuds dont le support
coupe l'interface \'e l'aide de la fonction $F^1$. Les erreurs sont
donn\'ees dans la troisi\'eme colonne du tableau~\ref{tab:1D}. Notons
que lorsque $\delta=0$ aucun noeud n'est enrichi car cela conduirait
\'e une matrice de raideur singuli\'ere; la fonction $F^1$ \'etant d\'ej\'e
contenue dans la base \'el\'ements finis classique. Les erreurs obtenues
avec X-FEM sont plus faibles que pour le calcul \'el\'ements finis
classique mais encore \'elev\'ees en regard de la simplicit\'e de la
solution exacte.


Consid\'erons une autre fonction
d'enrichissement\index{Enrichissement} $F^1(\bfx)$ r\'egularis\'ee
repr\'esent\'ee figure~\ref{fig:fct_enrich_bw}. Cette fonction co\'encide
avec $|ls(\bfx)|$ sur les \'el\'ements coup\'es par l'interface et est
constante en dehors de ces \'el\'ements. L'erreur \index{Erreur}
obtenue, colonne~4 du tableau~\ref{tab:1D}, est maintenant z\'ero
num\'eriquement. On peut d'ailleurs montrer que cet enrichissement
permet \'e l'approximation de repr\'esenter parfaitement la solution
exacte.

\begin{table}[htb]
\begin{center}

\begin{tabular}{c|c|cc} \hline
%& & \multicolumn{2}{c}{} \\
      & FEM &\multicolumn{2}{c}{X-FEM} \\
$\delta$ & & $F^1$ & $F^1$ + r\'egularisation \\ \hline
%% & & \\ \hline
0.00 & $3.0 \times 10^{-8}$  & $3.0 \times 10^{-8}$   &  $3.0 \times 10^{-8}$ \\ \hline
0.01 & $2.6 \times 10^{-2}$ &  $8.3 \times 10^{-2}$ & $3.0 \times 10^{-8}$  \\
0.05 & $4.9 \times 10^{-2}$ &  $1.6 \times 10^{-1}$ & $2.8 \times 10^{-8}$  \\
0.10 & $5.6 \times 10^{-2}$ &  $1.8 \times 10^{-1}$ & $2.1 \times 10^{-8}$  \\
0.15 & $5.1 \times 10^{-2}$   &  $1.8 \times 10^{-1}$ & $3.8 \times 10^{-8}$  \\
0.19 & $2.2 \times 10^{-2}$   &  $1.6 \times 10^{-1}$ & $3.6 \times 10^{-8}$  \\ \hline
\end{tabular}
\vspace*{0.1in}
\caption{Erreur relative en \'energie en fonction de la position
de l'interface (probl\'eme figure~\ref{fig:fct_enrich_bw})
pour le calcul \'el\'ements finis classique (FEM) et le calcul par
X-FEM pour deux types de fonction d'enrichissement.}
\label{tab:1D}
\end{center}
\end{table}


Consid\'erons maintenant un exemple o\`u l'interface mat\'eriau n'est pas
rectiligne. La figure~\ref{fig:bimaterial} montre le probl\'eme d'une
inclusion \'elastique circulaire. Le chargement en d\'eplacement est
normal \'e la surface ext\'erieure de rayon $b$
\cite{Sukumar:inclusion}. Dans le mod\'ele num\'erique, on consid\'ere un
carr\'e ($L \times L$, $L = 2$) avec une inclusion circulaire en son
centre de rayon $a=0.4$. Sur le bord de ce carr\'e, les tractions
exactes correspondant au probl\'eme figure~\ref{fig:bimaterial} (avec
$a = 0.4$ et $b = 2.0$) sont impos\'ees.


\onefigure{6cm}{bimaterial}{Le probl\'eme d'une inclusion circulaire \'elastique.}



Des maillages quasi-uniformes d'\'el\'ements
triangulaires \'e trois noeuds et de plus en plus raffin\'es sont
utilis\'es pour \'etudier la convergence de l'erreur
en \'energie. La figure \ref{fig:graph2Dinclusion} donne la
convergence \index{Convergence} en \'energie dans diff\'erents cas de figure.
La courbe ``FEM'' correspond \'e une approche \'el\'ements finis
classique dans laquelle le maillage respecte l'interface.
La convergence est d'ordre 1 conform\'ement \'e la pr\'evision
th\'eorique. La courbe ``FEM non conforming'' correspond \'e un
calcul \'el\'ements finis standard avec un maillage ne respectant
pas l'interface. La convergence est tr\'es m\'ediocre et explique la
n\'ecessit\'e dans l'approche \'el\'ements finis de mailler les
interfaces
mat\'eriaux.
La courbe ``XFEM 1 + smoothing'' correspond \'e l'enrichissement
avec $F^1$ r\'egularis\'e \index{Enrichissement!r\'egularis\'e}.
La fonction  $F^1$ r\'egularis\'ee
est \'egale \'e  la valeur absolue de la level set sur les
\'el\'ements coup\'es par l'interface mat\'eriau et est
``la plus constante possible'' en dehors de ces
\'el\'ements \cite{Sukumar:inclusion}.
Enfin, la courbe ``XFEM 2'' donne la convergence pour un nouvel
enrichissement propos\'e dans \cite{MoesCloirec02}.
La fonction d'enrichissement s'\'ecrit
\begin{equation}
F^2(\bfx) = \sum_i  |ls_i| N_i(\bfx)  - | \sum_i ls_i N_i(\bfx) |
\label{eq:5}
\end{equation}
Il s'agit de la diff\'erence entre l'interpol\'ee des valeurs
absolues nodales de la level set et la valeur absolue de la level
set.
Ce dernier enrichissement donne une convergence qui semble optimale.
Cette fonction d'enrichissement n'est non nulle que sur les
\'el\'ements coup\'es par l'interface. Elle est illustr\'ee sur la
figure~\ref{fig:fct_enrich_bw} (fonction $F^2$).



 \onefigure{13cm}{graph2Dinclusion_3}{Convergence de l'erreur
pour une inclusion circulaire. La valeur de $\alpha$ indique le taux de
convergence en fonction de la taille moyenne des \'el\'ements
(calcul\'ee par $h=\sqrt{2 A/N}$ o\`u $A$ est l'aire du domaine et N le
nombre d'\'el\'ements).}



La m\'eme \'etude de convergence a \'et\'e r\'ealis\'ee en 3D pour une
inclusion sph\'erique.
Les convergences observ\'ees sont donn\'ees sur la
figure~\ref{fig:graph3Dinclusion}. On note \'e nouveau que
l'enrichissement (\ref{eq:5}) semble donner une convergence optimale.


\onefigure{13cm}{graph3Dinclusion_4}{Convergence de l'erreur
pour une inclusion sph\'erique. La valeur de $\alpha$ indique le taux de
convergence en fonction de la taille moyenne des \'el\'ements
(calcul\'ee par $h=(6 V/N)^{1/3}$ o\`u $V$ est le volume du domaine
 du domaine et N le nombre d'\'el\'ements).}
